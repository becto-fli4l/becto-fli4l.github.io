% Last Update: $Id: chrony_main.tex 51350 2018-02-17 11:16:50Z kristov $
\marklabel{sec:opt-chrony}
{
\section {CHRONY - Network Time Protocol Server/Client}
}

OPT\_CHRONY erweitert fli4l um das \jump{url:chronyntporg}{Network Time
Protocol} (NTP). Dies ist nicht mit dem \emph{normalen} Time Protokoll zu
verwechseln, welches das alte OPT\_TIME bereitstellt. Die Protokolle sind nicht
kompatibel und somit werden gegebenenfalls neue Client-Programme, die NTP
beherrschen, benötigt. Falls man nicht auf das einfache Time Protokoll
verzichten kann, so läßt sich dieses Protokoll zusätzlich aktivieren.\\
OPT\_CHRONY arbeitet sowohl im  Server, als auch im Client Modus. In der
Funktion des Client gleicht es die Zeit des fli4l mit Zeitreferenzen (Time
Server) im Internet ab. In der Grundeinstellung nutzt OPT\_CHRONY bis zu drei
verschiedene Time Server aus dem Fundus von
\jump{url:chronypoolntporg}{pool.ntp.org}. Es ist jedoch auch möglich, über die
Konfigurationsdatei eine andere Auswahl an Time Servern zu treffen.
So ist es beispielsweise sinvoll Server aus Europa zu wählen. Möglich ist das,
indem man als Server de.pool.ntp.org angibt, wenn der Router bzw. der Provider 
in Deutschland ist. Weitere Informationen dazu auf der Webseite von 
\jump{url:chronypoolntporg}{pool.ntp.org}.\\

In der Funktion des Server dient OPT\_CHRONY als Zeitreferenz für das lokale
Netzwerk (LAN). NTP arbeitet auf Port 123.

Chrony zeichnet sich dadurch aus, dass es keine fortlaufende Verbindung zum
Internet benötigt. Sobald die Verbindung getrennt wird (offline), erhält
chrony hiervon Kenntnis und stellt den Zeitabgleich mit den Internet Time
Servern ein. Somit löst chrony keinen neuen Verbindungsaufbau aus. Weiterhin
verhindert chrony nicht die automatische Verbindungsunterbrechung, falls die
\var{HUP\_TIMEOUT}, also die Zeit, in der keine Daten mit dem Internet
ausgetauscht werden, erreicht wurde.

Damit der Zeitabgleich reibungslos funktioniert, sollte folgendes beachtet
werden:
\begin{itemize}
  \item Chrony erwartet, dass die BIOS-Uhr in der Zeitzone UTC läuft. \\
        UTC = Deutsche Zeit minus 1 (Winter) bzw. 2 (Sommer) Stunde(n)
  \item Seit der Version 2.1.12 setzt Chrony die Uhrzeit mit der ersten
        Verbindung zum Internet korrekt, auch wenn der Zeitunterschied 
	sehr groß sein sollte (beispielsweise bei defekter Mainboardbatterie).
  \item Sollte das BIOS Jahreszahlen nach 1999 nicht richtig darstellen können 
        (Year 2000 Bug) bzw. die Implementation der BIOS Uhr fehlerhaft sein,
	so muß \jump{OPTY2K}{\var{OPT\_Y2K='yes'}} aktiviert werden!
\end{itemize}

Es können nur Time Server im Internet über die Default Route (\var{0.0.0.0/0})
erreicht werden, da nur die Default Route Chrony in den online Zustand
versetzt. Ist der Router als LAN-Router konfiguriert, also keine DSL oder
ISDN Circuits definiert, ist Chrony permanent im online Zustand.

\textbf{Disclaimer: }\emph{Der Autor gibt weder eine Garantie auf die
Funktionsfähigkeit des OPT\_\-CHRONY, noch haftet er für Schäden, z.B.
Datenverlust, die durch den Einsatz von OPT\_\-CHRONY entstehen.}


\marklabel{sec:konfigchrony}{
\subsection {Konfiguration des OPT\_CHRONY}
}

Die Konfiguration erfolgt, wie bei allen fli4l Opts, durch Anpassung der Datei\\
\var{Pfad/fli4l-\version/$<$config$>$/chrony.txt} an die eigenen Anforderungen.
Jedoch sind fast alle Variablen des OPT\_\-CHRONY optional. Optional heißt, die
Variablen können, müssen aber nicht in der Konfigurationsdatei auftauchen.
Somit ist die chrony Konfigurationsdatei im Auslieferungszustand fast leer und
die optionalen Variablen sind sinnvoll vorbelegt. Möchte man dennoch eine
anderen Konfiguration, müssen die Variablen von Hand eingefügt werden. Im
weiteren erfolgt nun die Beschreibung der einzelnen Variablen: 


\begin{description}

\config {OPT\_CHRONY}{OPT\_CHRONY}{OPTCHRONY}

  Default: \var{OPT\_CHRONY='no'}

  Die Einstellung \var{'no'} deaktiviert das OPT\_CHRONY vollständig. Es werden
  keine Änderungen an dem fli4l Bootmedium bzw. dem Archiv \var{opt.img}
  vorgenommen. Weiter überschreibt das OPT\_CHRONY grundsätzlich keine anderen
  Teile der fli4l Installation, mit einer Ausnahme. Es wird die Filterdatei
  ausgetauscht, die dafür sorgt, das Anfragen von außen nicht als Traffic
  gewertet werden (fli4l legt sicher nach erreichen der Hangup Time auf). Die
  neue Filterdatei legt fest, dass der chrony-Traffic ebenfalls nicht mitgezählt
  wird, somit legt der Router weiterhin sicher auf.\\
  Um OPT\_CHRONY zu aktivieren, ist die Variable \var{OPT\_CHRONY} auf 
  \var{'yes'} zu setzen. 

  
\config {CHRONY\_TIMESERVICE}{CHRONY\_TIMESERVICE}{CHRONYTIMESERVICE}

  Default: \var{CHRONY\_TIMESERVICE='no'}

  Mit \var{CHRONY\_TIMESERVICE} kann ein weiteres Protokoll zur Zeitübermittlung
  aktiviert werden. Dieses ist nur dann nötig, wenn die lokalen Rechner nicht
  mit NTP arbeiten können. Das zusätzliche Protokoll ist RFC 868 konform und
  arbeitet auf Port 37. Wenn immer möglich, sollte NTP vorgezogen werden.

  Einen herzlichen Dank an Christoph Schulz, der das Programm \texttt{srv868}
  beigesteuert hat.


\config {CHRONY\_TIMESERVER\_N}{CHRONY\_TIMESERVER\_N}{CHRONYTIMESERVERN}

  Default: \var{CHRONY\_TIMESERVER\_N='3'}

  \var{CHRONY\_TIMESERVER\_N} legt die Anzahl der als Referenz benutzten Time
  Server fest.  Der Anzahl entsprechend sind \var{CHRONY\_TIMESERVER\_x}
  Variablen anzulegen. Der Index \var{x} muß fortlaufend bis zur Gesamtanzahl
  heraufgezählt werden.\\
  In der Grundeinstellung nutzt chrony drei Internet Time Server aus dem
  Fundus von \jump{url:chronypoolntporg}{pool.ntp.org}.
  

\config {CHRONY\_TIMESERVER\_x}{CHRONY\_TIMESERVER\_x}{CHRONYTIMESERVERx}

  Default: \var{CHRONY\_TIMESERVER\_x='pool.ntp.org'}
  
  Mit \var{CHRONY\_TIMESERVER\_x} kann eine eigene Liste von Internet Time
  Servern angelegt werden. Die Time Server können sowohl durch ihre IP als
  auch über ihren DNS Namen spezifiziert werden.

\config {CHRONY\_LOG}{CHRONY\_LOG}{CHRONYLOG}

  Default: \var{CHRONY\_LOG='/var/log/chrony'}

  \var{CHRONY\_LOG} bezeichnet das Verzeichnis, in dem chrony NTP-Messwerte
  und Informationen über vorgenommene Zeitkorrekturen ablegt.

\end{description}
  
\marklabel{sec:chronysupport}{
\subsection{Support}
}
Support wird nur im Rahmen der \jump{url:chronyfli4lnews}{fli4l
Newsgroups} geleistet. 
\subsection{Literatur}

\marklabel{url:chronysource}{
 Homepage von chrony: \altlink{http://chrony.tuxfamily.org/}
 }

\marklabel{url:chronyntporg}{
 NTP: The Network Time Protocol: \altlink{http://www.ntp.org/}
 }

\marklabel{url:chronypoolntporg}{
 pool.ntp.org: public ntp time server for everyone: \altlink{http://www.pool.ntp.org/de/}
 }

\marklabel{url:rfc1305}{
 RFC 1305 - Network Time Protocol (Version 3) Specification, Implementation:\\
 \indent\altlink{http://www.faqs.org/rfcs/rfc1305.html}
 }

\marklabel{url:chronyfli4lnews}{
 fli4l Newsgroups und ihre Spielregeln: \altlink{http://www.fli4l.de/hilfe/newsgruppen/}
}
