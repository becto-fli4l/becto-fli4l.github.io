% Do not remove the next line
% Synchronized to r51350

\marklabel{sec:opt-chrony}
{
\section {CHRONY~- Protocole serveur/client pour la diffusion de l'heure sur le réseau}
}

OPT\_CHRONY à été développé pour fli4l avec le protocole (NTP)
\jump{url:chronyntporg}{Network Time Protocol}. Ne pas confondre avec le
protocole \emph{normal} Time, lequel est disponible dans l'ancien OPT\_TIME. Les
deux protocoles ne sont pas compatibles, ainsi le nouveau protocole est
nécessaire si vous avez un programme client qui possède le pro. NTP. Si vous ne
voulez pas renoncer au protocole normal Time, il sera possible d'activer ce
protocole en plus du NTP. \\ OPT\_CHRONY fonctionne en mode serveur et en mode
client. En mode client, fli4l aura la même heure référencé sur le (serveur Time)
d'Internet. En utilisant le réglage de base, vous pouvez paramétrer dans
OPT\_CHRONY trois serveur-Time différents sur le site
\jump{url:chronypoolntporg}{pool.ntp.org}. Il est également possible, de
sélectionner dans le fichier de configuration d'autres serveurs-Time. Vous
pouvez choisir, par exemple le serveur Europe sinvoll. Il est possible avec le
serveur pool.ntp.org d'indiquer, si le routeur ou le fournisseur d'accès
est en Allemagne. Pour plus d'informations voir le site
\jump{url:chronypoolntporg}{pool.ntp.org}. \\

En mode serveur OPT\_CHRONY est utilisé comme horloge de référence pour le
réseau local (LAN). Le pro. NTP fonctionne sur le port 123.

Chrony se distingue par le faite qu'il n'est pas connecté en permanence sur
Internet. Dès que la connexion est arrêtée (off-line), chrony compare l'horloge
par rapport au serveur Time reçut sur Internet. Ainsi chrony ne déclenche aucune
nouvelle connexion. En outre, le temps de reconnexion automatique de chrony est
régler par la variable\var{HUP\_TIMEOUT}, c'est-à-dire durant la période qui est
indiqué dans celle-ci, aucune données ne sera échangées sur Internet.

Pour que l'heure de référence fonctionne sans problème, respectez les points
suivants~:

\begin{itemize}
  \item Chrony attend que l'heure du BIOS soit dans le fuseau horaire UTC. \\
        UTC = Heure Française, plus 1 heure en (hiver), plus 2 heures en (été)
  \item Depuis la version le 2.1.12, Chrony corrige l'heure avec la première
        connexion sur Internet, même si la différence de temps est très grande
        (batterie de la carte-mère défectueuse).
  \item Si le BIOS n'indique pas correctement les chiffres de l'année 1999
        (Bug de l'année 2000) ou l'implémentation défectueux de l'heure du BIOS,
        la variable \jump{OPTY2K}{\var{OPT\_Y2K='yes'}} doit être activés~!
\end{itemize}

Il est possible atteindre le serveur Time sur Internet avec le routage par
défaut (\var{0.0.0.0/0}), étant donné que le routage par défaut de Chrony
on-line change d'état. Si le routeur est configuré comme un routeur LAN, donc
sans Circuit DSL et RNIS défini, Chrony sera alors en permanence en état on-line.

\textbf{Mentions légales: }\emph{l'auteur ne donne aucune garantie sur la capacité
de fonctionnement d'OPT\_\-CHRONY, il ne sera pas responsable des dégâts, par
exemple de la perte de données qui peut apparaître avec l'utilisation d'OPT\_\-CHRONY.}


\marklabel{sec:konfigchrony}{
\subsection {Configuration de l'OPT\_CHRONY}
}

La configuration se fait, comme tous les autres Opts de fli4l, en paramétrant le
fichier.\\ \var{Disque/fli4l-\version/$<$config$>$/chrony.txt} selon vos besoin.
La plus par les variables d'OPT\_\-CHRONY sont optionnelles. Optionnelle veut dire
que les variables, peuvent ne pas apparaître dans le fichier de configuration.
C'est pour cela que le fichier de configuration chrony est presque vide, les
variables optionnelles sont logiquement préconfigurées. Pourtant si vous voulez,
une autre configuration, les variables doivent être insérées à la main. Nous 
pouvons voir maintenant, la description des variables séparéments~:

\begin{description}

\config {OPT\_CHRONY}{OPT\_CHRONY}{OPTCHRONY}

  Par défaut~: \var{OPT\_CHRONY='no'}

  Avec le paramètre \var{'no'} OPT\_CHRONY sera complètement désactive. Il n'y
  aura aucun changement sur le média de boot de fli4l ou dans l'archive
  \var{opt.img}. Sur le principe OPT\_CHRONY ne remplace aucun élément dans
  l'installation de fli4l, à une exception près. Il change le fichier de filtrage,
  et fait en sorte, que les demandes extérieur de chrony ne soit pas considéré
  comme du trafic (fli4l accéde au site sécurisé avant de raccroché).
  Le nouveau fichier de filtrage définis le trafic de chrony pour qu'il ne soit
  pas pris en compte par fli4l, ainsi le routeur raccrochera sûrement. \\
  Pour activer OPT\_CHRONY, mettez la variable \var{OPT\_CHRONY} Sur \var{'yes'}.

\config {CHRONY\_TIMESERVICE}{CHRONY\_TIMESERVICE}{CHRONYTIMESERVICE}

  Par défaut~: \var{CHRONY\_TIMESERVICE='no'}

  Avec la variable \var{CHRONY\_TIMESERVICE}, un autre Protocole peut être activé
  pour la transmission du temps de référence. Il est nécessaire de le modifier
  seulement si les ordinateurs locaux ne peuvent pas fonctionner avec NTP. Le
  protocole RFC 868 supplémentaire est compatible, il fonctionne sur le port 37.
  Si c'est possible toujours utiliser, le protocole NTP.

  Merci bien à Christoph Schulz qui a contribué au programme \texttt{srv868}.

\config {CHRONY\_TIMESERVER\_N}{CHRONY\_TIMESERVER\_N}{CHRONYTIMESERVERN}

  Par défaut~: \var{CHRONY\_TIMESERVER\_N='3'}

  Avec la variable \var{CHRONY\_TIMESERVER\_N} vous indiquez le nombre de serveur
  de temps de référence. Vous devez placer les noms de serveurs dans la variable
  \var{CHRONY\_TIMESERVER\_x}, les noms doivent correspondre à la quantité indiqué
  ici. Vous devez changer l'index \var{x} par rapport à la quantité croissante et
  totale des serveurs demandés.\\
  Dans le réglage de base chrony utilise trois serveurs de temps de référence
  sur Internet qui sont associés au site \jump{url:chronypoolntporg}{pool.ntp.org}.

\config {CHRONY\_TIMESERVER\_x}{CHRONY\_TIMESERVER\_x}{CHRONYTIMESERVERx}

  Par défaut~: \var{CHRONY\_TIMESERVER\_x='pool.ntp.org'}

  Avec la variable \var{CHRONY\_TIMESERVER\_x} vous indiquez votre propre liste
  de serveur de temps de référence sur Internet. Les serveurs de temps de
  référence peuvent être spécifiés par leurs adresses IP et aussi par leurs
  noms de DNS.

\config {CHRONY\_LOG}{CHRONY\_LOG}{CHRONYLOG}

  Par défaut~: \var{CHRONY\_LOG='/var/log/chrony'}

  Dans la variable \var{CHRONY\_LOG} vous indiquez le répertoire dans lequel chrony
  stockera les données métriques NTP et des informations sur la correction de l'horloge.

\end{description}


\marklabel{sec:chronysupport}{
\subsection{Aide}
}
support technique uniquement dans le cadre de fli4l \jump{url:chronyfli4lnews}{fli4l
Newsgroups}.


\subsection{Littératures}

\marklabel{url:chronysource}{
 Page d'accueil de chrony~: \altlink{http://chrony.tuxfamily.org/}
 }

\marklabel{url:chronyntporg}{
 NTP: The Network Time Protocol~: \altlink{http://www.ntp.org/}
 }

\marklabel{url:chronypoolntporg}{
 pool.ntp.org: public ntp time server for everyone~: \altlink{http://www.pool.ntp.org/fr/}
 }

\marklabel{url:rfc1305}{
 RFC 1305~- Network Time Protocol (Version 3) Specification, Implementation~:\\
 \indent\altlink{http://www.faqs.org/rfcs/rfc1305.html}
 }

\marklabel{url:chronyfli4lnews}{
 Les newsgroups de fli4l et sont règlement~: \altlink{http://www.fli4l.de/fr/aide/newsgroup/}
}
