% Do not remove the next line
% Synchronized to r29817
\section{Merci}

L'équipe fli4l tient à remercier Alexander Krause (\email{admin@erazor-zone.de})
qui a créé et maintenu ce paquetage un long moment. Sans sa collaboration
constructive et sa coopération, il n'aurait pas été possible de modifier
convenablement ce paquetage pour les nouvelles versions fli4l.

\section{Licence}

Copyright \copyright  2002-2004 Alexander Krause (\email{admin@erazor-zone.de}) \\
Copyright \copyright  2004-2010 Christoph Schulz (\email{fli4l@kristov.de}) \\
Copyright \copyright  2010-     Das fli4l-Team (\email{team@fli4l.de}) \\

Ce programme est un logiciel libre. Vous pouvez l'utiliser selon les conditions
du GNU General Public License, telle que publiée par Free Software Foundation,
pour les conditions de redistribution et/ou de modification, voir la version 2
de la licence, ou (si vous le souhaitez) toute version ultérieure.

La publication de ce programme est distribué dans l'espoir qu'il sera utile,
mais SANS AUCUNE GARANTIE - sans même la garantie implicite de COMMERCIALISATION
ou D'ADAPTATION À UN USAGE PARTICULIER. Les détails peuvent être trouvés dans
la GNU General Public License.

Vous devriez avoir une copie du GNU General Public License avec ce programme.
Sinon, écrivez à

\begin{verbatim}
		Free Software Foundation Inc.
		59 Temple Place
		Suite 330
		Boston MA 02111-1307 USA.
\end{verbatim}

Le texte GNU General Public License est également publié sur Internet
à cette adresse \altlink{http://www.gnu.org/licenses/gpl.txt}.
Une traduction non officielle en français se trouve à cette adresse
\altlink{http://www.linux-france.org/article/these/gpl.html}.
Cependant, cette traduction est destiné à mieux comprendre la GPL,
la version anglaise fait foi.
