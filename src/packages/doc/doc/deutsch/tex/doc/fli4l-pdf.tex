% Last Update: $Id: fli4l-pdf.tex 43077 2015-11-27 13:50:53Z cspiess $
\documentclass[twoside=false,index=totoc,listof=totoc,DIV12]{scrbook}
\usepackage[ngerman]{babel}
\usepackage[utf8]{inputenc}
\usepackage[T1]{fontenc}
\usepackage{lmodern}
\usepackage{makeidx}
\usepackage{graphicx}
\usepackage{longtable}
\usepackage{multirow}
\usepackage[pdftex,pdfstartview={XYZ null null 1},colorlinks=true,linkcolor=blue,urlcolor=blue,breaklinks=true,plainpages=false]{hyperref}
\usepackage{verbatimfiles}
\usepackage{alltt}
\usepackage{html}
\usepackage{upquote}
\usepackage{fancyvrb}
\usepackage{amssymb}
\usepackage{mathabx}

\hypersetup{bookmarksopen=true,bookmarksopenlevel=0}
\hypersetup{pdfdisplaydoctitle=true}

\input{version.tex}

%% Beschreibung einer Konfigurationsvariablen
%%
%% \config{name}{index eintrag}{label}{beschreibung}
%%
%% 1.Parameter = Name der Konfigurationsvariablen
%% 2.Parameter = Eintrag für den Index
%% 3.Parameter = Eintrag für entsprechendes Label, auch für hyperref
%% 4.Parameter = Beschreibung der Variablen
%% Bemerkungen: Legt auch ein Label für PDF- und HTML-Links an
%%

% Klammern einer Variablen im fließenden Text
% \var{variable}

% Einen Link erstellen, der die URL anzeigen
%
% \altlink{url}
%
% 1. Parameter: eine beliebige externe URL
%
% \link{url}
%
% 1. Parameter: eine beliebige externe URL
%
% N.B. \link ist 'deprecated'. Verwende stattdessen \altlink.

% Eine Warnung in den Text einfügen
%
% \achtung{text}
%
% 1. Parameter: Text der Warnung

% Label setzen, sowohl für Druck als auch PDF
%
% \marklabel{Name}{zugehöriger Text}
%
% 1. Parameter: Name des labels
% 2. Parameter: Text, z.B. der Anfang eines Kapitel, den man gleich
%               mit einem Label versehen möchte
% BSP: \marklabel{OPTSYSLOG}{\section{OPT\_SYSLOG - ...}}

% Link (Referenz) auf Label erzeugen
%
% \jump{Name}{Linktext}
% \smalljump{Name}{Linktext}
%
% 1. Parameter: Name eines mit marklabel erzeugten Labels
% 2. Parameter: Linktext
%
% Bsp: \jump{OPTSYSLOG}{siehe \var{OPT\_\-SYSLOG}}
%
% Anmerkung: \jump fügt beim pdf-Output zusätzlich zum Link noch die
%            Seitennummer hinzu

%\renewcommand{\index}{\index{\texttt}}
\newcommand{\configvar}[2]{\item[#1]#2}
\newcommand{\achtung}[1]{\vspace{1.0ex}\textbf{#1}\vspace{1.0ex}}
\newcommand{\wichtig}[1]{\vspace{1.0ex}\textbf{Wichtig:} \emph{#1}\vspace{1.0ex}}
\newcommand{\email}[1]{\mbox{E-Mail:~\href{mailto:#1}{\texttt{#1}}}}
\newcommand{\config}[4]{\index{#2}\item[#1]\html{\hypertarget{#3}{}\phantomsection}\label{#3}#4}
\newcommand{\configlabel}[2]{\index{#1}\html{\hypertarget{#2}{}\phantomsection}\label{#2}}
\newcommand{\altlink}[1]{\href{#1}{\small\texttt{\nolinkurl{#1}}}}
\newcommand{\link}[1]{\href{#1}{{\small\texttt{#1}}}}
\newcommand{\marklabel}[2]{\latex{#2\label{#1}}\html{\hypertarget{#1}{#2}\label{#1}}}
\newcommand{\jump}[2]{\smalljump{#1}{#2} \mbox{(Seite~\pageref{#1})}}
\newcommand{\smalljump}[2]{\latex{\hyperrefhyper[{#1}]{#2}}\html{\hyperlink{#1}{#2}}}
\newcommand{\enter}{$\dlsh$}
\newcommand{\pfeil}{$\Rightarrow$}
\newcommand{\var}[1]{{\small\texttt{#1}}}
\newcommand{\pdfbkmrk}[3]{\pdfbookmark[#1]{#2}{#3}}
\newcommand{\flhypersetup}[1]{\hypersetup{#1}}
\newenvironment{example}{\small}{\normalsize}
