% Synchronized to r31455
\marklabel{sec:opt-usercmd}
{
\section {OPT\_USERCMD - Execution Of User Defined Commands At System Start/Shutdown}
}

\subsection{Configuration}

Opt\_usercmd can execute arbitrary commands and scripts after boot or shutdown.
Possible examples would be:
\begin{itemize}
    \item Mounting of additional file systemf (i.e. when booting from a CD).
    \item Definiton of additional router beside the basic configuration.
    \item Ping a host (i.e. trigger automatic connections when using the option \var{DIALMODE}='auto').
\end{itemize}

You may include own files residing in the config directory into the fli4l image.

The following entries are used:

\begin{description}

\config{OPT\_USERCMD}{OPT\_USERCMD}{OPTUSERCMD}

        Default Setting: \var{OPT\_USERCMD}='no'

    \var{OPT\_USERCMD}='yes' activates the package. Only then the configured
    commands and scripts will be excuted on system start resp. shutdown.


\config{USERCMD\_BOOT\_N}{USERCMD\_BOOT\_N}{USERCMDBOOTN}

        Default Setting: \var{USERCMD\_BOOT\_N}='0'

    Number of commands/scripts to be executed after system start (boot).
    The rows will be processed consecutively.

\config{USERCMD\_BOOT\_x}{USERCMD\_BOOT\_x}{USERCMDBOOTx}

        Default Setting: \var{USERCMD\_BOOT\_x}='echo user-defined boot-command here'

    Command to be executed at system startup.

    \wichtig{For each entry only one command may be entered. If more commands are placed
    in a row, the have to be separated with a semicolon (See examples)}


\config{USERCMD\_HALT\_N}{USERCMD\_HALT\_N}{USERCMDHALTN}

        Default Setting: \var{USERCMD\_HALT\_N}='0'

    Number of commands/scripts to be executed when shutting down the system (halt, shutdown, reboot).
    The rows will be processed consecutively.

\config{USERCMD\_HALT\_x}{USERCMD\_HALT\_x}{USERCMDHALTx}

        Default Setting: \var{USERCMD\_HALT\_x}='echo user-defined halt-command here'

    Command to be executed at system shutdown.

    \wichtig{For each entry only one command may be entered. If more commands are placed
    in a row, the have to be separated with a semicolon (See examples)}

\config{USERCMD\_FILE\_N}{USERCMD\_FILE\_N}{USERCMDFILEN}

        Default Setting: \var{USERCMD\_FILE\_N}='0'

   Sometimes it may be necessary to include own files in fli4l's
   opt.img (it is currently not possible to integrate files in rootfs.img)
   without separate opt-package. For these cases a small number of files
   may be included directly from \var{$<$config$>$/etc/usercmd}
   into the fli4l image.

   \var{USERCMD\_FILE\_N}='x' indicates the number of files to be integrated.

\config{USERCMD\_FILE\_x\_SRC}{USERCMD\_FILE\_x\_SRC}{USERCMDFILExSRC}

    Filename of the source file in \var{$<$config$>$/etc/usercmd}. Only files
    residing in this directory will be integrated!

\config{USERCMD\_FILE\_x\_DST}{USERCMD\_FILE\_x\_DST}{USERCMDFILExDST}

    Fully qualified filename to set the destination of the file in
    fil4l's image. A filename specified like this may i.e. on system boot
    be executed by a\\
    \var{USERCMD\_BOOT\_x}='/usr/bin/mystuff.sh'.

\config{USERCMD\_FILE\_x\_MODE}{USERCMD\_FILE\_x\_MODE}{USERCMDFILExMODE}

    File mode of the file in fli4l's image. Unix file modes stick to the
    usual Unix conventions. For details see the developer documentation
    'List of files to copy' (see also \ref{sec:opt_txt}) or check Wikipedia
    \altlink{http://en.wikipedia.org/wiki/Filesystem_permissions\#Traditional_Unix_permissions}.

\config{USERCMD\_FILE\_x\_FLAGS}{USERCMD\_FILE\_x\_FLAGS}{USERCMDFILExFLAGS}

    Complies with flags= indication from the opt/<package>.txt file.
    See also \ref{sec:Dateiformate}.

    The following options exist to convert text file formats before inclusion in the image:
    \newline\newline
    \begin{tabular}{lp{6cm}}
        \emph{utxt} & Conversion to Unix format\\
        \emph{dtxt} & Conversion to DOS format\\
        \emph{sh}   & Shell scripts: Conversion to Unix format, removal of unnecessary characters,
        recommended for shell scripts
    \end{tabular}

\end{description}

\subsection{Examples}

\begin{verbatim}
USERCMD_BOOT_x='fli4lctrl dial pppoe'
\end{verbatim}
Establishes a DSL Internet connection after booting.


\begin{verbatim}

USERCMD_BOOT_x='sleep 60; ip link set tr0 down; ip link set tr0 up'
\end{verbatim}
Sets a one-minute break after system startup, then switches the interface tr0 down
and up again. Can be used to provide time for a Token Ring switch
to allow for booting and then restart the network.

\begin{verbatim}

USERCMD_BOOT_N='1'
USERCMD_BOOT_1='cp /data/log/imond.log /var/log/'
USERCMD_HALT_N='2'
USERCMD_HALT_1='cp /var/log/sys.log /data/log/sys`date +%Y%m%d`.log'
USERCMD_HALT_2='fli4lctrl hangup pppoe ; sleep 2 ; cp /var/log/imond.log /data/log/'
\end{verbatim}
Restores a previously saved imond log file at the end of the boot process.
At shutdown, the syslog is saved with the current date in the file name,
the DSL Internet connection is disconnected and the imond log file is saved.

\subsection{Support}
Support is provided only in the context of the fli4l Newsgroups.

\subsection{Comments}
\begin{itemize}
    \item Using the character \glqq{}'\grqq{} in command lines is not possible.
    \item A command lines must not start with the character \glqq{}-\grqq{}.
\end{itemize}
