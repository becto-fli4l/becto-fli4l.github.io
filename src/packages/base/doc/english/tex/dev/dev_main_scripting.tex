% Synchronized to r35952

\section{Creating Scripts for fli4l}


The following is \emph{not} a general introduction to shell scripts,
everyone can read about this topic on the Internet. It is only about the flil4-specific
things. Further information is available in the various *nix/Linux help pages.
The following links may be used as entry points to this topic:
\begin{itemize}
\item Introduction to shell scripts:
  \begin{itemize}
  \item \altlink{http://linuxcommand.org/writing_shell_scripts.php}
  \end{itemize}
 \item
   Help pages online:
   \begin{itemize}
   \item \altlink{http://linux.die.net/}
   \item \altlink{http://heapsort.de/man2web}
   \item \altlink{http://man.he.net/}
   \item \altlink{http://www.linuxcommand.org/superman_pages.php}
   \end{itemize}
\end{itemize}

\subsection{Structure}

    In the *nix world, it is necessary to begin a script with the name of the interpreter,
    hence the first line is:
\begin{example}
\begin{verbatim}
      #!/bin/sh
\end{verbatim}
\end{example}

    To easily recognize later what a script does and who created it,
    this line should now be followed by a short header, like so:

\begin{example}
\begin{verbatim}
      #--------------------------------------------------------------------
      # /etc/rc.d/rc500.dummy - start my cool dummy server
      #
      # Creation:     19.07.2001 Sheldon Cooper  <sheldon@nerd.net>
      # Last Update:  11.11.2001 Howard Wolowitz <howard@nerd.net>
      #--------------------------------------------------------------------
\end{verbatim}
\end{example}

    Now for the real stuff to start...



\marklabel{dev:sec:config-variables}{
\subsection{Handling of Configuration Variables}
}


    Packages are configured via the file \texttt{config/<PACKAGE>.txt}.
    The \jump{subsec:dev:var-check}{active variables} contained there are
    transferred to the file \texttt{rc.cfg} during creation of the medium.
    This file is processed during router boot before any rc-script
    (scripts under \texttt{/etc/rc.d/}) gets started. The script then
    may access all configuration variables by reading the content of
    \var{\$<variable name>}.

    If the values of configuration variables are needed after booting
    the may be extracted from \texttt{/etc/rc.cfg} to which the configuration
    of the boot medium was written during the boot process. If for example
    the value of the variable \texttt{OPT\_DNS} should be processed in a script
    this can be achieved as follows:

\begin{example}
\begin{verbatim}
    eval $(grep "^OPT_DNS=" /etc/rc.cfg)
\end{verbatim}
\end{example}

    This works efficiently also with multiple variables (with calling
    the \texttt{grep} program only once):

\begin{example}
\begin{verbatim}
    eval $(grep "^\(HOSTNAME\|DOMAIN_NAME\|OPT_DNS\|DNS_LISTEN_N\)=" /etc/rc.cfg)
\end{verbatim}
\end{example}

\marklabel{dev:sec:persistent-data}{
\subsection{Persistent Data Storage}
}

Occasionally a package needs the possibility to store data persistent,
surviving the reboot of the router. For this purpose, the function
\texttt{map2persistent} exists and can be called from a script in
\texttt{/etc/rc.d/}. It expects a variable that contains a path
and a subdirectory. The idea is that the variable is either describing
an actual path~-- then this path is used because the user explicitely
has done so, or the string ``auto''~-- then a subdirectory on a persistent
medium is created corresponding to the second parameter. The function returns
the result in the variable passed by name as the first parameter.
An example will make this clear. \var{VBOX\_SPOOLPATH} is a variable,
that 'contains a path or the string ``auto'' . The call

\begin{example}
\begin{verbatim}
    begin_script VBOX "Configuring vbox ..."
    [...]
    map2persistent VBOX_SPOOLPATH /spool
    [...]
    end_script
\end{verbatim}
\end{example}

results in the variable \var{VBOX\_SPOOLPATH} either not being changed at all
(if it contains a path), or being changed to \texttt{/var/lib/persistent/vbox/spool}
(if it contains the string ``auto''). \texttt{/var/lib/persistent} then points\footnote{
by the use of a so-called ``bind''-mount} to a directory on a non-volatile and writable
storage medium, \var{<SCRIPT>} is the name of the calling script in lowercase
(this name is derived from the first argument of the
\jump{subsec:dev:bug-searching}{\texttt{begin\_script}-call}).
If no suitable medium should exist (which may well be),
\texttt{/var/lib/persistent} is a directory in the RAM disk.

Please note that the path returned by \texttt{map2persistent} is \emph{not}
created automatically~-- The caller has to do that by himself (ie. by calling
\texttt{mkdir -p <path>}).

The file \texttt{/var/run/persistent.conf} allows for checking if
persistent data storage is possible. Example:

\begin{example}
\begin{verbatim}
    . /var/run/persistent.conf
    case $SAVETYPE in
    persistent)
        echo "persistent data storage is possible!"
        ;;
    transient)
        echo "persistent data storage is NOT possible!"
        ;;
    esac
\end{verbatim}
\end{example}

\marklabel{subsec:dev:bug-searching}{
\subsection{Debugging}
}

    For startup scripts it is often useful to run them in debug mode
    in a shell when you are in need to determine where they fail.
    For this purpose, insert the following at the beginning and at the end:

\begin{example}
\begin{verbatim}
      begin_script <OPT-Name> "start message"
      <script code>
      end_script
\end{verbatim}
\end{example}

At the start and at the end of the script the specified text will now appear,
preceded by ``finished''.

If you want to debug the scripts, you must do two things:

\begin{enumerate}

\item You have to set \jump{DEBUGSTARTUP}{\var{DEBUG\_\-STARTUP}} to ``yes''.
\item You have to activate debugging for the OPT. This is usually done by the
  entry
\begin{example}
\begin{verbatim}
      <OPT-Name>_DO_DEBUG='yes'
\end{verbatim}
\end{example}
in the config file.\footnote{Sometimes multiple start-scripts are used,
  which then have different names for their debug-variables. Have a quick look at the scripts for clarification.}
    What happens during runtime is now displayed in detail on screen.
\end{enumerate}


\subsubsection{Further Variables helpful for Debugging}

\begin{description}

  \config{DEBUG\_ENABLE\_CORE}{DEBUG\_ENABLE\_CORE}{DEBUGENABLECORE}

  This variable allows the creation of ``Core-Dumps''. If a program crashes
  due to an error an image of the current state in the file system is stored
  which can be used to analyse the problem. The core dumps are stored under
  \texttt{/var/log/dumps/}.

  \config{DEBUG\_IP}{DEBUG\_IP}{DEBUGIP}

  Activating this variable will log all calls of the program \texttt{ip}.

  \config{DEBUG\_IPUP}{DEBUG\_IPUP}{DEBUGIPUP}

  Setting this variable to ``yes'' will log all executed instructions of the
  \texttt{ip-up}- and \texttt{ip-down}-scripts to syslog.

  \config{LOG\_BOOT\_SEQ}{LOG\_BOOT\_SEQ}{LOGBOOTSEQ}

  Setting this variable to ``yes'' will cause \texttt{bootlogd} to log all console
  output during boot to the file \texttt{/var/tmp/boot.log}. This variable has
  ``yes'' as a default value.

  \config{DEBUG\_KEEP\_BOOTLOGD}{DEBUG\_KEEP\_BOOTLOGD}{DEBUGKEEPBOOTLOGD}

  Normally \texttt{bootlogd} is terminated at the end of the boot process.
  Activating this variable prevents this and thus allows for logging console
  output during the whole runtime.

  \config{DEBUG\_MDEV}{DEBUG\_MDEV}{DEBUGMDEV}

  Setting this variable generates a logfile for the \texttt{mdev}-daemon,
  which is responsible for creating device nodes under \texttt{/dev}.

\end{description}
\subsection{Hints}
\begin{itemize}
\item  It is \emph{always} better to use curly brackets ``\{\ldots\}'' instead of normal
      ones``(\ldots)''. However, care must be taken to ensure
      that after the opening bracket a space or a new line follows before
      the next command and before the closing brackets a Semicolon
      or a new line. For example:

\begin{example}
\begin{verbatim}
        { echo "cpu"; echo "quit"; } | ...
\end{verbatim}
\end{example}

      \noindent is equal to:

\begin{example}
\begin{verbatim}
        {
                echo "cpu"
                echo "quit"
        } | ...
\end{verbatim}
\end{example}


      \item A script may be stopped by an ``exit'' at any time. This is deadly for
	start scripts (\texttt{opt/etc/boot.d/...},
        \texttt{opt/etc/rc.d/...}), stop-scripts (\texttt{opt/etc/rc0.d/...}) and\\
        \texttt{ip-up}/\texttt{ip-down}-scripts (\texttt{opt/etc/ppp/...})
        because the following scripts will not be run as well. If in doubt,
        keep your fingers away.


      \item KISS~-- Keep it small and simple. You want to use Perl for
        scripting? The scripting abilities of fli4l are not enough for you?
        Rethink your attitude! Is your OPT really necessary? fli4l after all
        is ``only'' a router and a router should not really offer server services.


      \item The error message ``: not found'' usually means the script is
        still in DOS format. Another source of errors: the script is not executable.
        In both cases the \texttt{opt/<PACKAGE>.txt} file should be checked whether
        it contains the correct options for ``mode'', ``gid'', ``uid'' and Flags.
        If the script is created during boot, execute ``chmod +x <script name>''.

      \item Use the path \texttt{/tmp} for temporary files. However, it is important
	to keep in mind that there is little space there because it is situated in the
	rootfs-RAM disc! If more space is needed, you have to create and mount your own
	ramdisk. Detailed informations on this topic can be found in the section ``RAM-Disks''
        of this documentation.

    \item In order to create temporary files with unique names you should always append
      the current process-ID stored in the shell variable ``\$'' to the file name.
      \texttt{/tmp/<OPT-Name>.\$\$} hence is a perfect file name,
      \texttt{/tmp/<OPT-Name>} rather not (\texttt{<OPT-Name>} of course has
      to be replaced by the according OPT-Name).

\end{itemize}
