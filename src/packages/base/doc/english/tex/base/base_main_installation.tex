% Synchronized to r30469

\chapter{Setup and Configuration}

\section{Unpacking the archives}

Under Linux:

\begin{verse}\texttt{tar xvfz fli4l-\version.tar.gz}\end{verse}

\noindent If this does not work, try the following:

\begin{verse}\texttt{gzip -d < fli4l-\version.tar.gz | tar xvf -}\end{verse}

If you unpack the current version into a directory which already contains fli4l
files from a previous installation, you should execute \texttt{mkfli4l.sh -c}:

\begin{verse}
    \texttt{cd fli4l-\version}\\
    \texttt{sh mkfli4l.sh -c}
\end{verse}

However, we recommend to use a fresh directory for a new version as you can
easily take over the configuration with a file comparing tool.

If you use a MS~Windows system, you can extract the compressed tar archive e.g.
with WinZip. You have to check, however, that all files are extracted together
\emph{with} their corresponding directories (there is a WinZip setting to
achieve this). Also, you have to disable the so-called ``Smart TAR CR
conversion''  under \emph{Settings \pfeil Configuration}, as some
important files are corrupted during extraction otherwise.

Alternatively, we recommend using the OpenSource application 7-Zip
(\altlink{http://www.7-zip.org/}) which is as powerful as WinZip.

The following files are installed in the subdirectory \texttt{fli4l-\version/}:

\begin{itemize}
\item Documentation:
  \begin{itemize}
  \item doc/deutsch/* German documentation
  \item doc/english/* English documentation
  \item doc/french/*  French documentation
  \end{itemize}

\item Configuration:
  \begin{itemize}
  \item config/*.txt Configuration files, these ones have to be edited to suit your needs
  \end{itemize}

\item Scripts/Procedures:
  \begin{itemize}
  \item mkfli4l.sh Creates boot media or files: Linux/Unix version
  \item mkfli4l.bat Creates boot media: Windows version
  \end{itemize}

\item Kernel/Boot files:
  \begin{itemize}
  \item img/kernel Linux kernel
  \item img/boot*.msg bootscreen texts
  \end{itemize}

\item Additional packages:
  \begin{itemize}
  \item opt/*.txt These ones describe which files will be included in the opt.img archive.
  \item opt/... Optional kernel modules, files, and programs
  \end{itemize}

\item Source code:
  \begin{itemize}
  \item src/* Source code/tools for Linux, see src/README
  \end{itemize}

\item Programs:
  \begin{itemize}
  \item unix/mkfli4l* Creates boot medium: Linux/Unix version
  \item windows/* Creates boot medium: Windows version
  \item unix/imonc* imond client for Unix/Linux
  \item windows/imonc/* imond client for Windows
  \end{itemize}
\end{itemize}

\section{Configuration}
\subsection{Editing the configuration files}

To configurate fli4l, you only have to adapt the files config/*.txt.
We recommend to make a copy of the ``config'' directory and perform
the adaption within this copy. So you will be able to compare your own
configuration with the distributed one later and you are also
able to manage multiple configurations. Comparing two configurations
is relatively simple by using an adequate tool (i.e. ``diff'' under *nix).
Let's assume that your own config files reside in a directory named
my\_config under the fli4l directory. In this case you could execute:
\begin{example}
\begin{verbatim}
    ~/src/fli4l> diff -u {config,my_config}/build/rc.cfg | grep '^[+-]'
    --- config/build/rc.cfg    2014-02-18 15:34:39.085103706 +0100
    +++ my_config/build/rc.cfg        2014-02-18 15:34:31.094317441 +0100
    -PASSWORD='/P6h4iOIN5Bbc'
    +PASSWORD='3P8F3KbjYgzUc'
    -NET_DRV_1='ne2k-pci'
    +NET_DRV_1='pcnet32'
    -START_IMOND='no'
    +START_IMOND='yes'
    -OPT_PPPOE='no'
    +OPT_PPPOE='yes'
    -PPPOE_USER='anonymous'
    -PPPOE_PASS='surfer'
    +PPPOE_USER='me'
    +PPPOE_PASS='my-passwd'
    -OPT_SSHD='no'
    +OPT_SSHD='yes'
\end{verbatim}
\end{example}

You can also see by this example that a simple DSL router can be configured
without much effort, even if you feel at first overwhelmed by the sheer amount of
possible settings.

\subsection{Configuration via a special configuration file}

Due to the module concept of fli4l, the configuration is distributed across
different files. As editing these separate files may become tedious, it is
possible to store the configuration into a single file called
\emph{$<$config~directory$>$/\_fli4l.txt}. This file is read in addition to the
other configuration files and its contents override any settings
found in the other configuration files. Recall the example above: In order to
configure a simple DSL router, we could simply write the following lines into
this file:

\begin{example}
\begin{verbatim}
    PASSWORD='3P8F3KbjYgzUc'
    NET_DRV_N='1'
    NET_DRV_1='pcnet32'
    START_IMOND='yes'
    OPT_PPPOE='yes'
    PPPOE_USER='me'
    PPPOE_PASS='my-passwd'
    OPT_SSHD='yes'
\end{verbatim}
\end{example}

You should avoid to mix both flavours of configuration.

\subsection{Variables}

  You will notice that teh lines of some variables are prefixed with a '\#'.
  and thus are commented. If this is the case a reasonable default setting is 
  already in effect. Those defaults are documented for each variable. If you wish
  to set another value delete the '\#' at the beginning of the line and 
  put your value between the apostrophs.

\marklabel{VARIANTEN}{\section{Setup flavours}}

Previous versions of fli4l only supported booting from a floppy disk
which is not possible anymore due to causes already described. There
are many alternative possible nowadays, amongst them using an USB stick.

Many other boot media (CD, HD, network, Compact-Flash, DoC, \ldots)
exist and fli4l may also be installed permanently on some of them
(obviously only the read-write ones). fli4l may be booted in three
different ways:

\begin{description}
\item [Single Image] The boot loader loads the linux kernel and then fli4l
in a single image. After that, fli4l is able to continue the boot process
without the need to access other boot media. Examples are the boot types
\emph{integrated}, \emph{attached}, \emph{netboot}, and \emph{cd}.
\item [Split Image] The boot loader loads the linux kernel and then a
rudimental fli4l image which mounts the boot medium in a first step, then
loads the configuration and the remaining files from an archive residing
on that mounted medium. Examples for this are the boot types \emph{hd (Type A)},
\emph{ls120}, \emph{attached}, and \emph{cd-emul}.
\item [Installation on a Medium] The boot loader loads the linux kernel and
then a rudimental fli4l image which mounts an existing fli4l installation
without the need to extract any further archives. An example for this is a
type B hard disk installation.
\end{description}

Before you try the more advanced installation procedures you should make yourself
comfortable with fli4l by setting up a minimal version. If you want to use your fli4l
as an answering machine or a HTTP-proxy later on, you already feel confident and have
the experience of setting up a basic running system.

That said, four variants of installations are possible:

\begin{description}
\item[USB-Stick] Router on an USB stick
\item[CD-router] Router on a CD
\item[Netzwerk] Network boot
\item[HD-Installation Typ A] Router on a hard disk, CF, DoC~-- only one FAT-Partition
\item[HD-Installation Typ B] Router on a hard disk, CF, DoC~-- one FAT- and one ext3-Partition
\end{description}

\marklabel{INSTALLTYP0}{\subsection{Router on a USB-Stick}}

USB-Sticks are addressed as harddisks by Linux hence the same explanations
as for the harddisk installation are valid here. Please note that the
according drivers for the USB port have to be loaded via \var{OPT\_\-USB}
in order to access the stick with \var{OPT\_\-HDINSTALL}.

\marklabel{INSTALLTYP0}{\subsection{Router on a CD, or network boot}}

All necessary files are on the boot medium and are extracted to a dynamically
sized RAM disk while booting. Using a minimalistic configuration, it is
possible to run the router with only 64 MiB of RAM. The maximum setup
is only limited by the capacity of the boot medium and available RAM.

\marklabel{INSTALLTYPA}{\subsection{Type A: Router on hard disk---only one FAT partition}}

This corresponds to the CD version, with the only difference of the
files residing on a hard disk instead, the term ``hard disk''
also enclosing Compact Flash from 8 MiB upwards and other devices which are
accessed like hard disks under Linux. As of fli4l 2.1.4, you can also use
DiskOnChip Flash memory from M-Sys or SCSI hard disks.

The limit for the archive opt.img is removed by disk capacity, but all these files
have to be installed into a RAM disk of suitable size during the boot process.
This increases the necessary amount of RAM if you use many software packages.

In order to update software packages (i.e. the archive opt.img and the
configuration rc.cfg over the network), the FAT partition has to provide
enough space for the kernel, the RootFS and TWICE the size of the opt.img
archive! If you also want to enable the recovery option, the required space
is increased one more time by the size of the opt.img archive.

\marklabel{INSTALLTYPB}{\subsection{Type B: Router on hard disk---one FAT and one ext3 partition}}

In contrast to type A, most of the files are not put into the RAM disk.
Instead, they are copied from the opt.img archive to the ext3 partition
on the hard disk at the very first start after the initial installation or an
update. On successive reboots they are loaded from the ext3 partition.
Using this type of installation, the amount of RAM needed for running the
router is the smallest, such that running the router with very low memory is
possible in the majority of cases.

You can find further information on the hard disk installation in the
documentation of the HD package (a separate download) starting
at the description of the configuration variable \var{OPT\_\-HDINSTALL}.
