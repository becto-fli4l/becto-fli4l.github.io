% Synchronized to r54214

\marklabel{CONSOLESETTINGS}{\section{Console settings}}

fli4l can be operated on different hardware platforms. On many of these
platforms it is possible to have a keyboard and a monitor connected to 
interact with fli4l; this inpu/output combination is generally \emph{console}.

fli4l can also be used completely without keyboard and graphics card. 
In order to use the router as well without network access to it and 
to see all kernel boot messages, it is possible to use a console on a 
serial port catching inputs from the serial interface or sending output there. 
This requires the variables
\jump{SERCONSOLE}{\var{SER\_CONSOLE}},
\jump{SERCONSOLEIF}{\var{SER\_CONSOLE\_IF}} und
\jump{SERCONSOLERATE}{\var{SER\_CONSOLE\_RATE}} to be set resp. adapted.

It is also possible to use a console on both keyboard and monitor as well as 
via the serial port.

In general, fli4l provides the option to login and thus a \emph{Shell} on
\emph{any} console giving you the ability to login as the user ``fli4l'' 
with the password defined in \jump{PASSWORD}{\var{PASSWORD}}.

\begin{description}
  \config{CONSOLE\_BLANK\_TIME}{CONSOLE\_BLANK\_TIME}{CONSOLEBLANKTIME}
  
  Defaut Setting: \var{CONSOLE\_BLANK\_TIME=''}
  
  Typically, the Linux kernel activates the console's screen saver after some
  time without console input activity. The variable \var{CONSOLE\_BLANK\_TIME}
  allows you to configure the timeout to be used or to disable the screen saver
  completely (\var{CONSOLE\_BLANK\_TIME}='0').

  \config{BEEP}{BEEP}{BEEP}
  
  Defaut Setting: \var{BEEP='yes'}
  
  {Causes a beep at the end of the boot or shutdown process.

    If you enter `yes' here, there will be a beep at the end of the boot or
    shutdown process. If you suffer from an extreme shortage of space on your
    boot media or if you don't like your router to beep, use `no' instead.}

  \config{SER\_CONSOLE}{SER\_CONSOLE}{SERCONSOLE}
  
    Default Setting: \var{SER\_CONSOLE='no}'

    This variable enables or disables a console on a serial port. The serial 
    console can be operated in three modes:

      \begin{tabular}[h!]{|l|p{9cm}|}
        \hline
        \var{SER\_CONSOLE} & console input/outpute \\
        \hline
        no & Input and output (only) via keyboard and monitor (tty0) \\
        yes & Input and output (only) via serial interface (ttyS0) \\
        primary & Input and output via serial console as well as via 
        keyboard and monitor, output of kernel messages tty0 \\
        secondary & Input and output via serial console as well 
        as via keyboard and monitor, output of kernel messages ttyS0 \\
        \hline
      \end{tabular}
    
    Changing the value of \var{SER\_CONSOLE} affects the router only if you
    also update your boot media or if you perform a remote update of the
    syslinux.cfg file.

    \wichtig{When turning off the serial console, be sure to keep an alternate
    access to the router (SSH or directly from the keyboard and monitor)!}

    You will find further information in the appendix under
    \jump{SERIALCONSOLE}{Serial console}.


  \config{SER\_CONSOLE\_IF}{SER\_CONSOLE\_IF}{SERCONSOLEIF}
  
    Default Setting: \var{SER\_CONSOLE\_IF='0'}
  
  {
    Number of the serial interface for the serial console.

    Enter the number of the interface to which the serial console is connected.
    0 corresponds to ttyS0 under Linux or COM1 under Microsoft Windows.
  }

  \config{SER\_CONSOLE\_RATE}{SER\_CONSOLE\_RATE}{SERCONSOLERATE}
  
  Default Setting: \var{SER\_CONSOLE\_RATE='9600'}
  
  {Transmission rate of the serial port for console output.

    This variable contains the Baud rate to use for transmitting data over
    the serial port. Reasonable values are: 4800, 9600, 19200, 38400, 57600,
    115200.}

\end{description}
