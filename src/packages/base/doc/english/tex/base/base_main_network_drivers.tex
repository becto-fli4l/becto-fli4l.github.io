% Synchronized to r49626

\section{Ethernet network adapter drivers}

\begin{description}
  \config{NET\_DRV\_N}{NET\_DRV\_N}{NETDRVN}

  {Number of needed network adapter drivers.

    If you use the router with ISDN, you typically have one network adapter
    only, hence the default value is 1. However, if you use DSL, you will
    have two network adapters in the majority of cases.

    You have to separate two cases:
    \begin{enumerate}
    \item Both adapters are of the same type. Then you will have to specify
      only one driver communicating with both adapters, hence
      \var{NET\_\-DRV\_\-N}='1'.

    \item The types of the adapters used differ. Then you have to set the
      variable to `2' and to configure the drivers separately for both adapters.
    \end{enumerate}
    }


  \configlabel{NET\_DRV\_x\_OPTION}{NETDRVxOPTION}
  \config{NET\_DRV\_x NET\_DRV\_x\_OPTION}{NET\_DRV\_x}{NETDRVx}

  {This variable contains the name of the driver to be used for the network
   adapter. The default for \var{NET\_\-DRV\_\-1} is to load the driver for a NE2000
    compatible network adapter. More available drivers for a large amount of
    families of network adapters are included in the tables \ref{tabkartentreiber}
    and \ref{tabwlankartentreiber}.

    The 3COM EtherLinkIII network adapter (3c509) has to be configured by the
    DOS tool 3c509cfg.exe, available under:

    \altlink{ftp://ftp.ihg.uni-duisburg.de/Hardware/3com/3C5x9n/3C5X9CFG.EXE}

    It should be used for setting the IRQ and I/O port and, if necessary, the
    type of connection (BNC/TP). The entry \var{NET\_\-DRV\_\-x\_\-OPTION}=''
    can normally be left empty.

    Some ISA adapters require the driver to have additional information in
    order to find the adapter, e.g. the I/O address. This is the case e.g. for
    NE2000 compatible ISA adapters and the EtherExpress16.

    In such a case, you can set
\begin{example}
\begin{verbatim}
        NET_DRV_x_OPTION='io=0x340'
\end{verbatim}
\end{example}

    (or the corresponding numerical value).

    If no options are required, you can leave this variable empty.

    If you need to specify more than one option, you have to separate them
    by blanks, e.g.
\begin{example}
\begin{verbatim}
        NET_DRV_x_OPTION='irq=9 io=0x340'
\end{verbatim}
\end{example}

    If you use two identical network adapters, e.g. NE2000 ISA adapters,
    you have to separate the different I/O ports by commas:

\begin{example}
\begin{verbatim}
        NET_DRV_x_OPTION='io=0x240,0x300'
\end{verbatim}
\end{example}

    No space is allowed before or after the comma!

    This does not work with all network adapter drivers. Some of them need to
    be loaded twice, i.e. you have to use \var{NET\_\-DRV\_\-N}='2'. In this
    case you will have to assign different names to the adapters by using the
    ``-o'' option, e.g.

\begin{example}
\begin{verbatim}
          NET_DRV_N='2'
          NET_DRV_1='3c503'
          NET_DRV_1_OPTION='-o 3c503-0 io=0x280'
          NET_DRV_2='3c503'
          NET_DRV_2_OPTION='-o 3c503-1 io=0x300'
\end{verbatim}
\end{example}

    We recommend to try the ``comma'' method first before falling back to
    loading the driver multiple times.

    Some more examples:

    \begin{itemize}
    \item 1 x NE2000 ISA
\begin{example}
\begin{verbatim}
          NET_DRV_1='ne'
          NET_DRV_1_OPTION='io=0x340'
\end{verbatim}
\end{example}

    \item  1 x 3COM EtherLinkIII (3c509)
\begin{example}
\begin{verbatim}
          NET_DRV_1='3c509'
          NET_DRV_1_OPTION=''
\end{verbatim}
\end{example}
      For this adapter, see also:

    \begin{raggedright}
      \altlink{http://extern.fli4l.de/fli4l_faqengine/faq.php?display=faq&faqnr=132&catnr=7&prog=1}\\
      \altlink{http://extern.fli4l.de/fli4l_faqengine/faq.php?display=faq&faqnr=133&catnr=7&prog=1}\\
      \altlink{http://extern.fli4l.de/fli4l_faqengine/faq.php?display=faq&faqnr=135&catnr=7&prog=1}\par
    \end{raggedright}

    \item 2 x NE2000 ISA
\begin{example}
\begin{verbatim}
          NET_DRV_1='ne'
          NET_DRV_1_OPTION='io=0x320,0x340'
\end{verbatim}
\end{example}
      Here, you will often need to specify the IRQ values:

\begin{example}
\begin{verbatim}
          NET_DRV_1_OPTION='io=0x320,0x340 irq=3,5'
\end{verbatim}
\end{example}

      You should first try the configuration without specifying any IRQs
      and enter IRQs only if the network adapters are not found otherwise.

    \item 2 x NE2000 PCI
\begin{example}
\begin{verbatim}
          NET_DRV_1='ne2k-pci'
          NET_DRV_1_OPTION=''
\end{verbatim}
\end{example}
    \item  1 x NE2000 ISA, 1 x NE2000 PCI
\begin{example}
\begin{verbatim}
          NET_DRV_1='ne'
          NET_DRV_1_OPTION='io=0x340'
          NET_DRV_2='ne2k-pci'
          NET_DRV_2_OPTION=''
\end{verbatim}
\end{example}
    \item 1 x SMC WD8013, 1 x NE2000 ISA
\begin{example}
\begin{verbatim}
          NET_DRV_1='wd'
          NET_DRV_1_OPTION='io=0x270'
          NET_DRV_2='ne2k'
          NET_DRV_2_OPTION='io=0x240'
\end{verbatim}
\end{example}
    \end{itemize}

    You can find the complete list of available drivers in the documentation
    of the respective kernel package.

    \emph{If you need a dummy device, use 'dummy' as your \var{NET\_DRV\_x} and\\
    \jump{IPNETxDEV}{\var{IP\_NET\_x\_DEV}}='dummy$<$number$>$' as your device.}

   }
\end{description}
