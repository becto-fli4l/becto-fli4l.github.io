% Synchronized to r29817

\section{Additional routes (optional)}

\begin{description}

  \config{IP\_ROUTE\_N}{IP\_ROUTE\_N}{IPROUTEN}

  {Number of additional network routes. Additional network routes are mandatory
  if e.g. other routers in the LAN exist which have to be used to access
  other networks.

  Normally, you do not need to specify any other network routes.
  
    Default setting: \var{IP\_\-ROUTE\_\-N}='0'}

  \config{IP\_ROUTE\_x}{IP\_ROUTE\_x}{IPROUTEx}

  {The additional routes \var{IP\_\-ROUTE\_\-1}, \var{IP\_\-ROUTE\_\-2}, \ldots
    are structured as follows:

\begin{example}
\begin{verbatim}
        network/netmaskbits gateway
\end{verbatim}
\end{example}

    In this case, \texttt{network} is the network address, \texttt{/netmaskbits} the net mask
    using the \jump{tab:cidr}{CIDR} notation and \texttt{gateway} the address of
    the router needed for accessing the other network. Obviously, the gateway
    and the fli4l router are required to be in the same network!
    For example, if the network 192.168.7.0 with net mask 55.255.255.0 can
    be accessed through the gateway 192.168.6.99 you have to add the
    following entry:

\begin{example}
\begin{verbatim}
        IP_ROUTE_N='1'
        IP_ROUTE_1='192.168.7.0/24 192.168.6.99'
\end{verbatim}
\end{example}}

    If you use the fli4l router as a pure Ethernet router and not for routing
    Internet traffic, you can use some IP\_ROUTE\_x variable for specifying
    a default route. In order to achieve this, you have to specify `0.0.0.0/0'
    for `network/netmaskbits', as can be seen in the following example.

\begin{example}
\begin{verbatim}
        IP_ROUTE_N='3'
        IP_ROUTE_1='192.168.1.0/24 192.168.6.1'
        IP_ROUTE_2='10.73.0.0/16 192.168.6.1'
        IP_ROUTE_3='0.0.0.0/0 192.168.6.99'
\end{verbatim}
\end{example}

  \end{description}
