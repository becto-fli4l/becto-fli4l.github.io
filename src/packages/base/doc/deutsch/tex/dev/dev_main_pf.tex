% Last Update: $Id: dev_main_pf.tex 49532 2017-12-13 13:16:20Z kristov $

\providecommand{\fwaction}[1]{{\small\textsf{#1}}}
\providecommand{\fwchain}[1]{\texttt{#1}}
\providecommand{\fwtable}[1]{\textsc{#1}}
\providecommand{\fwmatch}[1]{\texttt{#1}}
\providecommand{\fwpktstate}[1]{\texttt{#1}}
\providecommand{\fwloglevel}[1]{\texttt{#1}}

\section{Arbeit mit dem Paketfilter}
\subsection{Hinzufügen von eigenen Ketten und Regeln}

Zur Manipulation des Paketfilters steht eine Reihe von Routinen zur
Verfügung, mit deren Hilfe man Ketten (engl. "`Chains"') und Regeln hinzufügen
und wieder löschen kann. Eine Kette ist eine benannte und geordnete Liste von
Regeln. Es gibt einen Satz vordefinierter Ketten (\fwchain{PREROUTING},
\fwchain{INPUT}, \fwchain{FORWARD}, \fwchain{OUTPUT}, \fwchain{POSTROUTING});
mit Hilfe dieser Funktionen können weitere Ketten nach Bedarf erstellt werden.

\begin{description}
\item [\texttt{add\_chain/add\_nat\_chain <chain>}:]
  Fügt eine Kette zur "`filter"'- oder "`nat"'-Ta\-bel\-le hinzu.
\item [\texttt{flush\_chain/flush\_nat\_chain <chain>}:]
  Entfernt alle Regeln aus einer Kette der "`filter"'- oder "`nat"'-Tabelle.
\item [\texttt{del\_chain/del\_nat\_chain <chain>}:]
  Entfernt eine Kette aus der "`filter"'- oder "`nat"'-Tabelle. Ketten müssen
  leer sein, bevor sie gelöscht werden können, und es darf auch keine Referenz
  mehr auf sie geben. Eine solche Referenz ist z.\,B. eine \fwaction{JUMP}-Aktion,
  deren Ziel eben diese Kette ist.
\item[\texttt{add\_rule/ins\_rule/del\_rule}:] Fügt Regeln
  am Ende einer Kette (\texttt{add\_rule}) bzw. an beliebiger Stelle in einer
  Kette (\texttt{ins\_rule}) ein bzw. löscht Regeln aus einer Kette
  (\texttt{del\_rule}). Ein Aufruf sieht wie folgt aus:

\begin{example}
\begin{verbatim}
    add_rule <table> <chain> <rule> <comment>
    ins_rule <table> <chain> <rule> <position> <comment>
    del_rule <table> <chain> <rule> <comment>
\end{verbatim}
\end{example}

  \noindent wobei die Parameter folgende Bedeutung haben:
  \begin{description}
  \item[table] Die Tabelle, in der sich die Kette befindet
  \item[chain] Die Kette, in welche die Regel eingefügt werden soll
  \item[rule] Die Regel, die eingefügt werden soll; das Format
    entspricht dem in der Konfigurationsdatei verwendeten
  \item[position] Die Position, an der die Regel eingefügt werden soll (nur
    bei \texttt{ins\_rule})
  \item[comment] Ein Kommentar, der zusammen mit der Regel angezeigt
    werden soll, wenn sich jemand den Paketfilter ansieht.
  \end{description}
\end{description}


\subsection{Einordnen in bestehende Regeln}

fli4l konfiguriert den Paketfilter mit einem gewissen
Standardregelsatz. Will man eigene Regeln einfügen, wird man diese in
der Regel nach dem Standardregelsatz einfügen wollen. Ebenfalls wird
man wissen wollen, was denn die vom Nutzer gewünschte Aktion beim
Verwerfen eines Paketes ist. Diese Informationen bekommt man für die
\fwchain{FORWARD}- und \fwchain{INPUT}-Ketten durch Aufruf zweier Funktionen,
\texttt{get\_defaults} und \texttt{get\_count}. Nach Aufruf von

\begin{example}
\begin{verbatim}
    get_defaults <chain>
\end{verbatim}
\end{example}

erhält man die folgenden Ergebnisse:

\begin{description}
\item[\var{drop}:] Diese Variable enthält die Kette, in die verzweigt wird,
  wenn ein Paket verworfen wird.
\item[\var{reject}:] Diese Variable enthält die Kette, in die verzweigt wird,
  wenn ein Paket abgelehnt wird.
\end{description}

Nach Aufruf von

\begin{example}
\begin{verbatim}
    get_count <chain>
\end{verbatim}
\end{example}

erhält man in der Variable \var{res} die Anzahl der Regeln in der Kette
\texttt{<chain>}. Diese Position ist insofern wichtig, als man \emph{nicht}
einfach \texttt{add\_rule} verwenden kann, um eine Regel am Ende der
vordefinierten "`filter"'-Ketten \fwchain{INPUT}, \fwchain{FORWARD} und
\fwchain{OUTPUT} einzufügen. Dies liegt daran, dass diese Ketten mit einer
Standardregel abgeschlossen werden, welche alle verbliebenen Pakete behandelt,
je nach Belegung der \var{PF\_<Kette>\_POLICY}-Variablen. Ein Einfügen
\emph{hinter} dieser letzten Regel hat also keine Auswirkungen. Die Funktion
\texttt{get\_count} erlaubt es nun hingegen, die Stelle direkt \emph{vor}
dieser letzten Regel zu ermitteln und die Position dann an die
\texttt{ins\_rule}-Funktion im Parameter \texttt{<position>} zu übergeben, um
die Regel wie gewünscht am Ende der jeweiligen Kette, aber vor der letzten
Auffang-Regel einzubauen.

Ein Beispiel aus dem Skript \texttt{opt/etc/rc.d/rc390.dns\_dhcp} des Pakets
"`dns\_dhcp"' soll dies verdeutlichen:

\begin{example}
\begin{verbatim}
    case $OPT_DHCPRELAY in
        yes)
            begin_script DHCRELAY "starting dhcprelay ..."

            idx=1
            interfaces=""
            while [ $idx -le $DHCPRELAY_IF_N ]
            do
                eval iface='$DHCPRELAY_IF_'$idx

                get_count INPUT
                ins_rule filter INPUT "prot:udp  if:$iface:any 68 67 ACCEPT" \
                    $res "dhcprelay access"

                interfaces=$interfaces' -i '$iface
                idx=`expr $idx + 1`
            done
            dhcrelay $interfaces $DHCPRELAY_SERVER

            end_script
        ;;
  esac
\end{verbatim}
\end{example}

Hier sieht man inmitten der Schleife einen Aufruf von \texttt{get\_count},
gefolgt von einem Aufruf der \texttt{ins\_rule}-Funktion, der unter anderem
die \var{res}-Variable als \texttt{position}-Parameter übergeben wird.


\subsection{Erweiterung der Paketfilter-Tests}

fli4l verwendet in den Paketfilterregeln die Syntax \fwmatch{match:params},
um zusätzliche Bedingungen an die Pakete zu stellen
(siehe \fwmatch{mac:}, \fwmatch{limit:}, \fwmatch{length:}, \fwmatch{prot:},
\ldots). Will man zusätzliche
Tests hinzufügen, wird das folgendermaßen gemacht:

\begin{enumerate}
\item Festlegen eines passenden Namens. Dieser Name muss mit einem
Kleinbuchstaben im Bereich a-z beginnen und ansonsten aus beliebigen
Buchstaben und Ziffern bestehen.

\achtung{Wenn der Paketfilter-Test in IPv6-Regeln verwendet werden soll,
dann muss darauf geachtet werden, dass der Name keine gültige
IPv6-Adresskomponente ist!}

\item Anlegen einer Datei \texttt{opt/etc/rc.d/fwrules-<name>.ext}.
In dieser Datei steht in etwa Folgendes:

\begin{example}
\begin{verbatim}
    # IPv4 extension is available
    foo_p=yes

    # the actual IPv4 extension, adding matches to match_opt
    do_foo()
    {
        param=$1
        get_negation $param
        match_opt="$match_opt -m foo $neg_opt --fooval $param"
    }

    # IPv6 extension is available
    foo6_p=yes

    # the actual IPv6 extension, adding matches to match_opt
    do6_foo()
    {
        param=$1
        get_negation6 $param
        match_opt="$match_opt -m foo $neg_opt --fooval $param"
    }
\end{verbatim}
\end{example}

Der Paketfilter-Test muss nicht zwingend sowohl für IPv4 als auch für IPv6
implementiert sein (obwohl dies zu bevorzugen ist, falls er für beide
Layer-3-Protokolle sinnvoll ist).

\item Testen der Erweiterung:

\begin{example}
\begin{verbatim}
    $ cd opt/etc/rc.d
    $ sh test-rules.sh 'foo:bar ACCEPT'
    add_rule filter FORWARD 'foo:bar ACCEPT'
    iptables -t filter -A FORWARD -m foo --fooval bar -s 0.0.0.0/0 \
        -d 0.0.0.0/0 -m comment --comment foo:bar ACCEPT -j ACCEPT
\end{verbatim}
\end{example}

\item Aufnahme der Erweiterung und aller noch benötigten Dateien
(\texttt{iptables}-Komponenten) ins Archiv über einen
der bekannten Mechanismen.
\item Zulassen der Erweiterung in der Konfiguration durch Erweitern
von \var{FW\_GENERIC\_MATCH} und/oder \var{FW\_GENERIC\_MATCH6} in einer
exp-Datei, z.\,B. wie folgt:

\begin{example}
\begin{verbatim}
    +FW_GENERIC_MATCH(OPT_FOO) = 'foo:bar' : ''
    +FW_GENERIC_MATCH6(OPT_FOO) = 'foo:bar' : ''
\end{verbatim}
\end{example}
\end{enumerate}
