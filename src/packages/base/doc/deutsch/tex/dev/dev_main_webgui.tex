% Last Update: $Id$

\section{CGI-Erstellung für das \emph{httpd}-Paket}

\subsection{Allgemeines zum Webserver}
Der Webserver, der bei fli4l verwendet wird, ist der \texttt{mini\_httpd} von ACME
Labs. Die Quellen können unter
\altlink{http://www.acme.com/software/mini_httpd/} heruntergeladen werden.
Allerdings wurden für fli4l ein paar Änderungen vorgenommen. 
Die Anpassungen befinden sich im \emph{src}-Paket im Verzeichnis
\texttt{src/""fbr/""buildroot/""package/""mini\_httpd}.

\subsection{Skriptnamen}

Der Skriptname sollte möglichst vielsagend sein, damit er von
anderen Skripten leichter zu unterscheiden ist und es keine
Namensüberschneidungen bei verschiedenen OPTs gibt.

Damit die Skripte ausführbar gemacht werden und DOS-Zeilenumbrüche in
Unix-Zeilen\-um\-brüche umgewandelt werden, muss in der \texttt{opt/<PAKET>.txt}
ein entsprechender Eintrag gemacht werden, siehe
Tabelle~\jump{table:options}{\ref{table:options}}.

\subsection{Menü-Einträge}

Um einen Eintrag im Menü vorzunehmen, muss eine Eintragung in der Datei
\texttt{/etc/httpd/menu} vorgenommen werden. Dieser Mechanismus erlaubt es OPTs,
auch im laufendem Betrieb Änderungen am Menü vorzunehmen. Dies sollte nur mit
dem Skript \texttt{httpd-menu.sh} gemacht werden, da dieses darauf achtet, dass
das Dateiformat dieser Datei immer konsistent ist. Um neue Menüpunkte
einzufügen, wird es folgendermaßen aufgerufen:

\begin{example}
\begin{verbatim}
    httpd-menu.sh add [-p <priority>] <link> <name> [section] [realm]
\end{verbatim}
\end{example}

So wird ein Eintrag mit dem Namen \texttt{<name>} in den Abschnitt
\texttt{[section]} eingetragen. Wenn \texttt{[section]} weggelassen wird, wird
es standardmäßig in den Abschnitt "`OPT-Pakete"' eingetragen. \texttt{<link>}
gibt das Ziel des neuen Links an. \texttt{<priority>} spezifiziert die Priorität
eines Menüeintrags in seinem Abschnitt. Wird sie nicht angegeben, wird die
Standardpriorität 500 benutzt. Die Priorität sollte eine dreistellige Nummer
sein. Je kleiner die Priorität, desto weiter oben steht der Link in dem
Abschnitt. Soll ein Eintrag möglichst weit nach unten, so ist z.\,B. die Priorität
900 zu wählen. Bei gleicher Priorität werden die Einträge nach dem Ziel des
Links sortiert. Bei \texttt{[realm]} wird der Bereich angegeben, für den ein
angemeldeter Benutzer mindestens die Berechtigung \emph{view} haben muss, damit
der Menüpunkt angezeigt wird. Wird \texttt{[realm]} nicht angegeben, wird der
Menüpunkt immer angezeigt. Siehe hierzu auch den Abschnitt
\jump{sec:rights}{"`Benutzerrechte"'}.

Beispiel:

\begin{example}
\begin{verbatim}
    httpd-menu.sh add "neuedatei.cgi" "Hier klicken" "Tools" "tools"
\end{verbatim}
\end{example}

Dieses Beispiel erzeugt im Abschnitt "`Tools"' einen Link mit dem Titel 
"`Hier klicken"' und dem Link-Ziel "`neuedatei.cgi"' und legt den Abschnitt,
falls dieser nicht vorhanden ist, an.

Das Skript kann auch Einträge aus dem Menü wieder entfernen:

\begin{example}
\begin{verbatim}
    httpd-menu.sh rem <link>
\end{verbatim}
\end{example}

Mit diesem Aufruf wird der Eintrag mit dem Link \texttt{<link>} wieder
entfernt.

\wichtig{Wenn mehrere Menüeinträge auf die gleiche Datei verweisen,
werden alle diese Einträge aus dem Menü entfernt.}

Da Abschnitte auch Prioritäten haben können, können diese auch manuell angelegt
werden. Wird ein Abschnitt automatisch beim Hinzufügen eines Eintrages
angelegt, erhält er automatisch die Priorität 500. Der Syntax zum Anlegen von
Abschnitten lautet:

\begin{example}
\begin{verbatim}
    httpd-menu.sh addsec <priority> <name>
\end{verbatim}
\end{example}

Auch hier sollte \texttt{<priority>} nur dreistellige numerische Werte annehmen.

Um sinnvolle Prioritäten in Erfahrung zu bringen, lohnt es sich, auf
einem laufenden fli4l in die Datei \texttt{/etc/httpd/menu} zu schauen, die
Prioritäten stehen in der zweiten Spalte.

Der Vollständigkeit halber wird hier kurz auf das Dateiformat der Menüdatei
eingegangen. Wem die Funktion von \texttt{httpd-menu.sh} reicht, der kann diesen
Abschnitt überspringen. Die Datei \texttt{/etc/httpd/menu} hat den folgenden
Aufbau: Sie ist in vier Spalten eingeteilt. In der ersten Spalte steht ein
Kennbuchstabe, der Überschriften und Einträge unterscheidet. In der zweiten
Spalte steht die Sortierungspriorität. Die dritte Spalte enthält bei Einträgen
das Ziel des Links und bei Überschriften einen Bindestrich, da dieses Feld bei
Überschriften keine Bedeutung hat. Im Rest der Zeile steht der Text, der
nachher im Menü erscheint.

Überschriften benutzen den Kennbuchstaben "`t"', dann wird ein neuer
Menüabschnitt begonnen. Normale Menüeinträge benutzen den Kennbuchstaben "`e"'.
Ein Beispiel:

\begin{example}
\begin{verbatim}
    t 300 - Mein tolles OPT
    e 200 meinopt.cgi Mach etwas Tolles
    e 500 meinopt.cgi?mehr=ja Mach mehr Tolles
\end{verbatim}
\end{example}

Beim Bearbeiten dieser Datei muss man darauf achten, dass das
\texttt{httpd-menu.sh}-Skript die Datei immer sortiert abspeichert. Die
einzelnen Abschnitte sind sortiert und die Einträge pro Abschnitt sind in diesem
Abschnitt sortiert. Der genaue Sortieralgorithmus kann aus
\texttt{httpd-menu.sh} übernommen werden~-- besser wäre allerdings, dieses
Skript um mögliche neue Funktionen zu erweitern, damit alle Menü-Bearbeitungen
an zentraler Stelle passieren.

\subsection{Aufbau eines CGI-Skriptes}

\subsubsection{Die Kopfzeilen}
Alle Skripte des Webservers sind einfache Shell-Skripte (Interpreter wie z.\,B.
Perl, PHP, etc.\ sind viel zu groß für fli4l). Sie sollten mit dem
obligatorischen Skript-Header anfangen (Verweis auf den Interpreter,
Name, Sinn des Skriptes, Autor, Lizenz).

\subsubsection{Das Hilfsskript "'cgi-helper"'}
Gleich nach den Kopfzeilen sollte dann schon das Hilfsskript \texttt{cgi-helper}
mit folgendem Aufruf eingebunden werden:

\begin{example}
\begin{verbatim}
    . /srv/www/include/cgi-helper
\end{verbatim}
\end{example}

Wichtig ist ein Leerzeichen zwischen Punkt und Schrägstrich!

Dieses Skript stellt diverse Hilfsfunktionen bereit, die das Erstellen von CGIs
für fli4l wesentlich vereinfachen sollen. Außerdem werden mit dem Einbinden
des \texttt{cgi-helper} auch noch Standardaufgaben ausgeführt, wie
beispielsweise das Parsen von Variablen, die mit Formularen oder über die URL
übergebenen wurden, oder das Laden von Sprach- und CSS-Dateien.

Tabelle~\ref{tab:dev:cgi-helper} gibt einen Überblick über die Funktionen des
\texttt{cgi-helper}-Skriptes.

\begin{table}[htbp]
  \centering
  \caption{Funktionen des \texttt{cgi-helper}-Skriptes}
  \label{tab:dev:cgi-helper}
  \begin{small}
    \begin{tabular}{|l|p{0.8\textwidth}|}
      \hline
      Name                         & Funktion         \\
      \hline
      \texttt{check\_rights}      & Überprüfung der Benutzerrechte \\
      \texttt{http\_header}       & Ausgabe eines Standard-HTTP-Headers oder eines speziellen HTTP-Headers, beispielsweise zum Download von Dateien\\
      \texttt{show\_html\_header} & Ausgabe des kompletten Seitenheaders (inkl. HTTP-Header, Kopfzeile und Menü)\\
      \texttt{show\_html\_footer} & Ausgabe des Abschlusses der HTML-Seite \\
      \texttt{show\_tab\_header}  & Ausgabe eines Inhalt-Fensters mit Reitern\\
      \texttt{show\_tab\_footer}  & Ausgabe des Abschlusses des Inhaltsfensters\\
      \texttt{show\_error}        & Ausgabe einer Box für Fehlermeldungen (Hintergrundfarbe: rot)\\
      \texttt{show\_warn}         & Ausgabe einer Box für Warnmeldungen (Hintergrundfarbe: gelb)\\
      \texttt{show\_info}         & Ausgabe einer Box für Informationen oder Erfolgsmeldungen (Hintergrundfarbe: grün)\\
      \hline
    \end{tabular}
  \end{small}
\end{table}

\subsubsection{Der Inhalt eines CGI-Skriptes}

Um ein einheitliches Design und vor allem die Kompatibilität
mit zukünftigen fli4l-Versionen zu gewährleisten, ist es sehr zu empfehlen, die
Funktionen des Hilfsskriptes \texttt{cgi-helper} zu benutzen, auch wenn man in
einem CGI theoretisch alle Ausgaben selbst generieren kann.

Eine einfaches CGI-Skript könnte wie folgt aussehen:

\begin{example}
\begin{verbatim}
    #!/bin/sh
    # --------------------
    # Header (c) Autor Datum
    # --------------------
    # get main helper functions
    . /srv/www/include/cgi-helper

    show_html_header "Mein erstes CGI"
    echo '   <h2>Willkommen</h2>'
    echo '   <h3>Dies ist ein Beispiel-CGI-Skript</h3>'
    show_html_footer
\end{verbatim}
\end{example}

\subsubsection{Die Funktion \texttt{show\_html\_header}}

Die Funktion \texttt{show\_html\_header} erwartet eine Zeichenkette als
Parameter. Diese Zeichenkette stellt den Titel der zu generierenden Seite dar.
Die Funktion generiert automatisch das Menü und bindet ebenso automatisch zum
Skript gehörende CSS- und Sprachdateien ein. Voraussetzung dafür ist, dass
diese sich in den Verzeichnissen \texttt{/srv/www/css}
bzw.\ \texttt{/srv/www/lang} befinden und denselben Namen (aber natürlich eine
andere Endung) wie das Skript haben. Ein Beispiel:

\begin{example}
\begin{verbatim}
    /srv/www/admin/OpenVPN.cgi
    /srv/www/css/OpenVPN.css
    /srv/www/lang/OpenVPN.de
\end{verbatim}
\end{example}

Sowohl das Benutzen von Sprachdateien als auch von CSS-Dateien ist optional.
Immer eingebunden werden \texttt{css/main.css} und \texttt{lang/main.<lang>},
wobei \texttt{<lang>} der gewählten Sprache entspricht.

Der Funktion \texttt{show\_html\_header} können aber neben dem Titel noch
weitere Parameter übergeben werden. Ein Aufruf mit allen möglichen Parametern
könnte so aussehen:

\begin{example}
\begin{verbatim}
    show_html_header "Titel" "refresh=$time;url=$url;cssfile=$cssfile;showmenu=no"
\end{verbatim}
\end{example}

Alle zusätzlichen Parameter müssen, wie im Beispiel gezeigt, mit
Anführungszeichen umschlossen und durch ein Semikolon getrennt werden. Eine
Überprüfung der Syntax erfolgt \emph{nicht}! Es ist also notwendig, sehr genau
auf die Parameterübergabe zu achten.

Hier eine kurze Übersicht über die Funktion der Parameter:

\begin{itemize}
 \item \texttt{refresh=}\emph{time}: Zeit in Sekunden in der die Seite vom
    Browser neu geladen werden soll
 \item \texttt{url=}\emph{url}: die URL, die bei einem Refresh neu geladen wird
 \item \texttt{cssfile=}\emph{cssfile}: Name einer CSS-Datei, wenn diese vom
    Namen des CGIs abweicht
 \item \texttt{showmenu=no}: unterdrückt die Anzeige des Menüs und des Headers
\end{itemize}

Sonstige Richtlinien zum Inhalt:

\begin{itemize}
 \item Fasst euch kurz :-)
 \item Schreibt sauberes HTML (SelfHTML\footnote{siehe
    \altlink{http://de.selfhtml.org/}} ist da ein guter Ansatzpunkt).
 \item Verzichtet auf hochmodernen Schnick-Schnack (JavaScript ist i.\,O., wenn
    es nicht stört, sondern den Benutzer unterstützt, das Ganze muss auch ohne
    JavaScript funktionieren).
\end{itemize}

\subsubsection{Die Funktion \texttt{show\_html\_footer}}

Die Funktion \texttt{show\_html\_footer} schließt den Block im CGI-Skript ab,
der durch die Funktion \texttt{show\_html\_header} eingeleitet wurde.

\subsubsection{Die Funktion \texttt{show\_tab\_header"'}}

Damit der Inhalt eurer mit Hilfe des CGIs erzeugten Webseite auch hübsch
geordnet aussieht, könnt ihr die \texttt{cgi-helper}-Funktion
\texttt{show\_tab\_header} nutzen. Sie erzeugt dann anklickbare Reiter ("`Tabs"'
genannt), so dass ihr eure Seite in mehrere logisch voneinander getrennte
Bereiche unterteilen könnt.

Der \texttt{show\_tab\_header}-Funktion werden Parameter immer in Paaren
übergeben. Der erste Wert entspricht dem Titel eines Reiters, der zweite dem
Link. Wird als Link die Zeichenkette "`no"' übergeben, wird lediglich der Titel
ausgegeben und der Reiter ist nicht anklickbar (und blau).

Im folgenden Beispiel wird ein "`Fenster"' mit dem Titel "`Ein tolles Fenster"'
erzeugt. Im Fenster steht "`foo bar"':

\begin{example}
\begin{verbatim}
    show_tab_header "Ein tolles Fenster" "no"
    echo "foo"
    echo "bar"
    show_tab_footer
\end{verbatim}
\end{example}

Im nächsten Beispiel werden zwei anklickbare Reiter generiert, die dem
Skript die Variable \var{action} mit verschiedenen Werten übergibt.

\begin{example}
\begin{verbatim}
    show_tab_header "1. Auswahltab" "$myname?action=machdies" \
                    "2. Auswahltab" "$myname?action=machjenes"
    echo "foo"
    echo "bar"
    show_tab_footer
\end{verbatim}
\end{example}

Nun kann das Skript den Inhalt der Variablen \var{FORM\_action} (siehe weiter
unten zur Variablenauswertung) auswerten und je nachdem unterschiedliche Inhalte
bereitstellen. Damit der angeklickte Reiter auch ausgewählt erscheint und nicht
mehr angeklickt werden kann, müsste der Funktion statt dem Link wie schon
erwähnt ein "`no"' übergeben werden. Das geht aber auch einfacher, wenn man sich
an die Konvention in folgendem Beispiel hält:

\begin{example}
\begin{verbatim}
    _opt_machdies="1. Auswahltab"
    _opt_machjenes="2. Auswahltab"
    show_tab_header "$_opt_machdies" "$myname?action=opt_machdies" \
                    "$_opt_machjenes" "$myname?action=opt_machjenes"
    case $FORM_action in
        opt_machdies) echo "foo" ;;
        opt_machjenes) echo "bar" ;;
    esac
    show_tab_footer
\end{verbatim}
\end{example}

Wird also für den Titel eine Variable verwendet, deren Namen dem Inhalt der
Variablen \var{action} mit führendem Unterstrich (\texttt{\_}) entspricht, wird
der entsprechende Reiter ausgewählt dargestellt.

\subsubsection{Die Funktion \texttt{show\_tab\_footer}}

Die Funktion \texttt{show\_tab\_footer} schließt den Block im CGI-Skript ab,
der durch die Funktion \texttt{show\_tab\_header} eingeleitet wurde.

\subsubsection{Mehrsprachfähigkeit}

Das Hilfsskript \texttt{cgi-helper} enthält weiterhin Funktionen, um CGI-Skripte
mehrsprachfähig zu machen. Dazu müssen "`nur"' für alle Textausgaben Variablen
mit führenden Unterstrichen (\texttt{\_}) verwendet und diese Variablen in den
entsprechenden Sprachdateien definiert werden.

Beispiel:

\texttt{lang/opt.de} enthalte:

\begin{example}
\begin{verbatim}
    _opt_machdies="Eine Ausgabe"
\end{verbatim}
\end{example}

\texttt{lang/opt.en} enthalte:

\begin{example}
\begin{verbatim}
    _opt_machdies="An Output"
\end{verbatim}
\end{example}

\texttt{admin/opt.cgi} enthalte:

\begin{example}
\begin{verbatim}
    ...
    echo $_opt_machdies
    ...
\end{verbatim}
\end{example}


\subsubsection{Formular-Auswertung}

Um Formulare zu verarbeiten, muss man noch einige Dinge wissen. Egal ob die
Formular-Methode \var{GET} oder \var{POST} verwendet wird, die Parameter finden
sich nach dem Einbinden des \texttt{cgi-helper}-Skripts (welches wiederum das
Hilfsprogramm proccgi aufruft) in den Variablen \var{FORM\_<Parameter>} wieder.
Wenn das Formularfeld also den Namen "`eingabe"' hatte, kann im CGI-Skript
mit \var{\$FORM\_eingabe} darauf zugegriffen werden.

Weitere Informationen zum Programm \texttt{proccgi} finden sich unter
\altlink{http://www.fpx.de/fp/Software/ProcCGI.html}.

\marklabel{sec:rights}{
    \subsubsection{Benutzerrechte: Die Funktion \texttt{check\_rights}}
}

Um zu prüfen, ob der Benutzer ausreichende Rechte zur Nutzung eines CGI-Skripts
besitzt, muss am Anfang des CGI-Skripts die Funktion \texttt{check\_rights}
wie folgt aufgerufen werden:

\begin{example}
\begin{verbatim}
    check_rights <Bereich> <Aktion>
\end{verbatim}
\end{example}

Das CGI-Skript wird dann nur ausgeführt, wenn der aktuell angemeldete Benutzer
\begin{itemize}
\item alle Rechte hat (\verb+HTTPD_RIGHTS_x='all'+), oder
\item alle Rechte für den angegebenen Bereich hat
    (\verb+HTTPD_RIGHTS_x='<Bereich>:all'+), oder
\item das Recht hat, die angegebene Aktion im angegebenen Bereich auszuführen\\
    (\verb+HTTPD_RIGHTS_x='<Bereich>:<Aktion>'+).
\end{itemize}

% More examples?

\subsubsection{Die Funktion \texttt{show\_error}}

Diese Funktion gibt eine Fehlermeldung in einer roten Box aus. Sie erwartet
zwei Parameter: einen Titel sowie eine Meldung. Beispiel:

\begin{example}
\begin{verbatim}
    show_error "Error: No key" "No key was specified!"
\end{verbatim}
\end{example}

\subsubsection{Die Funktion \texttt{show\_warn}}

Diese Funktion gibt eine Warnmeldung in einer gelben Box aus. Sie erwartet
zwei Parameter: einen Titel sowie eine Meldung. Beispiel:

\begin{example}
\begin{verbatim}
    show_info "Warnung" "Derzeit besteht keine Verbindung!"
\end{verbatim}
\end{example}

\subsubsection{Die Funktion \texttt{show\_info}}

Diese Funktion gibt eine Informations- oder Erfolgsmeldung in einer grünen Box
aus. Sie erwartet zwei Parameter: einen Titel sowie eine Meldung. Beispiel:

\begin{example}
\begin{verbatim}
    show_info "Info" "Aktion wurde erfolgreich ausgeführt!"
\end{verbatim}
\end{example}

\subsubsection{Das Hilfsskript "'cgi-helper-ip4"'}

Gleich nach den Hilfsskript "cgi-helper" kann dann das Hilfsskript cgi-helper-ip4
mit folgendem Aufruf eingebunden werden:

\begin{example}
\begin{verbatim}
. /srv/www/include/cgi-helper-ip4
\end{verbatim}
\end{example}

Wichtig ist ein Leerzeichen zwischen Punkt und Schrägstrich!

Dieses Skript stellt Hilfsfunktionen bereit, um Prüfungen von IPv4-Adressen
vornehmen zu können.

\subsubsection{Die Funktion \texttt{ip4\_isvalidaddr}}

Diese Funktion überprüft, ob eine gültige IPv4-Adresse übergeben wurde.
Beispiel:

\begin{example}
\begin{verbatim}
    if ip4_isvalidaddr ${FORM_inputip}
    then
        ...
    fi
\end{verbatim}
\end{example}

\subsubsection{Die Funktion \texttt{ipv4\_normalize}}

Diese Funktion entfernt aus der übergebenen IPv4-Adresse führende Nullen.
Beispiel:

\begin{example}
\begin{verbatim}
    ip4_normalize ${FORM_inputip}
    IP=$res
    if [ -n "$IP" ]
    then
        ...
    fi
\end{verbatim}
\end{example}

\subsubsection{Die Funktion \texttt{ipv4\_isindhcprange}}

Diese Funktion prüft, ob die übergebene IPv4-Adresse sich im Bereich der
übergebenen Start- und Zieladresse befindet. Beispiel:
 
\begin{example}
\begin{verbatim}
    if ip4_isindhcprange $FORM_inputip $ip_start $ip_end
    then
        ...
    fi
\end{verbatim}
\end{example}

\subsection{Sonstiges}

Dies und das (ja, das ist auch noch wichtig!):

\begin{itemize}
 \item Der \texttt{mini\_httpd} schützt Unterverzeichnisse nicht mit einem
    Passwort. Es muss für jedes Verzeichnis eine eigene \texttt{.htaccess}-Datei
    oder ein Link auf eine andere \texttt{.htaccess}-Datei angelegt werden.
 \item KISS - Keep it simple, stupid!
 \item Diese Angaben können sich jederzeit ohne Vorankündigung ändern!
\end{itemize}

\subsection{Fehlersuche}

Um die Fehlersuche innerhalb eines CGI-Skripts zu erleichtern, kann man vor
der Einbindung des \texttt{cgi-helper}-Skripts den Debugging-Modus aktivieren.
Dazu muss die Variable \var{set\_debug} auf den Wert "`yes"' gesetzt werden.
Dies führt zur Erstellung der Datei \texttt{debug.log}, die über die URL
\texttt{http://<fli4l-Host>/admin/debug.log} heruntergeladen werden kann.
Diese enthält alle Aufrufe des CGIs. Die \texttt{set\_debug}-Variable ist nicht
global, muss also in jedem Problem-CGI erneut gesetzt werden. Beispiel:

\begin{example}
\begin{verbatim}
    set_debug="yes"
    . /srv/www/include/cgi-helper
\end{verbatim}
\end{example}

Alternativ kann auch die jeweilige CGI-URL um den Parameter \texttt{debug=yes}
ergänzt werden, etwa \texttt{http://<fli4l-Host>/admin/meinopt.cgi?debug=yes}.

Des Weiteren eignet sich cURL\footnote{siehe
\altlink{http://de.wikipedia.org/wiki/CURL}} hervorragend zur Fehlersuche,
insbesondere wenn die HTTP-Kopfzeilen nicht korrekt zusammengesetzt werden oder
der Browser nur weiße Seiten anzeigt. Auch ist das Cachingverhalten moderner
Webbrowser bei der Fehlersuche hinderlich.

Beispiel: Mit dem folgenden Aufruf wird der HTTP-Header ("`\emph{d}ump"',
\texttt{-D}) sowie die normale Ausgabe des CGIs \texttt{admin/mein.cgi}
ausgegeben. Es soll der Benutzername ("`\emph{u}ser"', \texttt{-u}) "`admin"'
verwendet werden.

\begin{example}
\begin{verbatim}
    curl -D - http://fli4l/admin/mein.cgi -u admin
\end{verbatim}
\end{example}
