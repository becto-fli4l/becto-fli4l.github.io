% Last Update: $Id: base_main_packages.tex 38958 2015-05-07 20:22:00Z kristov $
\chapter{Pakete}

  Neben der Basisinstallation (BASE) gibt es weitere Pakete. Diese enthalten ein
  oder mehrere ``OPTs''\footnote{Abk. für ``OPTionales Modul''}, die bei Bedarf
  zu BASE hinzuinstalliert werden können. Einige dieser Pakete sind Bestandteil
  des Basis-Paketes, andere kommen separat. Eine Übersicht über die vom
  fli4l-Team bereitgestellten Pakete finden Sie auf der Download-Seite
  (\altlink{http://www.fli4l.de/download/stabile-version/}), die von anderen
  Autoren bereitgestellten Pakete sind in der OPT-Datenbank
  (\altlink{http://extern.fli4l.de/fli4l_opt-db3/}) zu finden. Im
  folgenden werden die vom fli4l-Team bereitgestellten Pakete
  beschrieben.

  \section{Werkzeuge im Basispaket}

  Im Basispaket befinden sich die folgenden OPTs:
  \begin{center}
  \begin{tabular}{@{}lp{12cm}@{}}\hline
    Name               & Beschreibung \\\hline
    \var{OPT\_SYSLOGD} & \jump{OPTSYSLOGD}{Werkzeug zum Protokollieren von Systemmeldungen}\\
    \var{OPT\_KLOGD}   & \jump{OPTKLOGD}{Werkzeug zum Protokollieren von Kernelmeldungen}\\
    \var{OPT\_LOGIP}   & \jump{OPTLOGIP}{Werkzeug zum Protokollieren von WAN-IP-Adressen}\\
    \var{OPT\_Y2K}     & \jump{OPTY2K}{Datumskorrektur bei nicht Y2K-festen Rechnern}\\
    \var{OPT\_PNP}     & \jump{OPTPNP}{Installation der isapnp-Werkzeuge}\\
    \var{OPT\_HOTPLUG\_PCI} & \jump{OPTHOTPLUGPCI}{Aktivierung von PCI-Hotplugging}\\\hline
  \end{tabular}
  \end{center}

\marklabel{OPTSYSLOGD}{\subsection{OPT\_SYSLOGD~-- Protokollieren von Systemmeldungen}}\index{OPT\_SYSLOGD}

  Viele Programme verwenden die Syslog-Schnittstelle, um
  Meldungen auszugeben. Damit diese auch auf der Konsole
  sichtbar werden, muss in diesem Falle der Daemon syslogd gestartet
  werden.

  Sind Debug-Meldungen gewünscht, stellt man \var{OPT\_\-SYSLOGD} auf `yes',
  ansonsten auf `no'.

  Siehe auch \jump{ISDNCIRCxDEBUG}{\var{ISDN\_CIRC\_x\_DEBUG}} und
  \jump{PPPOEDEBUG}{\var{PPPOE\_DEBUG}}.


  Standard-Einstellung: \var{OPT\_\-SYSLOGD}='no'

  \begin{description}
    \config{SYSLOGD\_RECEIVER}{SYSLOGD\_RECEIVER}{SYSLOGDRECEIVER}

  Mit \var{SYSLOGD\_RECEIVER} kann man festlegen, ob fli4l Syslog-Nachrichten
  vom Netzwerk empfangen soll oder nicht.

    \configlabel{SYSLOGD\_DEST\_x}{SYSLOGDDESTx}
    \config{SYSLOGD\_DEST\_N SYSLOGD\_DEST\_x}{SYSLOGD\_DEST\_N}{SYSLOGDDESTN}

  Mit \var{SYSLOGD\_\-DEST\_\-x} gibt man Ziele an, wohin die System-Meldungen, die
  von syslogd entgegengenommen werden, ausgegeben werden. Im
  Normalfall ist dies die Konsole von fli4l, also

\begin{example}
\begin{verbatim}
        SYSLOGD_DEST_1='*.* /dev/console'
\end{verbatim}
\end{example}

  Möchte man eine Datei als Ziel verwenden, ist z.B. einzutragen:

\begin{example}
\begin{verbatim}
        SYSLOGD_DEST_1='*.* /var/log/messages'
\end{verbatim}
\end{example}

  Ist ein sog. Loghost im Netz vorhanden, können die Meldungen auch
  auf diesen Rechner umgeleitet werden~-- unter Angabe der IP-Adresse.

  Beispiel:
\begin{example}
\begin{verbatim}
        SYSLOGD_DEST_1='*.* @192.168.4.1'
\end{verbatim}
\end{example}

  Das @-Zeichen ist dann der IP-Adresse voranzustellen.

  Wenn die Systemmeldungen an mehrere Ziele ``ausgeliefert'' werden sollen,
  ist es nötig, die Variable \var{SYSLOGD\_DEST\_N} (Anzahl der Ziele) entsprechend zu erhöhen
  und die Variablen \var{SYSLOG\_DEST\_1}, \var{SYSLOG\_DEST\_2} usw. zu füllen.

  Der Parameter `*.*' bedeutet, dass sämtliche Meldungen protokolliert
  werden. Man kann jedoch die Meldungen für bestimmte Ziele
  über sog. ``Prioritäten'' einschränken. In diesem Fall ersetzt man
  das Sternchen (*) hinter dem Punkt (.) durch eines der folgenden
  Schlüsselwörter:
  \begin{itemize}
  \item debug
  \item info
  \item notice
  \item warning (veraltet: warn)
  \item err (veraltet: error)
  \item crit
  \item alert
  \item emerg (veraltet: panic)
  \end{itemize}

  Die Reihenfolge in der Liste spiegelt dabei das ``Gewicht'' der
  Meldungen wider. Die Schlüsselwörter ``error'', ``warn'' und ``panic'' sind
  veraltet und sollten nicht mehr verwendet werden.

  Vor dem Punkt kann eine sog. ``Facility'' statt des Sternchens (*)
  eingetragen werden. Eine Erklärung würde aber hier zu weit gehen.
  Der geneigte Leser kann hierfür eine Suchmaschine seiner Wahl
  bemühen. Eine Übesicht über mögliche Facilities
  finden sich auf den Manpages der syslog.conf:

  \enlargethispage{\baselineskip}
  \noindent \altlink{http://linux.die.net/man/5/syslog.conf}

  Normalerweise ist das Sternchen aber vollkommen ausreichend. Beispiel:
\begin{example}
\begin{verbatim}
          SYSLOGD_DEST_1='*.warning @192.168.4.1'
\end{verbatim}
\end{example}
  Nicht nur Unix-/Linux-Rechner können als Loghost dienen, sondern
  auch Windows-Rechner. Auf \altlink{http://www.fli4l.de/sonstiges/links/} findet man
  Verweise auf entsprechende Software. Die Verwendung eines
  Loghosts wird dringend empfohlen, wenn eine detaillierte
  Protokollierung gewünscht ist. Auch hilft die Protokollierung bei
  der Fehlersuche. Auch imonc ``versteht'' als
  Windows-Client das Syslog-Protokoll und kann die Meldungen in einem
  Fenster ausgeben.

  Leider lassen sich die Boot-Meldungen von fli4l nicht über syslogd
  umlenken. Jedoch kann man fli4l auch so konfigurieren, dass die
  Konsole ein serielles Terminal (bzw. Terminalemulation) ist. Wie das
  geht, steht in dem Abschnitt \jump{CONSOLESETTINGS}{Konsolen-Einstellungen}.

  \config{SYSLOGD\_ROTATE}{SYSLOGD\_ROTATE}{SYSLOGDROTATE}

  Mit \var{SYSLOGD\_ROTATE} kann man festlegen, ob fli4l Syslog-Nachrichten
  einmal täglich rotiert. Es werden dabei die Meldungen der letzten x Tage
  archiviert.

  \config{SYSLOGD\_ROTATE\_DIR}{SYSLOGD\_ROTATE\_DIR}{SYSLOGDROTATEDIR}

  Mit der optionalen Variable \var{SYSLOGD\_ROTATE\_DIR} kann man festlegen,
  dass die rotierten Syslog-Dateien nicht in /var/log/ sondern in das angegebene
  Verzeichnis rotiert werden.

  \config{SYSLOGD\_ROTATE\_MAX}{SYSLOGD\_ROTATE\_MAX}{SYSLOGDROTATEMAX}

  Mit der optionalen Variablen \var{SYSLOGD\_ROTATE\_MAX} legt man die Anzahl der
  archivierten/rotierten Syslog-Dateien fest.

  \config{SYSLOGD\_ROTATE\_AT\_SHUTDOWN}{SYSLOGD\_ROTATE\_AT\_SHUTDOWN}{SYSLOGDROTATEATSHUTDOWN}

  Mit der optionalen Variable \var{SYSLOGD\_ROTATE\_AT\_SHUTDOWN} kann man den das Rotieren
  beim Herunterfahren des Routers deaktivieren. Dies sollte man jedoch nur einstellen, wenn
  die syslog-Dateien bereits auf ein nichtflüchtiges Ziel geschrieben werden.

\end{description}


\marklabel{OPTKLOGD}{\subsection{OPT\_KLOGD~-- Protokollieren von Kernelmeldungen}}\index{OPT\_KLOGD}

  Viele der auftretenden Fehler~-- z.B. fehlgeschlagene Einwahl --
  werden vom Linux-Kernel direkt auf die Konsole geschrieben. Mit
  \var{OPT\_\-KLOGD}='yes' werden diese Meldungen an den syslogd umgelenkt,
  welcher diese dann entweder in Dateien protokollieren oder an einen
  Loghost weiterleiten kann, s.o. Dann hat man auf der fli4l-Konsole
  (fast) Ruhe.

  \noindent Empfehlung: Wenn Sie \var{OPT\_\-SYSLOGD}='yes' benutzen, sollte man auch
  \var{OPT\_\-KLOGD}='yes' setzen.

  Standard-Einstellung: \var{OPT\_\-KLOGD}='no'


\marklabel{OPTLOGIP}{\subsection{OPT\_LOGIP~-- Protokollieren von WAN-IP-Adressen}}\index{OPT\_LOGIP}

  Mit LOGIP ist es möglich, die WAN-IP in einer Log-Datei festzuhalten. Mit Angabe von
  \var{OPT\_LOGIP}='yes' wird die Funktion aktiviert.

  Standard-Einstellung: \var{OPT\_LOGIP}='no'

\begin{description}
  \config{LOGIP\_LOGDIR}{LOGIP\_LOGDIR}{LOGIPLOGDIR}{Verzeichnis der Log-Datei festlegen}

  Mit \var{LOGIP\_LOGDIR} wird das Verzeichnis festgelegt, in welchem die
  Log-Datei angelegt wird oder 'auto' für autodetect.

  Standard-Einstellung: \var{LOGIP\_LOGDIR}='auto'
\end{description}

\marklabel{OPTY2K}{\subsection{OPT\_Y2K~-- Datumskorrektur bei nicht Y2K-festen Rechnern}}\index{OPT\_Y2K}

  Meist werden fli4l-Router aus alten Hardware-Teilen zusammengesetzt.
  Dabei kann das BIOS des Mainboards nicht Y2K-fest sein. Das kann dazu führen,
  dass bei einer Einstellung des Datums auf den 27.05.2000 im BIOS beim
  nächsten Booten der 27.05.2094 im BIOS zu finden ist! Linux zeigt dann
  übrigens den 27.05.1994 an.

  Eigentlich kann das eingestellte Datum für den fli4l-Router egal
  sein und sollte deshalb keine Rolle spielen. Wird fli4l jedoch als
  Least-Cost-Router eingesetzt, kann dies sehr wohl eine Rolle spielen.

  Grund: Der 27.05.1994 war ein Freitag, der 27.05.2000 jedoch ein
  Samstag. Und am Wochenende gibt's günstigere Tarife bzw. günstigere
  Provider \ldots

  Eine erste Lösung lautet: Das BIOS-Datum wird vom 27.05.2000 auf den
  28.05.1994 gestellt. Das war ebenso ein Samstag. Damit ist das Problem aber
  noch nicht gänzlich gelöst, denn fli4l verwendet nicht nur den Wochentag und
  die momentane Uhrzeit für das LC-Routing, sondern berücksichtigt auch die
  Feiertage.

\begin{description}
  \config{Y2K\_DAYS}{Y2K\_DAYS}{Y2KDAYS}{~-- N Tage auf Systemdatum addieren}

  Da das gesetzte Datum genau 2191 Tage zum tatsächlichen Datum differiert,
  werden bei Angabe von

\begin{example}
\begin{verbatim}
        Y2K_DAYS='2191'
\end{verbatim}
\end{example}

  auf das BIOS-Datum 2191 Tage addiert und dieses dann als Linux-Datum
  gesetzt. Das BIOS-Datum bleibt davon jedoch unberührt, sonst würde
  beim nächsten Booten das Datum wieder das Jahr 2094 (bzw. 1994) aufweisen.
\end{description}
  Es gibt noch eine Alternative:

  Mit dem Zugriff auf einen Time-Server kann sich fli4l die aktuelle
  Datum/Uhrzeit aus dem Internet holen. Dafür steht das Paket \jump{sec:opt-chrony}{\var{CHRONY}} zur
  Verfügung. Beide Einstellungen lassen sich kombinieren. Das ist
  sinnvoll, um mit \var{Y2K\_DAYS} zunächst das Datum schon einmal zu
  korrigieren und anschließend über den Time-Server die genaue Uhrzeit
  einzustellen.

  Wer mit Y2K keine Probleme hat: Variable \var{OPT\_\-Y2K}='no' setzen und
  einfach nicht mehr darüber nachdenken \ldots


\marklabel{OPTPNP}{\subsection{OPT\_PNP~-- Installation von isapnp tools}}\index{OPT\_PNP}

  Teilweise müssen ISAPnP-Karten über das Werkzeug ``isapnp''
  konfiguriert werden. Dies betrifft insb. ISDN-Karten mit \var{ISDN\_\-TYPE}
  7, 12, 19, 24, 27, 28, 30 und 106~-- aber nur, wenn es sich auch wirklich um
  ISAPnP-Karten handelt.

  Zur Konfiguration ist die Erstellung einer Konfigurations-Datei
  etc/isapnp.conf notwendig.

  Hier eine Kurzanleitung zur Erstellung:

  \begin{itemize}
  \item In $<$config$>$/base.txt die Variable \var{OPT\_\-PNP}='yes' und
    \var{MOUNT\_\-BOOT}='rw' setzen
  \item fli4l booten~-- die ISAPnP-Karte wird wahrscheinlich nicht erkannt
  \item Auf fli4l-Konsole eingeben:
\begin{example}
\begin{verbatim}
        pnpdump -c >/boot/isapnp.conf
        umount /boot
\end{verbatim}
\end{example}
    Damit ist die Konfiguration auf dem Boot-Medium gespeichert.
  \end{itemize}
  Weiter auf PC (Unix/Linux/Windows):
  \begin{itemize}
  \item Die Datei isapnp.conf vom Boot-Medium nach $<$config$>$/etc/isapnp.conf
    kopieren
  \item isapnp.conf mit Editor öffnen, bearbeiten und abspeichern\\
        Die vorgegebenen Werte kann man hier beibehalten oder auch
        durch andere ersetzen, die man aus den möglichen Werten wählt.
        Relevant sind dabei die folgenden Zeilen im folgenden Beispiel:

\begin{example}
\begin{verbatim}
            #     Start dependent functions: priority acceptable
            #       Logical device decodes 16 bit IO address lines
            #             Minimum IO base address 0x0160
            #             Maximum IO base address 0x0360
            #             IO base alignment 8 bytes
            #             Number of IO addresses required: 8
       1)      (IO 0 (SIZE 8) (BASE 0x0160))
            #       IRQ 3, 4, 5, 7, 10, 11, 12 or 15.
            #             High true, edge sensitive interrupt (by default)
       2)      (INT 0 (IRQ 10 (MODE +E
\end{verbatim}
\end{example}

    \begin{description}
      \item 1)~-- Hier kann als \glqq{}BASE\grqq{} eine Adresse zwischen die
                 angegebene Minimum und Maximum eingegeben werden, wobei man
                 das \glqq{}base alignment\grqq{} in betracht ziehen muss.\\
                 Bei mehr als einer ISA-Karte im System muss immer darauf
                 geachtet werden, dass es hier keine Überschneidungen gibt,
                 achte dabei auch auf die benötigte Anzahl Adressen (number
                 of addresses required).

      \item 2)~-- Hier kann aus der angezeigten Liste ein IRQ eingesetzt
                 werden.
                 Dabei ist 2(9), 3, 4, 5 und 7 eher eine schlechte Wahl, da
                 sich diese normalerweise mit den Seriellen und Parallelen
                 Schnittstellen bzw der Cascadierung ins Gehege kommen.\\
                 ISA Karten können IRQs nicht teilen, deshalb darf ein für
                 diese Karte verwendeter nicht anderweitig belegt sein.
    \end{description}
  \item Die entsprechenden Daten (IRQ/IO) in die $<$config$>$/isdn.txt
    übernehmen
  \item Es ist nötig in der $<$config$>$/base.txt die \var{OPT\_\-PNP}
    Einstellung auf `yes' zu belassen, anderenfalls werden die erforderliche
    Dateien nicht mit auf das Boot-Medium kopiert. Die Einstellung
    \var{MOUNT\_BOOT} kann man beliebig ändern.
  \item Neues Boot-Medium erzeugen
  \end{itemize}
  \achtung {Die automatisch generierte Datei ist im Unix-Format
    gespeichert und enthält keine CRs. Startet man unter Windows den
    Notepad-Editor, zeigt dieser alle Zeilen in einer einzigen Zeile
    an. Der DOS-Editor ``edit'' kann jedoch mit Unix-Dateien umgehen.
    Er speichert sie dann als DOS-Datei (mit CRs) ab.}

  Abhilfe:
  \begin{itemize}
  \item DOS-Box starten
  \item In das Verzeichnis $<$config$>$/etc wechseln
  \item Eingeben:
    edit isapnp.conf
  \item Datei bearbeiten und abspeichern
  \end{itemize}
  Anschließend kann die Datei auch wieder mit Notepad bearbeitet
  werden.

  Man kann auch unter Windows einfach den Wordpad-Editor verwenden.

  Die zusätzlich generierten CRs werden beim Booten von fli4l wieder
  herausgefiltert. Sie stören also nicht.

  Zunächst sollte man versuchen, ohne \var{OPT\_\-PNP} auszukommen. Wird die
  Karte nicht erkannt, sollte wie oben beschrieben vorgegangen werden.

  Bei einem Update auf eine neuere fli4l-Version kann die früher
  erstellte Datei \hbox{isapnp.conf} weiter verwendet werden.

  Standard-Einstellung: \var{OPT\_\-PNP}='no'

\marklabel{OPTHOTPLUGPCI}{\subsection{OPT\_HOTPLUG\_PCI~-- Aktivieren von PCI-Hotplugging}}\index{OPT\_HOTPLUG\_PCI}

  Mit \var{OPT\_HOTPLUG\_PCI='yes'} werden Module auf den fli4l kopiert und
  beim Booten geladen, die PCI-Hotplugging aktivieren, d.h. die Unterstützung
  für das Hinzufügen und Entfernen von PCI-Adaptern zur Laufzeit. Für diese
  Funktionalität muss ein passender PCI-Hotplug-Controller vorhanden sein.

  Für das Hinzufügen und Entfernen von virtuellen Geräten in
  \emph{Virtualisierungsumgebungen} wie KVM muss diese Option \emph{nicht}
  aktiviert sein, da dies über ACPI-Mechanismen geschieht und der zugehörige
  ACPI-Treiber dauerhaft im Kernel aktiviert ist.
