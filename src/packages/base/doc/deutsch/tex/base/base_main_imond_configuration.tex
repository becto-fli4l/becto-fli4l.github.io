% Last Update: $Id$
\section{imond-Konfiguration}
\begin{description}

    \config{OPT\_IMOND}{OPT\_IMOND}{OPTIMOND}

    Standard-Einstellung: \var{OPT\_\-IMOND='no'}

    {Mit \var{OPT\_\-IMOND} kann man einstellen, ob der imond-Server
      aktiviert werden soll. imond übernimmt dabei das
      Monitoring/Controlling und Least-Cost-Routing des fli4l-Routers.
      Der \jump{IMONDSCHNITTSTELLE}{Beschreibung von imond} ist deshalb
      ein extra Kapitel gewidmet (s.u.).

      Wichtig: Die LC-Routing-Features von fli4l können nur mit imond
      genutzt werden. Ein zeitabhängiges Umschalten von Verbindungen
      ist ohne imond nicht möglich!

      Für ISDN- und DSL-Routing ist imond ab Version 1.5 zwingend
      erforderlich.  In diesem Fall ist \var{OPT\_\-IMOND}='yes'
      einzustellen.

      Wird fli4l lediglich als Router zwischen 2 Netzwerken
      eingesetzt, sollte \var{OPT\_\-IMOND}='no' eingestellt werden.}


    \config{IMOND\_PORT}{IMOND\_PORT}{IMONDPORT}

    {TCP/IP-Port, auf dem imond auf Verbindungen horcht.  Der
      Standard-Wert `5000' sollte nur in Ausnahmefällen geändert
      werden.}


    \config{IMOND\_PASS}{IMOND\_PASS}{IMONDPASS}
    
    Standard-Einstellung: \var{IMOND\_\-PASS=''}
    
    {Hier kann ein spezielles User-Password für imond gesetzt werden.
      Meldet sich ein Client auf Port 5000 an, erwartet imond (und
      damit auch seine Clients) die Eingabe dieses Passworts, bevor er
      irgendeinen Befehl korrekt beantwortet. Ausnahme: Befehle
      ``quit'', ``help'' und ``pass''.  Ist \var{IMOND\_\-PASS} leer, wird
      kein Password benötigt.

      Ob der Client im User-Modus bestimmte Steuerbefehle, wie Dial,
      Hangup, Reboot, Umschalten der Default-Route bereits ausführen
      kann oder dafür die Eingabe des Admin-Passwords zwingend
      notwendig ist, wird über die Variablen
      \begin{itemize}
      \item \jump{IMONDENABLE}{IMOND\_ENABLE},
      \item \jump{IMONDDIAL}{IMOND\_DIAL},
      \item \jump{IMONDROUTE}{IMOND\_ROUTE} und
      \item \jump{IMONDREBOOT}{IMOND\_REBOOT}
      \end{itemize}

      eingestellt, siehe unten.
      }


    \config{IMOND\_ADMIN\_PASS}{IMOND\_ADMIN\_PASS}{IMONDADMINPASS}

    Standard-Einstellung: \var{IMOND\_ADMIN\_\-PASS=''}
    
    { Mit Hilfe der Admin-Passwords erhält der Client alle Rechte und
      kann so sämtliche Steuerfunktionen des imond-Servers nutzen~-- und
      zwar unabhängig von den Variablen \var{IMOND\_\-ENABLE}, \var{IMOND\_\-DIAL} usw.
      Lässt man \var{IMOND\_\-ADMIN\_\-PASS} leer, so reicht die Eingabe des
      User-Passwords, um sämtliche Rechte zu erhalten!
    }


    \config{IMOND\_LED}{IMOND\_LED}{IMONDLED}

    {imond kann den Online/Offline-Status nun über eine LED anzeigen.
      Diese wird folgendermaßen an einen COM-Port angeschlossen:

      Verbindung 25-polig:

\begin{example}
\begin{verbatim}
        20 DTR  -------- 1kOhm ----- >| ---------- 7 GND
\end{verbatim}
\end{example}


      Verbindung  9-polig:
\begin{example}
\begin{verbatim}
         4 DTR  -------- 1kOhm ----- >| ---------- 5 GND
\end{verbatim}
\end{example}

      Ist eine ISDN- oder DSL-Verbindung aufgebaut, leuchtet die LED.
      Ansonsten ist sie ausgeschaltet. Sollte es genau umgekehrt sein,
      ist die Leuchtdiode umzupolen. Sollte die LED zu schwach
      leuchten, kann der Vorwiderstand bis auf 470 Ohm reduziert
      werden.

      Es ist auch möglich, zwei verschiedenfarbige LEDs anzuschließen.
      Dann ist die zweite LED ebenso über einen Vorwiderstand zwischen
      DTR und GND anzuschließen, jedoch genau umgekehrt. Dann leuchtet
      je nach Zustand die eine oder die andere LED. Oder man verwendet
      direkt eine DUO-LED (zweifarbig, drei Anschlussbeinchen).

      Im Moment verhält sich der RTS-Anschluss der seriellen
      Schnittstelle genauso wie DTR. Hier könnte also noch eine
      weitere LED angechlossen werden, die den Online/Offline-Zustand
      anzeigt. Das könnte sich jedoch in einer zukünftigen
      fli4l-Version ändern.

      Als Wert von \var{IMOND\_\-LED} muss ein COM-Port angegeben werden, also
      'com1', 'com2', 'com3' oder 'com4'. Ist keine LED angeschlossen,
      sollte die Variable leer gelassen werden.}


    \config{IMOND\_BEEP}{IMOND\_BEEP}{IMONDBEEP}

    {Mit \var{IMOND\_\-BEEP}='yes' gibt imond einen Zweiklang-Ton über den
      PC"=Lautsprecher aus, wenn der Zustand von Offline nach Online
      wechselt und umgekehrt. Im ersten Fall wird zuerst ein tiefer,
      dann ein hoher Ton ausgegeben. Beim Wechsel in den
      Offline-Status zurück wird zuerst der höhere, dann der tiefere
      Ton ausgegeben.}


    \config{IMOND\_LOG}{IMOND\_LOG}{IMONDLOG}

    Standard-Einstellung: \var{IMOND\_\-LOG='no'}
    
    {Wird \var{IMOND\_LOG}='yes' benutzt, werden in der
      Datei \verb+/var/log/imond.log+ die Verbindungen protokolliert. Diese Datei kann z.B.
      für Statistikzwecke per scp auf einen Rechner im LAN kopiert
      werden. Für den scp-Zugriff ist aber dann noch das Paket
      sshd zu installieren und so zu konfigurieren, dass es auch scp zur
      Verfügung stellt.

      Das Format der Logdateieinträge ist in Tabelle \ref{tab:imondlog}
      beschrieben.
      \begin{table}[htbp]
        \small
        \centering
        \caption{Format der Imond-Logdatei}\label{tab:imondlog}
        \begin{tabular}{lp{12cm}}
          \hline
          Eintrag & Bedeutung \\
          \hline
          Circuit & der Name des Circuits, für den der Eintrag erzeugt
          wurde \\
          Startzeit & Datum und Uhrzeit der Einwahl dieses Circuits \\
          Stopzeit & Datum und Uhrzeit des Auflegens dieses Circuits \\
          Online-Zeit & die Zeit, die dieser Circuit online war \\
          Abgerechnete Zeit & die Zeit, die der Provider abrechnen wird
          (hängt vom Takt ab) \\
          Kosten & die Kosten, die der Provider für die Zeit in Rechnung
          stellt \\
          Bandbreite & die genutzte Bandbreite getrennt nach in und out
          (in zuerst), dargestellt als zwei vorzeichenlose
          Integerzahlen, für die gilt: Bandbreite =\newline
          \emph{4GiB~$*<$erste~Zahl$>+<$zweite~Zahl$>$} \\
          Device & das Gerät, über das kommuniziert wurde \\
          Abrechnungstakt & der Takt, der vom Provider zur
          Abrechnung herangezogen wird (Daten der Circuit-Konfiguration)\\
          Taktgebühren & die Gebühren, die pro Takt fällig werden
          (Daten der Circuit-Konfiguration)\\
          \hline
        \end{tabular}
      \end{table}

      Die Kosten werden in Euro ausgegeben. Wichtig ist dabei die
      korrekte Definition der entsprechenden Circuit-Variablen
      \jump{ISDNCIRCxTIMES}{\var{ISDN\_CIRC\_x\_TIMES}}.
      }

    \config{IMOND\_LOGDIR}{IMOND\_LOGDIR}{IMONDLOGDIR}

    {Ist das Protokollieren eingeschaltet, kann über \var{IMOND\_\-LOGDIR} ein
      alternatives Verzeichnis statt /var/log angegeben werden, z.B.
      '/boot'. Dann wird die Log-Datei imond.log auf dem Bootmedium
      angelegt. Dazu muss dieses aber auch Read/Write
      \glqq{}gemounted\grqq{} sein. Default ist 'auto' was den Speicherort automatisch
      bestimmt. Je nach weiterer Konfiguration liegt das dann unter /boot/persistent/base
      oder an einem anderen durch \var{FLI4L\_UUID} bestimmten Pfad. Ist /boot nicht Read/Write
      und \var{FLI4L\_UUID} nicht gesetzt, befindet sich das File unter /var/run.}

    \configlabel{IMOND\_DIAL}{IMONDDIAL}
    \configlabel{IMOND\_ROUTE}{IMONDROUTE}
    \configlabel{IMOND\_REBOOT}{IMONDREBOOT}
    \config{IMOND\_ENABLE  IMOND\_DIAL  IMOND\_ROUTE  IMOND\_REBOOT}{IMOND\_ENABLE}{IMONDENABLE}

    {Durch diese Variablen werden bestimmte Kommandos, die von
      imonc-Clients zum imond-Server gesendet werden, bereits im
      User-Modus freigeschaltet.

      Hiermit kann man einstellen, ob der imond-Server die
      ISDN-Schnittstelle ein- bzw. ausschalten, wählen/einhängen, eine neue
      Default-Route setzen und/oder den Rechner booten darf.


      Standard-Einstellungen:

    \begin{example}
      \begin{verbatim}
        IMOND_ENABLE='yes'
        IMOND_DIAL='yes'
        IMOND_ROUTE='yes'
        IMOND_REBOOT='yes'
      \end{verbatim}
    \end{example}

      Alle weiteren Features der Client-/Server-Schnittstelle von imond sind in
      einem \jump{IMONDSCHNITTSTELLE}{eigenen Kapitel} beschrieben.}
\end{description}
