% Last Update: $Id$
\section{Verwendung einer eigenen /etc/inittab}

  Man kann von Init zusätzliche Programme auf
  zusätzlichen Konsolen starten lassen oder die Standardkommandos der
  Init-Konfigurationsdatei verändern. Ein Eintrag sieht wie folgt
  aus:

  \begin{example}
    \begin{verbatim}
      device:runlevel:action:command
    \end{verbatim}
  \end{example}

  Das \emph{device} ist das Gerät, über das das Programm seine
  Ein-/Ausgaben machen soll. Möglich sind hier die normalen Terminals
  tty1-tty4 oder serielle Terminals ttyS0-ttySn mit $n <$ Anzahl der
  vorhandenen seriellen Schnittstellen.

  Als \emph{action} kommen in der Regel \emph{askfirst} oder \emph{respawn} 
  in Frage. Bei askfirst wird auf einen Tastendruck
  gewartet, bevor das Kommando ausgeführt wird, bei respawn wird es
  automatisch neu gestartet, wenn sich das Programm beendet.

  \emph{command} ist das Programm, das ausgeführt werden soll. Es
  ist mit vollständiger Pfadangabe zu spezifizieren.

  Die Busybox-Dokumentation unter \altlink{http://www.busybox.net} enthält eine
  genaue Beschreibung des inittab Formats.

  Die normale inittab sieht wie folgt aus:

  \begin{example}
    \begin{verbatim}
      ::sysinit:/etc/rc
      ::respawn:cttyhack /usr/local/bin/mini-login
      ::ctrlaltdel:/sbin/reboot
      ::shutdown:/etc/rc0
      ::restart:/sbin/init
    \end{verbatim}
  \end{example}

  Diese könnte man z.B. um den Eintrag

  \begin{example}
    \begin{verbatim}
      tty2::askfirst:cttyhack /usr/local/bin/mini-login
    \end{verbatim}
  \end{example}

  erweitern, um ein zweites Login auf Terminal zwei zu bekommen. Dazu
  einfach die Datei opt/etc/inittab nehmen, nach $<$config
  verzeichnis$>$/etc/inittab kopieren und mit einem Editor bearbeiten.
