% Last Update: $Id: base_main_general_circuit_configuration.tex 48716 2017-08-16 06:53:23Z florian $
\marklabel{sect:circuits}{\section{Circuit-Konfiguration}}

\newcommand{\circstatetrans}[2]{\emph{#1} $\rightarrow$ \emph{#2}}
\newcommand{\dialmodetrans}[2]{\emph{#1} $\rightarrow$ \emph{#2}}

\subsection{Circuits allgemein}\label{sec:circuits:general}

Der fli4l-Router erlaubt ab Version 4.0, Verbindungen nach ``außen'' flexibel
über so genannte ``Circuits'' zu konfigurieren. Der Begriff ``Circuit'' kommt
aus dem Englischen und bedeutet in diesem Zusammenhang so viel wie ``Leitung''.
Seine Verwendung in fli4l entstammt dem Wählen und Auflegen von
ISDN-Verbindungen, das seit der ersten fli4l-Version möglich ist; da ISDN
ein leitungsvermittelnder (``circuit-switched'') Dienst ist, hat sich der
Begriff etabliert und wird heute in allen anderen Verbindungs-Situationen
verwendet, auch wenn es meistens nicht mehr um leitungsvermittelnde, sondern um
paketvermittelnde Dienste geht. Auch in dieser Dokumentation wird der Begriff
des Circuits durchgängig verwendet.

Ein konfigurierter Circuit erlaubt es dem fli4l-Router, irgendeine Form der
Verbindung zwischen dem Router und einem anderen Netzwerk-Kommunikationspartner
herzustellen. Meistens, aber nicht immer, geht es dabei um die Herstellung
einer Internet-Anbindung. Im Folgenden wird eine kurze Übersicht darüber
gegeben, welche Circuit-Typen der fli4l-Router beherrscht (die meisten davon
werden jedoch von anderen Paketen angeboten, dies ist aber entsprechend in der
Tabelle vermerkt).

\begin{longtable}{|l|l|l|p{0.4\textwidth}|}
    \hline
    \multicolumn{1}{|l}{\textbf{Typ}} &
    \multicolumn{1}{|l}{\textbf{Paket}} &
    \multicolumn{1}{|l}{\textbf{OPT}} &
    \multicolumn{1}{|l|}{\textbf{Beschreibung}} \\
    \hline
    \endhead
    \hline
    \endfoot
    \endlastfoot

route & base & - &
Mit Hilfe von route-Circuits können Routen in andere Netze konfiguriert werden.
Dies entspricht im Wesentlichen der Funktionalität der Variablen
\var{IP\_ROUTE\_\%} (siehe \jump{IPROUTEN}{\var{IP\_ROUTE\_N}}), geht jedoch
etwas darüber hinaus. Intern werden alle per \var{IP\_ROUTE\_\%} konfigurierte
Routen in route-Circuits abgebildet.
    \\
    \hline

dhcp & dhcp\_client & \var{OPT\_DHCP\_CLIENT} &
Mit Hilfe von dhcp-Circuits lassen sich IPv4- und IPv6-Adressinformationen von
einem DHCP-Server ermitteln. Dies ist vor allem dann sinnvoll, wenn der fli4l
nicht selbsttätig eine Internet-Verbindung aufbaut, sondern sich hinter einem
anderen Router befindet, der dies für einen erledigt, etwa hinter einem
Kabelmodem.
    \\
    \hline

isdn & isdn & \var{OPT\_ISDN} &
Mit Hilfe von isdn-Circuits ist eine Einwahl in ein anderes Netz (z.\,B. ein
Firmennetz) über ISDN möglich.
    \\
    \hline

ppp & ppp (+ diverse) & \var{OPT\_PPP} (+ diverse) &
Mit Hilfe von ppp-Circuits ist eine Einwahl ins Internet oder ein Firmennetz
über eine Reihe diverser Kanäle mit Hilfe des Point-to-Point-Protokolls (PPP)
möglich. Mehr dazu ist in Abschnitt \ref{sect:ppp-circuits} zu finden.
    \\
    \hline

    \caption{Verfügbare Circuit-Typen}\marklabel{circuit:types}{}
\end{longtable}

Alle Circuits werden der Übersichtlichkeit halber in der Datei
\texttt{circuits.txt} konfiguriert. Die Anzahl der konfigurierten Circuits wird
dabei in der Variable \var{CIRC\_N} festgehalten:

\begin{description}

\config{CIRC\_N}{CIRC\_N}{CIRCN}

Diese Variable gibt die Anzahl der konfigurierten Circuits an.

Standard-Einstellung: \verb+CIRC_N='0'+

Beispiel: \verb+CIRC_N='4'+

\end{description}

Darüber hinaus besitzt jeder Circuit, egal von welchem Typ, einige allgemeine
Attribute. Zuerst operiert \emph{jeder} Circuit auf einer
\emph{Netzwerk-Schnittstelle}. Je nach Circuit-Typ ist diese Schnittstelle
statisch vorhanden (dies gilt z.\,B. für Ethernet-Schnittstellen wie ``eth0''),
oder sie wird dynamisch beim Einwählen erzeugt (das ist z.\,B. bei den
``pppX''-Schnittstellen von ppp-Circuits der Fall), oder sie wird von einem
anderen Circuit erzeugt und hier nur referenziert (das ist z.\,B. bei
DHCPv6-over-PPPoE der Fall). Auf Grund dieser verschiedenen
Möglichkeiten gibt es keine einheitliche Variable zur Angabe der Schnittstelle,
stattdessen kümmert sich jeder Circuit-Typ selbst um die Konfiguration der
Schnittstelle (und einige wie ppp verbieten die Angabe einer Schnittstelle
gänzlich, da sie automatisch erzeugt und verwaltet wird).

Weitere allgemeine Attribute werden über die folgenden Variablen konfiguriert.

\begin{description}

\config{CIRC\_x\_NAME}{CIRC\_x\_NAME}{CIRCxNAME}

Jeder Circuit hat einen Namen. Dieser Name kann aus Buchstaben, Ziffern und dem
Bindestrich (`-') bestehen und hilft dabei, den Circuit in der Web-GUI, in
Protokollen etc. zu identifizieren. Der Name muss unter allen Circuits und
Circuit-Klassen (siehe \jump{CIRCCLASSxNAME}{\var{CIRC\_CLASS\_x\_NAME}})
eindeutig sein.

Beispiel: \verb+CIRC_x_NAME='DSL-Telekom'+

\config{CIRC\_x\_TYPE}{CIRC\_x\_TYPE}{CIRCxTYPE}

Jeder Circuit hat einen Typ. Dieser Typ bestimmt, wie das Wählen und Auflegen
funktioniert. Intern entscheidet der Typ, welches Skript zum Wählen und
Auflegen verwendet wird.

Es sind immer nur die Typen der aktivierten Pakete verfügbar. Wenn Sie also
z.\,B. einen dhcp-Circuit verwenden möchten, dann müssen Sie das Paket
dhcp\_client herunterladen und DHCP mit \verb+OPT_DHCP_CLIENT='yes'+
aktivieren. Andernfalls bekommen Sie bei der Verwendung des Typs ``dhcp'' beim
Bauen der Installationsarchive eine Fehlermeldung.

Beispiel: \verb+CIRC_x_TYPE='dhcp'+

\config{CIRC\_x\_ENABLED}{CIRC\_x\_ENABLED}{CIRCxENABLED}

Die Variable \verb+CIRC_x_ENABLED+ aktiviert einen Circuit zur
Konfigurationszeit. Damit der Circuit überhaupt Beachtung findet, muss diese
Variable auf \verb+'yes'+ gesetzt werden. Gilt hingegen
\verb+CIRC_x_ENABLED='no'+, dann wird der Circuit auf dem fli4l nicht
konfiguriert. Auch kann zur Konfigurationszeit der Circuits nicht mit seinem
Namen angesprochen werden, etwa in Firewall-Regeln.

Standard-Einstellung: \verb+CIRC_x_ENABLED='no'+

Beispiel: \verb+CIRC_x_ENABLED='yes'+

\config{CIRC\_x\_DIALMODE}{CIRC\_x\_DIALMODE}{CIRCxDIALMODE}

Jeder Circuit kann einen individuellen initialen \jump{sect:dialmode}{Wählmodus}
erhalten, der dann über diese Variable konfiguriert werden kann.

Standard-Einstellung: \verb+CIRC_x_DIALMODE='auto'+

Beispiel: \verb+CIRC_x_DIALMODE='manual'+

\configlabel{CIRC\_x\_NETS\_IPV4\_N}{CIRCxNETSIPV4N}
\config{CIRC\_x\_NETS\_IPV4\_y}{CIRC\_x\_NETS\_IPV4\_y}{CIRCxNETSIPV4y}

Ein Circuit ist nur dann sinnvoll, wenn er auch zu einer Netzanbindung führt.
Dazu ist es in der Regel notwendig, dass Routen konfiguriert werden, sobald
eine erfolgreiche Einwahl stattgefunden hat. Welche IPv4-Netze über den Circuit
geroutet werden, kann über diese Variablen festgelegt werden.

Im häufigsten Fall der Circuit-Nutzung, der Internet-Anbindung, muss die
Default-Route über den Circuit gehen. Dazu muss das Netz 0.0.0.0/0 in die Liste
eingetragen werden.

Nur in Ausnahmefällen müssen keine Netze angegeben werden, etwa bei
Server-Circuits, bei denen keine Routen zurück zu den Clients installiert werden
sollen.

Standard-Einstellung: \verb+CIRC_x_NETS_IPV4_N='0'+

Beispiel 1:
\begin{example}
\begin{verbatim}
    CIRC_x_NETS_IPV4_N='1'
    CIRC_x_NETS_IPV4_1='0.0.0.0/0'
\end{verbatim}
\end{example}

Beispiel 2:
\begin{example}
\begin{verbatim}
    CIRC_x_NETS_IPV4_N='2'
    CIRC_x_NETS_IPV4_1='10.15.16.0/24'
    CIRC_x_NETS_IPV4_2='10.16.0.0/16'
\end{verbatim}
\end{example}

\configlabel{CIRC\_x\_NETS\_IPV6\_N}{CIRCxNETSIPV6N}
\config{CIRC\_x\_NETS\_IPV6\_y}{CIRC\_x\_NETS\_IPV6\_y}{CIRCxNETSIPV6y}

Hier werden analog zu \var{CIRC\_x\_NETS\_IPV4\_y} IPv6-Netze angegeben, die
nach der Einwahl über den Circuit geroutet werden sollen. Diese Variablen können
nur bei aktivierter IPv6-Unterstützung (Paket ipv6, \verb+OPT_IPV6='yes'+)
verwendet werden.

Im häufigsten Fall der Circuit-Nutzung, der Internet-Anbindung, muss die
Default-Route über den Circuit gehen. Dazu muss das Netz ::/0 in die Liste
eingetragen werden.

Nur in Ausnahmefällen müssen keine Netze angegeben werden, etwa bei
Server-Circuits, bei denen keine Routen zurück zu den Clients installiert werden
sollen.

Standard-Einstellung: \verb+CIRC_x_NETS_IPV6_N='0'+

Beispiel 1:
\begin{example}
\begin{verbatim}
    CIRC_x_NETS_IPV6_N='1'
    CIRC_x_NETS_IPV6_1='::/0'
\end{verbatim}
\end{example}

Beispiel 2:
\begin{example}
\begin{verbatim}
    CIRC_x_NETS_IPV6_N='2'
    CIRC_x_NETS_IPV6_1='2001:db8:1::/48'
    CIRC_x_NETS_IPV6_2='2001:db8:2::/48'
\end{verbatim}
\end{example}

\config{CIRC\_x\_PROTOCOLS}{CIRC\_x\_PROTOCOLS}{CIRCxPROTOCOLS}

In der Regel werden die Layer-3-Protokolle, die von einem Circuit unterstützt
werden (IPv4 oder IPv6) aus den konfigurierten Netzen
(\var{CIRC\_x\_NETS\_IPV4\_y} und \var{CIRC\_x\_NETS\_IPV6\_y}, siehe oben)
abgeleitet. In manchen Fällen werden jedoch keine Netze angegeben, weil keine
Routen aufgebaut werden sollen. Dies ist beispielsweise bei Server-Circuits der
Fall. In solchen Fällen ist es nötig, die zu verwendenden Layer-3-Protokolle
explizit einzustellen. Dazu wird in dieser Variable eine Liste von Protokollen
notiert, die durch Leerzeichen voneinander getrennt sind. Erlaubte Protokolle
sind \texttt{ipv4} und \texttt{ipv6}.

Standard-Einstellung: abgeleitet aus \var{CIRC\_x\_NETS\_IPV4\_y} und
\var{CIRC\_x\_NETS\_IPV6\_y}

Beispiel: \verb+CIRC_x_PROTOCOLS='ipv4 ipv6'+

\config{CIRC\_x\_UP}{CIRC\_x\_UP}{CIRCxUP}

Ist \verb+CIRC_x_UP='yes'+, dann wird der entsprechende Circuit beim Booten
aktiviert. Je nach \jump{sect:dialmode}{Wählmodus} kann dies bereits beim
Booten zu einer Einwahl führen, oder die Einwahl kann später über die GUI oder
das \texttt{fli4lctrl}-Programm veranlasst werden. Bei \verb+CIRC_x_UP='no'+
muss der Circuit erst mit Hilfe der GUI oder mit dem \texttt{fli4lctrl}-Programm
aktiviert werden, bevor eine Einwahl gestartet werden kann.

Falls Circuits aktiviert werden, die sich in einem oder mehreren gerouteten
Netzen überlappen, ist dies ein Fehler, der bereits beim Versuch, das
Installationsarchiv zu bauen, von \texttt{mkfli4l} gemeldet
wird.\footnote{In einer späteren Version ist angedacht, in einem solchen Fall
den Datenverkehr für diese Netze gleichmäßig auf alle diese Circuits zu
verteilen.}

Diese Einstellung ist nur relevant mit \verb+OPT_CIRCD='no'+.

Standard-Einstellung: \verb+CIRC_x_UP='no'+

Beispiel: \verb+CIRC_x_UP='yes'+

\config{CIRC\_x\_PRIORITY}{CIRC\_x\_PRIORITY}{CIRCxPRIORITY}

Diese Variable gibt die Priorität des Circuits an. Höhere Werte bedeuten eine
niedrigere Priorität. Prioritäten dienen dazu, Circuits nach Eignung zu
gruppieren: Bei der Auswahl von Circuits\footnote{durch den Hintergrundprozess
\texttt{circd}} werden alle Circuits prioritätsweise abgearbeitet. Nur wenn in
der höchsten Prioritätsklasse keine Circuits in Frage kommen, kommen die
Circuits der nächstniedrigeren Prioritätsklasse zum Zuge. Mehr zum
Auswahlalgorithmus des \texttt{cird} finden Sie im Abschnitt
\jump{sect:circd}{``Das Programm \texttt{circd}''}.

Diese Einstellung ist nur relevant mit \verb+OPT_CIRCD='yes'+.

Standard-Einstellung: \verb+CIRC_x_PRIORITY='1'+

Beispiel: \verb+CIRC_x_PRIORITY='2'+

\config{CIRC\_x\_TIMES}{CIRC\_x\_TIMES}{CIRCxTIMES}

Ist \verb+OPT_CIRCD='yes'+, dann enthält diese Variable eine
Zeitspezifikation, die angibt, wann der Circuit aktiviert werden soll und wann
nicht, und wie viel der Circuit bei einer erfolgreichen Einwahl pro Minute
kostet. Dadurch wird es möglich, zu verschiedenen Zeiten verschiedene Circuits
zu verwenden (Least-Cost-Routing). Dabei kontrolliert der Dämon-Prozess
\texttt{circd} das Aktivieren und Deaktivieren der Circuits.

Der Inhalt der Variablen ist wie folgt aufgebaut:
\begin{example}
\begin{verbatim}
    CIRC_x_TIMES='W1-W2:hh-hh:Kosten:Typ [W1-W2:hh-hh:Kosten:Typ [...]]'
\end{verbatim}
\end{example}

Jedes Feld besteht aus vier Unterfeldern, die mit Hilfe eines Doppelpunkts (':')
voneinander getrennt sind:

\begin{itemize}

\item Feld \emph{W1--W2}: Wochentag-Zeitraum, z.\,B. Mo--Fr oder Sa--Su usw.
Sowohl die deutsche als auch die englische Schreibweise sind erlaubt. Soll ein
einzelner Wochentag eingetragen werden, ist W1--W1 zu schreiben, also z.\,B.
Su--Su.

\item Feld \emph{hh--hh}: Stunden-Bereich, z.\,B. 09--18 oder auch 18--09.
18--09 ist gleichbedeutend mit 18--24 plus 00--09. 00--24 meint den ganzen Tag.
Stunden müssen immer zweistellig (also notfalls mit führenden Nullen) angegeben
werden.

\item Feld \emph{Kosten}: Hier werden in Euro die Kosten pro Minute angegeben,
z.\,B. 0.032 für 3,2 Cent pro Minute. Diese werden unter Berücksichtigung der
Taktzeit in die tatsächlich anfallenden Kosten umgerechnet, welche dann im
imon-Client oder der WebGUI angezeigt werden.

\item Feld \emph{Typ}: Dieses Feld gibt den Typ des Zeitraums an:

\begin{tabular}[h!]{lp{11cm}}
    Y oder J: & Im angegebenen Zeitraum wird der Circuit aktiviert,
                unabhängig von den anfallenden Kosten.\\
    L:        & Im angegebenen Zeitraum wird der Circuit aktiviert,
                wenn er zu den günstigsten Circuits gehört. (`L' steht für
                `Least cost', also ``niedrigste Kosten''.)\\
    N:        & Der angegebene Zeitbereich dient nur zum Berechnen
                von Kosten, der Circuit wird in diesem Zeitraum jedoch
                nicht aktiviert. Dies kann sinnvoll sein, wenn der
                Circuit \emph{manuell} aktiviert wird, etwa wenn es sich um
                einen Circuit zur Verbindung mit einem Firmennetz handelt,
                der nur bei Bedarf hinzugeschaltet wird.
\end{tabular}

 Der Typ kann weggelassen werden, in diesem Fall wird `L' angenommen.
\end{itemize}

Standard-Einstellung: \verb+CIRC_2_TIMES='Mo-Su:00-24:0.0:N'+

Beispiel 1:

\begin{example}
\begin{verbatim}
    CIRC_1_TIMES='Mo-Fr:09-18:0.049:N Mo-Fr:18-09:0.044:L Sa-Su:00-24:0.039:Y'
\end{verbatim}
\end{example}

Beispiel 2 für diejenigen, die eine Flatrate nutzen:

\begin{example}
\begin{verbatim}
    CIRC_2_TIMES='Mo-Su:00-24:0.0:Y'
\end{verbatim}
\end{example}

\wichtig{Wenn die Zeitbereiche aller aktivierten Circuits mit einer
Default-Route zusammengenommen nicht die komplette Woche beinhalten, gibt es zu
diesen Lückenzeiten keine Default-Route. Damit ist dann das Surfen im Internet
zu diesen Zeiten ausgeschlossen!}

Und noch eine letzte Bemerkung: \emph{Feiertage werden wie Sonntage behandelt.}

\config{CIRC\_x\_CHARGEINT}{CIRC\_x\_CHARGEINT}{CIRCxCHARGEINT}

Mit dieser Variable wird der Zeittakt in Sekunden angegeben. Dieser
wird dann für die Kosten-Berechnung verwendet.

Die meisten Provider rechnen minutengenau ab. In diesem Fall ist der Wert '60'
richtig. Bei Providern mit sekundengenauer Abrechnung setzt man die Variable
entsprechend auf '1'.

Diese Variable ist nur möglich bzw. sinnvoll bei Circuits, die tatsächlich
Kosten verursachen. Bei Circuits, die keine Einwahl im herkömmlichen Sinne
durchführen (route, dhcp), ist diese Einstellung nicht möglich.

Standard-Einstellung: \verb+CIRC_x_CHARGEINT='0'+

Beispiel: \verb+CIRC_x_CHARGEINT='60'+

\config{CIRC\_x\_HUP\_TIMEOUT}{CIRC\_x\_HUP\_TIMEOUT}{CIRCxHUPTIMEOUT}

Hier kann die Zeit in Sekunden angegeben werden, nach welcher die Verbindung
beendet werden soll, wenn kein Datenverkehr mehr über den Circuit läuft. Dabei
steht ein Timeout von '0' für ``kein Timeout'', d.\,h. der Router legt nicht
auf und wählt sich nach einem Zwangsauflegen auch sofort wieder neu ein.

Momentan wird ein Hangup-Timeout > 0 nur für ppp-Circuits unterstützt.

Diese Eigenschaft unterscheidet generell zwischen ``normalen'' Circuits, die
sich bei der \texttt{fli4lctrl dial}-Operation sofort einwählen
(Hangup-Timeout gleich null), und ``dial-on-demand''-Circuits, die nach der
\texttt{fli4lctrl dial}-Operation nur bereit für eine (kommende) Einwahl sind
(Hangup-Timeout größer null). Mehr Informationen hierzu finden sich in den
Abschnitten \jump{sect:circstates}{``Circuit-Zustände''} und
\jump{sect:fli4lctrl}{``Das Programm \texttt{fli4lctrl}''}.

Standard-Einstellung: \verb+CIRC_x_HUP_TIMEOUT='0'+

Beispiel: \verb+CIRC_x_HUP_TIMEOUT='600'+

\config{CIRC\_x\_USEPEERDNS}{CIRC\_x\_USEPEERDNS}{CIRCxUSEPEERDNS}

Hiermit wird festgelegt, ob die vom Provider bei der Einwahl übergebenen
DNS-Namensserver für die Dauer der Verbindung in die Konfigurationsdatei des
lokalen DNS-Servers (dnsmasq) eingetragen werden sollen.

Sinnvoll ist die Nutzung dieser Option also nur bei Circuits, die entsprechende
Informationen liefern. Dies betrifft i.\,d.\,R. Internet-Anbindungen via PPP
und DHCP-Circuits.

Diese Option bietet den Vorteil, immer mit den am nächsten liegenden
DNS-Namensser\-vern arbeiten zu können, sofern der Provider die korrekten
IP-Adressen übermittelt~-- dadurch geht die Namensauflösung schneller.

Im Falle eines Ausfalls eines DNS-Servers beim Provider werden in der Regel die
übergebenen DNS-Server-Adressen sehr schnell vom Provider korrigiert.

Trotz allem ist vor jeder ersten Einwahl die Angabe eines gültigen
Namensservers in \var{DNS\_\-FORWARDERS} zwingend erforderlich, da sonst die
erste Anfrage nicht korrekt aufgelöst werden kann. Außerdem wird beim Beenden
der Verbindung die originale Konfiguration des lokalen Namensservers wieder
hergestellt.

Standard-Einstellung: \verb+CIRC_x_USEPEERDNS='no'+

Beispiel: \verb+CIRC_x_USEPEERDNS='yes'+

\config{CIRC\_x\_WAIT}{CIRC\_x\_WAIT}{CIRCxWAIT}

In der Regel wird das Wählen eines Circuits im Hintergrund durchgeführt. Will
man jedoch beim Booten sicherstellen, dass sich ein Circuit erfolgreich
eingewählt hat, kann man diese Variable auf die Anzahl der maximal zu wartenden
Sekunden setzen. Bei `0' wird nicht gewartet.

Ein Wert größer null kann bei ppp-Circuits (d.\,h. wenn
\verb+CIRC_x_TYPE='ppp'+ gesetzt ist)
nur verwendet werden, wenn es sich um keinen Dial-on-demand-Circuit handelt,
wenn also \verb+CIRC_x_HUP_TIMEOUT='0'+ gesetzt ist (bzw. die Variable gar
nicht definiert wird). Der Grund hierfür ist, dass ein Warten auf einen
Dial-on-demand-Circuit wenig Sinn hat, weil der Wählvorgang erst bei
entsprechender Netzwerkaktivität erfolgt.

Standard-Einstellung: \verb+CIRC_x_WAIT='0'+

Beispiel: \verb+CIRC_x_WAIT='15'+

\config{CIRC\_x\_DEBUG}{CIRC\_x\_DEBUG}{CIRCxDEBUG}

Mit dieser Variable können zusätzliche Debug-Ausgaben eingeschaltet werden.
Dies ist Circuit-spezifisch und hat nicht zwangsläufig bei jedem Circuit-Typ
sichtbare Auswirkungen.

Standard-Einstellung: \verb+CIRC_x_DEBUG='no'+

Beispiel: \verb+CIRC_x_DEBUG='yes'+

\config{CIRC\_x\_DEPS}{CIRC\_x\_DEPS}{CIRCxDEPS}

Mit dieser Variable können Abhängigkeiten zwischen Circuits spezifiziert
werden. Ein Circuit A, der von einem Circuit B abhängig ist, kann nur dann
online gehen, wenn Circuit B ebenfalls online ist. Dies ist insbesondere dann
nützlich, wenn Circuit A über seine Netzanbindung Infrastruktur zur Verfügung
stellt, die Circuit B benötigt, etwa wenn Circuit A eine IPv4-Internetanbindung
herstellt und Circuit B einen 6in4-Tunnel aufbaut.

Standard-Einstellung: \verb+CIRC_x_DEPS=''+

Beispiel 1: \verb+CIRC_x_DEPS='internet'+

Gelegentlich muss nicht der \emph{gesamte} Circuit mit all seinen konfigurierten
Layer-3-Proto\-kollen online sein, damit eine Abhängigkeit erfüllt ist, sondern
nur ein bestimmtes Protokoll, z.B. IPv4 oder IPv6. In diesem Fall kann man das
Protokoll hinter dem Circuit angeben, mit einem Schrägstrich abgetrennt.
Das folgende Beispiel zeigt eine Abhängigkeit zu einem Tag oder Circuit namens
``internet'', wobei es ausreicht, wenn dessen IPv4-Anbindung online ist. Dies
ist z.B. für 6in4-Tunnel völlig ausreichend, denn die IPv6-Konnektivität spielt
bei 6in4 naturgemäß keine Rolle (schließlich stellt ein 6in4-Tunnel gerade eine
IPv6-Anbindung über eine IPv4-Anbindung her).

Beispiel 2: \verb+CIRC_x_DEPS='internet/ipv4'+

\configlabel{CIRC\_x\_CLASS\_N}{CIRCxCLASSN}
\config{CIRC\_x\_CLASS\_y}{CIRC\_x\_CLASS\_y}{CIRCxCLASSy}

Mit diesen Variablen können einem Circuit \jump{sec:circuits:classes}{Klassen}
zugeordnet werden. Durch Klassen sich Circuits logisch gruppieren, es wird
somit eine Abstraktion geschaffen, die auf vielfältige Weise ausgenutzt werden
kann. Eine mögliche Anwendung ist die Nutzung innerhalb von Abhängigkeiten
zwischen Circuits (\var{CIRC\_x\_DEPS}, siehe oben), wenn mehrere Circuits eine
Abhängigkeit erfüllen können.

Jede hier angegebene Klasse muss via
\jump{CIRCCLASSxNAME}{\var{CIRC\_CLASS\_x\_NAME}} definiert werden. Ist dies
nicht der Fall, wird eine Fehlermeldung ausgegeben.

Standard-Einstellung: \verb+CIRC_x_CLASS_N='0'+

Beispiel:

\begin{example}
\begin{verbatim}
    CIRC_1_NAME='DHCP-LAN'
    CIRC_1_TYPE='dhcp'
    CIRC_1_ENABLED='yes'
    CIRC_1_DHCP_DEV='eth0'
    CIRC_1_NETS_IPV4_N='1'
    CIRC_1_NETS_IPV4_1='0.0.0.0/0'
    CIRC_1_CLASS_N='1'
    CIRC_1_CLASS_1='internet-v4'

    CIRC_2_NAME='DSL-Telekom'
    CIRC_2_TYPE='ppp'
    CIRC_2_ENABLED='yes'
    CIRC_2_PPP_TYPE='ethernet'
    CIRC_2_PPP_USERID='anonymer'
    CIRC_2_PPP_PASSWORD='surfer'
    CIRC_2_PPP_ETHERNET_TYPE='kernel'
    CIRC_2_PPP_ETHERNET_DEV='eth1'
    CIRC_2_NETS_IPV4_N='1'
    CIRC_2_NETS_IPV4_1='0.0.0.0/0'
    CIRC_2_CLASS_N='1'
    CIRC_2_CLASS_1='internet-v4'

    CIRC_3_NAME='IPv6-Tunnel'
    CIRC_3_TYPE='tun6in4-he'
    CIRC_3_ENABLED='yes'
    CIRC_3_NETS_IPV6_N='1'
    CIRC_3_NETS_IPV6_1='::/0'
    CIRC_3_CLASS_N='1'
    CIRC_3_CLASS_1='internet-v6'
    CIRC_3_DEPS='internet-v4'
\end{verbatim}
\end{example}

In diesem Beispiel ist ein 6in4-HE-Tunnel (siehe Paket ipv6) abhängig von
einem Circuit der Klasse ``internet-v4''. Ob das zur Laufzeit dann die
DSL- (siehe Paket dsl) oder DHCP-Anbindung (siehe Paket dhcp\_client) ist, ist
egal~-- sobald einer der beiden Circuits online ist, kann der Tunnel ebenfalls
online gehen.

\config{CIRC\_x\_BUNDLE}{CIRC\_x\_BUNDLE}{CIRCxBUNDLE}

Ist \verb+CIRC_x_BUNDLE+ nicht leer und referenziert es einen anderen gültigen
Circuit, dann wird der referenzierende Circuit Teil eines so genannten
``Bündels''. Gebündelte Circuits bilden zusammen \emph{eine} logische
Verbindung. Dies wird momentan nur vom Paket ppp unterstützt, siehe
Abschnitt \jump{sect:multilink-ppp}{``Multilink PPP''}.

Beispiel: \verb+CIRC_x_BUNDLE='internet-mp'+

\end{description}

\marklabel{sect:circstates}{\subsection{Circuit-Zustände}}

Jeder Circuit hat, während der Router läuft, einen der folgenden Zustände:

\begin{longtable}{|l|l|p{0.6\textwidth}|}
    \hline
    \multicolumn{1}{|l}{\textbf{Zustand}} &
    \multicolumn{1}{|l}{\textbf{Interner Name}} &
    \multicolumn{1}{|l|}{\textbf{Beschreibung}} \\
    \hline
    \endhead
    \hline
    \endfoot
    \endlastfoot

\emph{inaktiv} & inactive &
Ein Circuit ist \emph{inaktiv}, wenn er nicht zur Einwahl herangezogen werden
kann.
    \\
    \hline
\emph{aktiv}   & active   &
Ein Circuit ist \emph{aktiv}, wenn er zur Einwahl herangezogen werden kann.
    \\
    \hline
\emph{bereit}  & ready    &
Ein Circuit ist \emph{bereit}, wenn er sich bei Netzwerk-Aktivität automatisch
einwählt
(``dial-on-demand'').
    \\
    \hline
\emph{online}  & online   &
Ein Circuit ist \emph{online}, wenn die Verbindung erfolgreich aufgebaut werden
konnte und die Netzwerk-Anbindung erfolgt ist.
    \\
    \hline
\emph{ausgefallen}  & failed   &
Ein Circuit ist \emph{ausgefallen}, wenn festgestellt wurde, dass die
Verbindung über diesen Circuit nicht funktioniert. Dieser Zustand entspricht
fast komplett dem Zustand \emph{inaktiv}, mit dem einzigen Unterschied, dass
ein ausgefallener Zustand von \texttt{circd} ignoriert wird (siehe hierzu
den Abschnitt \jump{sect:circd}{``Das Programm \texttt{circd}''}). Dieser
Zustand ist somit vor allem für einen automatisierten Fallback-Mechanismus
gedacht.
    \\
    \hline

    \caption{Circuit-Zustände}\marklabel{circuit:states}{}
\end{longtable}

Die \emph{Übergänge} zwischen diesen Zuständen werden teilweise mit
\jump{sect:fli4lctrl}{\texttt{fli4lctrl}} durchgeführt, teilweise
erfolgen sie automatisch. Ihre Bedeutung ist wie folgt:

\begin{longtable}{|l|p{0.7\textwidth}|}
    \hline
    \multicolumn{1}{|l}{\textbf{Zustandsübergang}} &
    \multicolumn{1}{|l|}{\textbf{Beschreibung}} \\
    \hline
    \endhead
    \hline
    \endfoot
    \endlastfoot

\circstatetrans{inaktiv}{aktiv} &
Ein Circuit wird aktiviert und kann sich je nach Wählmodus manuell oder
automatisch einwählen. Zu diesem Zeitpunkt können noch keine Daten über den
Circuit transportiert werden.

Dieser Zustandsübergang kann in allen Wählmodi erfolgen. Er wird durch
\texttt{fli4lctrl up} ausgelöst.
    \\
    \hline
\circstatetrans{aktiv}{bereit} &
Ein \emph{aktiver} Circuit wird in den Zustand \emph{bereit} versetzt, in dem
eine Einwahl auf Grund von Netzwerk-Aktivität möglich ist. Zu diesem Zeitpunkt
können noch keine Daten über den Circuit transportiert werden. In der Regel
werden bei diesem Zustandsübergang Hintergrundprozesse gestartet, die für die
folgenden Zustandsübergänge verantwortlich sind.

Im Wählmodus \emph{auto} erfolgt dieser Zustandsübergang direkt nach dem
Übergang \circstatetrans{inaktiv}{aktiv} oder \circstatetrans{bereit}{aktiv}.
Im Wählmodus \emph{manual} muss dieser Zustandsübergang explizit durch
\texttt{fli4lctrl dial} ausgelöst werden. Im Wählmodus \emph{off} ist dieser
Zustandsübergang nicht möglich.
    \\
    \hline
\circstatetrans{bereit}{online} &
Über den Circuit im Zustand \emph{bereit} findet eine Einwahl statt. Nach deren
erfolgreichem Abschluss können Daten über den Circuit transportiert werden,
sofern für den Circuit entsprechende zu routende Netze (siehe
\jump{CIRCxNETSIPV4y}{\var{CIRC\_x\_NETS\_IPV4\_y}} und
\jump{CIRCxNETSIPV6y}{\var{CIRC\_x\_NETS\_IPV6\_y}}) konfiguriert sind.

Je nach \jump{CIRCxHUPTIMEOUT}{Hangup-Timeout} erfolgt dieser Zustandsübergang
direkt nach dem Zustandsübergang \circstatetrans{aktiv}{bereit}
(Hangup-Timeout = 0), oder er wird durch eine Netzwerk-Aktivität ausgelöst
(Hangup-Timeout > 0).
    \\
    \hline
\circstatetrans{online}{bereit} &
Die Wählverbindung wird beendet und die Netzwerk-Anbindung abgebaut. Danach
können keine Daten mehr über den Circuit transportiert werden.

Je nach \jump{CIRCxHUPTIMEOUT}{Hangup-Timeout} wird dieser Zustandsübergang
entweder explizit vom Benutzer via \texttt{fli4lctrl hangup} angefordert
(Hangup-Timeout = 0), oder er erfolgt automatisch nach einer gewissen
Zeitspanne der Netzwerk-Inaktivität (Hangup-Timeout > 0).
    \\
    \hline
\circstatetrans{bereit}{aktiv} &
Ein Circuit im Zustand \emph{bereit} wird wieder in den Zustand \emph{aktiv}
versetzt. Danach ist eine automatische (Wieder-)Einwahl auf Grund von
Netzwerkaktivität nicht mehr möglich. In der Regel werden bei diesem
Zustandsübergang Hintergrundprozesse beendet, die beim Zustandsübergang
\circstatetrans{aktiv}{bereit} gestartet wurden.

Im Wählmodus \emph{manual} erfolgt dieser Zustandsübergang direkt nach dem
Übergang \circstatetrans{online}{bereit}. Im Wählmodus \emph{auto} muss
dieser Zustandsübergang explizit durch \texttt{fli4lctrl hangup} ausgelöst
werden. (Auf Grund der Semantik vom Wählmodus \emph{auto} wird sofort wieder in
den Zustand \emph{bereit} gewechselt, siehe die Beschreibung von
\circstatetrans{aktiv}{bereit} weiter oben.)
    \\
    \hline
\circstatetrans{aktiv}{inaktiv} &
Ein Circuit wird deaktiviert und kann künftig nicht mehr zur Einwahl
herangezogen werden.

Dieser Zustandsübergang kann in allen Wählmodi erfolgen. Er wird durch
\texttt{fli4lctrl down} ausgelöst.
    \\
    \hline

    \caption{Circuit-Zustandsübergänge}\marklabel{circuit:transitions}{}
\end{longtable}

Nicht jeder Circuit unterscheidet effektiv zwischen \emph{bereit} und
\emph{online}. So fallen diese Konzepte z.\,B. bei DHCP zusammen, weil es dort
nicht möglich ist, einen Hangup-Timeout > 0 zu konfigurieren.

\marklabel{sect:dialmode}{\subsection{Wählmodus (DIALMODE)}}

Der Wählmodus steuert, ob und auf welche Art und Weise der fli4l-Router für die
Einwahl verantwortlich ist. Der Wählmodus ist zum einen eine globale
Eigenschaft, die das generelle Wählverhalten des fli4l kontrolliert. Es gibt
drei Varianten:

\begin{longtable}{|l|p{0.75\textwidth}|}
    \hline
    \multicolumn{1}{|l}{\textbf{Wählmodus}} &
    \multicolumn{1}{|l|}{\textbf{Beschreibung}} \\
    \hline
    \endhead
    \hline
    \endfoot
    \endlastfoot

off &
In diesem Modus sind alle Circuits \emph{inaktiv} oder \emph{aktiv}. Der fli4l
wählt weder von alleine (Netzwerkaktivität, \texttt{circd} etc.) noch auf
Benutzerwunsch (\texttt{fli4lctrl dial}, WebGUI etc.).
    \\
    \hline

manual &
In diesem Modus können \emph{aktive} Circuits auf explizite Benutzeranfrage
(via \texttt{fli4ctrl dial} oder über die WebGUI) in den Zustand \emph{online}
wechseln. Sowohl bei \texttt{fli4lctrl hangup} als auch bei einem
konfigurierten Hangup-Timeout > 0 und entsprechend langer Netzwerk-Inaktivität
wird aufgelegt, der Circuit wechselt dabei wieder in den Zustand \emph{aktiv}.

In diesem Wählmodus ignoriert \texttt{circd} die Zeitspezifikationen und
schaltet keine Circuits um.
    \\
    \hline

auto &
In diesem Modus werden \emph{aktive} Circuits automatisch in den Zustand
\emph{bereit} versetzt. Je nach konfiguriertem Hangup-Timeout wechselt ein
solcher Circuit entweder sofort (Hangup-Timeout = 0) oder erst bei Bedarf
(Hangup-Timeout > 0) in den Zustand \emph{online}, nämlich wenn
Netzwerk-Aktivität verzeichnet wird. Bei einem konfigurierten Hangup-Timeout
> 0 wird bei entsprechend langer Netzwerk-Inaktivität aufgelegt, der Circuit
wechselt dabei wieder in den Zustand \emph{bereit}. Der Befehl
\texttt{fli4lctrl dial} steuert hierbei \emph{nicht} direkt die Einwahl,
sondern nur, ob der Circuit \emph{bereit} oder nicht \emph{bereit} (d.\,h. nur
\emph{aktiv}) ist. Bei externen Verbindungsabbrüchen oder beim expliziten
Auflegen wird danach gleich wieder eine erneute Einwahl versucht.

In diesem Wählmodus schaltet \texttt{circd} je nach Zeitspezifikation die
Circuits automatisch um. \emph{Ausgefallene} Circuits werden dabei jedoch
ignoriert.
    \\
    \hline

    \caption{Verfügbare Wählmodi}\marklabel{circuit:dialmodes}{}
\end{longtable}

Sowohl im Wählmodus \emph{manual} als auch im Wählmodus \emph{auto} muss ein
Circuit via \texttt{fli4lctrl up} aktiviert werden, bevor er (automatisch oder
manuell via \texttt{fli4lctrl dial}) in den Zustand bereit wechseln kann.

Zum anderen ist der Wählmodus eine lokale Eigenschaft, die pro Circuit verwaltet
wird. Der \emph{effektive} Wählmodus ist dann das Minimum beider Wählmodi, mit
der Ordnung \emph{off} < \emph{manual} < \emph{auto}. Damit lässt sich z.\,B.
erreichen, dass Circuits generell automatisch (z.\,B. ins Internet), einige
Circuits (z.\,B. in die Firma) aber nur manuell gewählt werden; ein schnelles
Auflegen aller Circuits ist weiterhin durch das Ändern des globalen Wählmodus
auf \emph{off} einfach zu realisieren.

Der lokale Wählmodus eines jeden Circuits ist, sofern nicht explizit via
\jump{CIRCxDIALMODE}{\var{CIRC\_x\_DIALMODE}} anders konfiguriert, initial
``auto'', so dass anfangs nur der globale Wählmodus eine Rolle spielt.

\begin{description}

\config{DIALMODE}{DIALMODE}{DIALMODE}

Mit dieser Variable wird der initiale globale Wählmodus beim Booten festgelegt.
Eine Änderung ist im Nachhinein mit Hilfe des \texttt{fli4lctrl}-Programms, der
WebGUI oder dem imon-Client möglich.

Standard-Einstellung: \verb+DIALMODE='auto'+

Beispiel: \verb+DIALMODE='manual'+

\end{description}

\subsection{Circuit-Klassen}\label{sec:circuits:classes}

Circuit-Klassen bieten eine Möglichkeit, ``artverwandte'' Circuits zu
gruppieren. Wenn man z.\,B.\ mehrere Möglichkeiten der Anbindung des Routers
ans Internet hat, so ist doch all diesen Circuits die Default-Route gemeinsam,
d.\,h. dass \var{CIRC\_x\_NETS\_IPV4\_y='0.0.0.0/0'} (für IPv4) bzw.
\var{CIRC\_x\_NETS\_IPV6\_y='::/0'} (für IPv6) gilt. Dies kann man
``herausmultiplizieren'' und daraus eine eigene Klasse definieren, die man
z.\,B.\ sinnigerweise ``Internet'' nennt:

\begin{example}
\begin{verbatim}
CIRC_CLASS_N='1'
CIRC_CLASS_1_NAME='Internet'
CIRC_CLASS_1_NETS_IPV4_N='1'         # Circuits dieser Klasse installieren die
CIRC_CLASS_1_NETS_IPV4_1='0.0.0.0/0' # Default-Route für IPv4...
CIRC_CLASS_1_NETS_IPV6_N='1'
CIRC_CLASS_1_NETS_IPV6_1='::/0'      # ...und für IPv6
\end{verbatim}
\end{example}

Ein Circuit kann dieser Klasse ``Internet'' mit Hilfe der Variable
\jump{CIRCxCLASSy}{\var{CIRC\_x\_CLASS\_y}} zugeordnet werden.

Neben Routen können auch Firewall-Regeln in einer Klasse sinnvoll sein. So kann
z.\,B.\ eine Port-Weiterleitung zentral für alle Internet-Circuits in der
Klasse ``Internet'' konfiguriert werden:

\begin{example}
\begin{verbatim}
# Fortsetzung von oben
CIRC_CLASS_1_PF_PREROUTING_N='1'
CIRC_CLASS_1_PF_PREROUTING_1='tmpl:http DNAT:@web-server'
CIRC_CLASS_1_PF_FORWARD_N='1'
CIRC_CLASS_1_PF_FORWARD_1='tmpl:http @web-server ACCEPT'
\end{verbatim}
\end{example}

Diese Regeln sind dann für alle Circuits gültig, die zur Klasse ``Internet''
gehören.

\begin{description}

\config{CIRC\_CLASS\_N}{CIRC\_CLASS\_N}{CIRCCLASSN}

Diese Variable gibt die Anzahl der konfigurierten Circuit-Klassen an.

Standard-Einstellung: \verb+CIRC_CLASS_N='0'+

Beispiel: \verb+CIRC_CLASS_N='2'+

\config{CIRC\_CLASS\_x\_NAME}{CIRC\_CLASS\_x\_NAME}{CIRCCLASSxNAME}

Jede Circuit-Klasse hat einen Namen. Dieser Name kann aus Buchstaben, Ziffern
und dem Bindestrich (`-') bestehen. Der Name muss unter allen Circuits
(siehe \jump{CIRCxNAME}{\var{CIRC\_x\_NAME}}) und Circuit-Klassen eindeutig
sein.

Beispiel: \verb+CIRC_CLASS_x_NAME='Internet'+

\end{description}

\marklabel{sect:fli4lctrl}{\subsection{Das Programm \texttt{fli4lctrl}}}

Das Programm \texttt{fli4lctrl} ist der Zugang zum Circuit-System über die
Kommandozeile. Es kann verwendet werden, um die Zustandsübergänge der Circuits
herbeizuführen und um den Wählmodus zu verändern. Die möglichen Befehle lauten:

\begin{longtable}{|p{0.3\textwidth}|p{0.6\textwidth}|}
    \hline
    \multicolumn{1}{|l}{\textbf{Kommando}} &
    \multicolumn{1}{|l|}{\textbf{Beschreibung}} \\
    \hline
    \endhead
    \hline
    \endfoot
    \endlastfoot

\texttt{up} <Circuit> &
Der Circuit wird vom Zustand \emph{inaktiv} in den Zustand \emph{aktiv}
überführt. Im Wählmodus \emph{auto} schließt sich unmittelbar ein \texttt{dial}
<Circuit> an, s.\,u.
    \\
    \hline
\texttt{dial} <Circuit> &
Der Circuit wird vom Zustand \emph{aktiv} in den Zustand \emph{bereit}
überführt. Bei einem Hangup-Timeout = 0 folgt unmittelbar ein Wählvorgang mit
dem Ziel, den Circuit in den Zustand \emph{online} zu überführen. Bei einem
Hangup-Timeout > 0 wird nicht sofort gewählt, sondern auf entsprechende
Netzwerk-Aktivität gewartet.

Im Wählmodus \emph{off} wird das Kommando ignoriert.
    \\
    \hline
\texttt{autodial} <Circuit> &
Der Circuit wird genauso wie beim Befehl \texttt{fli4lctrl dial} vom Zustand
\emph{aktiv} in den Zustand \emph{bereit} versetzt, aber \emph{nur}, wenn der
effektive Wählmodus des Circuits \emph{auto} ist. In allen anderen Fällen wird
der Befehl ignoriert.
    \\
    \hline
\texttt{dial} &
Es findet die Initiierung der Einwahl auf allen aktiven Circuits statt.
    \\
    \hline
\texttt{hangup} <Circuit> &
Der Circuit wird von den Zuständen \emph{online} und \emph{bereit} in den
Zustand \emph{aktiv} überführt. Im Wählmodus \emph{auto} erfolgt anschließend
ein \texttt{dial} <Circuit>.
    \\
    \hline
\texttt{hangup} &
Es findet ein Auflegen aller Circuits statt, die in den Zuständen \emph{bereit}
oder \emph{online} sind.
    \\
    \hline
\texttt{down} <Circuit> &
Der Circuit wird von allen anderen Zuständen in den Zustand \emph{inaktiv}
überführt, unabhängig vom Wähl\-modus.

Falls der Circuit vorher im Zustand \emph{bereit} oder \emph{online} war, wird
vorher ein \texttt{hangup} <Circuit> ausgeführt, s.\,o.
    \\
    \hline
\texttt{fail} <Circuit> &
Der Circuit wird von allen anderen Zuständen in den Zustand \emph{ausgefallen}
überführt, unabhängig vom Wählmodus.

Falls der Circuit vorher im Zustand \emph{bereit} oder \emph{online} war, wird
vorher ein \texttt{hangup} <Circuit> ausgeführt, s.\,o.
    \\
    \hline
\texttt{dialmode global} <Wählmodus> &
Der angegebene Wählmodus wird global eingestellt. Pro Circuit wird der
\emph{effektive} Wählmodus als Minimum des globalen und lokalen (s.\,u.)
Wählmodus bestimmt. Folgende Zustandsübergänge finden statt:

\begin{itemize}
\item Ein effektiver Moduswechsel \dialmodetrans{*}{off} impliziert ein
\texttt{hangup} auf allen Circuits in den Zuständen \emph{bereit} oder
\emph{online} (dies entspricht somit einem \texttt{fli4lctrl hangup}).
\item Ein effektiver Moduswechsel \dialmodetrans{*}{manual} impliziert ein
\texttt{hangup} auf allen Circuits im Zustand \emph{bereit}. Circuits im
Zustand \emph{online} legen \emph{nicht} auf; hier greift der neue Wählmodus
erst \emph{nach} dem nächsten Auflegen.
\item Ein effektiver Moduswechsel \dialmodetrans{*}{auto} impliziert ein
\texttt{dial} auf allen \emph{aktiven} Circuits (dies entspricht somit einem
\texttt{fli4lctrl dial}).
\end{itemize}
    \\
    \hline
\texttt{dialmode local} <Circuit> <Wählmodus> &
Der angegebene Wählmodus wird lokal für den Circuit eingestellt. Für die sich
daraus ergebenden Zustandsänderungen siehe oben.
    \\
    \hline

    \caption{\texttt{fli4lctrl}-Befehle}\marklabel{fli4lctrl:modifying-commands}{}
\end{longtable}

\begin{longtable}{|l|p{0.55\textwidth}|}
    \hline
    \multicolumn{1}{|l}{\textbf{Kommando}} &
    \multicolumn{1}{|l|}{\textbf{Beschreibung}} \\
    \hline
    \endhead
    \hline
    \endfoot
    \endlastfoot

\texttt{status} <Circuit> &
Es wird der aktuelle Zustand des zugehörigen Circuits ausgegeben.
    \\
    \hline
\texttt{status} &
Der aktuelle Online-Status des Routers wird ausgegeben. Der Rückgabecode ist 0,
falls der Router \emph{online} ist, und 1, falls er \emph{offline} ist oder ein
Fehler aufgetreten ist. Wann genau der Router \emph{online} ist, wird im
Abschnitt \jump{sect:router-online}{Wann ist mein Router online?} erläutert.
    \\
    \hline
\texttt{dialmode global} &
Der aktuelle globale Wählmodus wird zurückgegeben.
    \\
    \hline
\texttt{dialmode local} <Circuit> &
Der aktuelle lokale Wählmodus für den angegebenen Circuit wird zurückgegeben.
    \\
    \hline
\texttt{dialmode effective} <Circuit> &
Der aktuelle effektive Wählmodus für den angegebenen Circuit wird zurückgegeben.
    \\
    \hline
\texttt{list states} &
Eine Liste aller Circuits mit den zugehörigen Zuständen wird ausgegeben.
    \\
    \hline
\texttt{list dialmodes} &
Eine Liste aller Circuits mit den zugehörigen lokalen und effektiven Wählmodi
wird ausgegeben.
    \\
    \hline
\texttt{list classes} &
Eine Liste aller Circuits mit den zugehörigen Klassen wird ausgegeben.
    \\
    \hline
\texttt{list deps} &
Eine Liste aller Circuits mit den zugehörigen Abhängigkeiten wird ausgegeben.
Erfüllte Abhängigkeiten werden entsprechend markiert.
    \\
    \hline
\texttt{show} &
Ein Alias für \texttt{list states}.
    \\
    \hline

    \caption{\texttt{fli4lctrl}-Statusbefehle}\marklabel{fli4lctrl:query-commands}{}
\end{longtable}

Zu beachten ist, dass statt des Identifikators eines Circuits (z.\,B. ``circ1'')
alternativ auch
\begin{itemize}
\item sein Alias (z.\,B. ``ppp0''),
\item sein Name (z.\,B. ``T-Com DSL'') oder
\item ein aktives, ihm zugeordnetes Schlagwort (z.\,B. ``internet-v4'')
\end{itemize}
verwendet werden kann.

\marklabel{sect:circd}{\subsection{Das Programm \texttt{circd}}}

Der \texttt{circd} ist ein Dämon, der anhand von Zeit-Spezifikationen Circuits
automatisch im Hintergrund umschaltet. Dabei werden neben der Zeit- und
Kosten-Spezifikation (\var{CIRC\_x\_TIMES}) auch die Priorität eines Circuits
(\var{CIRC\_x\_PRIORITY}) sowie dessen Zustand (\emph{ausgefallen} oder
nicht) berücksichtigt.

Für einen beliebigen Zeitpunkt funktioniert das Auswählen wie folgt:

\begin{enumerate}
\item Zuerst werden alle Circuits gesucht, die für den betreffenden Zeitpunkt
aktiviert sind (Typ `Y').
\item\label{alg:circprio} Aus dieser Menge werden zuerst all jene betrachtet,
die zur höchsten verwendeten Prioritätsklasse gehören (deren
\var{CIRC\_x\_PRIORITY}-Wert also am niedrigsten ist).
\item\label{alg:circfailed} Aus dieser Menge werden alle Circuits entfernt, die
nicht nutzbar sind, die also im Zustand \emph{ausgefallen} sind.
\item Wenn die resultierende Menge nicht leer ist, ist der
Algorithmus beendet, und die resultierenden Circuits werden ausgewählt. Wenn
die resultierende Menge leer ist, werden die Schritte \ref{alg:circprio} und
\ref{alg:circfailed} für die nächst niedrigere Prioritätsklasse wiederholt.
\item Falls alle via `Y' aktivierten Circuits nicht nutzbar sind, werden alle
Circuits gesucht, die für den betreffenden Zeitpunkt als LCR-Circuits
vorgemerkt sind (Typ `L').
\item\label{alg:lcrcircprio} Aus dieser Menge werden zuerst all jene betrachtet,
die zur höchsten verwendeten Prioritätsklasse gehören (deren
\var{CIRC\_x\_PRIORITY}-Wert also am niedrigsten ist).
\item\label{alg:lcrcircfailed} Aus dieser Menge werden alle Circuits entfernt,
die nicht nutzbar sind, die also im Zustand \emph{ausgefallen} sind.
\item\label{alg:lcrcirccosts} Aus dieser Menge werden diejenigen Circuits
bestimmt, die am wenigsten kosten.
\item Wenn die resultierende Menge nicht leer ist, ist der Algorithmus beendet,
und die resultierenden Circuits werden ausgewählt. Wenn die resultierende Menge
leer ist, werden die Schritte \ref{alg:lcrcircprio} bis \ref{alg:lcrcirccosts}
für die nächst niedrigere Prioritätsklasse wiederholt.
\item Falls danach immer noch keine Circuits übrig bleiben, wird kein Circuit
ausgewählt.
\end{enumerate}

Einige Anmerkungen zum Algorithmus:
\begin{itemize}
\item Mit `Y' aktivierte Circuits ``drängeln'' sich immer vor alle mit `L'
aktivierten Circuits. Damit ist der Nutzer in der Lage, für bestimmte Zeiträume
die Least-cost-Funktionalität zu deaktivieren und die Menge der zu verwendenden
Circuits explizit einzustellen. Dies ist immer dann nützlich, wenn man aus
irgendwelchen Gründen auf einen bestimmten Circuit angewiesen ist, etwa weil
man die Dienste des Providers in Anspruch nehmen will, auf die man nur dann
Zugriff hat, wenn man sich über den Zugang des Providers einwählt.

\item Bei mit `Y' aktivierten Circuits werden die Kosten bei der Auswahl
ignoriert, es werden somit immer \emph{alle} Circuits derselben Prioritätsklasse
\emph{gleichzeitig} ausgewählt. Das ist konsistent mit der Bedeutung des Typs
`Y' (aktiviere den Circuit unabhängig von den Kosten). Will man teurere Circuits
nicht mit günstigeren gleichzeitig aktivieren, muss man den teureren Circuits
eine niedrigere Priorität zuordnen.

\item Die Priorität wird \emph{vor} den Kosten berücksichtigt. Damit ist der
Nutzer in der Lage, auch im Least-cost-Betrieb unabhängig von den tatsächlichen
Kosten Circuits zu ordnen, da die Priorität eine benutzerdefinierte und nach
Belieben veränderbare Einstellung ist, die Kosten jedoch vom Provider
vorgegeben sind und für eine korrekte Abrechnung keine Veränderungen zulässig
sind. Circuits zuerst nach Priorität zu ordnen ist insbesondere dann
vorteilhaft, wenn ein Circuit zwar günstiger ist als andere (oder gar umsonst
ist), aber dafür die Datenübertragungsrate sehr langsam ist. In diesem Fall
möchte man den langsamen Circuit aller Voraussicht nach nur als Fallback
nutzen, also falls alle schnelleren (und teureren) Circuits ausfallen. Bei
einer umgekehrten Reihenfolge (erst Kosten, dann Priorität berücksichtigen)
würde immer der kostenlose und langsame Circuit ausgewählt, ohne dass man mit
der Priorität irgendetwas daran verändern könnte.

\item Der Algorithmus kann durchaus mehrere Circuits als Ergebnis liefern. Dies
ist unproblematisch, falls die von den Circuits gerouteten Netzwerke sich nicht
überlappen. Überlappenden Netzwerke führen zu einem Fehler, und es wird nur
ein Circuit aktiviert (siehe hierzu die Beschreibung der Variablen
\var{CIRC\_x\_UP}); eine automatische Verteilung des Datenverkehrs auf
mehrere Circuits (Stichwort ``Load Balancing'') ist momentan noch nicht möglich.
\end{itemize}

\marklabel{sect:router-online}{\subsection{Wann ist mein Router online?}}

Was auf den ersten Blick trivial erscheint, ist auf den zweiten Blick eine
recht knifflige Frage: Wann ist ein fli4l-Router ``online''? Ist nur ein
Circuit definiert, ist die Frage relativ einfach zu beantworten: Der Router
ist genau dann \emph{online}, wenn der Circuit im Zustand \emph{online} ist.
Bei mehreren Circuits, die verschiedene Netze routen, sich teilweise
gegenseitig ausschließen und auch u.\,U. keine Wählverbindung eingehen ist die
Frage schon schwieriger zu beantworten. Auch einige auf der Hand liegende
Ansätze liefern nicht immer die gewünschte Antwort:

\begin{enumerate}
\item\label{rule:atleastone} \emph{Der Router ist online, wenn mindestens ein
Circuit online ist.} Diese Regelung ist ungünstig, weil bereits die Existenz
einer einzigen Route den Router in den Zustand \emph{online} versetzt, da Routen
intern über Circuits abgewickelt werden. Und ein Router ohne Routen ist nicht
sehr interessant\dots
\item \emph{Der Router ist online, wenn alle Circuits online sind.}
Diese Regelung ist immer dann falsch, wenn sich Circuits gegenseitig
ausschließen, wenn also beispielsweise zwei DSL-Circuits konfiguriert werden,
die beide eine Internet-Anbindung herstellen und deshalb nicht parallel
aktiviert werden können.
\item \emph{Der Router ist online, wenn alle Dialup-Circuits online sind.}
Diese Regelung ist besser als Regelung \ref{rule:atleastone}, weil sie
Route-Circuits ausschließt, aber auch sie löst nicht das Problem der sich
ausschließenden Circuits.
\item \emph{Der Router ist online, wenn mindestens ein Dialup-Circuit online
ist.} Dies versetzt den Router in den Zustand \emph{online}, wenn z.\,B. zwar
eine Wählverbindung in die Firma besteht, aber die Internet-Verbindung inaktiv
ist. Das kann gewünscht sein, muss aber nicht.
\item\label{rule:default} \emph{Der Router ist online, wenn ein Circuit mit
Default-Route online ist.} Diese Regelung entspricht dem Zustand vor fli4l 4.0.
Dies versetzt den Router in den Zustand \emph{offline}, wenn eine Wählverbindung
in die Firma besteht, aber die Internet-Verbindung inaktiv ist. Das kann
gewünscht sein, muss aber nicht.
\item \emph{Der Router ist online, wenn für jedes konfigurierte, zu routende
Netz ein Circuit online ist.} Auch diese Regelung versetzt den Router in den
Zustand \emph{offline}, wenn eine Wählverbindung in die Firma besteht, aber die
Internet-Verbindung inaktiv ist. Das kann gewünscht sein, muss aber nicht.
\end{enumerate}

Anstatt nun eine feste Regel vorzugeben und im Zweifelsfall genau das zu tun,
was der Benutzer \emph{nicht} erwartet, wurde ein zweistufiges Verfahren
gewählt. Der Standard-Fall wählt die rückwärtskompatible Regelung
\ref{rule:default}. Alternativ kann über die Variable \var{CIRC\_ONLINE}
eine Menge von Circuits (oder Circuit-Schlagwörtern) spezifiziert werden, die
für die Online-Bewertung berücksichtigt werden sollen. Sollen also sowohl
Internet- als auch Firma-Anbindungen berücksichtigt werden, und werden die
entsprechenden Circuits mit den Schlagwörtern ``internet'' und ``firma''
gekennzeichnet, so ist bei \verb+CIRC_ONLINE='internet firma'+ der Router
\emph{online}, wenn der ``internet''- oder der ``firma''-Circuit (oder beide)
online sind; bei \verb+CIRC_ONLINE='internet'+ hat der Router
den Zustand \emph{offline}, wenn keine Internet-Verbindung aktiv ist, unabhängig
davon, ob eine Verbindung zur Firma besteht oder nicht.

\begin{description}

\config{CIRC\_ONLINE}{CIRC\_ONLINE}{CIRCONLINE}

Diese Variable beinhaltet eine Liste von Circuits (oder Schlagwörtern), die
bei der Bestimmung des Online-Zustands des Routers berücksichtigt werden. Ist
die Variable vorhanden und die Liste nicht leer, ist der Router genau dann
\emph{online}, wenn mindestens ein Circuit aus der Liste \emph{online} ist bzw.
wenn mindestens ein Schlagwort aus der Liste aktiv ist. Ist die Liste leer oder
die Variable nicht vorhanden, greift die rückwärtskompatible Regelung, nach
welcher der Router \emph{online} ist, wenn die Default-Route aktiv ist.

Standard-Einstellung: \verb+CIRC_ONLINE=''+

\end{description}

\subsection{Sonstige Einstellungen}

\begin{description}

\config{IP\_DYN\_ADDR}{IP\_DYN\_ADDR}{IPDYNADDR}

Wird eine Verbindung mit dynamischer Adressvergabe verwendet, ist
\var{IP\_\-DYN\_\-ADDR} auf `yes' zu stellen, ansonsten auf `no'. Die
meisten Internet-Provider verwenden eine dynamische Adressvergabe.

Standard-Einstellung: \verb+IP_DYN_ADDR='yes'+

\config{OPT\_CIRCUIT\_STATUS}{OPT\_CIRCUIT\_STATUS}{OPTCIRCUITSTATUS}

Diese Variable aktiviert einen Circuit-Status-Monitor auf der dritten
fli4l-Console. Diese lässt sich mit der Tastenkombination \texttt{Alt+F3}
aktivieren, ein Wechseln zurück zur Login-Konsole wird mit Hilfe der
Tastenkombination \texttt{Alt+F1} bewerkstelligt. Der Monitor zeigt jeden
Circuit mit Namen, Typ, Alias, Schnittstelle und Status an.

Standard-Einstellung: \verb+OPT_CIRCUIT_STATUS='no'+

Beispiel: \verb+OPT_CIRCUIT_STATUS='yes'+

\end{description}
