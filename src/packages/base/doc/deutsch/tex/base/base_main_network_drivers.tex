% Last Update: $Id: base_main_network_drivers.tex 43043 2015-11-27 10:26:37Z cspiess $
\section{Ethernet-Netzwerkkarten-Treiber}

\begin{description}
  \config{NET\_DRV\_N}{NET\_DRV\_N}{NETDRVN}

    Standard-Einstellung: \var{NET\_DRV\_N='1'}

    {Anzahl der benötigten Netzwerkkarten-Treiber.

    Wird der Router nur für ISDN verwendet, so gibt es in der Regel
    nur eine Netzwerkkarte, der Standard-Wert ist also 1.

    Bei DSL werden jedoch meistens zwei Netzwerkkarten verwendet.

    Hierbei muss man zwei Fälle unterscheiden:
    \begin{enumerate}
      \item Beide Karten sind vom gleichen Typ. Dann muss nur ein
        Treiber geladen werden, der dann beide Karten anspricht, also
        \var{NET\_DRV\_N}='1'.
      \item Es werden zwei verschiedene Karten verwendet, dann muss man
        hier `2' angeben und den Treiber für die zweite Karte angeben.
    \end{enumerate}
    }


  \config{NET\_DRV\_x}{NET\_DRV\_x}{NETDRVx}

    Standard-Einstellung: \var{NET\_DRV\_1='ne2k-pci'}

    {Hier wird der Treiber für die Netzwerkkarte angegeben.
    Über die Variable \var{NET\_DRV\_1} wird standardmäßig der
    Treiber für eine NE2000"=kompatible Netzwerkkarte geladen. Weitere
    verfügbare Treiber für
    Netzwerkkartenfamilien sind in den Tabellen \ref{tabkartentreiber} und
    \ref{tabwlankartentreiber} eingetragen.

    Die 3COM EtherLinkIII (3c509) muss über das Dostool 3c509cfg.exe
    (beziehbar von
    \altlink{ftp://ftp.ihg.uni-duisburg.de/Hardware/3com/3C5x9n/3C5X9CFG.EXE})

    konfiguriert werden. Dabei sollten IRQ und I/O-Port und
    gegebenenfalls auch der Anschluss (BNC/TP) eingestellt werden.}

  \config{NET\_DRV\_x\_OPTION}{NET\_DRV\_x\_OPTION}{NETDRVxOPTION}

    Standard-Einstellung: \var{NET\_DRV\_x\_OPTION=''}

    {Der Eintrag kann in der Regel leer bleiben.

    Bei manchen ISA-Karten braucht der Treiber zusätzliche
    Informationen, um die Karte zu finden, z.B. die I/O-Adresse. Bei
    NE2000-kompatiblen ISA-Karten und bei der EtherExpress16 ist dies
    zum Beispiel der Fall.

    Hier ist z.B.
    \begin{example}
    \begin{verbatim}
        NET_DRV_x_OPTION='io=0x340'
    \end{verbatim}
    \end{example}

    zu setzen (oder der entsprechende numerische Wert).

    Ist keine Option nötig, kann die Variable leer gelassen werden.

    Sind mehrere Optionen nötig, sind diese durch Leerzeichen zu trennen, z.B.
    \begin{example}
    \begin{verbatim}
        NET_DRV_x_OPTION='irq=9 io=0x340'
    \end{verbatim}
    \end{example}

    Werden zwei identische Netzwerkkarten verwendet, z.B.
    NE2000-ISA-Karten, müssen die verschiedenen I/O-Werte durch Komma
    getrennt werden, also

    \begin{example}
    \begin{verbatim}
        NET_DRV_x_OPTION='io=0x240,0x300'
    \end{verbatim}
    \end{example}

    Die beiden IO-Werte müssen durch Komma ohne Blank getrennt werden!

    Dieses funktioniert nicht bei allen Netzwerkkarten-Treibern.
    Einige muss man auch doppelt laden, also dann \var{NET\_\-DRV\_\-N}='2'. In
    diesem Fall müssen aber mit der Option ``-o'' verschiedene Namen
    vergeben werden, z.B.

    \begin{example}
    \begin{verbatim}
          NET_DRV_N='2'
          NET_DRV_1='3c503'
          NET_DRV_1_OPTION='-o 3c503-0 io=0x280'
          NET_DRV_2='3c503'
          NET_DRV_2_OPTION='-o 3c503-1 io=0x300'
    \end{verbatim}
    \end{example}

    Unser Tip: erst die Komma-Methode ausprobieren, danach das mehrfache Laden mit Option
    ``-o'' versuchen.

    Nachfolgend noch einige Beispiele von Netzwerkkarten:

    \begin{itemize}
    \item 1 x NE2000 ISA
      \begin{example}
      \begin{verbatim}
          NET_DRV_1='ne'
          NET_DRV_1_OPTION='io=0x340'
      \end{verbatim}
      \end{example}

    \item  1 x 3COM EtherLinkIII (3c509)
      \begin{example}
      \begin{verbatim}
          NET_DRV_1='3c509'
          NET_DRV_1_OPTION=''
      \end{verbatim}
      \end{example}
      Siehe zu dieser Karte auch:

    \begin{raggedright}
      \altlink{http://extern.fli4l.de/fli4l_faqengine/faq.php?display=faq&faqnr=132&catnr=7&prog=1}\\
      \altlink{http://extern.fli4l.de/fli4l_faqengine/faq.php?display=faq&faqnr=133&catnr=7&prog=1}\\
      \altlink{http://extern.fli4l.de/fli4l_faqengine/faq.php?display=faq&faqnr=135&catnr=7&prog=1}\par
    \end{raggedright}

    \item 2 x NE2000 ISA
      \begin{example}
      \begin{verbatim}
          NET_DRV_1='ne'
          NET_DRV_1_OPTION='io=0x320,0x340'
      \end{verbatim}
      \end{example}
      Oft muss man hier noch die IRQ-Werte mit angeben, also

      \begin{example}
      \begin{verbatim}
          NET_DRV_1_OPTION='io=0x320,0x340 irq=3,5'
      \end{verbatim}
      \end{example}

      Man sollte es hier zunächst ohne Angabe der Interrupts probieren
      und lediglich dann die Interrupts eintragen, wenn ohne Angabe
      der Interrupts die Netzwerkkarten nicht gefunden werden.

    \item 2 x NE2000 PCI
      \begin{example}
      \begin{verbatim}
          NET_DRV_1='ne2k-pci'
          NET_DRV_1_OPTION=''
      \end{verbatim}
      \end{example}
    \item  1 x NE2000 ISA, 1 x NE2000 PCI
      \begin{example}
      \begin{verbatim}
          NET_DRV_1='ne'
          NET_DRV_1_OPTION='io=0x340'
          NET_DRV_2='ne2k-pci'
          NET_DRV_2_OPTION=''
      \end{verbatim}
      \end{example}
    \item 1 x SMC WD8013, 1 x NE2000 ISA
      \begin{example}
      \begin{verbatim}
          NET_DRV_1='wd'
          NET_DRV_1_OPTION='io=0x270'
          NET_DRV_2='ne2k'
          NET_DRV_2_OPTION='io=0x240'
      \end{verbatim}
      \end{example}
    \end{itemize}

    Die vollständige Liste der verfügbaren Treiber finden Sie in der
    Dokumentation des jeweiligen Kernel-Pakets.

    \emph{Falls Sie ein dummy-Device brauchen, verwenden Sie 'dummy' für \var{NET\_DRV\_x} und\\
    \jump{IPNETxDEV}{\var{IP\_NET\_x\_DEV}}='dummy$<$Nummer$>$' als Device-Name.}

    }
\end{description}
