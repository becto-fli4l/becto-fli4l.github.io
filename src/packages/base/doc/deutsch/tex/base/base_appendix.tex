% Last Update: $Id: base_appendix.tex 54276 2018-10-22 10:42:22Z sklein $
\marklabel{NULLMODEMKABEL}{\section{Nullmodemkabel}}

    Für die Verwendung des optionalen Programmpakets \jump{sec:optppp}{PPP}
    benötigt man ein Null\-modem\-kabel.

    Dieses muss mindestens drei Adern haben. Hier die Anschluss-Belegung:
%    Dieses muss mindestens sieben Adern haben. Hier die Anschluss-Belegung:

\begin{example}
%\begin{verbatim}
%                      female                   female
%                    9pol  25pol             25pol   9pol
%
%                     3      2 -------------- 3        2
%                     2      3 -------------- 2        3
%                     7      4 -------------- 5        8
%                     8      5 -------------- 4        7
%                     4     20 -------------- 6,8    6,1
%                   6,1    6,8 -------------- 20       4
%                     5      7 -------------- 7        5
%\end{verbatim}
%\end{example}
\begin{verbatim}

Stift-  Buchsen-                     Buchsen-  Stift-
    leiste                                leiste
25pol     9pol                        9pol     25pol

    8  +- 1                             1  -+   8
       |                                    |
    3  |  2 ------------\ /------------ 2   |   3
       |                 X                  |
    2  |  3 ------------/ \------------ 3   |   2
       |                                    |
   20  +- 4                             4  -+  20
       |                                    |
    7  |  5 --------------------------- 5   |   7
       |                                    |
    6  +- 6                             6  -+   6

    4  +- 7                             7  -+   4
       |                                    |
    5  +- 8                             8  -+   5


\end{verbatim}
\end{example}


    Bei den Steckern müssen die im Schaltbild gezeigte Brücken 
    eingelötet werden.
%    Bei 25-poligen Steckern muss also eine Brücke zwischen Pin 6 und
%    8, bei 9-poligen Steckern zwischen 6 und 1 gelötet werden.


\marklabel{SERIALCONSOLE}{\section{Serielle Konsole}}

    fli4l kann ohne Monitor und Tastatur eingesetzt werden. Ein Nachteil 
    davon ist, dass eventuelle Fehlermeldungen nicht bemerkt werden, weil 
    sich nicht alle Meldungen über die syslog-Schnittstelle umleiten lassen.

    Eine Möglichkeit ist die Umlenkung der Konsole-Meldungen auf seinen PC 
    oder auf ein klassisches Terminal, nämlich über die serielle 
    Schnittstelle. Die Konfiguration erfolgt über die Variablen
    \jump{SERCONSOLE}{\var{SER\_CONSOLE}},
    \jump{SERCONSOLEIF}{\var{SER\_CONSOLE\_IF}} und
    \jump{SERCONSOLERATE}{\var{SER\_CONSOLE\_RATE}}.

    Rechner mit älteren Mainboards/Karten unterstützen keine höheren 
    Geschwindigkeiten als 38400 Baud. Deshalb sollte man es zunächst mit 
    höchstens 38400 Baud probieren, bevor man sich an höhere Geschwindigkeiten 
    heranwagt. Da lediglich Text-Ausgaben über die Konsole gehen, sind höhere 
    Geschwindigkeiten eigentlich auch gar nicht notwendig.

    Sämtliche Meldungen, die normalerweise auf der Konsole ausgegeben werden, 
    werden nun auf die serielle Schnittstelle gelenkt~-- auch die Meldungen 
    des Bootvorgangs!

    Als Kabel zum Terminal oder PC mit Terminalemulation kommt ein 
    \jump{NULLMODEMKABEL}{Nullmodemkabel} zum Einsatz. Wir raten aber davon ab,
    ein Standard-Nullmodemkabel zu verwenden, weil dort normalerweise auch
    die Handshake-Leitungen verdrahtet sind. Ist das Terminal bzw. der PC
    abgeschaltet (oder die Terminalemulation nicht geladen), kann es bei
    Verwendung eines Standard-Nullmodemkabels zu einem Einfrieren von fli4l
    kommen!

    Deshalb ist hier eine spezielle Verdrahtung notwendig, um fli4l auch mit 
    abgeschaltetem Terminal betreiben zu können. Es wird dafür ein dreiadriges 
    Kabel benötigt, wobei einige Kontakte an den Steckern überbrückt werden 
    müssen. Siehe dazu bei \jump{NULLMODEMKABEL}{Nullmodemkabel}.

%\begin{verbatim}
%                      female                   female
%                    9pol  25pol             25pol   9pol
%                     3      2 -------------- 3        2
%                     2      3 -------------- 2        3
%                     7      4 -+          +- 5        8
%                               |          |
%                     8      5 -+          +- 4        7
%                     6      6 -+          +- 6        6
%                               |          |
%                     1      8 -+          +- 8        1
%                               |          |
%                     4     20 -+          +- 20       4
%                     5      7 -------------- 7        5
%\end{verbatim}



    \section{Programme}

    Um Platz auf dem Bootmedium zu sparen, wird unter anderem das Paket
    ``BusyBox'' verwendet. Das Programm ist ein einzelnes Exceutable,
    welches die Standard-Unix-Programme

\begin{example}
\begin{verbatim}
        [, [[, arping, ash, base64, basename, bbconfig, blkid, bunzip2, bzcat, bzip2,
        cat, chgrp, chmod, chown, chroot, cmp, cp, cttyhack, cut, date, dd, df,
        dirname, dmesg, dnsdomainname, echo, egrep, expr, false, fdflush, fdisk, find,
        findfs, grep, gunzip, gzip, halt, hdparm, head, hostname, inetd, init, insmod,
        ip, ipaddr, iplink, iproute, iprule, iptunnel, kill, killall, klogd, less, ln,
        loadkmap, logger, ls, lsmod, lzcat, makedevs, md5sum, mdev, mkdir, mknod,
        mkswap, modprobe, mount, mv, nameif, nice, nslookup, ping, ping6, poweroff,
        ps, pscan, pwd, reboot, reset, rm, rmmod, sed, seq, sh, sleep, sort, swapoff,
        swapon, sync, sysctl, syslogd, tail, tar, test, top, tr, true, tty, umount,
        uname, unlzma, unxz, unzip, uptime, usleep, vi, watch, xargs, xzcat, zcat
\end{verbatim}
\end{example}

    \noindent nachbildet. Zumeist sind es jedoch ``minimalistische''
    Implementationen, welche nicht den vollen Funktionsumfang
    abdecken, aber vollauf den bescheidenen Anforderungen von fli4l
    genügen.

    BusyBox steht unter GPL und ist als Source komplett erhältlich
    unter

    \altlink{http://www.busybox.net/}


    \section{Andere i4l-Tools}

    Es gibt für isdn4linux viele weitere Tools, die auch fli4l
    bereichern würden. Das Problem ist leider der Platz! Bestimmt wäre
    isdnlog als Tool zum Berechnen der Online-Gebühren wesentlich
    geeigneter, jedoch ist isdnlog einfach zu fett!

    imond braucht weniger als 10\% des Platzes, übernimmt dabei Monitoring,
    Controlling und LC-Routing, wenn auch nicht alles ganz perfekt
    ist.

    \section{Fehlersuche}

    Hilfreich bei der Fehlersuche sind natürlich einmal die
    Konsole-Outputs.  Diese rauschen aber oft einfach so durch, dass
    man gar nicht mehr mitlesen kann. Hinweis: Mit SHIFT-BILD-RAUF
    kann man zurück-, mit SHIFT-BILD-RUNTER wieder vorblättern.

    Falls im Betrieb des Routers Fehlermeldungen ``try-to-free-pages''
    auftreten ist zuwenig für Programme nutzbares RAM übrig. Als
    Abhilfe stehen dann folgende Optionen zur Verfügung:
    \begin{itemize}
    \item mehr RAM einbauen
    \item weniger Opt-Pakete einsetzen
    \item eine Festplatteninstallation nach \jump{INSTALLTYPB}{Typ B}
      durchführen
    \end{itemize}

    Auch proc-Dateien können bei der Fehlersuche helfen. z.B. gibt der
    Befehl

\begin{example}
\begin{verbatim}
                cat /proc/interrupts
\end{verbatim}
\end{example}
    
    die von den Treibern verwendeten Interrupts aus~-- nicht die
    tatsächlich von der Hardware belegten!
    
    Weitere interessante Dateien unter /proc sind devices, dma,
    ioports, kmsg, meminfo, modules, uptime, version und pci (falls
    der Router einen PCI-Bus hat).
    
    Meist liegt ein Verbindungsproblem bei ipppd vor, insbesondere bei
    der Authentifizierung. Dann helfen oft die Variablen

\begin{example}
\begin{verbatim}
        OPT_SYSLOGD='yes'
\end{verbatim}
\end{example}

\begin{example}
\begin{verbatim}
        OPT_KLOGD='yes'
\end{verbatim}
\end{example}

    in config/base.txt und

\begin{example}
\begin{verbatim}
        ISDN_CIRC_x_DEBUG='yes'
\end{verbatim}
\end{example}

    in config/isdn.txt weiter.


    \section{Literaturhinweise}

    \begin{itemize}
    \item Computer Networks, Andy Tanenbaum
    \item TCP/IP Netzanbindung von PCs, Craig Hunt
    \item TCP/IP, Kevin Washburn, Jim Evans, Verlag: Addison-Wesley, \\ISBN: 3-8273-1145-4
    \item TCP/IP Netzanbindung von PCs, ISBN 3-930673-28-2
    \item TCP/IP Netzwerk Administration, ISBN 3-89721-110-6
    \item Linux-Anwenderhandbuch, ISBN 3-929764-06-7
    \item TCP/IP im Detail:\\
      \altlink{http://www.nickles.de/c/s/ip-adressen-112-1.htm}
    \item Generell das online Linuxanwenderhandbuch von Lunetix unter:\\
      \altlink{http://www.linux-ag.com/LHB/}
    \item Einführung in die Linux-Firewall:
      \altlink{http://www.little-idiot.de/firewall/}
    \end{itemize}

    \section{Präfixe}

    Bei den Einheiten richten die Präfixe in dieser Doku sich nach \verb+IEC 60027-2+.\\
    Siehe: \altlink{http://physics.nist.gov/cuu/Units/binary.html}.

    \section{Gewähr und Haftung}

    Natürlich kann für das gesamte fli4l-Paket oder für Teile
    davon keine Gewähr dafür übernommen werden, dass es überhaupt
    funktioniert oder dass irgendeine Dokumentation in diesem
    Verzeichnis oder einem der Unterverzeichnisse korrekt ist.
    
    Auch ist jede Haftung wegen evtl. entstandender Schäden oder 
    Kosten ausgeschlossen!

    \section{Danke}
    \newcommand{\membermail}[3]{\multicolumn{2}{l}{#1 (\emph{#2})}\\\nopagebreak & \email{#3}\\}
    \newcommand{\member}[2]{#1 (\emph{#2})\\}
    \newcommand{\personmail}[2]{#1 & \email{#2}\\}
    \newcommand{\person}[1]{#1\\}
    
    In diesem Abschnitt der Dokumentation soll all denen durch namentliche Nennung
    gedankt werden, die zur Entwicklung von fli4l beitragen bzw. beigetragen haben.

    \subsection{Projektgründung}
    
    \begin{tabular}{ll}
      \person{Meyer, Frank}
    \end{tabular}\latex{\\}

    \noindent\latex{\parbox{\textwidth}}{
    Frank startete am 04.05.2000 das Projekt fli4l!\\
    Siehe: \altlink{http://www.fli4l.de/home/eigenschaften/historie/}}
  
    \subsection {Entwickler- und Testteam}

    \noindent \textbf{Das fli4l-Team bilden (in alphabetischer Reihenfolge):}

    \begin{tabular}{l}
      \member{Charrier, Bernard}     {französische Übersetzung}
      \member{Eckhofer, Felix}       {Dokumentation, Howtos}
      \member{Franke, Roland}        {OW, FBR}
      \member{Hilbrecht, Claas}      {VPN, Kernel}
      \member{Klein, Sebastian}      {Kernel, Wlan}
      \member{Knipping, Michael}     {Accounting}
      \member{Krister, Stefan}       {Opt-Cop, lcd4linux}
      \member{Miksch, Gernot}        {LCD}
      \member{Schiefer, Peter}       {fli4l-CD, Opt-Cop, Webseite, Releasemanagement}
      \member{Schliesing, Manfred}   {Tester}
      \member{Schulz, Christoph}     {FBR, IPv6, Kernel}
      \member{Siebmanns, Harvey}     {Dokumentation, englische Übersetzung}
      \member{Spieß, Carsten}        {Dsltool, Hwsupp, Rrdtool, Webgui}
      \member{Vosselman, Arwin}      {LZS-Kompression, Dokumentation}
      \member{Weiler, Manuela}       {CD-Versand, Kassenwart}
      \member{Weiler, Marcel}        {Qualitätsmanagement}
      \member{Wolters, Florian}      {Firmware, Kernel}
    \end{tabular}

    \subsection {Entwickler- und Testteam (nicht mehr aktive)}

    \begin{tabular}{l}
      \member{Arndt, Kai-Christian}  {USB}
      \member{Bauer, Jürgen}         {LCD-Package, fliwiz}
      \member{Behrends, Arno}        {Support}
      \member{Blokland, Kees}        {Englische Übersetzung}
      \member{Bork, Thomas}          {lpdsrv}
      \member{Bußmann, Lars}         {Tester}
      \member{Cerny, Carsten}        {Webseite, fliwiz}
      \member{Dawid, Oliver}         {dhcp, uClibc}
      \member{Ebner, Hannes}         {QoS}
      \member{Fischer, Joerg}        {Tester}
      \member{Frauenhoff, Peter}     {Tester}
      \member{Grabner, Hans-Joerg}   {imonc}
      \member{Grammel, Matthias}     {Englische Übersetzung}
      \member{Gruetzmacher, Tobias}  {Mini-httpd, imond, proxy}
      \member{Hahn, Joerg}           {IPSEC}
      \member{Hanselmann, Michael}   {Mac OS X/Darwin}
      \member{Hoh, Jörg}             {Newsletter, NIC-DB, Veranstaltungen}
      \member{Hornung, Nicole}       {Verein}
      \member{Horsmann, Karsten}     {Mini-httpd, WLAN}
      \member{Janus, Frank}          {LCD}
      \member{Kaiser, Gerrit}        {Logo}
      \member{Karner, Christian}     {PPTP-Package}
      \member{Klein, Marcus}         {Problemfeedback}
      \member{Lammert, Gerrit}       {HTML-Dokumentation}
      \member{Lanz, Ulf}             {LCD}
      \member{Lichtenfeld, Nils}     {QoS}
      \member{Neis, Georg}           {fli4l-CD, Dokumentation}
      \member{Peiser, Steffen}       {FAQ}
      \member{Peus, Christoph}       {uClibc}
      \member{Pohlmann, Thorsten}    {Mini-httpd}
      \member{Raschel, Tom}          {IPX}
      \member{Reinard, Louis}        {CompactFlash}
      \member{Resch, Robert}         {PCMCIA, WLAN}
      \member{Schäfer, Harald}       {HDD-Support}
      \member{Schmitts, Jupp}        {Tester}
      \member{Strigler, Stefan}      {GTK-Imonc, Opt-DB, NG}
      \member{Wallmeier, Nico}       {Windows-Imonc}
      \member{Walter, Gerd}          {UMTS}
      \member{Walter, Oliver}        {QoS}
      \member{Wolter, Jean}          {Paketfilter, uClibc}
      \member{Zierer, Florian}       {Wunschliste}
    \end{tabular}

    \subsection {Sponsoren}

    \noindent Auch ist mittlerweile  fli4l als Wort-/Bildmarke eingetragen.
    Folgende fli4l-Anwender (neben einigen, die nicht genannt werden wollten) haben
    geholfen das dafür nötige Geld zusammen zu bekommen:\\

    \begin{tabular}{l}
      \person{Bebensee, Norbert}
      \person{Becker, Heiko}
      \person{Behrends, Arno}
      \person{Böhm, Stefan}
      \person{Brederlow, Ralf}
      \person{Groot, Vincent de}
      \person{Hahn, Olaf}
      \person{Hogrefe, Paul}
      \person{Holpert, Christian}
      \person{Hornung, Nicole}
      \person{Kuhn, Robert}
      \person{Lehnen, Jens}
      \person{Ludwig, Klaus-Ruediger}
      \person{Mac Nelly, Christa}
      \person{Mahnke, Hans-Jürgen}
      \person{Menck, Owen}
      \person{Mende, Stefan}
      \person{Mücke, Michael}
      \person{Roessler, Ingo}
      \person{Schiele, Michael}
      \person{Schneider, Juergen}
      \person{Schönleber, Suitbert}
      \person{Sennewald, Matthias}
      \person{Sternberg, Christoph}
      \person{Vollmar, Thomas}
      \person{Walter, Oliver}
      \person{Wiebel, Christian}
      \person{Woelk, Fabian}
    \end{tabular}\latex{\\}

    \noindent Seit einiger Zeit hat fli4l nun auch seine eigenen Sponsoren, die
    mit Ihrer \mbox{(Hardware-)Spende} die Weiterentwicklung von fli4l
    unterstützen. Dabei handelt es sich um Adapter, CompactFlash und
    Ethernetkarten.\\

    \noindent Hardwarespender (in alphabetischer Reihenfolge):\\

    \begin{tabular}{l}
      \person{Baglatzis, Stephanos}
      \person{Bauer, Jürgen}
      \person{Dross, Heiko}
      \person{Kappenhagen, Wenzel}
      \person{Kipka, Joachim}
      \person{Klopfer, Tom}
      \person{Peiser, Steffen}
      \person{Reichelt, Detlef}
      \person{Reinard, Louis}
      \person{Stärkel, Christopher}
    \end{tabular}\latex{\\\\}

    \noindent\latex{\parbox{\textwidth}}{
    Weitere Sponsoren sind auf der fli4-Homepage gelistet:\\
    \altlink{http://www.fli4l.de/sonstiges/sponsoren/}}

    \section{Feedback}

    Kritik, Feedback und Zusammenarbeit ist jederzeit willkommen.

    Die primäre Anlaufstelle dafür sind die fli4l-Newsgroups.  Wer Probleme bei
    der Einrichtung eines fli4l-Routers hat, sollte sich erst FAQ, Howtos und
    NG-Archiv anschauen, bevor er sich an die Newsgroups wendet.  Informationen
    über die verschiedenen Gruppen und die Netiquette findet man auf der
    fli4l-Webseite:

           \altlink{http://www.fli4l.de/hilfe/newsgruppen/}\\
    \indent\altlink{http://www.fli4l.de/hilfe/faq/}\\
    \indent\altlink{http://www.fli4l.de/hilfe/howtos/}\\

    Gerade weil für fli4l-Router meist ältere Hardware eingesetzt
    wird, kann es immer wieder mal zu Problemen kommen. 
    Informationen können anderen fli4l-Usern bei Problemen mit der
    Hardware weiterhelfen, denn es gibt immer wieder Probleme mit den
    PC-Karten bzgl. I/O-Adressen, Interrupts und so weiter.

    Auf der fli4l-Webseite wurde eine Netzwerkkarten-Datenbank eingerichtet,
    in die man z.B. die passenden Treiber für eine bestimmte Karte eintragen
    kann.

        \altlink{http://www.fli4l.de/hilfe/nic-db/}

    \bigskip

    Viel Spaß mit fli4l!
