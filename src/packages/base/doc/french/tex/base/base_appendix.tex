% Do not remove the next line
% Synchronized to r54273

\marklabel{NULLMODEMKABEL}{\section{Câble Null-modem}}

    Pour utilisez le programme facultatif \jump{sec:optppp}{PPP} vous
    aurez besoin d'un câble null-modem.

    Il faut avoir branchées au moins trois fils sur le connecteur
    voici le schéma~:
%    Dieses muß mindestens 7 Adern haben. Hier die Anschluß-Belegung:

\begin{example}
%\begin{verbatim}
%                      female                   female
%                    9pol  25pol             25pol   9pol
%
%                     3      2 -------------- 3        2
%                     2      3 -------------- 2        3
%                     7      4 -------------- 5        8
%                     8      5 -------------- 4        7
%                     4     20 -------------- 6,8    6,1
%                   6,1    6,8 -------------- 20       4
%                     5      7 -------------- 7        5
%\end{verbatim}
%\end{example}
\begin{verbatim}

Broches utilisées                Broches utilisées
  Connecteur                          Connecteur
25 bro |  9 bro                       9 bro |  25 bro

    8  +- 1                             1  -+   8
       |                                    |
    3  |  2 ------------\ /------------ 2   |   3
       |                 X                  |
    2  |  3 ------------/ \------------ 3   |   2
       |                                    |
   20  +- 4                             4  -+  20
       |                                    |
    7  |  5 --------------------------- 5   |   7
       |                                    |
    6  +- 6                             6  -+   6

    4  +- 7                             7  -+   4
       |                                    |
    5  +- 8                             8  -+   5
\end{verbatim}
\end{example}

    Vous devez souder les fils sur les broches du connecteur en suivant le schéma.
%    Bei 25-poligen Steckern muss also eine Brücke zwischen Pin 6 und
%    8, bei 9-poligen Steckern zwischen 6 und 1 gelötet werden.


\marklabel{SERIALCONSOLE}{\section{Console par câble Série}}

    fli4l peut être utilisé sans écran et sans clavier. L'inconvénient,
    vous ne pourrez pas voir les messages d'erreur du boot, et les messages
    ne peuvent pas être réorientés vers l'interface de syslog.

    Une possibilité est de réorienter les messages de la console sur son
    PC ou un terminal classique en utilisant l'interface série. La configuration
    s'effectue avec les variables suivantes
    \jump{SERCONSOLE}{\var{SER\_CONSOLE}},
    \jump{SERCONSOLEIF}{\var{SER\_CONSOLE\_IF}} et
    \jump{SERCONSOLERATE}{\var{SER\_CONSOLE\_RATE}}.

    Les ordinateurs avec des cartes mères anciennes ne soutiennent pas des
    vitesses supérieures à 38400 Baud. Par conséquent, il faudra d'abord
    essayer avec 38400 Baud, avant d'utiliser des vitesses plus élevées.
    Puisque seules des sorties texte sont écritent sur la console, des vitesses
    plus élevées ne sont pas nécessaires.

    Maintenant, tous les messages sont envoyés sur la console par le port série,
    ainsi que les messages de Boot (ou démarrage)~!

    Le câble \jump{NULLMODEMKABEL}{Null modem} est utilisé entre l'émulation
    et le terminal ou le PC. Nous déconseillons toutefois d'utiliser un câble
    null modem standard, parce que normalement toutes les connexions de la prise
    serie sont branchés. Si le terminal ou le PC ne reçoit rien (ou l'émulateur
    n'arrive pas à émettre) avec la connexion fli4l, cela peut venir de
    l'utilisation du câble null modem standard~!

    Par conséquent, un câblage spécial est nécessaire, pour pouvoir arrêter
    fli4l avec le terminal du PC. Pour cela il suffit de brancher uniquement
    les 3 broches dans le connecteur, tous les autres contacts du connecteur
    ne sont pas utilisés (pas de parasite). Voir le câblage
    du \jump{NULLMODEMKABEL}{câble Null-modem}.

%\begin{verbatim}
%                      female                   female
%                    9pol  25pol             25pol   9pol
%                     3      2 -------------- 3        2
%                     2      3 -------------- 2        3
%                     7      4 -+          +- 5        8
%                               |          |
%                     8      5 -+          +- 4        7
%                     6      6 -+          +- 6        6
%                               |          |
%                     1      8 -+          +- 8        1
%                               |          |
%                     4     20 -+          +- 20       4
%                     5      7 -------------- 7        5
%\end{verbatim}


    \section{Programmes}

    Pour économiser de la place sur le média on utilise le paquetage
    "BusyBox". Qui est un programme exécutable standard Unix unique,
    dans lequel est incorporé~:

\begin{example}
\begin{verbatim}
        [, [[, arping, ash, base64, basename, bbconfig, blkid, bunzip2, bzcat, bzip2,
        cat, chgrp, chmod, chown, chroot, cmp, cp, cttyhack, cut, date, dd, df,
        dirname, dmesg, dnsdomainname, echo, egrep, expr, false, fdflush, fdisk, find,
        findfs, grep, gunzip, gzip, halt, hdparm, head, hostname, inetd, init, insmod,
        ip, ipaddr, iplink, iproute, iprule, iptunnel, kill, killall, klogd, less, ln,
        loadkmap, logger, ls, lsmod, lzcat, makedevs, md5sum, mdev, mkdir, mknod,
        mkswap, modprobe, mount, mv, nameif, nice, nslookup, ping, ping6, poweroff,
        ps, pscan, pwd, reboot, reset, rm, rmmod, sed, seq, sh, sleep, sort, swapoff,
        swapon, sync, sysctl, syslogd, tail, tar, test, top, tr, true, tty, umount,
        uname, unlzma, unxz, unzip, uptime, usleep, vi, watch, xargs, xzcat, zcat
\end{verbatim}
\end{example}

    \noindent Ce sont principalement des "mini-programmes", ils ne couvrent
    pas toutes les fonctions, cependant ils suffisent à remplir les demandes
    modestes de fli4l.

    BusyBox est sous licence GPL et les fichiers sources sont complètement
    accessibles.

    \altlink{http://www.busybox.net/}


    \section{Autre outils-i4l}

    Il y a beaucoup d'autres outils, pour isdn4linux, et aussi pour enrichir
    fli4l. Le problème est malheureusement un manque de place~! On pourrait
    utiliser isdnlog comme outil largement plus approprié pour le calcul
    des connexions en ligne, mais isdnlog est simplement trop gros pour une
    installation sur un média~!

    Imond a besoin d'au moins 10\% de place sur le média, pour
    l'utilisation de contrôles et du Routing-LC, même si se n'est
    pas tout à fait parfait.


    \section{Dépannage}

    On peut dépister les erreurs en les lisent sur l'écran de contrôle,
    après le boot de fli4l ils sont affichées uniquement sur la dernière page
    de l'écran. Pour pouvoir lire les pages précédentes ou suivantes, vous devez
    utiliser les touches MAJUSCULE [PAGE PREC] et MAJUSCULE [PAGE SUIV].

    A l'installation du routeur si vous avez un message d'erreur du genre
    "try-to-free-pages" qui apparait, ce message indique que vous n'avez pas
    assez de mémoire RAM pour les programmes utilisés. Comme solution les
    options suivantes sont alors disponibles~:
    \begin{itemize}
    \item augmenter la mémoire RAM
    \item utiliser moins de paquetage-Opt à l'installation
    \item effectuer une installation sur le disque dur avec \jump{INSTALLTYPB}{Type B}
    \end{itemize}

    Le fichier proc peut également aider à dépister des erreurs, par exemple~:

\begin{example}
\begin{verbatim}
                cat /proc/interrupts
\end{verbatim}
\end{example}

    Avec le paramètre Interrupts on peut visualiser les pilotes
    matériels et ceux qui ne sont pas activés~!

    Voici d'autres paramètres intéressants avec la commande /proc~:
    dma, ioports, kmsg, meminfo, modules, uptime, version et pci
    (si le routeur a un Bus-PCI).

    Le plus souvent il s'agit d'un problème de connexion avec ipppd,
    en particulier lors de l'authentification, vous pouvez utiliser
    les variables dans config/base.txt

\begin{example}
\begin{verbatim}
        OPT_SYSLOGD='yes'
\end{verbatim}
\end{example}

\begin{example}
\begin{verbatim}
        OPT_KLOGD='yes'
\end{verbatim}
\end{example}

    et dans config/isdn.txt

\begin{example}
\begin{verbatim}
        ISDN_CIRC_x_DEBUG='yes'
\end{verbatim}
\end{example}

    pour essayer de résoudre certains problèmes.


    \section{Références}

    \begin{itemize}
    \item Computer Networks, Andy Tanenbaum
    \item TCP/IP Netzanbindung von PCs, Craig Hunt
    \item TCP/IP, Kevin Washburn, Jim Evans, Verlag: Addison-Wesley, \\ISBN: 3-8273-1145-4
    \item TCP/IP Netzanbindung von PCs, ISBN 3-930673-28-2
    \item TCP/IP Netzwerk Administration, ISBN 3-89721-110-6
    \item Linux-Anwenderhandbuch, ISBN 3-929764-06-7
    \item TCP/IP im Detail:\\
      \altlink{http://www.nickles.de/c/s/ip-adressen-112-1.htm}
    \item Generell das online Linuxanwenderhandbuch von Lunetix unter:\\
      \altlink{http://www.linux-ag.com/LHB/}
    \item Einführung in die Linux-Firewall:
      \altlink{http://www.little-idiot.de/firewall/}
    \end{itemize}


    \section{Préfixe}

    Les unités préfixer, abordé dans ce présent document sont après la norme
    \verb+IEC 60027-2+.\\
    Voir~: \altlink{http://physics.nist.gov/cuu/Units/binary.html}.
    Pour les unités en français voir~:
    \altlink{http://fr.wikipedia.org/wiki/Octet}


    \section{Aucune responsabilité et de garantie}

    Naturellement on ne garantit pas que tous les paquetages-fli4l
    fonctionnent ou que tous les dossiers ou sous dossiers de cette
    documentation soit correcte.
    
    Toute responsabilité pour les dommages causés et éventuellement pour
    les frais engager seront déclinés~!


    \section{Merci}
    \newcommand{\membermail}[3]{\multicolumn{2}{l}{#1 (\emph{#2})}\\\nopagebreak & \email{#3}\\}
    \newcommand{\member}[2]{#1 (\emph{#2})\\}
    \newcommand{\personmail}[2]{#1 & \email{#2}\\}
    \newcommand{\person}[1]{#1\\}

    Dans cette partie de cette documentation, je remercie toutes les personnes
    qui ont contribuées ou beaucoup plus contribuées au développement de fli4l.
    Voici ceux qui mon autorisé à mentionner leurs noms.

    \subsection{Fondateur du Projet}

    \begin{tabular}{ll}
      \person{Meyer, Frank}
    \end{tabular}\latex{\\}
    
    \noindent\latex{\parbox{\textwidth}}{
    Frank a commencé le projet fli4l le 04.05.2000~!\\
    Voir~: \altlink{http://www.fli4l.de/fr/fli4l/caracteristique/historique/}}

    \subsection {L'équipe de développeurs et de testeurs}

    \noindent \textbf{L'équipe fli4l de développeurs est formée (dans l'ordre alphabétique)~:}

    \begin{tabular}{l}
      \member{Charrier, Bernard}     {traduction française}
      \member{Eckhofer, Felix}       {Documentation, Howtos}
      \member{Franke, Roland}        {OW, FBR}
      \member{Hilbrecht, Claas}      {VPN, Kernel}
      \member{Klein, Sebastian}      {Kernel, Wlan}
      \member{Knipping, Michael}     {Accounting}
      \member{Krister, Stefan}       {Opt-Cop, lcd4linux}
      \member{Miksch, Gernot}        {LCD}
      \member{Schiefer, Peter}       {fli4l-CD, Opt-Cop, site Web, gestion des versions}
      \member{Schliesing, Manfred}   {testeur}
      \member{Schulz, Christoph}     {FBR, IPv6, Kernel}
      \member{Siebmanns, Harvey}     {Documentation, Traduction anglaise}
      \member{Spieß, Carsten}        {Dsltool, Hwsupp, Rrdtool, Webgui}
      \member{Vosselman, Arwin}      {LZS-Compression, Documentation}
      \member{Weiler, Manuela}       {Copie de CD, trésorière}
      \member{Weiler, Marcel}        {Gestion de la qualité}
      \member{Wolters, Florian}      {Firmware, Kernel}
    \end{tabular}

    \subsection {L'équipe de développeurs et de testeurs (qui ne sont plus actifs)}

    \begin{tabular}{l}
      \member{Arndt, Kai-Christian}  {USB}
      \member{Bauer, Jürgen}         {LCD-Package, fliwiz}
      \member{Behrends, Arno}        {Support}
      \member{Blokland, Kees}        {Traduction anglaise}
      \member{Bork, Thomas}          {lpdsrv}
      \member{Bußmann, Lars}         {testeur}
      \member{Cerny, Carsten}        {Site Web, fliwiz}
      \member{Dawid, Oliver}         {dhcp, uClibc}
      \member{Ebner, Hannes}         {QoS}
      \member{Fischer, Joerg}        {testeur}
      \member{Frauenhoff, Peter}     {testeur}
      \member{Grabner, Hans-Joerg}   {imonc}
      \member{Grammel, Matthias}     {Traduction anglaise}
      \member{Gruetzmacher, Tobias}  {Mini-httpd, imond, proxy}
      \member{Hahn, Joerg}           {IPSEC}
      \member{Hanselmann, Michael}   {Mac OS X/Darwin}
      \member{Hoh, Jörg}             {Newsletter, NIC-DB, manifestation}
      \member{Hornung, Nicole}       {Verein}
      \member{Horsmann, Karsten}     {Mini-httpd, WLAN}
      \member{Janus, Frank}          {LCD}
      \member{Kaiser, Gerrit}        {Logo}
      \member{Karner, Christian}     {PPTP-Package}
      \member{Klein, Marcus}         {Problèmes réactions}
      \member{Lammert, Gerrit}       {HTML-Documentation}
      \member{Lanz, Ulf}             {LCD}
      \member{Lichtenfeld, Nils}     {QoS}
      \member{Neis, Georg}           {fli4l-CD, Documentation}
      \member{Peiser, Steffen}       {FAQ}
      \member{Peus, Christoph}       {uClibc}
      \member{Pohlmann, Thorsten}    {Mini-httpd}
      \member{Raschel, Tom}          {IPX}
      \member{Reinard, Louis}        {CompactFlash}
      \member{Resch, Robert}         {PCMCIA, WLAN}
      \member{Schäfer, Harald}       {HDD-Support}
      \member{Schmitts, Jupp}        {testeur}
      \member{Strigler, Stefan}      {GTK-Imonc, Opt-DB, NG}
      \member{Wallmeier, Nico}       {Windows-Imonc}
      \member{Walter, Gerd}          {UMTS}
      \member{Walter, Oliver}        {QoS}
      \member{Wolter, Jean}          {Paketfilter, uClibc}
      \member{Zierer, Florian}       {Liste de souhaits}
    \end{tabular}

    \subsection {Sponsor}

    \noindent Le nom et le logo de fli4l sont enregistrée comme marque déposée. Les
    utilisateurs suivants (et ceux qu'ils ne veulent pas être nommés) ont
    aidé financièrement au développement de fli4l~:\\

    \begin{tabular}{l}
      \person{Bebensee, Norbert}
      \person{Becker, Heiko}
      \person{Behrends, Arno}
      \person{Böhm, Stefan}
      \person{Brederlow, Ralf}
      \person{Groot, Vincent de}
      \person{Hahn, Olaf}
      \person{Hogrefe, Paul}
      \person{Holpert, Christian}
      \person{Hornung, Nicole}
      \person{Kuhn, Robert}
      \person{Lehnen, Jens}
      \person{Ludwig, Klaus-Ruediger}
      \person{Mac Nelly, Christa}
      \person{Mahnke, Hans-Jürgen}
      \person{Menck, Owen}
      \person{Mende, Stefan}
      \person{Mücke, Michael}
      \person{Roessler, Ingo}
      \person{Schiele, Michael}
      \person{Schneider, Juergen}
      \person{Schönleber, Suitbert}
      \person{Sennewald, Matthias}
      \person{Sternberg, Christoph}
      \person{Vollmar, Thomas}
      \person{Walter, Oliver}
      \person{Wiebel, Christian}
      \person{Woelk, Fabian}
    \end{tabular}\latex{\\}

    \noindent Depuis un certain temps, fli4l a maintenant ses propres sponsors, ils
    soutiennent le développement de fli4l par \mbox{(des dons de matériels)}.
    Il s'agit d'adaptateurs de Compact Flash et de cartes Ethernet.\\

    \noindent Donateurs de matériel (dans l'ordre alphabétique)~:\\

    \begin{tabular}{l}
      \person{Baglatzis, Stephanos}
      \person{Bauer, Jürgen}
      \person{Dross, Heiko}
      \person{Kappenhagen, Wenzel}
      \person{Kipka, Joachim}
      \person{Klopfer, Tom}
      \person{Peiser, Steffen}
      \person{Reichelt, Detlef}
      \person{Reinard, Louis}
      \person{Stärkel, Christopher}
    \end{tabular}\latex{\\\\}

    \noindent\latex{\parbox{\textwidth}}{
    Une liste des autres sponsors est sur la page d'accueil de fli4l~:\newline
    \altlink{http://www.fli4l.de/fr/divers/sponsors/}}

    \section{Réaction}

    Les critiques et les réactions sont toujours les bienvenues
    pour la collaboration de fli4l.

    Pour les services d'aide, adressez-vous sur les Newsgroups de fli4l.
    Si vous avez des problèmes d'installation avec le routeur fli4l,
    voir avant de s'adresser au Newsgroups, les FAQ, Howtos et les archives
    Newsgroups. On trouve sur le site Web fli4l différentes informations et
    en plus d'autres sites internet au sujet de fli4l~:

           \altlink{http://www.fli4l.de/fr/aide/newsgroup/}\\
    \indent\altlink{http://www.fli4l.de/fr/aide/faq/}\\
    \indent\altlink{http://www.fli4l.de/fr/aide/howtos/}\\

    C'est justement parce qu'on utilise en général du vieux matériel
    pour le routeur fli4l, que l'on peut avoir des problèmes avec ce genre de
    matériel. Ces informations peuvent aider d'autres utilisateurs fli4l à
    résoudre les problèmes de matériel, car il y a sans cesse des problèmes
    avec les cartes installées dans le PC par rapport aux adresses I/O,
    aux Interruptions, et autres.

    Sur le site Web fli4l une banque de données pour les cartes réseau et
    wireless, sur laquelle on peut écrire les information, par exemple,
    le pilote correspondant à une carte déterminée et la compatibilitée
    avec fli4l. Voici l'adresse du site~:

        \altlink{http://www.fli4l.de/fr/aide/bd-cartes-reseaux/}

    \bigskip

    Amusez-vous bien avec fli4l~!
