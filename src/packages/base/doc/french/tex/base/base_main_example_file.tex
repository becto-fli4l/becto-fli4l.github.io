% Do not remove the next line
% Synchronized to r29818

\marklabel{beispielbase}{\section{Exemple de fichier}}\index{Exemple de fichier (base.txt)}\index{base.txt}

L'exemple fichier de \verb+base.txt+ qui est dans le répertoire \verb+config/+ a
le contenu suivant:~:

\begin{example}
\verbatimfile{\basedir/config/base.txt}
\end{example}

\medskip

Ce fichier est enregistré sous le format DOS. Cela signifie, qu'à l'extrémité
de chaque ligne, il y a un retour chariot (CR). J'ai décidé d'utiliser ce format
car la plupart des éditeurs Unix ne rencontreront aucun problème avec. Le
bloc-notes de Windows, ne peut pas manipuler ces fichiers sans CRs~!

Si vous avez, des problèmes avec votre éditeur Unix/Linux favori, vous pouvez
employer la commande suivante avant d'éditer le fichier au format Unix~:

\begin{example}
\begin{verbatim}
        sh unix/dtou config/base.txt
\end{verbatim}
\end{example}

Lors de la création du support de boot, il n'y a aucune importance si le fichier
contient ou pas des CRs en fin de lignes. Lorsque le fichier sera écrit sur le
support de boot ou sur le disque dur, tout les CRs et tous les commentaires,
seront complètement ignorés.

Maintenant nous pouvons commencer \ldots