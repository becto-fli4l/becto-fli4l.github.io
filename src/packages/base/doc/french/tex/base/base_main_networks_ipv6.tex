% Do not remove the next line
% Synchronized to r48928

\section{Configuration du réseau (IPv6)}

\subsection{Introduction}

En plus de la configuration présentée dans le dernier paragraphe du réseau IPv4,
il est également possible à bien des égards de rendre le routeur fli4l compatible IPv6.
La configuration les adresses IPv6 sur le routeur gère les sous-réseaux IPv6, les routes IPv6
prédéfinis et les règles de pare-feu des paquets IPv6. L'IPv6 est le successeur de protocole
Internet IPv4. Il a été conçu principalement pour augmenter sans équivoque la quantité
relativement faible des adresses Internet~: Alors que l'IPv4 supporte environ \(2^{32}\) 
adresses,\footnote{certaines adresses servent à des fins spécifiques, comme pour le Broad et
le Multicasting,} pour l'IPv6 il y a déjà \(2^{128}\) adresses. Ainsi chaque hôte peut être
associé à une adresse IPv6 unique et on n'est plus dépendant des techniques d'instructions.
tels que NAT, PAT, masquage, etc.

Outre cet aspect, des questions telles que l'auto-configuration et la sécurité joue un rôle
dans le développement du protocole IPv6. Ce sera repris dans les paragraphes suivantes.

Le plus gros problème avec l'IPv6 est la diffusion~: Actuellement l'IPv6 est~-- par rapport
à l'IPv4 très peu utilisé. La raison est que l'IPv6 et l'IPv4 ne sont pas techniquement
compatibles les une avec les autres et donc tous les composants logiciels et matériels qui
sont impliqués dans la transmission de paquets sur Internet, doivent être modernisés pour
l'IPv6. En outre, certains services tels que le DNS (Domain Name System) doivent être améliorés
en conséquence pour l'IPv6.

Un cercle vicieux est ouvert~: La faible utilisation de l'IPv6 par les fournisseurs d'accès
sur Internet conduisent au désintérêt de la part des fabricants de routeur pour équiper leurs
appareils pour des fonctionnalités IPv6, qui à leur tour, les fournisseurs d'accès évitent
la transition vers l'IPv6, car ils craignent que cet effort n'en vaut pas la peine. Lentement
le vent tourne en faveur de l'IPv6, car le stock des adresses IPv4 se fait rare.\footnote{Pendant
ce temps, les derniers blocs d'adresses IPv4 ont été attribuées par l'IANA.}

\subsection{Format de l'adresse}

Une adresse IPv6 se compose de huit groupes de deux octets, ils sont écrit dans le format
hexadécimal~:

\emph{Exemple 1~:} \verb*?2001:db8:900:551:0:0:0:2?

\emph{Exemple 2~:} \verb*?0:0:0:0:0:0:0:1? (adresse IPv6 de bouclage)

Pour rendre les adresses un peu plus clair, les zéros consécutifs sont fusionnés, en les
retirant il nous restera seulement deux points consécutif. Les adresses ci-dessus peuvent
donc également être écrites comme ceci~:

\emph{Exemple 1 (compacté)~:} \verb*?2001:db8:900:551::2?

\emph{Exemple 2 (compacté)~:} \verb*?::1?

Cependant, une telle réduction est permise seulement une fois, c'est pour éviter des ambiguïtés.
L'adresse \verb*?2001:0:0:1:2:0:0:3? peut être compacté soit \verb*?2001::1:2:0:0:3? ou soit
\verb*?2001:0:0:1:2::3?, mais pas \verb*?2001::1:2::3?, il serait difficile de savoir comment
distribué respectivement les quatres zéros, de manière à rassembler chaque partie de l'adresse.

Une autre ambiguïté quand une adresse IPv6 est combiner avec un port (TCP ou UDP)~: Dans ce
cas, vous ne devez pas coller directement les deux-points avec la valeur du port, parce que
les deux-points sont déjà utilisés dans l'adresse et donc dans certains cas cela ne seraient
pas claires, cars l'indication du port représente peut-être un élément de l'adresse. Par conséquent,
dans ce cas là, l'adresse IPv6 doit être spécifié entre des crochets. C'est également la syntaxe,
pour une demande d'URL (par exemple lorsque vous utilisez une adresse IPv6 numérique dans
le navigateur).

\emph{Exemple 3~:} \verb*?[2001:db8:900:551::2]:1234?

Sans l'utilisation des crochets, l'adresse \verb*?2001:db8:900:551::2:1234?, est produite
intégralement, l'adresse \verb*?2001:db8:900:551:0:0:2:1234? correspond donc à aucune
spécification du port.

Si vous ne pouvez pas utiliser un accès Internet avec une connexion IPv6 native, vous allez
avoir besoin d'un tunnel IPv6 avec un fournisseur IPv6. Ceci est rendue possible avec le paquetage
"IPv6". S'il vous plaît lire la documentation du paquetage "ipv6" pour plus de détails.

\subsection{Configuration}

\subsubsection{Paramètre général}

Il comprend tout d'abord dans le paramètre général l'activation du support IPv6 et d'autre part,
la distribution optionnelle d'une adresse IPv6 par le routeur.

\begin{description}
\config{OPT\_IPV6}{OPT\_IPV6}{OPTIPV6}{
Cette variable active le support IPv6.

Paramètre par défaut~:
}
\verb*?OPT_IPV6='no'?

\config{HOSTNAME\_IP6}{HOSTNAME\_IP6}{HOSTNAMEIP6}{(Optionnelle)
Avec cette variable vous rendez explicite l'adresse IPv6 du routeur. Si la variable n'est pas
paramétrée, l'adresse IPv6 sera configurée avec l'adresse du premier sous-réseau IPv6 paramétrée
dans la variable (\var{IPV6\_NET\_x}, voir ci-dessous).

Exemple~:
}
\verb*?HOSTNAME_IP6='IPV6_NET_1_IPADDR'?
\end{description}

\subsubsection{Configuration du sous-réseau}

Ce paragraphe décrit la configuration d'un ou plusieurs sous-réseaux IPv6 présent. On appelle
un sous-réseau IPv6, un espace d'adressage IPv6 qui est spécifié par un préfixe qui est lié
à une interface réseau spécifique. d'autres paramètres affectent la publication du préfixe,
le service DNS pour le sous-réseau et le nom du routeur facultatif pour le sous-réseau.

\begin{description}
\config{IPV6\_NET\_N}{IPV6\_NET\_N}{IPV6NETN}{

Dans cette variable vous indiquez le nombre de sous-réseaux IPv6. Au moins un sous-réseau IPv6
doit être défini pour pouvoir utiliser l'IPv6 sur le réseau local.

Paramètre par défaut~:
}
\verb*?IPV6_NET_N='0'?

\config{IPV6\_NET\_x}{IPV6\_NET\_x}{IPV6NETx}{
Dans cette variable vous indiquez l'adresse IPv6 du routeur ainsi que la taille du masque de
sous-réseau en notation CIDR pour un sous-réseau IPv6 spécifique. Si le sous-réseau routé est
public, il provient généralement du fournisseur d'accès Internet ou d'un tunnel.

\wichtig{Si l'auto-configuration sans état est activé pour le sous-réseau, (voir le
paragraphe de la variable \var{IPV6\_NET\_x\_ADVERTISE} ci-dessous), le préfixe du sous-réseau
doit avoir une longueur de 64 bits~!}

Si le sous-réseau est connecté à un tunnel ou si vous utilisez un DHCPv6 pour obtenir le préfixe,
dans ce cas la partie de l'adresse du routeur peut être spécifié ici, mais le préfixe du sous-réseau
ne doit \emph{pas} appartenir à la classe du tunnel ou du serveur DHCPv6, car le préfixe et cette
adresse seront combinées~! Consultez la documentation du (Tunnel) "ipv6" ou du "dns\_dhcp"
dans le paquerage (DHCPv6) pour plus de détails.

Exemple~:
}

\begin{example}
\begin{verbatim}
IPV6_NET_1='2001:db8:1743:42::1/64'       # Statique~: adresse complète
IPV6_NET_2='{he}+0:0:0:42::1/64'          # Tunnel-HE~: adresse partielle
IPV6_NET_3='{dhcpv6}+0:0:0:43::1/64'      # DHCPv6~: adresse partielle
\end{verbatim}
\end{example}

\config{IPV6\_NET\_x\_DEV}{IPV6\_NET\_x\_DEV}{IPV6NETxDEV}{
Dans cette variable vous indiquez le nom de l'interface réseau sur lequel le réseau IPv6 sera
attaché, pour le sous-réseau IPv6 spécifique. Ce nom n'est \emph{pas} incompatible avec les
interfaces réseaux attribués dans la configuration de la base (\texttt{base.txt}), une interface
réseau peuvent être affectée à la fois pour les adresses IPv4 et IPv6.

Exemple~:
}
\verb*?IPV6_NET_1_DEV='eth0'?

\config{IPV6\_NET\_x\_ADVERTISE}{IPV6\_NET\_x\_ADVERTISE}{IPV6NETxADVERTISE}{
Avec cette variable vous déterminez, si vous voulez utiliser attribution par le "Router Advertisements"
du préfixe pour le sous-réseau sélectionné dans le LAN (ou réseau local). Cette méthode utilise une
soi-disante "auto-configuration" (configuration automatique sans état), il ne doit pas être confondue
avec le DHCPv6. Les valeurs possibles sont "yes" ou "no".

Il est recommandé d'activer cette variable, à moins que toutes les adresses du réseau soient affectées
de manière statique ou qu'un autre routeur est déjà responsable de l'attribution du préfixe pour
le sous-réseau.

\wichtig{La distribution automatique du sous-réseau ne fonctionne que si le masque de sous-réseau
est à /64, cela veut dire que la longueur du préfixe du sous-réseau est de 64 bits~! La raison est,
que les adresses IPv6 sont calculées à partir du préfixe et de l'adresse MAC de chaque hôte sur le
réseau, cela ne fonctionne pas, si la partie hôte n'est pas à 64 bits. Si l'auto-configuration échoue,
vous devez vérifier si le préfixe du sous-réseau est peut-être incorrectement spécifié (par exemple, /48).}

Paramètre par défaut~:
}
\verb*?IPV6_NET_1_ADVERTISE='yes'?

\config{IPV6\_NET\_x\_ADVERTISE\_DNS}{IPV6\_NET\_x\_ADVERTISE\_DNS}{IPV6NETxADVERTISEDNS}{
Avec cette variable vous déterminez, si vous voulez activer le service DNS local sur le sous-réseau
IPv6 en utilisant le "Router Advertisements" (option RDNSS) pour l'attribution du préfixe. Cela
fonctionne que si le service DNS IPv6 est activé avec la variable \var{DNS\_SUPPORT\_IPV6}='yes'.
Les valeurs possibles sont "yes" ou "no".

Paramètre par défaut~:
}
\verb*?IPV6_NET_1_ADVERTISE_DNS='no'?

\config{IPV6\_NET\_x\_NAME}{IPV6\_NET\_x\_NAME}{IPV6NETxNAME}{(optional)
Dans cette variable vous pouvez indiquer un nom d'hôte spécifique à l'interface du routeur pour
le sous-réseau IPv6 correspondante.

Exemple~:
}
\verb*?IPV6_NET_1_NAME='fli4l-subnet1'?

\end{description}
