% Do not remove the next line
% Synchronized to r34546

\section{Routes supplémentaire (IPv6)}

Les routes IPv6 sont des chemins que doivent prendre les paquets IPv6. Ainsi, le routeur
sait où il doit envoyer le paquet entrant, il se fie pour cela à la table de routage sur
laquel se trouve exactement cette information. Au sujet de l'IPv6, il est important
de savoir où les paquets IPv6 sont envoyés, s'ils ne sont pas pour le réseau local.

\begin{description}
\config{IPV6\_ROUTE\_N}{IPV6\_ROUTE\_N}{IPV6ROUTEN}{
Dans cette variable vous indiquez le nombre de route IPv6 à installer. En règle générale,
il n'est pas nécessaire d'installer de route IPv6 supplémentaires.

Paramètre par défaut~:
}
\verb*?IPV6_ROUTE_N='0'?

\config{IPV6\_ROUTE\_x}{IPV6\_ROUTE\_x}{IPV6ROUTEx}{
Dans cette variable vous devez indiquer la route sous la forme 'réseau de destination gateway'
le réseau de destination est écrit avec la notation CIDR. Pour utiliser une route
par défaut, vous devez indiquer \var{::/0} elle sera le réseau de destination.

Exemple~:
}
\verb*?IPV6_ROUTE_1='2001:db8:1743:44::/64 2001:db8:1743:44::1'?
\end{description}
