% Do not remove the next line
% Synchronized to r34546

\section{Filtrage de paquets (IPv6)}

Comme pour le filtrage IPv4 il est également nécessaire de filtrer le réseau
IPv6 avec un pare-feu, donc le monde extérieur ne pourra pas atteindre chaque
ordinateur du réseau local. Cela est d'autant plus important, car chaque ordinateur
reçoit normalement une adresse IPv6 unique au monde, cette adresse sera affecté
en permanence à l'ordinateur, elle est basée sur l'adresse MAC de la carte réseau
utilisée.\footnote{Une exception existe, si les hôtes du LAN sont dits "Privacy
Extensions" (extension de la vie privée) alors, une partie de l'adresse IPv6
sera générée de façon aléatoire et temporaire. Par définition, ces adresses ne sont
pas connus du monde extérieur et donc elles ne sont partiellement ou pas du tout
pertinentes pour la configuration du pare-feu.} Par conséquent, le pare-feu interdira
en une seule fois tout accès externe et ensuite, il ouvrira un par un des ordinateurs
configurés selon les besoins.

La configuration du pare-feu IPv6 correspond globalement à la configuration du
pare-feu IPv4. Les particularités et les différences seront examinées séparément.

\begin{description}
\config{PF6\_LOG\_LEVEL}{PF6\_LOG\_LEVEL}{PF6LOGLEVEL} Dans la variable
\var{PF6\_LOG\_LEVEL} vous pouvez régler le niveau de journalisation de l'ensemble
des chaînes suivantes, vous pouvez indiquer les valeurs~: debug, info, notice,
warning, err, crit, alert, emerg.

\config{PF6\_INPUT\_POLICY}{PF6\_INPUT\_POLICY}{PF6INPUTPOLICY}{
Cette variable spécifie la stratégie par défaut pour les paquets entrants sur le routeur
en utilisant la (chaîne INPUT). Les valeurs possibles sont "REJECT" (par défaut, pour tous
les paquets), "DROP" (jeter secrètement tous les paquets) et "ACCEPT" (accepte tous
les paquets). Pour une description plus détaillée, voir la documentation de la variable
\var{PF\_INPUT\_POLICY}.

Configuration par défaut~:
}
\verb*?PF6_INPUT_POLICY='REJECT'?

\config{PF6\_INPUT\_ACCEPT\_DEF}{PF6\_INPUT\_ACCEPT\_DEF}{PF6INPUTACCEPTDEF}{
Cette variable permet d'utiliser les règles prédéfinies pour la chaîne INPUT du pare-feu
IPv6. Les valeurs possibles sont "yes" ou "no".

Elle ouvre la présélection des règles du pare-feu pour les pings ICMPv6 entrant
(un ping par seconde comme limite) ainsi que pour les paquets NDP (Neighbor Discovery
Protocol), qui sont nécessaires pour la configuration automatique sans état des réseaux IPv6.
La connexions localhost ainsi que la réponse des paquets locaux des connexions initiées
sont également autorisés. Enfin, le pare-feu IPv4 sera ajusté à cette effet, pour que chaque
tunnel qui encapsule les paquets IPv6-in-IPv4 seront acceptées à l'extrémité du tunnel.

Configuration par défaut~:
}
\verb*?PF6_INPUT_ACCEPT_DEF='yes'?

\config{PF6\_INPUT\_LOG}{PF6\_INPUT\_LOG}{PF6INPUTLOG}{
Cette variable vous permet d'enregistrer dans un fichier jounal tous les paquets entrants
rejetés. Les valeurs possibles sont "yes" ou "no". Pour une description plus détaillée, voir
la documentation de la variable \var{PF\_INPUT\_LOG}.

Configuration par défaut~:
}
\verb*?PF6_INPUT_LOG='no'?

\config{PF6\_INPUT\_LOG\_LIMIT}{PF6\_INPUT\_LOG\_LIMIT}{PF6INPUTLOGLIMIT}{
Dans cette variable vous configurez la limite du fichier journal de la chaîne INPUT du pare-feu
IPv6, ainsi, le fichier journal sera lisible un certain temps. Pour une description
plus détaillée, voir la documentation de la variable \var{PF\_INPUT\_LOG\_LIMIT}.

Configuration par défaut~:
}
\verb*?PF6_INPUT_LOG_LIMIT='3/minute:5'?

\config{PF6\_INPUT\_REJ\_LIMIT}{PF6\_INPUT\_REJ\_LIMIT}{PF6INPUTREJLIMIT}{
Dans cette variable vous définissez la limite du rejet des paquets TCP entrants. Si le paquet TCP
dépasse cette limite, ce paquet sera tranquillement écarté (DROP). Pour une description plus
détaillée, voir la documentation de la variable \var{PF\_INPUT\_REJ\_LIMIT}.

Configuration par défaut~:
}
\verb*?PF6_INPUT_REJ_LIMIT='1/second:5'?

\config{PF6\_INPUT\_UDP\_REJ\_LIMIT}{PF6\_INPUT\_UDP\_REJ\_LIMIT}{PF6INPUTUDPREJLIMIT}{
Dans cette variable vous définissez la limite du rejet des paquets UDP entrants. Si le paquet
UDP dépasse cette limite, ce paquet sera tranquillement écarté (DROP). Pour une description plus
détaillée, voir la documentation de la variable \var{PF\_INPUT\_UDP\_REJ\_LIMIT}.

Configuration par défaut~:
}
\verb*?PF6_INPUT_UDP_REJ_LIMIT='1/second:5'?

\config{PF6\_INPUT\_ICMP\_ECHO\_REQ\_LIMIT}{PF6\_INPUT\_ICMP\_ECHO\_REQ\_LIMIT}{PFI6NPUTICMPECHOREQLIMIT}{
Dans cette variable vous définissez le nombre de fois que vous voulez répondre à une demande
écho ICMPv6. La fréquence de restriction avec une période définie, est décrite selon la
fonction 'n/temps\-période' par exemple '3/minute:5'. Si la limite est dépassée, le paquet
est tout simplement ignoré (DROP). Si cette variable est vide, la valeur par défaut utilisée
est '1/second:5', si vous indiquez 'none' aucune limite n'est affectée.

Configuration par défaut~:
}
\verb*?PF6_INPUT_ICMP_ECHO_REQ_LIMIT='1/second:5'?

\config{PF6\_INPUT\_ICMP\_ECHO\_REQ\_SIZE}{PF6\_INPUT\_ICMP\_ECHO\_REQ\_SIZE}{PF6INPUTICMPECHOREQSIZE}{
Vous pouvez indiquer dans cette variable la taille (en octets) d'une demande d'écho ICMPv6.
Vous devez pendre en considération pour indiquer cette valeur l'en-tête du paquet. La valeur
par défaut est de 150 octets.

Configuration par défaut~:
}
\verb*?PF6_INPUT_ICMP_ECHO_REQ_SIZE='150'?

\config{PF6\_INPUT\_N}{PF6\_INPUT\_N}{PF6INPUTN}{
Vous indiquez dans cette variable le nombre de règles du pare-feu IPv6 pour les paquets
entrants (chaîne INPUT). Par défaut, deux règles sont activées~: la première permet à tous
les hôtes du réseau locaux d'accéder au routeur via la dite adresse du niveau du lien,
la seconde permet aux hôtes du réseau local de communiquer sur le premier sous-réseau
IPv6 défini.

Si vous indiquez plusieurs sous-réseaux IPv6 locaux, les règles suivantes doivent être reproduites
autant de fois. Voir le fichier de configuration.

Exemple~:
}
\verb*?PF6_INPUT_N='2'?

\config{PF6\_INPUT\_x}{PF6\_INPUT\_x}{PF6INPUTx}{
Dans cette variable vous indiquez la règle pour la chaîne INPUT du pare-feu IPv6. Pour
une description plus détaillée, voir la documentation de la variable \var{PF\_INPUT\_x}.

Différence avec le pare-feu IPv4~:
\begin{itemize}
\item Au lieu de \var{IP\_NET\_x} vous indiquez ici \var{IPV6\_NET\_x}.
\item Au lieu de \var{IP\_ROUTE\_x} vous indiquez ici \var{IPV6\_ROUTE\_x}.
\item Les adresses IPv6 doivent être placées entre des crochets (y compris le masque
	de sous-réseau, si disponible).
\item Toutes les informations de l'adresse IPv6 (y compris \var{IPV6\_NET\_x} etc.)
	doivent être placées entre des crochets, si vous indiquez un port ou une plage
	de ports.
\end{itemize}

Exemple~:
}

\begin{example}
\begin{verbatim}
PF6_INPUT_1='[fe80::0/10] ACCEPT'
PF6_INPUT_2='IPV6_NET_1 ACCEPT'
PF6_INPUT_3='tmpl:samba DROP NOLOG'
\end{verbatim}
\end{example}

\config{PF6\_INPUT\_x\_COMMENT}{PF6\_INPUT\_x\_COMMENT}{PF6INPUTxCOMMENT}{
Dans cette variable vous pouvez faire une description ou un commentaire sur la règle
INPUT correspondante.

Exemple~:
}
\verb*?PF6_INPUT_3_COMMENT='no samba traffic allowed'?

\config{PF6\_FORWARD\_POLICY}{PF6\_FORWARD\_POLICY}{PF6FORWARDPOLICY}{
Cette variable détermine la stratégie par défaut de la (chaîne FORWARD) pour les paquets
redirigés par le routeur. Les valeurs possibles sont "REJECT" (par défaut, pour tous les
paquets), "DROP" (jeter secrètement tous les paquets) et "ACCEPT" (accepte tous les paquets).
Pour une description plus détaillée, voir la documentation de la variable \var{PF\_FORWARD\_POLICY}.

Configuration par défaut~:
}
\verb*?PF6_FORWARD_POLICY='REJECT'?

\config{PF6\_FORWARD\_ACCEPT\_DEF}{PF6\_FORWARD\_ACCEPT\_DEF}{PF6FORWARDACCEPTDEF}{
Cette variable permet d'utiliser les règles prédéfinies pour la chaîne FORWARD du pare-feu
IPv6. Les valeurs possibles sont "yes" ou "no".

Les règles prédéfinies ouvrent également le pare-feu pour les pings ICMPv6 sortants (un ping
par seconde comme limite). Une répondre aux paquets avec une connexion déjà accepté est également
autorisée.

Configuration par défaut~:
}
\verb*?PF6_FORWARD_ACCEPT_DEF='yes'?

\config{PF6\_FORWARD\_LOG}{PF6\_FORWARD\_LOG}{PF6FORWARDLOG}{
Cette variable vous permet d'enregistrer dans un fichier jounal tous les paquets redirigés rejetés.
Les valeurs possibles sont "yes" ou "no". Pour une description plus détaillée, voir la documentation
de la variable \var{PF\_FORWARD\_LOG}.

Configuration par défaut~:
}
\verb*?PF6_FORWARD_LOG='no'?

\config{PF6\_FORWARD\_LOG\_LIMIT}{PF6\_FORWARD\_LOG\_LIMIT}{PF6FORWARDLOGLIMIT}{
Dans cette variable vous configurez la limite du fichier journal de la chaîne FORWARD du
pare-feu IPv6, ainsi, le fichier journal sera lisible un certain temps. Pour une description
plus détaillée, voir la documentation de la variable \var{PF\_FORWARD\_LOG\_LIMIT}.

Configuration par défaut~:
}
\verb*?PF6_FORWARD_LOG_LIMIT='3/minute:5'?

\config{PF6\_FORWARD\_REJ\_LIMIT}{PF6\_FORWARD\_REJ\_LIMIT}{PF6FORWARDREJLIMIT}{
Dans cette variable vous définissez la limite du rejet des paquets TCP redirigés. Si le paquet
TCP dépasse cette limite, ce paquet sera tranquillement écarté (DROP). Pour une description plus
détaillée, voir la documentation de la variable \var{PF\_FORWARD\_REJ\_LIMIT}.

Configuration par défaut~:
}
\verb*?PF6_FORWARD_REJ_LIMIT='1/second:5'?

\config{PF6\_FORWARD\_UDP\_REJ\_LIMIT}{PF6\_FORWARD\_UDP\_REJ\_LIMIT}{PF6FORWARDUDPREJLIMIT}{
Dans cette variable vous définissez la limite du rejet des paquets UDP redirigés. Si le paquet
UDP dépasse cette limite, ce paquet sera tranquillement écarté (DROP). Pour une description plus
détaillée, voir la documentation de la variable \var{PF\_FORWARD\_UDP\_REJ\_LIMIT}.

Configuration par défaut~:
}
\verb*?PF6_FORWARD_UDP_REJ_LIMIT='1/second:5'?

\config{PF6\_FORWARD\_N}{PF6\_FORWARD\_N}{PF6FORWARDN}{
Vous indiquez dans cette variable le nombre de règles du pare-feu IPv6 pour les paquets redirigés
(chaîne FORWARD). Par défaut, deux règles sont activées~: la première empêche la redirection
de tous les paquets Samba du réseau local vers un réseau non local, la seconde permet la redirection
de tous les autres paquets du réseau local sur le premier sous-réseau IPv6 défini.

Si vous indiquez plusieurs sous-réseaux IPv6 locaux, les règles suivantes doivent être reproduites
autant de fois. Voir le fichier de configuration.

Exemple~:
}
\verb*?PF6_FORWARD_N='2'?

\config{PF6\_FORWARD\_x}{PF6\_FORWARD\_x}{PF6FORWARDx}{
Dans cette variable vous indiquez la règle pour la chaîne FORWARD du pare-feu IPv6. Pour
une description plus détaillée, voir la documentation de la variable \var{PF\_FORWARD\_x}.

Différence avec le pare-feu IPv4~:
\begin{itemize}
\item Au lieu de \var{IP\_NET\_x} vous indiquez ici \var{IPV6\_NET\_x}.
\item Au lieu de \var{IP\_ROUTE\_x} vous indiquez ici \var{IPV6\_ROUTE\_x}.
\item Les adresses IPv6 doivent être placées entre des crochets (y compris le masque
	de sous-réseau, si disponible).
\item Toutes les informations de l'adresse IPv6 (y compris \var{IPV6\_NET\_x} etc.)
	doivent être placées entre des crochets, si vous indiquez un port ou une plage
	de ports.
\end{itemize}

Exemple~:
}

\begin{example}
\begin{verbatim}
PF6_FORWARD_1='tmpl:samba DROP'
PF6_FORWARD_2='IPV6_NET_1 ACCEPT'
\end{verbatim}
\end{example}

\config{PF6\_FORWARD\_x\_COMMENT}{PF6\_FORWARD\_x\_COMMENT}{PF6FORWARDxCOMMENT}{
Dans cette variable vous pouvez faire une description ou un commentaire sur la règle
FORWARD correspondante.

Exemple~:
}
\verb*?PF6_FORWARD_1_COMMENT='no samba traffic allowed'?

\config{PF6\_OUTPUT\_POLICY}{PF6\_OUTPUT\_POLICY}{PF6OUTPUTPOLICY}{
Cette variable spécifie la stratégie par défaut pour les paquets sortants du routeur
en utilisant la (chaîne OUTPUT). Les valeurs possibles sont "REJECT" (par défaut, pour tous
les paquets), "DROP" (jeter secrètement tous les paquets) et "ACCEPT" (accepte tous
les paquets). Pour une description plus détaillée, voir la documentation de la variable
\var{PF\_OUTPUT\_POLICY}.

Configuration par défaut~:
}
\verb*?PF6_OUTPUT_POLICY='REJECT'?

\config{PF6\_OUTPUT\_ACCEPT\_DEF}{PF6\_OUTPUT\_ACCEPT\_DEF}{PF6OUTPUTACCEPTDEF}{
Cette variable permet d'utiliser les règles prédéfinies pour la chaîne OUTPUT du pare-feu
IPv6. Les valeurs possibles sont "yes" ou "no". Actuellement, il y a une règle prédéfinie.

Configuration par défaut~:
}
\verb*?PF6_OUTPUT_ACCEPT_DEF='yes'?

\config{PF6\_OUTPUT\_LOG}{PF6\_OUTPUT\_LOG}{PF6OUTPUTLOG}{
Cette variable vous permet d'enregistrer dans un fichier jounal tous les paquets sortants rejetés.
Les valeurs possibles sont "yes" ou "no". Pour une description plus détaillée, voir la documentation
de la variable \var{PF\_OUTPUT\_LOG}.

Configuration par défaut~:
}
\verb*?PF6_OUTPUT_LOG='no'?

\config{PF6\_OUTPUT\_LOG\_LIMIT}{PF6\_OUTPUT\_LOG\_LIMIT}{PF6OUTPUTLOGLIMIT}{
Dans cette variable vous configurez la limite du fichier journal de la chaîne OUTPUT du
pare-feu IPv6, ainsi, le fichier journal sera lisible un certain temps. Pour une description
plus détaillée, voir la documentation de la variable \var{PF\_OUTPUT\_LOG\_LIMIT}.

Configuration par défaut~:
}
\verb*?PF6_OUTPUT_LOG_LIMIT='3/minute:5'?

\config{PF6\_OUTPUT\_REJ\_LIMIT}{PF6\_OUTPUT\_REJ\_LIMIT}{PF6OUTPUTREJLIMIT}{
Dans cette variable vous définissez la limite du rejet des paquets TCP sortants. Si le paquet
TCP dépasse cette limite, ce paquet sera tranquillement écarté (DROP). Pour une description plus
détaillée, voir la documentation de la variable \var{PF\_OUTPUT\_REJ\_LIMIT}.

Configuration par défaut~:
}
\verb*?PF6_OUTPUT_REJ_LIMIT='1/second:5'?

\config{PF6\_OUTPUT\_UDP\_REJ\_LIMIT}{PF6\_OUTPUT\_UDP\_REJ\_LIMIT}{PF6OUTPUTUDPREJLIMIT}{
Dans cette variable vous définissez la limite du rejet des paquets UDP sortants. Si le paquet
UDP dépasse cette limite, ce paquet sera tranquillement écarté (DROP). Pour une description plus
détaillée, voir la documentation de la variable \var{PF\_OUTPUT\_UDP\_REJ\_LIMIT}.

Configuration par défaut~:
}
\verb*?PF6_OUTPUT_UDP_REJ_LIMIT='1/second:5'?

\config{PF6\_OUTPUT\_N}{PF6\_OUTPUT\_N}{PF6OUTPUTN}{
Vous indiquez dans cette variable le nombre de règles du pare-feu IPv6 pour les paquets sortants
(chaîne OUTPUT). Par défaut, aucune régle n'est définie.

Si vous indiquez plusieurs sous-réseaux IPv6 locaux, les règles suivantes doivent être reproduites
autant de fois. Voir le fichier de configuration.

Exemple~:
}
\verb*?PF6_OUTPUT_N='1'?

\config{PF6\_OUTPUT\_x}{PF6\_OUTPUT\_x}{PF6OUTPUTx}{
Dans cette variable vous indiquez la règle pour la chaîne OUTPUT du pare-feu IPv6. Pour
une description plus détaillée, voir la documentation de la variable \var{PF\_OUTPUT\_x}.

Différence avec le pare-feu IPv4~:
\begin{itemize}
\item Au lieu de \var{IP\_NET\_x} vous indiquez ici \var{IPV6\_NET\_x}.
\item Au lieu de \var{IP\_ROUTE\_x} vous indiquez ici \var{IPV6\_ROUTE\_x}.
\item Les adresses IPv6 doivent être placées entre des crochets (y compris le masque
	de sous-réseau, si disponible).
\item Toutes les informations de l'adresse IPv6 (y compris \var{IPV6\_NET\_x} etc.)
	doivent être placées entre des crochets, si vous indiquez un port ou une plage
	de ports.
\end{itemize}

Exemple~:
}

\begin{example}
\begin{verbatim}
PF6_OUTPUT_1='tmpl:ftp IPV6_NET_1 ACCEPT HELPER:ftp'
\end{verbatim}
\end{example}

\config{PF6\_OUTPUT\_x\_COMMENT}{PF6\_OUTPUT\_x\_COMMENT}{PF6OUTPUTxCOMMENT}{
Dans cette variable vous pouvez faire une description ou un commentaire sur la règle
OUTPUT correspondante.

Exemple~:
}
\verb*?PF6_OUTPUT_3_COMMENT='no samba traffic allowed'?

\config{PF6\_USR\_CHAIN\_N}{PF6\_USR\_CHAIN\_N}{PF6USRCHAINN}{
Vous indiquez dans cette variable le nombre de règles du pare-feu IPv6 définie par l'utilisateur.
Pour une description plus détaillée, voir la documentation de la variable \var{PF\_USR\_CHAIN\_N}.

Configuration par défaut~:
}
\verb*?PF6_USR_CHAIN_N='0'?

\config{PF6\_USR\_CHAIN\_x\_NAME}{PF6\_USR\_CHAIN\_x\_NAME}{PF6USRCHAINxNAME}{
Vous indiquez dans cette variable le nom de la règle du pare-feu IPv6 définie par l'utilisateur.
Pour une description plus détaillée, voir la documentation de la variable \var{PF\_USR\_CHAIN\_x\_NAME}.

Exemple~:
}
\verb*?PF6_USR_CHAIN_1_NAME='usr-myvpn'?

\config{PF6\_USR\_CHAIN\_x\_RULE\_N}{PF6\_USR\_CHAIN\_x\_RULE\_N}{PF6USRCHAINxRULEN}{
Vous indiquez dans cette variable le nombre de règles du pare-feu IPv6 qui est associè au nom
de la règle définie par l'utilisateur. Pour une description plus détaillée, voir la
documentation de la variable \var{PF\_USR\_CHAIN\_x\_RULE\_N}.

Exemple~:
}
\verb*?PF6_USR_CHAIN_1_RULE_N='0'?

\config{PF6\_USR\_CHAIN\_x\_RULE\_x}{PF6\_USR\_CHAIN\_x\_RULE\_x}{PF6USRCHAINxRULEx}{
Dans cette variable vous indiquez la règle définie par l'utilisateur pour le pare-feu IPv6.
Pour une description plus détaillée, voir la documentation de la variable \var{PF\_USR\_CHAIN\_x\_RULE\_x}.

Différence avec le pare-feu IPv4~:
\begin{itemize}
\item Au lieu de \var{IP\_NET\_x} vous indiquez ici \var{IPV6\_NET\_x}.
\item Au lieu de \var{IP\_ROUTE\_x} vous indiquez ici \var{IPV6\_ROUTE\_x}.
\item Les adresses IPv6 doivent être placées entre des crochets (y compris le masque
	de sous-réseau, si disponible).
\item Toutes les informations de l'adresse IPv6 (y compris \var{IPV6\_NET\_x} etc.)
	doivent être placées entre des crochets, si vous indiquez un port ou une plage
	de ports.
\end{itemize}
}

\config{PF6\_USR\_CHAIN\_x\_RULE\_x\_COMMENT}{PF6\_USR\_CHAIN\_x\_RULE\_x\_COMMENT}{PF6USRCHAINxRULExCOMMENT}{
Dans cette variable vous pouvez faire une description ou un commentaire sur la règle 
définie par l'utilisateur correspondante.

Exemple~:
}
\verb*?PF6_USR_CHAIN_1_RULE_1_COMMENT='some user-defined rule'?

\config{PF6\_POSTROUTING\_N}{PF6\_POSTROUTING\_N}{PF6POSTROUTINGN}{
Vous indiquez dans cette variable le nombre de règles du pare-feu IPv6 pour masquer
le réseau (chaîne POSTROUTING). Pour une description plus détaillée, voir la documentation
de la variable \var{PF\_POSTROUTING\_N}.

Exemple~:
}
\verb*?PF6_POSTROUTING_N='2'?

\configlabel{PF6\_POSTROUTING\_x\_COMMENT}{PF6POSTROUTINGxCOMMENT}
\config{PF6\_POSTROUTING\_x PF6\_POSTROUTING\_x\_COMMENT}{PF6\_POSTROUTING\_x}{PF6POSTROUTINGx}
\mbox{}\newline
Dans ces variables vous indiquez les règles pour masquer par le routeur les paquets IPv6
(ou les transmettre non masqué) et aussi faire une description ou un commentaire sur la règle
POSTROUTING correspondante. Pour une description plus détaillée, voir la documentation
de la variable \var{PF\_POSTROUTING\_x}.

\config{PF6\_PREROUTING\_N}{PF6\_PREROUTING\_N}{PF6PREROUTINGN}{
Vous indiquez dans cette variable le nombre de règles du pare-feu IPv6 pour le transférer le
paquet vers une autre destination (chaîne PREROUTING). Pour une description plus détaillée,
voir la documentation de la variable \var{PF\_PREROUTING\_N}.

Exemple~:
}
\verb*?PF6_PREROUTING_N='2'?

\configlabel{PF6\_PREROUTING\_x\_COMMENT}{PF6PREROUTINGxCOMMENT}
\config{PF6\_PREROUTING\_x PF6\_PREROUTING\_x\_COMMENT}{PF6\_PREROUTING\_x}{PF6PREROUTINGx}
\mbox{}\newline
Dans ces variables vous indiquez les règles pour transférer par le routeur les paquets IPv6
vers une autre destination et aussi faire une description ou un commentaire sur la règle
PREROUTING correspondante. Pour une description plus détaillée, voir la documentation de
la variable \var{PF\_PREROUTING\_x}.

\end{description}
