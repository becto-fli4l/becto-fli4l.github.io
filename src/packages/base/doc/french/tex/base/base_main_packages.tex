% Do not remove the next line
% Synchronized to r38958

  \chapter{Les paquetages}

  En plus de l'installation de la base (BASE) il existe d'autres paquetages. Il
  s'agit notamment de "OPTs"\footnote{abréviation pour "module OPTionnel"}
  supplémentaire, qui peuvent au besoin être installés dans la base. Certains
  de ces OPTs sont intégrés dans le paquetage base, les autres sont à télécharger
  part. Une vue d'ensemble les paquetages fournis par l'équipe fli4l peuvent
  être téléchargés sur cette page Web
  (\altlink{http://www.fli4l.de/fr/telechargement/version-stable/}). D'autres
  paquetages crées par des concepteurs privés peuvent être trouvés dans la banque
  de données OPT (\altlink{http://extern.fli4l.de/fli4l_opt-db3/}). Nous allons
  voir dans les paragraphes suivant une description des paquetages créés et
  utilisés par l'équipe fli4l.


  \section{Outils dans le paquetage base}

  Dans le paquetage base vous trouverez les OPTs suivants~:

  \begin{center}
  \begin{tabular}{@{}lp{12cm}@{}}\hline
    Nom                & Description \\\hline
    \var{OPT\_SYSLOGD} & \jump{OPTSYSLOGD}{Programme qui enregistre tous les messages} \\
    \var{OPT\_KLOGD}   & \jump{OPTKLOGD}{Programme qui enregistre les messages Kernel} \\
    \var{OPT\_LOGIP}   & \jump{OPTLOGIP}{Programme qui enregistre les protocoles IP WAN} \\
    \var{OPT\_Y2K}     & \jump{OPTY2K}{Correctif pour les ordinateurs avant l'année 2K} \\
    \var{OPT\_PNP}     & \jump{OPTPNP}{Outil pour l'installation des cartes ISAPnP} \\
    \var{OPT\_HOTPLUG\_PCI} & \jump{OPTHOTPLUGPCI}{Activation du PCI hot-plugging}\\\hline
  \end{tabular}
  \end{center}


\marklabel{OPTSYSLOGD}{\subsection{OPT\_SYSLOGD~- Enregistre tous les messages du système}}\index{OPT\_SYSLOGD}

  Beaucoup de programmes utilisent l'interface de syslogd pour
  visualiser les messages du système. Pour rendre les messages visibles,
  installer le démon syslogd.

  Si vous voulez voir les messages debug, placer \var{OPT\_\-SYSLOGD}
  sur 'yes', si vous ne voulez pas de message sur 'no'.

  Voir également \jump{ISDNCIRCxDEBUG}{\var{ISDN\_CIRC\_x\_DEBUG}} et
  \jump{PPPOEDEBUG}{\var{PPPOE\_DEBUG}}.

  Configuration par défaut~: \var{OPT\_\-SYSLOGD}='no'

  \begin{description}
    \config{SYSLOGD\_RECEIVER}{SYSLOGD\_RECEIVER}{SYSLOGDRECEIVER}

  Avec la variable \var{SYSLOGD\_RECEIVER} on peut définir, si fli4l doit
  recevoir ou non les messages syslog par le réseau.

    \configlabel{SYSLOGD\_DEST\_x}{SYSLOGDDESTx}
    \config{SYSLOGD\_DEST\_N SYSLOGD\_DEST\_x}{SYSLOGD\_DEST\_N}{SYSLOGDDESTN}

  Avec la variable \var{SYSLOGD\_\-DEST\_\-x} on indique les emplacements,
  où vous voulez voir les messages système enregistrés par l'interface syslogd.
  Normalement c'est sur la console de fli4l que l'on voit les messages~:

\begin{example}
\begin{verbatim}
        SYSLOGD_DEST_1='*.* /dev/console'
\end{verbatim}
\end{example}

  Si vous souhaitez utiliser un fichier pour enregistrer les messages~:

\begin{example}
\begin{verbatim}
        SYSLOGD_DEST_1='*.* /var/log/messages'
\end{verbatim}
\end{example}

  Si un hôte dans le réseau veut lire les messages, vous pouvez réorienter des
  messages vers cette ordinateur~-- en indiquant l'adresse IP.

  Exemple~:
\begin{example}
\begin{verbatim}
        SYSLOGD_DEST_1='*.* @192.168.4.1'
\end{verbatim}
\end{example}

  Il faut préfixer par le caractère @ avant d'écrire l'adresse IP ou le nom hôte.

  Si vous voulez envoyer les messages sur différent système, il est nécessaire
  d'augmenter le nombre dans la variable \var{SYSLOGD\_DEST\_N} (nombre de
  descrition) et de remplir les variables en conséquence, par ex.
  \var{SYSLOG\_DEST\_1}, \var{SYSLOG\_DEST\_2} etc.

  Les caractères '*.*' désignent l'ensemble des services et des
  priorités des messages, on peut limiter les priorités pour une
  "destination" déterminée. Dans ce cas, on remplace l'étoile
  après le point Par l'un des mots clés suivant~:

  \begin{itemize}
  \item debug
  \item info
  \item notice
  \item warning (obsolète~: warn)
  \item err (obsolète~: error)
  \item crit
  \item alert
  \item emerg (obsolète~: panic)
  \end{itemize}

  L'ordre dans la liste reflète le "poids" des annonces. Les mots
  clés "error", "warn" et "panic" sont obsolètes et ne devaient plus être
  utilisés ils sont remplacés par err, warning et emerg.

  Vous pouvez remplacer l'astérisque (*) devant le point par un soi-disant
  "sélecteur", cependant il seraient trop long d'expliquer ici tous des
  paramètres. Le lecteur peut essayer trouver toutes les informations
  nécessaires sur un moteur de recherche. Vous pouvez voir la configuration
  dans le manuel de syslog.conf~:

  \enlargethispage{\baselineskip}
  \noindent \altlink{http://linux.die.net/man/5/syslog.conf} ou\\
  \altlink{http://okki666.free.fr/docmaster/articles/linux068.htm}

  Normalement, l'astérisque, est tout à fait suffisant. Exemple~:
\begin{example}
\begin{verbatim}
          SYSLOGD_DEST_1='*.warning @192.168.4.1'
\end{verbatim}
\end{example}

  Non seulement les ordinateurs Unix et Linux, mais aussi les ordinateurs Windows
  peuvent servir d'hôte pour les logs (ou fichiers journal). Sur
  \altlink{http://www.fli4l.de/fr/divers/liens/} vous trouverez des liens
  pour avoir des logiciels appropriés. L'application d'un serveur log est recommandée,
  pour l'enregistrement détaillé des protocoles, l'enregistrement des protocoles
  aide également au dépistage des erreurs. Le protocole syslog est aussi
  compatible avec imonc client Windows et peut ainsi recevoir les messages log.

  Malheureusement, les informations de Boot fli4l ne peuvent pas être
  enregistrées avec le démon syslogd. Toutefois, on peut configurer fli4l pour
  que les informations de Boot puissent sortir sur une console de terminal
  série (voir \jump{CONSOLESETTINGS}{Configuration de la console}).

  \config{SYSLOGD\_ROTATE}{SYSLOGD\_ROTATE}{SYSLOGDROTATE}

  Vous pouvez définir avec la variable \var{SYSLOGD\_ROTATE} si fli4l doit faire
  une rotation des messages syslog une fois par jour. Ainsi les derniers messages
  seront enregistrés tous les x jours.

  \config{SYSLOGD\_ROTATE\_DIR}{SYSLOGD\_ROTATE\_DIR}{SYSLOGDROTATEDIR}

  La variable \var{SYSLOGD\_ROTATE\_DIR} est optionnelle, vous pouvez définir
  ici le répertoire pour l'enregistrement des fichiers syslog de rotation.
  Si cette variable est vide, le répertoire par défaut /var/log sera utilisé.

  \config{SYSLOGD\_ROTATE\_MAX}{SYSLOGD\_ROTATE\_MAX}{SYSLOGDROTATEMAX}

  La variable \var{SYSLOGD\_ROTATE\_MAX} est optionnelle, elle vous permet de
  spécifier un nombre d'enregistrements par rotation des fichiers syslog.

  \config{SYSLOGD\_ROTATE\_AT\_SHUTDOWN}{SYSLOGD\_ROTATE\_AT\_SHUTDOWN}{SYSLOGDROTATEATSHUTDOWN}

  La variable \var{SYSLOGD\_ROTATE\_AT\_SHUTDOWN} est optionnelle, elle vous
  permet de désactiver la rotation du fichier syslog lors d'un arrêt du routeur.
  Attention vous ne pouvait pas désactiver la rotation, si vos fichiers syslog
  sont écrit directement vers une destination permanente.

\end{description}

\marklabel{OPTKLOGD}{\subsection{OPT\_KLOGD~-- Messages du Kernel lors du boot}}\index{OPT\_KLOGD}

  Parfois des erreurs apparaissent lors du Boot de Linux Kernel, ils
  sont écrits directement sur la console (ou écran) et il est difficile
  de les visualiser. En utilisant \var{OPT\_\-KLOGD}='yes' ces messages
  sont réorientés sur le syslogd, ils peuvent être soit expédiés sur un
  client log ou écrits dans un fichier voir ci-dessus. Ainsi nous
  ne sommes pas obligés de surveiller la console.

  \noindent Il est recommandé de paramétrer~: \var{OPT\_\-SYSLOGD}='yes' et aussi
  de paramétrer \var{OPT\_\-KLOGD}='yes'.

  Configuration par défaut~: \var{OPT\_\-KLOGD}='no'


\marklabel{OPTLOGIP}{\subsection{OPT\_LOGIP~-- Journalisation des adresses IP WAN}}\index{OPT\_LOGIP}

  Avec LOGIP il est possible, d'enregistrer les messages IP WAN dans un fichier
  journal pour cela il faut activer la variable \var{OPT\_LOGIP}='yes'.

  Configuration par défaut~: \var{OPT\_LOGIP}='no'

\begin{description}
  \config{LOGIP\_LOGDIR}{LOGIP\_LOGDIR}{LOGIPLOGDIR}{~- Défini le répertoire des fichiers LOG}

  Avec la variable \var{LOGIP\_LOGDIR} on défini le répertoire, dans
  lequel les fichiers Log sont créés ou 'auto' pour l'autodétection.

  Configuration par défaut~: \var{LOGIP\_LOGDIR}='auto'
\end{description}


\marklabel{OPTY2K}{\subsection{OPT\_Y2K~-- Correctif pour avant l'année 2000}}\index{OPT\_Y2K}

  Dans la plupart des cas les routeurs fli4l sont assemblés avec du
  vieux matériel. Parfois les cartes mères ne sont pas compatible pour
  passer l'année 2000. Lorsque vous réglerez la date du 27/05/2000
  dans le BIOS, au prochain démarrage la date dans le BIOS sera peut être
  indiquée 27/05/2094~! Et dans Linux elle sera indiqué 27/05/1994~:-)

  Si votre date n'est pas correcte il n'y a pas vraiment d'importance pour
  le routeur fli4l. Mais si le routeur est utilisé en tant gestionnaire
  de coût (ou frais de connexion Internet), cela est important.

  La raison~: le 27/05/1994 est un vendredi et le 27/05/2000 est un samedi
  et les week-ends les prix des connexion Internet et/ou les fournisseurs sont
  meilleur marché. \ldots

  La première alternative~: vous indiquez dans la BIOS la date du 28/05/1994,
  au lieu du 27/05/2000, qui est un samedi. Mais le problème n'est pas
  complètement résolu. parce que fli4l utilise non seulement la semaine mais
  aussi l'heure actuelle pour le réglage du LC-routage, il prend également en
  compte les jours fériés.

\begin{description}
  \config{Y2K\_DAYS}{Y2K\_DAYS}{Y2KDAYS}{~-- Ajouter N jours à la date du système}

  Puisque la différence exact de la date est de 2191 jours par rapport à
  la date réelle, on peut indiquer~:

\begin{example}
\begin{verbatim}
        Y2K_DAYS='2191'
\end{verbatim}
\end{example}

  En ajoutant 2191 jours à la date du BIOS, la date dans Linux
  sera à jour. Cependant, la date du BIOS ne doit pas être modifier.
  Autrement la date sera remise à zéro à 2094 (ou à 1994) au
  prochain démarrage~:-)
\end{description}

  Il y a une autre alternative~:

  En accédant à un serveur de temps fli4l peut rechercher la date et
  l'heure exacte sur Internet. Pour cela vous avez le paquetage
  \jump{sec:opt-chrony}{\var{CHRONY}} qui fait cette recherche. Vous pouvez
  combiner les deux variables, ainsi la date sera corrigée en utilisant
  \var{Y2K\_DAYS} et l'heure exacte sera recherché sur le serveur de temps.

  Si vous n'avez aucun problème lié à Y2K, placer \var{OPT\_\-Y2K}='no'
  et oublier ce fonction \ldots


\marklabel{OPTPNP}{\subsection{OPT\_PNP~-- Installation des cartes ISAPnP}}\index{OPT\_PNP}

  Quelques cartes ISAPnP doivent être configurées en utilisant
  l'outil isapnp. cela concerne les cartes ISDN du \var{ISDN\_\-TYPE} 7, 12, 19,
  24, 27, 28, 30 et 106~-- Mais uniquement si vous avez vraiment une carte ISAPnP.

  Au démarrage, il est nécessaire de créer un fichier de configuration etc/isapnp.conf.

  Voici une description courte pour le créer~:

  \begin{itemize}
  \item Dans le fichier $<$config$>$/base.txt régler les variables
    comme ceci \var{OPT\_\-PNP}='yes' et \var{MOUNT\_\-BOOT}='rw'
  \item La carte ISAPnP ne sera probablement pas identifiée au boot (ou démarrage)
  \item Écrire sur la console du routeur fli4l~:
\begin{example}
\begin{verbatim}
        pnpdump -c >/boot/isapnp.conf
        umount /boot
\end{verbatim}
\end{example}
    Maintenant la configuration doit être sauvegardée sur un média de boot.
  \end{itemize}
  On continue la configuration sur le PC (Unix/Linux/Windows)~:
  \begin{itemize}
  \item Copier le fichier isapnp.conf depuis le média de boot dans le
    répertoire $<$config$>$/etc/isapnp.conf de votre PC
  \item Ouvrez isapnp.conf avec un éditeur de texte\\
        On peut garder les valeurs avancées ou remplacer ces valeurs
        par d'autres. Les lignes suivantes sont importantes dans
        l'exemple qui suit~:

\begin{example}
\begin{verbatim}
            #     Start dependent functions: priority acceptable
            #       Logical device decodes 16 bit IO address lines
            #             Minimum IO base address 0x0160
            #             Maximum IO base address 0x0360
            #             IO base alignment 8 bytes
            #             Number of IO addresses required: 8
       1)      (IO 0 (SIZE 8) (BASE 0x0160))
            #       IRQ 3, 4, 5, 7, 10, 11, 12 or 15.
            #             High true, edge sensitive interrupt (by default)
       2)      (INT 0 (IRQ 10 (MODE +E
\end{verbatim}
\end{example}

    \begin{description}
      \item 1)~-- Dans la \flqq{}BASE\frqq{} on indique les adresses
                 Minimum et Maximum utilisées, on doit toujours prendre
                 en considération \flqq{}l'alignement de base\frqq{}.\\
                 Si vous avez plus d'une carte ISA dans votre système,
                 vous devez toujours vérifier s'il n'y a pas d'intersection
                 entre les adresses et faire attention à la quantité
                 d'adresses nécessaire (number of addresses required).

      \item 2)~-- La liste des IRQ suivante est un mauvais choix pour le
                 paramètrage de la carte ISA.
                 2, (9), 3, 4, 5 et 7 ils sont normalement utilisé par le système,
                 les ports serie, le port parallèle, ect.\\
                 On ne peut pas diviser une IRQ pour plusieurs cartes ISA,
                 c'est pourquoi on ne doit jamais utiliser une IRQ dèjà occupée.
    \end{description}
  \item Copier les valeurs (IRQ/IO) et les écrire dans le fichier
    $<$config$>$/isdn.txt
  \item Il est nécessaire de régler la variable \var{OPT\_\-PNP}
    dans $<$config$>$/base.txt sur 'yes' autrement les fichiers ne seront
    pas copiés sur le média de boot. Vous pouvez remodifier la variable
    \var{MOUNT\_BOOT} selon votre choix.
  \item Créer un nouveau média de boot
  \end{itemize}

  \achtung {Le fichier qui a été généré automatiquement est sauvegardé dans le
    format Unix et ne contient aucun CRs (ou Retours Chariots). Si on lance
    l'éditeur Notepad sous Windows on verra le fichier sur une seul ligne.
    L'éditeur Notepad sous DOS "édite" et peut traiter des fichiers Unix. Il
    faut le sauvegarder comme un fichier DOS avec les CRs.}

  Remède~:
  \begin{itemize}
  \item Lancer la boite de commande DOS
  \item Charger le répertoire $<$config$>$/etc
  \item Entrer~: edit isapnp.conf
  \item Éditer le fichier et sauvegardez le
  \end{itemize}

  Ensuite, on peut travailler sur le fichier avec Notepad.

  On peut aussi utiliser simplement l'éditeur de Wordpad sous Windows.

  De plus les CRs générés sont filtrés, ils ne causeront pas de
  problème lors du Boot de fli4l.

  Au début, vous devriez essayer sans activer \var{OPT\_\-PNP}. Au cas où
  la carte ne serait pas identifiée, suivre la procédure décrite ci-dessus.

  Quand vous installez une nouvelle version fli4l, vous pouvez récupérer le
  fichier \hbox{isapnp.conf} qui a été créé, il ne doit pas être créé à nouveau,
  mais peut être réutilisé.

  Configuration par défaut~: \var{OPT\_\-PNP}='no'

\marklabel{OPTHOTPLUGPCI}{\subsection{OPT\_HOTPLUG\_PCI~-- Activation du PCI hot-plugging}}\index{OPT\_HOTPLUG\_PCI}

  Si vous activez cette variable \var{OPT\_HOTPLUG\_PCI='yes'} les
  modules seront copiés dans fli4l et chargés au démarrage ainsi le PCI hot-plugging
  (ou branchement PCI à chaud) sera activé, c'est à dire que vous pourrez ajouter
  ou retirer des cartes PCI pendant que le système est en marche. Pour que
  cela fonctionne un contrôleur PCI hot-plug doit être présent sur l'ordinateur.

  Cette option ne doit \emph{pas} être activé pour ajouter ou supprimer des périphériques
  virtuels dans un \emph{environnement de virtualisation} comme KVM, puisque cela
  se fait par l'intermédiaire du mécanisme ACPI et que les pilotes ACPI sont activés en
  permanence dans le Kernel.
