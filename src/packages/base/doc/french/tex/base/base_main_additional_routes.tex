% Do not remove the next line
% Synchronized to r29817

\section{Route supplémentaire (optionnel)}

\begin{description}
  \config{IP\_ROUTE\_N}{IP\_ROUTE\_N}{IPROUTEN}

    Configuration par défaut~: \var{IP\_\-ROUTE\_\-N}='0'

    {Vous indiquez ici le nombre de routes supplémentaires pour le réseau. Une
    route supplémentaire est nécessaire par exemple, lorsqu'on a dans le
    LAN un routeur supplémentaire ou une passerelle sur lequel est connecté un
    autre réseau, ce réseau doit être accessibles par le routeur fli4l.

    Normalement il n'est pas nécessaire d'indiquer de route supplémentaire
    pour le réseau.
    }

    \config{IP\_ROUTE\_x}{IP\_ROUTE\_x}{IPROUTEx}

    {Les routes supplémentaires \var{IP\_\-ROUTE\_\-1}, \var{IP\_\-ROUTE\_\-2},
    \ldots ont la structure suivante~:

\begin{example}
\begin{verbatim}
        network/netmaskbits gateway
\end{verbatim}
\end{example}

    Pour se connecter il faut l'adresse du réseau \texttt{network} et son masque de
    sous-réseau \texttt{/netmaskbits}, avec la notation \jump{tab:cidr}{CIDR} et
    l'adresse de la \texttt{gateway} (ou passerelle). Le routeur fli4l et la passerelle
    doivent être naturellement dans la même classe d'adresse IP, par exemple,
    pour que le réseau 192.168.7.0 avec son masque de sous-réseau 255.255.255.0
    accéde à la passerelle 192.168.6.99, on écrit alors~:

\begin{example}
\begin{verbatim}
        IP_ROUTE_N='1'
        IP_ROUTE_1='192.168.7.0/24 192.168.6.99'
\end{verbatim}
\end{example}}

    Si le routeur fli4l n'est pas installé comme routeur Internet mais
    seulement comme un pur routeur Ethernet (un pont), on peut indiquer dans
    IP\_ROUTE\_x une route par défaut. On enregistre alors 0.0.0.0/0 à la place
    de network/netmaskbits, voir l'exemple suivant~:

\begin{example}
\begin{verbatim}
        IP_ROUTE_N='3'
        IP_ROUTE_1='192.168.1.0/24 192.168.6.1'
        IP_ROUTE_2='10.73.0.0/16 192.168.6.1'
        IP_ROUTE_3='0.0.0.0/0 192.168.6.99'
\end{verbatim}
\end{example}
\end{description}

