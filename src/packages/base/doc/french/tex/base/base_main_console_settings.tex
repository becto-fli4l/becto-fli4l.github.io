% Do not remove the next line
% Synchronized to r54214

\marklabel{CONSOLESETTINGS}{\section{Configuration de la console}}

fli4l peut être exécuté sur différentes plates-formes matérielles. Sur bon
nombre de ces plates-formes, il est possible de connecter un clavier et un
moniteur pour interagir avec fli4l, cette combinaison d'entrées et de sorties
est généralement appelée \emph{console}.

fli4l peut également être utilisé sans clavier ni carte graphique. Si vous voulez
voir les messages de démarrage du noyau (kernel) du routeur et si vous n'avez pas
de connexion réseau, il est possible d'utiliser une console distante pour recevoir
les entrées et sorties en passant par l'interface série. Pour cela, il est nécessaire
de paramétrer les variables suivantes
\jump{SERCONSOLE}{\var{SER\_CONSOLE}},
\jump{SERCONSOLEIF}{\var{SER\_CONSOLE\_IF}} et
\jump{SERCONSOLERATE}{\var{SER\_CONSOLE\_RATE}}

Enfin, vous pouvez utiliser en parallèle une console avec clavier et moniteur
et aussi utiliser l’interface série.

En général, fli4l offre la possibilité de se connecter à \emph{n’importe} quelle
console et donc au \emph{Shell} (interpréteur de commandes), vous pouvez vous
connecter avec le nom d'utilisateur "fli4l" et le mot de passe configuré dans
la variable \jump{PASSWORD}{\var{PASSWORD}}

\begin{description}
  \config{CONSOLE\_BLANK\_TIME}{CONSOLE\_BLANK\_TIME}{CONSOLEBLANKTIME}

  Configuration par défaut~: \var{CONSOLE\_BLANK\_TIME=''}

  Lorsque vous n'utilisez pas la console du kernel Linux (de fli4l) pendant un
  certain temps, normalement l'économiseur d'écran s'active. Avec la variable
  \var{CONSOLE\_BLANK\_TIME} on peut désactiver complètement le mode économiseur
  d'écran, avec le paramètrage (\var{CONSOLE\_BLANK\_TIME}='0').

  \config{BEEP}{BEEP}{BEEP}

    Configuration par défaut~: \var{BEEP='yes'}

    {Signale sonore au démarrage et à l'arrêt de fli4l.

    Si vous placez 'yes' dans cette variable, un signal sonore retentira au démarrage
    et à l'arrêt du processus. S'il manque de la place sur le média de boot et
    aussi pour gagner quelques octets, ou si vous ne voulez pas que le signal sonore
    soit émit, vous peuvez indiquer 'no'.}

  \config{SER\_CONSOLE}{SER\_CONSOLE}{SERCONSOLE}

    Configuration par défaut~: \var{SER\_CONSOLE='no'}

	Cette variable active ou désactive la console sur le port série. La
	console série peut être configurée en trois modes différents~:

      \begin{tabular}[h!]{|l|p{9cm}|}
        \hline
        \var{SER\_CONSOLE} & Entrée/Sortie sur la console \\
        \hline
		no & Entrée et sortie (uniquement) par le clavier et le moniteur (tty0) \\
		yes & Entrée et sortie (uniquement) par l'interface série (ttyS0) \\
		primary & Entrée et sortie par la console série ainsi que par
		le clavier et le moniteur, sortie des messages du noyau sur tty0 \\
		secondary & Entrée et sortie par la console série ainsi que par
		le clavier et le moniteur, sortie des messages du noyau sur ttyS0 \\
		\hline
      \end{tabular}

    Si la valeur \var{SER\_CONSOLE} est modifiée, cette modification ne
    prendra effet lors de la création d'un nouveau support de démarrage ou
	lors de la mise à jour à distance du fichier syslinux.cfg.

    \wichtig{Lorsque vous coupez la console série, veillez à maintenir un accès
	alternatif au routeur avec (le SSH ou directement à partir du clavier et
	du moniteur)~!}

    Vous trouverais des informations complémentaires en cliquant sur
    \jump{SERIALCONSOLE}{Console serie}.

  \config{SER\_CONSOLE\_IF}{SER\_CONSOLE\_IF}{SERCONSOLEIF}

    Configuration par défaut~: \var{SER\_CONSOLE\_IF='0'}

   {Numéro de l'interface série pour la console série.
	
	Vous indiquez dans cette variable le numéro d'interface sur laquelle
	la console série est connectée. 0 correspond à ttyS0 sous Linux ou COM1
	sous Microsoft Windows.}

  \config{SER\_CONSOLE\_RATE}{SER\_CONSOLE\_RATE}{SERCONSOLERATE}

    Configuration par défaut~: \var{SER\_CONSOLE\_RATE='9600'}

    {Vitesse de transmission de l'interface série pour la console.

    Ici vous indiquez la vitesse en Baud avec laquelle les données seront
    transmises sur l'interface série. Les valeurs sont~: 4800, 9600, 19200,
    38400, 57600, 115200.}

\end{description}
