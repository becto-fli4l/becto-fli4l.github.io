% Do not remove the next line
% Synchronized to r29883

\section{Réseaux}

\begin{description}
  \config{IP\_NET\_N}{IP\_NET\_N}{IPNETN}

  Configuration par défaut~: \var{IP\_NET\_N='1'}

  {Dans cette variable on indique le nombre de réseaux qui sera associé au 
  Pro\-to\-cole IP, en général '1' réseau est déjà indiqué. S'il n'y a pas de
  réseaux ou s'ils sont configurés sur un autre chemin, alors la variable
  \var{IP\_NET\_N} sera placé sur '0'. Un message d'avertissement sera indiqué
  lors de la construction de archive, on peut annuler cette avertissement avec
  la variable \marklabel{IGNOREIPNETWARNING}{\var{IGNOREIPNETWARNING}='yes'}.}

  \config{IP\_NET\_x}{IP\_NET\_x}{IPNETx}

  Configuration par défaut~: \var{IP\_NET\_1='192.168.6.1/24'}

  {Présentation du dispositif pour l'adressage IP et du masque de sous-réseau
  avec CIDR\footnote{Classless inter-domaine Routing} dans le routeur fli4l. Si
  l'adresse IP est attribuée dynamiquement par le client DHCP, la valeur \texttt{'dhcp'}
  sera alors indiquer dans cette variable.

  Dans le tableau ci-dessous, vous pouvez voir les relations entre CIDR, le masque
  de sous-réseau et le nombre d'adresse IP

   \marklabel{tab:cidr}{
     \begin{tabular}[h!]{rcc}
       CIDR &  Masque réseau  & Nombre d'IPs \\
       \hline
       \hline
       /8   & 255.0.0.0       & 16777216   \\
       /16  & 255.255.0.0     & 65536      \\
       /23  & 255.255.254.0   & 512        \\
       /24  & 255.255.255.0   & 256        \\
       /25  & 255.255.255.128 & 128        \\
       /26  & 255.255.255.192 & 64         \\
       /27  & 255.255.255.224 & 32         \\
       /28  & 255.255.255.240 & 16         \\
       /29  & 255.255.255.248 & 8          \\
       /30  & 255.255.255.252 & 4          \\
       /31  & 255.255.255.254 & 2          \\
       /32  & 255.255.255.255 & 1
     \end{tabular}
  }

  \textbf{Remarque~:} Puisque l'on réserve respectivement une adresse IP pour
    le Broadcast et une pour le réseau, le calcul du nombre maximal des hôtes
    dans le réseau est le suivant~: \texttt{Nombre\_Hôtes = Nombre\_IPs - 2}.
    Le plus petit masque de sous-réseau est \texttt{/30}, correspondant à 4
    adresses IP - 2 reste 2 adresses IP pour les hôtes.
  }

  \config{IP\_NET\_x\_DEV}{IP\_NET\_x\_DEV}{IPNETxDEV}

  Configuration par défaut~: \var{IP\_\-NET\_\-1\_\-DEV='eth0'}

  {Requis~: le nom du périphérique de la carte réseau.

    Dès la version 2.1.8, le nom du périphérique utilisé est nécessaire~!
    Les noms des périphériques commencent dans la plupart des cas par
    \texttt{'eth'} et suivi par d'un chiffre. La première carte réseau
    reconnue par le système reçoit le nom \texttt{'eth0'}, la deuxième \texttt{'eth1'} etc...\\

    Exemple~:

\begin{example}
\begin{verbatim}
        IP_NET_1_DEV='eth0'
\end{verbatim}
\end{example}

    fli4l maîtrise aussi l'IP Aliasing, c'est l'attribution de plusieurs adresses
    IPs sur une carte réseau. on définit d'autre réseaux sur une même
    interface avec simplement des IPs supplémentaires. Lors de la vérification
    des information de configuration, "mkfli4l" indique qu'un alias est défini~--
    vous pouvez ignorer cet avertissement.

    Exemple~:

\begin{example}
\begin{verbatim}
        IP_NET_1='192.168.6.1/24'
        IP_NET_1_DEV='eth0'
        IP_NET_2='192.168.7.1/24'
        IP_NET_2_DEV='eth0'
\end{verbatim}
\end{example}
    }

  \config{IP\_NET\_x\_MAC}{IP\_NET\_x\_MAC}{IPNETxMAC}

  Configuration par défaut~: \var{IP\_\-NET\_\-1\_\-MAC=''}

  {Optionnel~: adresse MAC de la carte réseau.

    Avec cette variable on peut installer l'adresse (MAC) de la carte réseau.
    Par exemple, si vous voulez utiliser un fournisseur d'accès DHCP qui attend
    uniquement une adresse MAC déterminée. Si la variable \var{IP\_NET\_x\_MAC}
    est vide ou pas installée l'adresse MAC de la carte réseau préréglée sera
    installée automatiquement. La plupart des utilisateurs n'auront pas besoin
    de cette variable.

    Exemple~:

\begin{example}
\begin{verbatim}
        IP_NET_1_MAC='01:81:42:C2:C3:10'
\end{verbatim}
\end{example}
    }

  \config{IP\_NET\_x\_NAME}{IP\_NET\_x\_NAME}{IPNETxNAME}

  Configuration par défaut~: \var{IP\_\-NET\_\-x\_\-NAME=''}

  {Optionnelle~: On peut donner un nom à la carte réseau.

    Lors de la résolution de nom inverse, un nom apparaîtra à la place de
    l'adresse IP selon le nom par défaut sous la forme 'fli4l-ethx.$<$domain$>$'.
    Avec la variable \var{IP\_NET\_x\_NAME} vous pouvez indiquer le nom que
    vous voulez. Ce nom sera vu dans la résolution de nom inverse. Avec une
    adresse IP publique, on peut accéder au nom public de celle-ci,
    voir ci-dessous.

    Exemple~:

\begin{example}
\begin{verbatim}
        IP_NET_2='80.126.238.229/32'
        IP_NET_2_NAME='ajv.xs4all.nl'
\end{verbatim}
\end{example}
    }

  \config{IP\_NET\_x\_TYPE}{IP\_NET\_x\_TYPE}{}

  \config{IP\_NET\_x\_COMMENT}{IP\_NET\_x\_COMMENT}{IPNETxCOMMENT}

  Configuration par défaut~: \var{IP\_NET\_x\_COMMENT=''}

    {Optionnelle~: Cette variable sert à donner une indication à un
    périphérique avec un nom 'parlant'. Celui-ci peut être utilisé pour
    l'identification du réseau dans des paquetages comme par exemple opt rrdtool.
    }

\end{description}
