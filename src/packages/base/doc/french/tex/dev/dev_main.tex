% Synchronized to r53804
\chapter{Documentation pour développeur}

\section{Règles générales}

Certaines règles doivent être respectées, lorsque vous ajoutez un nouveau
paquetage dans la base de données OPT-fli4l, qui se trouve sur la page
d'accueil du site fli4l. Le paquetage qui ne respectent pas à ces règles,
sera supprimé de la base de données, sans avertissement préalable.

  \begin{enumerate}
    \item L'utilisateur ne doit faire, AUCUNE copie supplémentaire~! fli4l
    fournit un système sophistiqué, Les données des paquetages-fli4l sont
    décompressés dans les répertoires de l'installation, tous les fichiers
	qui font fonctionner le routeur sont dans le répertoire \texttt{opt/}.

    \item Les paquetages correctement empaqueter sont compressés~: de telle sorte,
    que les paquetages soient facilement décompresser dans le répertoire-fli4l.

    \item Les paquetages doivent être TOTALEMENT configurable dans le fichier de
    configuration. L'utilisateur ne doit pas faire de modifications sur
    d'autres fichiers de configurations. Vous ne devez pas mettre l'utilisateur
    en difficulté, sur des décisions difficiles à prendre, par exemple
    (à la fin du fichier de configuration avec une remarque en gros caractère~:
    ONLY MODIFY IF YOU KNOW WHAT YOU DO).

    \item Encore une remarque sur le fichier de configuration~: les nom des
    variables doivent être claires et l'on doit savoir à quelle OPT elles
    appartiennent, par exemple \var{OPT\_\-HTTPD} les nom des variables sont
    \var{OPT\_\-HTTPD}, \var{HTTPD\_\-USER\_\-N}, etc.

    \item S'il vous plaît, si vous avez compilé vous-même de petits (Programmes)
    binaire~! Et si vous traduisez vous-même le FBR, pensez de désactiver les
	fonctionnalités inutiles.

    \item Contrôler votre Copyright~! Si vous utilisé un modèle de fichier,
    merci de respecter les droit d'auteur. Le copyright doit est remplacé
    ici par votre propre nom si vous créez vos fichiers. vérifié en particulié,
    les fichiers dans config-, Check- et les fichiers textes dans opt-. Si vous
    copiez la documentation mot à mot, le copyright de l'auteur d'origine
    doit être naturellement gardé~!

    \item Merci de diffuser seulement des types d'archivages, utilisant des
    formats libres. Il s'agit notamment de~:
    \begin{itemize}
      \item ZIP  (\texttt{.zip})
      \item GZIP (\texttt{.tgz} ou \texttt{.tar.gz})
    \end{itemize}
    S'il vous plaît n'utilisez pas les autres formats tels que RAR, ACE, Blackhole,
	LHA, etc.
	Vous ne devez pas utiliser les fichiers d'installation Windows (\texttt{.msi}) ou
	les fichiers d'installation (\texttt{.exe}) et les archives auto-extractable.
  \end{enumerate}


\marklabel{sec:libc}{
  \section{Compiler les programmes}
}

Pour pouvoir compiler des programmes vous allez avoir besoin du paquetage
\og{}src\fg{} qui est disponible à part. Il y a également une documentation
pour compiler vos propre programme pour fli4l.

% Synchronized to r35857

\section{Module Concept}

As of version 2.0 fli4l is split into modules (packages), i.e.

\begin{itemize}
    \item fli4l-\version~~$<$--- The Base Package
    \item dns-dhcp
    \item dsl
    \item isdn
    \item sshd
\item and much more...
\end{itemize}


With the base package fli4l acts as a pure Ethernet router. For ISDN
and/or DSL the packages isdn and/or dsl have to be unpacked to the
fli4l directory. The same applies for the other packages.


\marklabel{mkfli4l}{\subsection{mkfli4l}}

Depending on the current configuration a file called \texttt{rc.cfg} and two
archives \texttt{rootfs.img} and \texttt{opt.img} will be generated
which contain all required configuration informations and files.
These files are generated using \var{mkfli4l} which reads the individual
package files and checks for configuration errors.

\var{mkfli4l} will accept the parameters listed in table \ref{tab:mkfli4l}.
If omitted the default values noted in brackets are used. A complete list
of all options (Table \ref{tab:mkfli4l}) is displayed when executing
\begin{verbatim}
    mkfli4l -h
\end{verbatim}
.

\begin{table}[htbp]
  \centering
  \caption{Parameters for \var{mkfli4l}}
  \begin{tabular}{|lp{2cm}|p{8cm}|}
    \hline
    \multicolumn{1}{|c|}{\textbf{Option}} & \multicolumn{2}{c|}{\textbf{Meaning}}\\
    \hline
    -c, -\,-config    & \multicolumn{2}{|p{11cm}|} { Declaration of the
      directory \var{mkfli4l} will scan for package config files (default: config)} \\
    -x, -\,-check     & \multicolumn{2}{|p{11cm}|} { Declaration of the
      directory \var{mkfli4l} will scan for files needed for package error checking
      (\texttt{<package>.txt}, \texttt{<package>.exp} and
      \texttt{<package>.ext}; default: check)} \\
    -l, -\,-log       & \multicolumn{2}{|p{11cm}|} { Declaration of the log
      file to which \var{mkfli4l} will log error messages and warnings
      (default: \texttt{img/mkfli4l.log})} \\
    -p, -\,-package   & \multicolumn{2}{|p{11cm}|} { Declaration of the packages
      to be checked, this option may be used more than once in case of
      a desired check for several packages in conjunction. If using -p,
      however, the file \texttt{<check\_dir>/base.exp} will always be read
      first to provide the common regular expressions provided by the base
      package. Hence, this file must exist.} \\
    -i, -\,-info       & \multicolumn{2}{|p{11cm}|} { Provides information on
      the check (which files are read, which tests are run, which uncommon things
      happened during the review process)} \\
    -v, -\,-verbose    & \multicolumn{2}{|p{11cm}|} { More verbose
      variant of option -i} \\
    -h, -\,-help       & \multicolumn{2}{|p{11cm}|} { Displays the help text} \\
    \html{\multirow{8}{*}{top}} \latex{\multirow{8}{*}{}}{-d, -\,-debug} &
      \multicolumn{2}{|p{11cm}|} { Allows for debugging the review process. This
      is meant to be a help for package developers wishing to know in detail how
      the process of package checking is working.} \\
    \cline{2-3}
    \latex{&} \multicolumn{1}{|c|}{\textbf{Debug Option}} & \multicolumn{1}{c|}{\textbf{Meaning}} \\
    \cline{2-3}
    \latex{&} \multicolumn{1}{|l|}{check} & show check process \\
    \latex{&} \multicolumn{1}{|l|}{zip-list} & show generation of zip list \\
    \latex{&} \multicolumn{1}{|l|}{zip-list-skipped} & show skipped files \\
    \latex{&} \multicolumn{1}{|l|}{zip-list-regexp} & show regular expressions for ziplist \\
    \latex{&} \multicolumn{1}{|l|}{opt-files} & check all files in \texttt{opt/<package>.txt} \\
    \latex{&} \multicolumn{1}{|l|}{ext-trace} & show trace of extended checks \\
    \hline
  \end{tabular}
  \label{tab:mkfli4l}
\end{table}


\subsection{Structure}

A package can contain multiple OPTs, if it contains only one, however, it is
appropriate to name the package like the OPT. Below \texttt{<PACKAGE>} is
to be replaced by the respective package name.
A package consists of the following parts:

\begin{itemize}
\item Administrative Files
\item Documentation
\item Developer Documentation
\item Client Programs
\item Source Code
\item More Files
\end{itemize}

The individual parts are described in more detail below.

\subsection{Configuration of Packages}

The user's changes to the package's configuration are made in the
file \texttt{config/<PACKAGE>.txt}. All the OPT's variables should begin with
the name of the OPT, for example:

\begin{example}
\begin{verbatim}
    #-------------------------------------------------------------------
    # Optional package: TELNETD
    #-------------------------------------------------------------------
    OPT_TELNETD='no'        # install telnetd: yes or no
    TELNETD_PORT='23'       # telnet port, see also FIREWALL_DENY_PORT_x
\end{verbatim}
\end{example}

An OPT should be prefixed by a header in the configuration file (see
above). This increases readability, especially as a package indeed can
contain multiple OPTs. Variables associated to the OPT should~--- again
in the interest of readability~--- not be indented further. Comments and
blank lines are allowed, with comments always starting in column 33.
If a variable including its content has more than 32 characters, the
comment should be inserted with a row offset, starting in column 33.
Longer comments are spread over multiple lines, each starting at column
33. All this increases easy review of the configuration file.

All values following the equal sign must be enclosed in quotes\footnote{Both
single and double quotes are valid. You can hence write
\texttt{FOO='bar'} as well as \texttt{FOO="bar"}. The use of double quotes should be an
exception and you should previously inform about how an *nix shell uses single and
double quotes.}
not doing so can lead to problems during system boot.

    Activated variables (see below), will be transferred to \texttt{rc.cfg},
    everything else will be ignored. The only exceptions are variables by
    the name of \var{<PACKAGE>\_DO\_DEBUG}. These are used for debugging and
    are transferred as is.

\marklabel{sec:opt_txt}{
  \subsection{List of Files to Copy}
}

    The file \texttt{opt/<PACKAGE>.txt} contains instructions that describe
\begin{itemize}
\item which files are owned by the OPT,
\item the preconditions for inclusion in the \texttt{opt.img} resp. \texttt{rootfs.img},
\item what User ID (uid), Group ID (gid) and rights will be applied to files,
\item which conversions have to be made before inclusion in the archive.
\end{itemize}

Based on this information \var{mkfli4l} will generate the archives needed.

Blank lines and lines beginning with ``\#'' are ignored. In one of the first lines
the version of the package file format should be noted as follows:

\begin{example}
\begin{verbatim}
    <first column>      <second column> <third column>
    opt_format_version  1                    -
\end{verbatim}
\end{example}

    The remaining lines have the following syntax:

\begin{example}
\begin{verbatim}
    <first column>  <second column> <third column> <columns following>
    Variable        Value            File           Options
\end{verbatim}
\end{example}

    \begin{enumerate}
    \item
        The first column contains the name of a variable which triggers
        inclusion of the file referenced in the third column depending on
        its value in the package's config file. The name of a variable may
        appear in the first column as often as needed if multiple files depend
        on it. Any variable that appears in the file \texttt{opt/<PACKAGE>.txt}
        is marked by \var{mkfli4l}.

        If multiple variables should be tested for the same value a list of
        variables (separated by commas) can be used instead. It is sufficient
        in this case if at least \emph{one} variable contains the value required
        in the second column. It is important \emph{not} to use spaces between
        the individual variables!

        In OPT variables (ie variables that begin with \var{OPT\_} and typically have
        the type \var{YESNO}), the prefix ``\var{OPT\_}'' can be omitted. It does not
        matter whether variables are noted in upper- or lowercase (or mixed).

      \item The second column contains a value. If the variable in the first column
	is identical with this value and is activated too (see below), the file referenced
	in the third column will be included. If the first column contains a \%-variable
        it will be iterated over all indices and checked whether the respective
        variable matches the value. If this is the case copying will be executed.
        In addition, the copy process based on the current value of the variable
        will be logged.

        It is possible to write a ``!'' in front of the value. In this case,
        the test is negated, meaning the file is only copied if the variable
        does \emph{not} contain the value.

      \item  In the third column a file name is referenced. The path must be given
	relative to the \texttt{opt} directory. The file must exist and be readable,
	otherwise an error is raised while generating the boot medium and the build process
	is aborted.

        If the file name is prefixed with a ``\texttt{rootfs:}'' the file is included in the
        list of files to be copied to the RootFS. The prefix will be stripped before.

        If the file is located below the current configuration directory it is added
        to the list of files to be copied from there, otherwise the file found below
        opt is taken. Those files are not allowed to have a \texttt{rootfs:} prefix.

        If the file to copy is a kernel module the actual kernel version may be substituted
        by \var{\$\{KERNEL\_VERSION\}}. \var{mkfli4l} will then pick the version from the configuration
        and place it there. Using this you may provide modules for several kernel versions
        for the package and the module matching the current kernel version will be copied to
        the router.
        For kernel modules the path may be omitted, \var{mkfli4l} will find the module
        using \texttt{modules.dep} and \texttt{modules.alias}, see the section
        \jump{subsec:automatic-dependencies}{``Automatically Resolving Dependencies for Kernel Modules''}.

        \begin{table}[ht!]
          \centering
          \small
          \caption{Options for Files}
          \label{table:options}
          \begin{tabular}{|p{2.5cm}|p{7.5cm}|p{3.5cm}|}
            \hline
            Option & Meaning & Default Value \\
            \hline
            type= & Type of the Entry:\newline\newline
            \begin{tabular}{ll}
            \emph{local} & Filesystem Object\\
            \emph{file} & File\\
            \emph{dir} & Directory\\
            \emph{node} & Device\\
            \emph{symlink} & (symbolic) Link
            \end{tabular}\newline\newline
            This option has to be placed in front when given. The type
            ``local'' represents the type of an object existing in the file system
            and hence matches ``file'', ``dir'', ``node'' or ``symlink'' (depends).
            All other types except for ``file'' can create entries in the archive
            that do not have to exist in the local file system. This can i.e. be used
            to create devices files in the RootFS archive. & local \\
            uid= & The file owner, either numeric or as a name from passwd & root \\
            gid= & File group, either numeric or as a name from group & root \\
            mode= & Access rights &
            Files and Devices:\newline\verb?rw-r--r--? (644)\newline
            Directories:\newline\verb?rwxr-xr-x? (755)\newline
            Links:\newline
            \verb?rwxrwxrwx? (777)\newline\\
            flags=\newline
            (type=file) & Conversions before inclusion in the archive:\newline\newline
            \begin{tabular}{lp{6cm}}
            \emph{utxt} & Conversion to *nix format\\
            \emph{dtxt} & Conversion to DOS format\\
            \emph{sh}   & Shell script: Conversion to *nix format, stripping of superfluous chars\\
            \emph{luac} & Lua script: Translation into byte code of the Lua VM
            \end{tabular}
            & \\
            name= & Alternative name for inclusion of the entry in the archive &  \\
            devtype=\newline
            (type=node) & Descibes the type of the device (``c'' for character and ``b'' für block
            oriented devices). Has to be placed in second position. & \\
            major=\newline
            (type=node) & Decribes the so-called
            ``Major'' number of the device file. Has to be placed in third position. & \\
            minor=\newline
            (type=node) & Decribes the so-called
            ``Minor'' number of the device file. Has to be placed in fourth position. & \\
            linktarget=\newline
            (type=symlink) & Describes the target of the
            symbolic link. Has to be placed in second position. & \\
            \hline

          \end{tabular}
        \end{table}

      \item the other columns may contain the options for owner, group, rights for files and conversion
	listed in table~\ref{table:options}.

    \end{enumerate}

    Some examples:
    \begin{itemize}
    \item copy file if \verb+OPT_TELNETD='yes'+, set its
uid/gid to root and the rights to 755 (\verb?rwxr-xr-x?)

\begin{example}
\begin{verbatim}
    telnetd     yes    usr/sbin/in.telnetd mode=755
\end{verbatim}
\end{example}

    \item copy file, set
uid/gid to root, the rights to 555 (\verb?r-xr-xr-x?) and convert the
file to *nix format while stripping all superfluous chars

\begin{example}
\begin{verbatim}
    base    yes     etc/rc0.d/rc500.killall mode=555 flags=sh
\end{verbatim}
\end{example}

         \item copy file if \verb+PCMCIA_PCIC='i82365'+, set
uid/gid to root and the rights to 644 (\verb?rw-r--r--?)

\begin{example}
\begin{verbatim}
    pcmcia_pcic i82365 lib/modules/${KERNEL_VERSION}/pcmcia/i82365.ko
\end{verbatim}
\end{example}

         \item copy file if one of the \var{NET\_DRV\_\%} variables matches the second field,
set uid/gid to root and the rights to 644 (\verb?rw-r--r--?)

\begin{example}
\begin{verbatim}
    net_drv_%   3c503  3c503.ko
\end{verbatim}
\end{example}

        \item copy file if the variable \var{POWERMANAGEMENT} does \emph{not}
        contain the value ``none'':

        \begin{example}
\begin{verbatim}
    powermanagement !none etc/rc.d/rc100.pm mode=555 flags=sh
\end{verbatim}
\end{example}

        \item copy file if any of the OPT variables \var{OPT\_MYOPTA}
        or \var{OPT\_MYOPTB} contains the value ``yes'':

\begin{example}
\begin{verbatim}
    myopta,myoptb yes usr/local/bin/myopt-common.sh mode=555 flags=sh
\end{verbatim}
\end{example}

        This example is only an abbreviation for:

\begin{example}
\begin{verbatim}
    myopta yes usr/local/bin/myopt-common.sh mode=555 flags=sh
    myoptb yes usr/local/bin/myopt-common.sh mode=555 flags=sh
\end{verbatim}
\end{example}

        And the latter is a shorthand notation for:

\begin{example}
\begin{verbatim}
    opt_myopta yes usr/local/bin/myopt-common.sh mode=555 flags=sh
    opt_myoptb yes usr/local/bin/myopt-common.sh mode=555 flags=sh
\end{verbatim}
\end{example}

        \item copy file \texttt{opt/usr/bin/beep.sh} to the RootFS archive,
        but rename it to \texttt{bin/beep} before:

\begin{example}
\begin{verbatim}
    base yes rootfs:usr/bin/beep.sh mode=555 flags=sh name=bin/beep
\end{verbatim}
\end{example}

    \end{itemize}

    The files will be copied only if the above conditions are met and
    \verb+OPT_PACKAGE='yes'+ of the corresponding package is set. What OPT variable is
    referenced is decribed in the file \texttt{check/<PACKAGE>.txt}.

    If a variable is referenced in a package that is not defined by the package
    itself, it may happen that the corresponding package is not installed. This
    would result in an error message from \var{mkfli4l}, as it expects that all of
    the variables referenced by \texttt{opt/<PACKAGE>.txt} are defined.

    To handle this situation correctly the ``weak'' declaration has been introduced.
    It has the following format:

\begin{example}
\begin{verbatim}
    weak        <Variable>    -
\end{verbatim}
\end{example}

    By this the variable it is defined (if not already existing) and its value is
    set to ``undefined''. Please note that the prefix ``\var{OPT\_}" \emph{must} be
    provided (if existing) because else a variable \emph{without} this prefix will
    be created.

    An example taken from \texttt{opt/rrdtool.txt}:
\begin{example}
\begin{verbatim}
    weak opt_openvpn -
    [...]
    openvpn    yes    usr/lib/collectd/openvpn.so
\end{verbatim}
\end{example}

Without the \texttt{weak} definition \var{mkfli4l} would display an error message
when using the package ``rrdtool'' while the ``openvpn'' package is not activated.
By using the \texttt{weak} definition no error message is raised even in the case that
the ``openvpn'' package does not exist.

\marklabel{subsec:konfigspezdatei}{
\subsubsection{Files adapted by Configuration}
}

In some situations it is desired to replace original files with configuration-specific
files for inclusion in the archive, i.e. host keys, own firewall scripts, \ldots{}
\var{mkfli4l} supports this scenario by checking whether a file can be found in the
configuration directory and, if so, including this one instead in the file list for
\texttt{opt.img} resp.\ \texttt{rootfs.img}.

Another option to add configuration-specific files to an archive is decribed
in the section \jump{subsec:addtoopt}{Extended Checks of the Configuration}.


\marklabel{subsec:automatic-dependencies}{
\subsubsection{Automatically Resolving Dependencies for Kernel Modules}}
Kernel modules may depend on other kernel modules. Those must be loaded before
and therefore also have to be added to the archive. \var{mkfli4l} resolves this
dependencies based on \texttt{modules.dep} and \texttt{modules.alias}
(two files generated during the kernel build), automatically including
all required modules in the archives. Thus, for example the following
entry

\begin{example}
\begin{verbatim}
    net_drv_%   ne2k-pci    ne2k-pci.ko
\end{verbatim}
\end{example}

triggers that both 8390.ko and crc32.ko are included in the archive
because ne2k\_pci depends on both of them.

The necessary entries from \texttt{modules.dep} and \texttt{modules.alias}
are included in the RootFS and can be used by \texttt{modprobe} for
loading the drivers.

\marklabel{subsec:dev:var-check}{
\subsection{Checking Configuration Variables}
}

By the help of \texttt{check/<PACKAGE>.txt} the content of variables can be
checked for validity. In earlier version of the program \var{mkfli4l} this
check was hard coded there but it was outsourced to the check files in
the course of modularizing fli4l. This file contains a line for each
variable in the config files. These lines consist of four to five columns
which have the following functions:

\begin{enumerate}

\item Variable: this column specifies the name of the configuration file
  variable to check. If this is an \emph{array variable}, it can appear
  multiple times with different indices, so instead of the index number a percent
  sign (\%) is added to the variable name. It is always used as ``\var{\_\%\_}''
  in the middle of a name resp.\ ``\var{\_\%}'' at the end of a name. The name
  may contain more than one percent sign allowing the use of multidimensional
  arrays. It is recommended (but not mandatory) to add some text between
  the percent signs to avoid weird names like ``\var{FOO\_\%\_\_\%}''.

  Often the problem occurs that certain variables describe options that are
  needed only in some situations. Therefore variables may be marked as optional.
  Optional variables are identified by the prefix ``+''. They may then exist,
  but do not have to. Arrays can also use a ``++'' prefix. Prefixed with a ``+''
  the array can exist or be entirely absent.  Prefixed with a``++'' in addition
  some elements of the array may be missing.

\item \var{OPT\_\-VARIABLE}: This column assigns the variable to a specific OPT.
  The variable is checked for validity only if the OPT variable is set to ``yes''.
  If there is no OPT variable a ``-'' indicates this. In this case,
  the variable must be defined in the configuration file, unless a default
  value is defined (see below). The name of the OPT variable may be arbitrary
  but should start with the prefix ``\var{OPT\_}''.

  If a variable does not depend on any OPT variables, it is considered
  \emph{active}. If it is depending on an OPT variable, it is precisely
  active if

  \begin{itemize}
  \item its OPT variable is active and
  \item its OPT variable contains the value ``yes''.
  \end{itemize}

  In all other cases the variable is inactive.

  \textbf{Hint:} Inactive OPT variables will be set back to ``no'' by
  \var{mkfli4l} if set to ``yes'' in the configuration file, an appropriate
  warning will be generated then (i.e. \verb+OPT_Y='yes'+ is ignored, because
  \verb+OPT_X='no'+). For transitive dependency chains (\var{OPT\_Z} depends
  on \var{OPT\_Y} which in turn depends on \var{OPT\_X}) this will only work
  reliable, if the names of all OPT-variables start with ``\var{OPT\_}''.

\item \var{VARIABLE\_\-N}: If the first column contains a variable with a
  ``\%'' in its name, it indicates the number of occurrences of the variable
  (the so-called \emph{N-variable}). In case of a multi-dimensional variable,
  the occurences of the last index are specified. If the variable depends on a
  certain OPT, the N-variable must be dependant on the same or no OPT. If the
  variable does not depend on any OPT, the N-variable also shouldn't.
  If no N-variable exists, specify ``-'' to indicate that.

  For compatibility with future versions of fli4l the variable specified here
  \emph {must} be identical with the variable in \var{OPT\_VARIABLE} where the last
  ``\%'' is replaced by an ``N'' and everything following is removed.
  An array \var{HOST\_\%\_IP4} must have the N-Variable \var{HOST\_N} assigned
  and an array  \var{PF\_USR\_CHAIN\_\%\_RULE\_\%} hence the N-variable
  \var{PF\_USR\_CHAIN\_\%\_RULE\_N}, and this N-variable itself is an
  array variable with the corresponding N-variable \var{PF\_USR\_CHAIN\_N}.
  \emph{All other namings of the N variables will be incompatible with future
  versions of fli4l!}

\item \var{VALUE}: This column provides the values a variable can hold.
  For example the following settings are possible:

  \begin{tabular}[ht!]{|l|l|}
    \hline
    Name & Meaning \\
    \hline
    \hline
    \var{NONE}     &  No error checking will be done\\
    \var{YESNO}    &  The variable must be  ``yes'' or ``no''\\
    \var{NOTEMPTY} &  The variable can't be empty\\
    \var{NOBLANK}  &  The variable can't contain spaces\\
    \var{NUMERIC}  &  The variable must be numeric\\
    \var{IPADDR}   &  The variable must be an IP address\\
    \var{DIALMODE} &  The variable must be ``on'', ``off'' or ``auto''\\
    \hline
  \end{tabular}
  \\

  I values are prefixed by ``\var{WARN\_}'' an illegal content will not raise
  an error message and abort the build by \var{mkfli4l}, but only display a warning.

  The possible checks are defined by regular expressions in \texttt{check/base.exp}.
  This file may be extended and now contains some new checking routines, for example:
  \var{HEX}, \var{NUMHEX}, \var{IP\_ROUTE}, \var{DISK} and \var{PARTITION}.

  The number of expressions may be extended at any time for the future needs of
  package developers. Provide feedback!

  In addition, regular expressions can also be directly defined in the check-files,
  even relations to existing expressions can be made. Instead of \var{YESNO} you
  could, for example also write
\begin{example}
\begin{verbatim}
    RE:yes|no.
\end{verbatim}
\end{example}
This is useful if a test is performed only once and is relatively easy. For
more details see the next chapter.

\item Default Setting: In this column, an optional default value for the variables
can be defined in the case that the variable is not specified in the configuration file.

\textbf{Hint:} At the moment this does not work for array variables. Additionally,
the variable can't be optional (no ``+'' in front of the variable name).

Example:
\begin{example}
\begin{verbatim}
    OPT_TELNETD     -      -      YESNO    "no"
\end{verbatim}
\end{example}

If \var{OPT\_TELNETD} is missing in the config file, ``no'' will be assumed
and written as a value to \texttt{rc.cfg}.

\end{enumerate}

    The percent sign thingie is best decribed with an example. Let's assume
    \texttt{check/base.txt} amongst others has the following content:
\begin{example}
\begin{verbatim}
    NET_DRV_N          -                  -                  NUMERIC
    NET_DRV_%          -                  NET_DRV_N          NONE
    NET_DRV_%_OPTION   -                  NET_DRV_N          NONE
\end{verbatim}
\end{example}

      This means that depending on the value of \var{NET\_\-DRV\_\-N} the variables \var{NET\_\-DRV\_\-N},
      \var{NET\_\-DRV\_\-1\_\-OPTION}, \var{NET\_\-DRV\_\-2\_\-OPTION}, \var{NET\_\-DRV\_\-3\_\-OPTION}, a.s.o.
      will be checked.

\subsection{Own Definitions for Checking the Configuration Variables}

\subsubsection{Introduction of Regular Expressions}

  In version 2.0 only the above mentioned value ranges for variable checks existed:
  \var{NONE}, \var{NOTEMPTY}, \var{NUMERIC}, \var{IPADDR}, \var{YESNO}, \var{NOBLANK},
  \var{DIALMODE}. Checking was hard-coded to \var{mkfli4l}, not expandable and
  restricted to essential ``data types'' which could be evaluated with reasonable
  efforts.

  As of version 2.1 this checking has been reimplemented. The aim of the
  new implementation is a more flexible testing of variables, that is also
  able to examine more complex expressions. Therefore, regular expressions
  are used that can be stored in one or more separate files. This on one hand
  makes it possible to examine variables that are not checked for the moment and
  on the other hand, developers of optional packages can now define own terms in
  order to check the configuration of their packages.

  A description of regular expressions can be found via ``man 7 regex''
  or i.e. here:\\ \altlink{http://unixhelp.ed.ac.uk/CGI/man-cgi?regex+7}.

\subsubsection{Specification of Regular Expressions}

  Specification of regular expressions can be accomplished in two ways:

  \begin{enumerate}
  \item Package specific exp files \texttt{check/<PACKAGE>.exp}

    This file can be found in the \texttt{check} directory and has the same name
    as the package containing it, i.e. \texttt{check/base.exp}. It contains
    definitions for expressions that can be referenced in the file \texttt{check/<PACKAGE>.txt}.
    \texttt{check/base.exp} for example at the moment contains definitions for the
    known tests and \texttt{check/isdn.exp} a definition for the variable
    \var{ISDN\_\-CIRC\_\-x\_ROUTE} (the absence of this check was the trigger
    for the changes).

The syntax is as follows (again, double quotes can be used if needed):
\begin{example}
\begin{verbatim}
    <Name> = '<Regular Expression>' : '<Error Message>'
\end{verbatim}
\end{example}
as an example \texttt{check/base.exp}:
\begin{example}
\begin{verbatim}
    NOTEMPTY = '.*[^ ]+.*'          : 'should not be empty'
    YESNO    = 'yes|no'             : 'only yes or no are allowed'
    NUMERIC  = '0|[1-9][0-9]*'      : 'should be numeric (decimal)'
    OCTET    = '1?[0-9]?[0-9]|2[0-4][0-9]|25[0-5]'
             : 'should be a value between 0 and 255'
    IPADDR   = '((RE:OCTET)\.){3}(RE:OCTET)' : 'invalid ipv4 address'
    EIPADDR  = '()|(RE:IPADDR)'
             : 'should be empty or contain a valid ipv4 address'
    NOBLANK  = '[^ ]+'              : 'should not contain spaces'
    DIALMODE = 'auto|manual|off'    : 'only auto, manual or off are allowed'
    NETWORKS = '(RE:NETWORK)([[:space:]]+(RE:NETWORK))*'
             : 'no valid network specification, should be one or more
                network address(es) followed by a CIDR netmask,
                for instance 192.168.6.0/24'
\end{verbatim}
\end{example}

The regular expressions can also include already existing definitions by a reference.
These are then pasted to substitute the reference. This makes it easier to construct regular
expressions. The references are inserted by '(RE: Reference)'. (See the definition of the term
\var{NETWORKS} above for an appropriate example.)

The error messages tend to be too long. Therefore, they may be displayed on multiple lines.
The lines afterwards always have to start with a space or tab then. When reading the file
\texttt{check/<PACKAGE>.exp} superfluous whitespaces are reduced to one and tabs are replaced
by spaces. An entry in \texttt{check/<PACKAGE>.exp} could look like this:

\begin{example}
\begin{verbatim}
    NUM_HEX         = '0x[[:xdigit:]]+'
                    : 'should be a hexadecimal number
                       (a number starting with "0x")'
\end{verbatim}
\end{example}

\item  Regular expressions directly in the check file \texttt{check/<PACKAGE>.txt}

Some expressions occur but once and are not worth defining a regular expression in a
\texttt{check/<PACKAGE>.exp} file. You can simply write this expression to the check
file for example:

\begin{example}
\begin{verbatim}
    # Variable      OPT_VARIABLE    VARIABLE_N     VALUE
    MOUNT_BOOT      -               -              RE:ro|rw|no
\end{verbatim}
\end{example}

\var{MOUNT\_\-BOOT} can only take the value ``ro'', ``rw'' or ``no'',
everything else will be denied.

If you want to refer to existing regular expressions, simply add
a reference via `(RE:...)''. Example:

\begin{example}
\begin{verbatim}
    # Variable      OPT_VARIABLE    VARIABLE_N     VALUE
    LOGIP_LOGDIR    OPT_LOGIP       -              RE:(RE:ABS_PATH)|auto
\end{verbatim}
\end{example}

\end{enumerate}


\subsubsection{Expansion of Existing Regular Expressions}

If an optional package adds an additional value for a variable
which will be examined by a regular expression, then the regular
expression has to be expanded. This is done simply by defining the
new possible values by a regular expression (as described above) and
complement the existing regular expression in a separate \texttt{check/<PACKAGE>.exp}
file. That an existing expression is modified is indicated by a leading ``+''.
The new expression complements the existing expression by appending the new value
to the existing value(s) as an alternative. If another expression makes use of the
complemented expression, the supplement is also there. The specified error message
is simply appended to the end of the existing one.

Using the Ethernet driver as an example this could look like here:

\begin{itemize}
\item The base packages provides a lot of Ethernet drivers and checks
  the variable \var{NET\_DRV\_x} using the regular expression \var{NET\_DRV},
  which is defined as follows:

\begin{example}
\begin{verbatim}
    NET_DRV         = '3c503|3c505|3c507|...'
                    : 'invalid ethernet driver, please choose one'
                      ' of the drivers in config/base.txt'
\end{verbatim}
\end{example}
\item The package ``pcmcia'' provides additional device drivers,
  and hence has to complement \var{NET\_DRV}. This is done as follows:

\begin{example}
\begin{verbatim}
    PCMCIA_NET_DRV = 'pcnet_cs|xirc2ps_cs|3c574_cs|...' : ''
    +NET_DRV       = '(RE:PCMCIA_NET_DRV)' : ''
\end{verbatim}
\end{example}
\end{itemize}

Now PCMCIA drivers can be chosen in addition.

\subsubsection{Extend Regular Expressions in Relation to \var{YESNO} Variables}

If you have extended \var{NET\_DRV} with the PCMCIA drivers as shown above, but
the package ``pcmcia'' has been deactivated, you still could select a PCMCIA driver
in \texttt{config/base.txt} without an error message generated when creating the archives.
To prevent this, you may let the regular expression depend on a \var{YESNO} variable in the
configuration. For this purpose, the name of the variable that determines whether the
expression is extended is added with brackets immediately after the name of the expression.
If the variable is active and has the value ``yes'', the term is extended,
otherwise not.

\begin{example}
\begin{verbatim}
    PCMCIA_NET_DRV       = 'pcnet_cs|xirc2ps_cs|3c574_cs|...' : ''
    +NET_DRV(OPT_PCMCIA) = '(RE:PCMCIA_NET_DRV)' : ''
\end{verbatim}
\end{example}

If specifying \verb+OPT_PCMCIA='no'+  and using i.e. the PCMCIA driver
\texttt{xirc2ps\_cs} in\\ \texttt{config/base.txt}, an error message will
be generated during archive build.\\

\textbf{Hint:} This does \emph{not} work if the variable is not set
explicitely in the configuration file but gets its value by a default
setting in \texttt{check/<PACKAGE>.txt}. In this case the variable hence has
to be set explicitely and the default setting has to be avoided if necessary.

\marklabel{sec:regexp-dependencies}{
  \subsubsection{Extending Regular Expressions Depending on other Variables}
}

Alternatively, you may also use arbitrary values of variables as conditions,
the syntax looks like this:

\begin{example}
\begin{verbatim}
    +NET_DRV(KERNEL_VERSION=~'^3\.18\..*$') = ...
\end{verbatim}
\end{example}

If \var{KERNEL\_VERSION} matches the given regular expression (if any of the
kernels of the 3.18 line is used) then the list of network driver allowed is
extended with the drivers mentioned.\\

\textbf{Hint:} This does \emph{not} work if the variable is not set
explicitely in the configuration file but gets its value by a default
setting in \texttt{check/<PACKAGE>.txt}. In this case the variable hence has
to be set explicitely and the default setting has to be avoided if necessary.

\subsubsection{Error Messages}

If the checking process detects an error, an error message of the following kind
is displayed:

\begin{example}
\begin{verbatim}
    Error: wrong value of variable HOSTNAME: '' (may not be empty)
    Error: wrong value of variable MOUNT_OPT: 'rx' (user supplied regular expression)
\end{verbatim}
\end{example}

For the first error, the term was defined in a \texttt{check/<PACKAGE>.exp} file and
an explanation of the error is displayed. In the second case the term was
specified directly in a \texttt{check/<PACKAGE>.txt} file, so there is no
additional explanation of the error cause.

\subsubsection{Definition of Regular Expressions}

Regular expressions are defined as follows:

Regular expression: One or more alternatives, separated by
'$|$', i.e. ``ro$|$rw$|$no''. If one option matches, the whole term
matches (in this case ``ro'', ``rw'' and ``no'' are valid expressions).

An alternative is a concatenation of several sections that are simply added.

A section is an ``atom'', followed by a single ``*'', ``+'',
``?'' or ``\{min, max\}''. The meaning is as follows:
\begin{itemize}
\item ``a*''~--- as many ``a''s as wished (allows also no ``a'' is existing at all)
\item   ``a+''~--- at least one``a''
\item   ``a?''~--- none or one ``a''
\item   ``a\{2,5\}''~--- two to five ``a''s
\item   ``a\{5\}''~--- exactly five ``a''s
\item   ``a\{2,\}''~--- at least two ``a''s
\item   ``a\{,5\}''~--- a maximum of five ``a''s
\end{itemize}

An ``atom'' is a
\begin{itemize}
\item  regular expression enclosed in brackets, for example ``(a$|$b)+''
          matches any string containing at least one ``a'' or ``b'', up to an arbitrary
          number and in any order
        \item   an empty pair of brackets stands for an ``empty'' expression
        \item   an expression in square brackets ``[\,]'' (see below)
        \item a dot ``.'', matching an arbitrary character, for example
          a ``.+'' matches any string containing at least one char
        \item a ``\^\,'' represents the beginning of a line, for example a ``\^\,a.*''
          matches a string beginning with an ``a'' followed by any char like in
          ``a'' or ``adkadhashdkash''
        \item a ``\$'' represents the end of a line
        \item a ``$\backslash$'' followed by one of the special characters
          \texttt{\^\,.\,[\,\$\,(\,)\,$|$\,*\,+\,?\,\{\,$\backslash$} stands for the second char
          without its special meaning
        \item  a normal char matches exactly this char, for example
          ``a'' matches exactly an ``a''.
\end{itemize}

An expression in square brackets indicates the following:
\begin{itemize}
\item ``x-y''~--- matches any char inbetween ``x'' and ``y'', for example ``[0-9]''
                  matches all chars between ``0'' and ``9''; ``[a-zA-Z]'' symbolizes all chars,
                  either upper- or lowercase.

                \item ``\^\,x-y''~--- matches any char \emph{not} contained in the given interval,
                  for example ``[\^\,0-9]'' matches all chars \emph{except} for digits.

                \item ``[:\emph{character-class}:]''~--- matches a char from \emph{character-class}.
                Relevant standard character classes are: \texttt{alnum}, \texttt{alpha}, \texttt{blank},
                \texttt{digit}, \texttt{lower}, \texttt{print}, \texttt{punct}, \texttt{space}, \texttt{upper}
                and \texttt{xdigit}. I.e. ``[\,[:alpha:]\,]'' stands for all upper- or lowercase chars and hence
                is identical with ``[\,[:lower:]\,[:upper:]\,]''.
\end{itemize}


\subsubsection{Examples for regular Expressions}

Let's have a look at some examples!

\var{NUMERIC}: A numeric value consists of at least one, but otherwise
any number of digits. ``At least one'' is expressed with a ``+'', one digit
was already in an example above. So this results in:

\begin{example}
\begin{verbatim}
    NUMERIC = '[0-9]+'
\end{verbatim}
\end{example}
or alternatively
\begin{example}
\begin{verbatim}
    NUMERIC = '[[:digit:]]+'
\end{verbatim}
\end{example}

\var{NOBLANK}: A value that does not contain spaces, is any
char (except for the char ``space'') and any number of them:

\begin{example}
\begin{verbatim}
    NOBLANK = '[^ ]*'
\end{verbatim}
\end{example}

or, if the value is not allowed to be empty:

\begin{example}
\begin{verbatim}
    NOBLANK = '[^ ]+'
\end{verbatim}
\end{example}

\var{IPADDR}: Let's have a look at an example with an IP4-address. An
ipv4 address consists of four ``Octets'', divided by dots (``.''). An
octet is a number between 0 and 255. Let's define an octet at first.
It may be\\

\begin{tabular}[ht!]{lr}
  a number between 0 and 9: &       [0-9]\\
  a number between 10 and 99: &     [1-9][0-9]\\
  a number between 100 and 199:&   1[0-9][0-9]\\
  a number between 200 and 249: &  2[0-4][0-9]\\
  a number between 250 and 255: & 25[0-5]\\
\end{tabular}\\

All are alternatives hence we concatenate them with ``$|$'' forming
one expression: ``[0-9]$|$[1-9][0-9]$|$1[0-9][0-9]$|$2[0-4][0-9]$|$25[0-5]'' and
get an octet. Now we compose an IP4 address, four octets divided by dots (the dot
must be masked with a \emph{backslash}, because else it would represent an arbitrary
char). Based on the syntax of an exp-file it would look like this:

\begin{example}
\begin{verbatim}
    OCTET  = '[0-9]|[1-9][0-9]|1[0-9][0-9]|2[0-4][0-9]|25[0-5]'
    IPADDR = '((RE:OCTET)\.){3}(RE:OCTET)'
\end{verbatim}
\end{example}


\subsubsection{Assistance for the Design of Regular Expressions}

If you want to design and test regular expressions, you can use the
``regexp'' program located in the \texttt{unix} or \texttt{windows}
directory of the package ``base''. It accepts the following
syntax:

\begin{example}
\begin{verbatim}
    usage: regexp [-c <check dir>] <regexp> <string>
\end{verbatim}
\end{example}

The parameters explained in short:
\begin{itemize}
\item \texttt{<check dir>} is the directory containing check and exp files.
These are read by ``regexp'' to use expressions already defined there.

\item \texttt{<regexp>} is the regular expression (enclosed in \verb+'...'+
or \verb+"..."+ if in doubt, with double quotes needed only if single
quotes are used in the expression itself)


\item \texttt{<string>} is the string to be checked
\end{itemize}

This may for example look like here:
\begin{example}
\begin{verbatim}
./i586-linux-regexp -c ../check '[0-9]' 0
adding user defined regular expression='[0-9]' ('^([0-9])$')
checking '0' against regexp '[0-9]' ('^([0-9])$')
'[0-9]' matches '0'

./i586-linux-regexp -c ../check '[0-9]' a
adding user defined regular expression='[0-9]' ('^([0-9])$')
checking 'a' against regexp '[0-9]' ('^([0-9])$')
regex error 1 (No match) for value 'a' and regexp '[0-9]' ('^([0-9])$')

./i586-linux-regexp -c ../check IPADDR 192.168.0.1
using predefined regular expression from base.exp
adding IPADDR='((RE:OCTET)\.){3}(RE:OCTET)'
 ('^(((1?[0-9]?[0-9]|2[0-4][0-9]|25[0-5])\.){3}(1?[0-9]?[0-9]|2[0-4][0-9]|25[0-5]))$')
'IPADDR' matches '192.168.0.1'

./i586-linux-regexp -c ../check IPADDR 192.168.0.256
using predefined regular expression from base.exp
adding IPADDR='((RE:OCTET)\.){3}(RE:OCTET)'
 ('^(((1?[0-9]?[0-9]|2[0-4][0-9]|25[0-5])\.){3}(1?[0-9]?[0-9]|2[0-4][0-9]|25[0-5]))$')
regex error 1 (No match) for value '192.168.0.256' and regexp
 '((RE:OCTET)\.){3}(RE:OCTET)'
(unknown:-1) wrong value of variable cmd_var: '192.168.0.256' (invalid ipv4 address)
\end{verbatim}
\end{example}


\subsection{Extended Checks of the Configuration}

    Sometimes it is necessary to perform more complex checks.
    Examples of such complex things would be i.e. dependencies
    between packages or conditions that must be satisfied only
    when variables take certain values. For example if a PCMCIA
    ISDN adapter is used the package ``pcmcia'' has to be installed, too.

    In order to perform these checks you may write small tests to
    \texttt{check/<PACKAGE>.ext} (also called ext-script). The language
    consists of the following elements:

    \begin{enumerate}
    \item Keywords:

      \begin{itemize}
      \item Control Flow:

        \begin{itemize}
        \item \texttt{if (\textit{expr}) then \textit{statement} else \textit{statement} fi}
        \item \texttt{foreach \textit{var} in \textit{set\_var} do \textit{statement} done}
        \item \texttt{foreach \textit{var} in \textit{set\_var\_1 ... set\_var\_n} do \textit{statement} done}
        \item \texttt{foreach \textit{var} in \textit{var\_n} do \textit{statement} done}
        \end{itemize}

      \item
        Dependencies:
        \begin{itemize}
        \item \texttt{provides \textit{package} version \textit{x.y.z}}
        \item \texttt{depends on \textit{package} version \textit{x1.y1 x2.y2.z2 x3.y3 \ldots}}
        \end{itemize}

      \item Actions:
        \begin{itemize}
        \item \texttt{warning "\textit{warning}"}
        \item \texttt{error   "\textit{error}"}
        \item \texttt{fatal\_error "\textit{fatal error}"}
        \item \texttt{set \textit{var} = \textit{value}}
        \item \texttt{crypt (\textit{variable})}
        \item \texttt{stat (\textit{filename}, \textit{res})}
        \item \texttt{fgrep (\textit{filename}, \textit{regex})}
        \item \texttt{split (\textit{string}, \textit{set\_variable}, \textit{character})}
        \end{itemize}
      \end{itemize}
    \item Data Types:      strings, positive integers, version numbers
    \item Logical Operations:    \texttt{<}, \texttt{==}, \texttt{>}, \texttt{!=}, \texttt{!}, \texttt{\&\&}, \texttt{||},
      \texttt{=}\verb+~+, \texttt{copy\_pending}, \texttt{samenet}, \texttt{subnet}
    \end{enumerate}

\marklabel{subsec:dev:data-types}{
\subsubsection{Data Types}
}

    Concerning data types please note that variables, based on the associated
    regular expression are permanently assigned to a data type:

\begin{itemize}
\item Variables, starting with type ``\var{NUM}'' are numeric and
      contain positive integers
\item Variables representing an N-variable for any kind of array are numeric as well
\item all other variables are treated as strings
\end{itemize}

    This means, among other things, that a variable of type \var{ENUMERIC}
    can \emph{not} be used as an index when accessing an array variable, even if
    you have checked at first that it is not empty.
    The following code thus does not work as expected:
\begin{example}
\begin{verbatim}
    # TEST should be a variable of type ENUMERIC
    if (test != "")
    then
        # Error: You can't use a non-numeric ID in a numeric
        #         context. Check type of operand.
        set i=my_array[test]
        # Error: You can't use a non-numeric ID in a numeric
        #         context. Check type of operand.
        set j=test+2
    fi
\end{verbatim}
\end{example}

    A solution for this problem is offered by \jump{subsec:split}{\texttt{split}}:
\begin{example}
\begin{verbatim}
    if (test != "")
    then
        # all elemente of test_% are numeric
        split(test, test_%, ' ', numeric)
        # OK
        set i=my_array[test_%[1]]
        # OK
        set j=test_%[1]+2
    fi
\end{verbatim}
\end{example}

\marklabel{subsec:dev:string-rewrite}{
\subsubsection{Substitution of Strings and Variables}
}

    At various points strings are needed, such as when a
    \jump{subsec:dev:print}{Warning} should be issued. In some cases
    described in this documentation, such a string is scanned for variables.
    If found, these are \emph{replaced} by their contents or other attributes.
    This replacement is called \emph{variable substitution}.

    This will be illustrated by an example. Assume this configuration:

\begin{example}
\begin{verbatim}
    # config/base.txt
    HOSTNAME='fli4l'
    # config/dns_dhcp.txt
    HOST_N='1' # Number of hosts
    HOST_1_NAME='client'
    HOST_1_IP4='192.168.1.1'
\end{verbatim}
\end{example}

    Then the character strings are rewritten as follows, if
    variable substitution is active in this context:

\begin{example}
\begin{verbatim}
    "My router is called $HOSTNAME"
    # --> "My router is called fli4l"
    "HOSTNAME is part of the package %{HOSTNAME}"
    # --> "HOSTNAME is part of the package base"
    "@HOST_N is $HOST_N"
    # --> " # Number of hosts is 1"
\end{verbatim}
\end{example}

    As you can see, there are basically three options for replacement:
    \begin{itemize}
    \item \texttt{\$<Name>} resp.\ \texttt{\$\{<Name>\}}: Replaces the
          variable name with the contents of the variable. This is the most common
          form of replacement. The name must be enclosed in \texttt{\{...\}} if in
          the string it is directly followed by a char that may be a valid part of
          a variable name (a letter, a digit, or an underscore). In all other cases,
          the use of curly brackets is possible, but not mandatory.

    \item \texttt{\%<Name>} resp.\ \texttt{\%\{<Name>\}}: Replaces the variable name
	  with the name of the package in which the variable is defined. This does
	  \emph{not} work with variables assigned in the script via
          \jump{subsec:dev:assignment}{\texttt{set}} or with counting
          variables of a \jump{subsec:dev:control}{\texttt{foreach}-loop}
          since such variables do not have a package and their syntax is different.

    \item \texttt{@<Name>} resp.\ \texttt{@\{<Name>\}}: Replaces the
          variable name with the comment noted in the configuration after the variable.
          Again, this does not make sense for variables defined by the script.
    \end{itemize}

    \textbf{Hint:} Elements of array variables can \emph{not} be integrated into
    strings this way, because there is no possibility to provide an index.

    In general, only \emph{constants} can be used for variable substitution,
    strings that come from a variable remain unchanged. An example will make
    this clear - assume the following configuration:

\begin{example}
\begin{verbatim}
    HOSTNAME='fli4l'
    TEST='${HOSTNAME}'
\end{verbatim}
\end{example}

    This code:

\begin{example}
\begin{verbatim}
    warning "${TEST}"
\end{verbatim}
\end{example}

    leads to the following output:

\begin{example}
\begin{verbatim}
    Warning: ${HOSTNAME}
\end{verbatim}
\end{example}

    It will \emph{not} display:

\begin{example}
\begin{verbatim}
    Warning: fli4l
\end{verbatim}
\end{example}

    In the following sections it will be explicitly noted under which conditions
    strings are subject of variable substitution.

\subsubsection{Definition of a Service with an associated
    Version Number: \texttt{provides}}

    For instance, an OPT may declare that it provides a Printer service or a
    Webserver service. Only one package can provide a certain service.
    This prevents i.e. that two web servers are installed in parallel, which
    is not possible for obvious reasons, since the two servers would both
    register port 80. In addition, the current version of the service is
    provided so that updates can be triggered. The version number consists
    of two or three numbers separated by dots, such as ``4.0'' or ``2.1.23''.

    Services typically originate from OPTs, not from packages. For example
    the package ``tools'' has a number of programs that each have their own
    \texttt{provides} statement defined if activated by \verb+OPT_...='yes'+.

    The syntax is as follows:

\begin{example}
\begin{verbatim}
    provides <Name> version <Version>
\end{verbatim}
\end{example}

    Example from package ``easycron'':

\begin{example}
\begin{verbatim}
    provides cron version 3.10.0
\end{verbatim}
\end{example}

    The version number should be incemented by the OPT-developer in the third
    component, if only functional enhancements have been made and the OPT's
    interface is still. The version number should be increased in the first or
    second component when the interface has changed in any incompatible way (eg.
    due to variable renaming, path changes, missing or renamed utilities, etc.).

\subsubsection{Definition of a Dependency to a Service with a
    specific Version: \texttt{depends}}

    If another service is needed to provide the own function (eg. a web server)
    this dependency to a specific version may be defined here. The version can
    be given with two (i.e. ``2.1'') or three numbers (i.e. ``2.1.11'')
    while the two-number version accepts all versions starting with this
    number and the three-number version only accepting just the specified one.
    A list of version numbers may also be specified if multiple versions of
    the service are compatible with the package.

    The syntax is as follows:

\begin{example}
\begin{verbatim}
    depends on <Name> version <Version>+
\end{verbatim}
\end{example}

    An example: Package ``server'' contains:
\begin{example}
\begin{verbatim}
    provides server version 1.0.1
\end{verbatim}
\end{example}

    A Package ``client'' with the following
    \texttt{depends}-instruction is given:\footnote{Of course only
    one at a time!}

\begin{example}
\begin{verbatim}
    depends on server version 1.0       # OK, '1.0' matches '1.0.1'
    depends on server version 1.0.1     # OK, '1.0.1' matches '1.0.1'
    depends on server version 1.0.2     # Error, '1.0.2' does not match with '1.0.1'
    depends on server version 1.1       # Error, '1.1' does not match with '1.0.1'
    depends on server version 1.0 1.1   # OK, '1.0' matches '1.0.1'
    depends on server version 1.0.2 1.1 # Error, neither '1.0.2' nor '1.1' are matching
                                        # '1.0.1'
\end{verbatim}
\end{example}

\marklabel{subsec:dev:print}{
\subsubsection{Communication with the User: \texttt{warning}, \texttt{error}, \texttt{fatal\_error}}
}

    Using these three functions users may be warned, signalized an
    errors or stop the test immediately. The syntax is as follows:

    \begin{itemize}
    \item \verb+warning "text"+
    \item \verb+error "text"+
    \item \verb+fatal_error "text"+
    \end{itemize}

    All strings passed to these funtions are subject of
    \jump{subsec:dev:string-rewrite}{variable substitution}.

\marklabel{subsec:dev:assignment}{
\subsubsection{Assignments}
}

    If for any reason a temporary variable is required it can be created by
    ``\texttt{set var [= value]}''. \emph{The variable can not be a configuration
    variable!} \footnote{This is a desired restriction: Check scripts are \emph{not}
    able to change the user configuration.} If you omit the ``= value'' part the
    variable is simply set to ``yes'' so it may be tested in an \texttt{if}-statement.
    If an assignment part is given, anything may be specified after the equal sign:
    normal variables, indexed variables, numbers, strings and version numbers.

    Please note that by the assignment also the \emph{type} of the temporary variable
    is defined. If a number is assigned \var{mkfli4l}  ``remembers'' that the variable
    contains a number and later on allows calculations with it. Trying to do
    calculations with variables of other types will fail.\\ Example:

\begin{example}
\begin{verbatim}
    set i=1   # OK, i is a numeric variable
    set j=i+1 # OK, j is a numeric variable and contains the value 2
    set i="1" # OK, i now is a string variable
    set j=i+1 # Error "You can't use a non-numeric ID in a numeric
              #         context. Check type of operand."
              # --> no calculations with strings!
\end{verbatim}
\end{example}

    You may also define temporary arrays (see below). Example:

\begin{example}
\begin{verbatim}
    set prim_%[1]=2
    set prim_%[2]=3
    set prim_%[3]=5
    warning "${prim_n}"
\end{verbatim}
\end{example}

    The number of array elements is kept by \var{mkfli4l} in the variable
    \var{prim\_n}. The code above hence leads to the following output:

\begin{example}
\begin{verbatim}
    Warning: 3
\end{verbatim}
\end{example}

    If the right side of an assignment is a string constant, it is subject of
    \jump{subsec:dev:string-rewrite}{variable substitution} at the time of
    assignment. The following example demonstrates this. The code:

\begin{example}
\begin{verbatim}
    set s="a"
    set v1="$s" # v1="a"
    set s="b"
    set v2="$s" # v2="b"
    if (v1 == v2)
    then
      warning "equal"
    else
      warning "not equal"
    fi
\end{verbatim}
\end{example}

    will output ``not equal'', because the variables \var{v1} and \var{v2}
    replace the content of the variable \var{s} already at the time of assignment.\\

    \textbf{Hint:} A variable set in a script is visible while processing further
    scripts ~-- currently there exists no such thing as local variables. Since the
    order of processing scripts of different packages is not defined, you should
    never rely on any variable having values defined by another package.

\subsubsection{Arrays}

    If you want to access elements of a \%-variable (of an array) you have to use
    the original name of the variable like mentioned in the file \texttt{check/<PACKAGE>.txt}
    and add an index for each ``\%'' sign by using ``[\emph{Index}]''.

    Example: If you want to access the elements of variable
    \var{PF\_USR\_CHAIN\_\%\_RULE\_\%} you need two indices because the
    variable has two ``\%'' signs. All elements may be printed for example
    using the following code (the \texttt{foreach}-loop
    is exlained in \jump{subsec:dev:control}{see below}):

\begin{example}
\begin{verbatim}
    foreach i in pf_usr_chain_n
    do
        # only one index needed, only one '%' in the variable's name
        set j_n=pf_usr_chain_%_rule_n[i]
        # Attention: a
        # foreach j in pf_usr_chain_%_rule_n[i]
        # is not possible, hence the use of j_n!
        foreach j in j_n
        do
            # two indices needed, two '%' in the variable's name
            set rule=pf_usr_chain_%_rule_%[i][j]
            warning "Rule $i/$j: ${rule}"
        done
    done
\end{verbatim}
\end{example}

    With this sample configuration

\begin{example}
\begin{verbatim}
    PF_USR_CHAIN_N='2'
    PF_USR_CHAIN_1_NAME='usr-chain_a'
    PF_USR_CHAIN_1_RULE_N='2'
    PF_USR_CHAIN_1_RULE_1='ACCEPT'
    PF_USR_CHAIN_1_RULE_2='REJECT'
    PF_USR_CHAIN_2_NAME='usr-chain_b'
    PF_USR_CHAIN_2_RULE_N='1'
    PF_USR_CHAIN_2_RULE_1='DROP'
\end{verbatim}
\end{example}

    the following output is printed:

\begin{example}
\begin{verbatim}
    Warning: Rule 1/1: ACCEPT
    Warning: Rule 1/2: REJECT
    Warning: Rule 2/1: DROP
\end{verbatim}
\end{example}

    Alternatively, you can iterate directly over all values of the array
    (but the exact indices of the entries are not always known, because this is not
    required):

\begin{example}
\begin{verbatim}
    foreach rule in pf_usr_chain_%_rule_%
    do
        warning "Rule %{rule}='${rule}'"
    done
\end{verbatim}
\end{example}

    That produces the following output with the sample configuration from above:

\begin{example}
\begin{verbatim}
    Warning: Rule PF_USR_CHAIN_1_RULE_1='ACCEPT'
    Warning: Rule PF_USR_CHAIN_1_RULE_2='REJECT'
    Warning: Rule PF_USR_CHAIN_2_RULE_1='DROP'
\end{verbatim}
\end{example}

    The second example nicely shows the meaning of the
    \texttt{\%{<Name>}}-syntax: Within the string
    \texttt{\%{rule}} is substitued by the \emph{name} of the variable in question
    (for example \var{PF\_USR\_CHAIN\_1\_RULE\_1}), while \texttt{\${rule}}
    is substituted by its \emph{content} (i.e. \var{ACCEPT}).

\subsubsection{Encryption of Passwords: \texttt{crypt}}

Some variables contain passswords that should not be noted in plain text in
\texttt{rc.cfg}. These variables can be encrypted by the use of \texttt{crypt}
and are transferred to a format also needed on the router. Use this like here:

\begin{example}
\begin{verbatim}
    crypt (<Variable>)
\end{verbatim}
\end{example}

The \texttt{crypt} function is the \emph{only} point at which a configuration
variable can be changed.

\marklabel{subsec:statdatei}{
\subsubsection{Querying File Properties: \texttt{stat}}
}

    \texttt{stat} is used to query file properties. At the moment only file
    size can be accessed. If checking for files under the current configuration
    directory you may use the internal variable \var{config\_dir}. The Syntax:

\begin{example}
\begin{verbatim}
    stat (<file name>, <key>)
\end{verbatim}
\end{example}

    The command looks like this (the
    parameters used are only examples):

\begin{example}
\begin{verbatim}
    foreach i in openvpn_%_secret
    do
       stat("${config_dir}/etc/openvpn/$i.secret", keyfile)
       if (keyfile_res != "OK")
       then
          error "OpenVPN: missing secretfile <config>/etc/openvpn/$i.secret"
       fi
    done
\end{verbatim}
\end{example}

    The example checks whether a file exists in the current configuration directory.\\
    If \verb+OPENVPN_1_SECRET='test'+ is set in the configuration file, the loop
    in the first run checks for the existence of the file \texttt{etc/openvpn/test.secret}
    in the current configuration directory.

    After the call two variables are defined:

    \begin{itemize}
    \item \texttt{<Key>\_res}: Result of the system call stat() (``OK'', if
      system call was successful, else the error message of the system call)
    \item \texttt{<Key>\_size}: File size
    \end{itemize}

    It may for example look like this:

\begin{example}
\begin{verbatim}
    stat ("unix/Makefile", test)
    if ("$test_res" == "OK")
    then
            warning "test_size = $test_size"
    else
            error "Error '$test_res' while trying to get size of 'unix/Makefile'"
    fi
\end{verbatim}
\end{example}

    A file name passed as a string constant is subject of
    \jump{subsec:dev:string-rewrite}{variable substitution}.

\marklabel{subsec:fgrepdatei}{
\subsubsection{Search files: \texttt{fgrep}}
}

    If you wish to search a file via ``grep''\footnote{``grep'' is a common
    command on *nix-like OSes for filtering text streams.}
    you may use the \texttt{fgrep} command. The syntax is:

\begin{example}
\begin{verbatim}
    fgrep (<File name>, <RegEx>)
\end{verbatim}
\end{example}

    If the file \texttt{<File name>} does not exist \var{mkfli4l}
    will abort with a fatal error! If it is not sure if the file exists,
    test this before with \texttt{stat}. After calling \texttt{fgrep} the
    search result is present in an array called \var{FGREP\_MATCH\_\%}, with
    its index \emph{x} as usual ranging from one to \var{FGREP\_MATCH\_N}.
    \var{FGREP\_MATCH\_1} points to the whole range of the line the regular
    expression has matched, while \var{FGREP\_MATCH\_2} to \var{FGREP\_MATCH\_N}
    contain the \emph{n-1} th part in brackets.

    A first example will illustrate the use. The file
    \texttt{opt/etc/shells} contains the line:

\begin{example}
\begin{verbatim}
/bin/sh
\end{verbatim}
\end{example}

    The following code

\begin{example}
\begin{verbatim}
    fgrep("opt/etc/shells", "^/(.)(.*)/")
    foreach v in FGREP_MATCH_%
    do
      warning "%v='$v'"
    done
\end{verbatim}
\end{example}

    produces this output:

\begin{example}
\begin{verbatim}
    Warning: FGREP_MATCH_1='/bin/'
    Warning: FGREP_MATCH_2='b'
    Warning: FGREP_MATCH_3='in'
\end{verbatim}
\end{example}

    The RegEx has (only) matched with ``/bin/'' (only this part of the
    line is contained in the variable \var{FGREP\_MATCH\_1}). The first bracketed
    part in the expression only matches the first char after the first ``/'',
    this is why only ``b'' is contained in \var{FGREP\_MATCH\_2}. The second
    bracketed part contains the rest after ``b'' up to the last ``/'',
    hence ``in'' is noted in variable \var{FGREP\_MATCH\_3}.

    The following second example demonstrates an usual use of \texttt{fgrep}
    taken from \texttt{check/base.ext}. It will be tested if all \texttt{tmpl:}-references
    given in \var{PF\_FORWARD\_x} are really present.

\begin{example}
\begin{verbatim}
    foreach n in pf_forward_n
    do
      set rule=pf_forward_%[n]
      if (rule =~ "tmpl:([^[:space:]]+)")
      then
        foreach m in match_%
        do
          stat("$config_dir/etc/fwrules.tmpl/$m", tmplfile)
          if(tmplfile_res == "OK")
          then
            add_to_opt "etc/fwrules.tmpl/$m"
          else
            stat("opt/etc/fwrules.tmpl/$m", tmplfile)
            if(tmplfile_res == "OK")
            then
              add_to_opt "etc/fwrules.tmpl/$m"
            else
              fgrep("opt/etc/fwrules.tmpl/templates", "^$m[[:space:]]+")
              if (fgrep_match_n == 0)
              then
                error "Can't find tmpl:$m for PF_FORWARD_${n}='$rule'!"
              fi
            fi
          fi
        done
      fi
    done
\end{verbatim}
\end{example}

    Both a filename value as well as a regular expression passed as a string constant are subject to
    \jump{subsec:dev:string-rewrite}{variable substitution}.

\marklabel{subsec:split}{
\subsubsection{Splitting Parameters: \texttt{split}}
}

    Often variables can be assigned with several parameters, which then
    have to be split apart again in the startup scripts. If it is desired
    to split these previously and perform tests on them \texttt{split}
    can be used. The syntax is like this:

\begin{example}
\begin{verbatim}
    split (<String>, <Array>, <Separator>)
\end{verbatim}
\end{example}

    The string can be specified by a variable or directly as a
    constant. \var{mkfli4l} splits it where a separator is found
    and generates an element of the array for each part. You may iterate
    over these elements later on and perform tests. If nothing is found between
    two separators an array element with an empty string as its value is created.
    The exception is `` '': Here all spaces are deleted and no empty variable is
    created.

    If the elements generated by such a split should be in a numeric context (e.g.
    as indices) this has to be specified when calling \texttt{split}. This is done by
    the additional attribute ``numeric''. Such a call looks as follows:

\begin{example}
\begin{verbatim}
    split (<String>, <Array>, <Separator>, numeric)
\end{verbatim}
\end{example}

   An example:

\begin{example}
\begin{verbatim}
    set bar="1.2.3.4"
    split (bar, tmp_%, '.', numeric)
    foreach i in tmp_%
    do
            warning "%i = $i"
    done
\end{verbatim}
\end{example}

    the output looks like this:

\begin{example}
\begin{verbatim}
    Warning: TMP_1 = 1
    Warning: TMP_2 = 2
    Warning: TMP_3 = 3
    Warning: TMP_4 = 4
\end{verbatim}
\end{example}

    \textbf{Hint:} If using the ``numeric'' variant \var{mkfli4l} will \emph{not}
    check the generated string parts for really being numeric! If you use such a
    non-numeric construct later in a numeric context (i.e. in an addtion) \var{mkfli4l}
    will raise a fatal error. Example:

\begin{example}
\begin{verbatim}
    set bar="a.b.c.d"
    split (bar, tmp_%, '.', numeric)
    # Error: invalid number 'a'
    set i=tmp_%[1]+1
\end{verbatim}
\end{example}

    A string constant passed to \texttt{split} in the first parameter is subject of
    \jump{subsec:dev:string-rewrite}{variable substitution}.

\marklabel{subsec:addtoopt}{
\subsubsection{Adding Files to the Archives: \texttt{add\_to\_opt}}
}

    The function \texttt{add\_to\_opt} can add additional files to the
    Opt- or RootFS-Archives. \emph{All} files under \texttt{opt/} or from
    the configuration directory may be chosen. There is no limitation to only
    files from a specific package. If a file is found under \texttt{opt/} as
    well as in the configuration directory, \texttt{add\_to\_opt} will prefer
    the latter. The function \texttt{add\_to\_opt} is typically used if complex
    logical rules decide if and what files have to be included in the archives.\\

    The syntax looks like this:
\begin{example}
\begin{verbatim}
    add_to_opt <File> [<Flags>]
\end{verbatim}
\end{example}

    Flags are optional. The defaults from table ~\ref{table:options}
    are used if no flags are given.

    See an example from the package ``sshd'':

\begin{example}
\begin{verbatim}
    if (opt_sshd)
    then
       foreach pkf in sshd_public_keyfile_%
       do
         stat("$config_dir/etc/ssh/$pkf", publickeyfile)
         if(publickeyfile_res == "OK")
         then
             add_to_opt "etc/ssh/$pkf" "mode=400 flags=utxt"
         else
             error "sshd: missing public keyfile %pkf=$pkf"
         fi
       done
    fi
\end{verbatim}
\end{example}

    \jump{subsec:statdatei}{\texttt{stat}} at first checks for the file existing
    in the configuration directory. If it is, it will be included in the archive,
    if not, \var{mkfli4l} will abort with an error message.\\

    \textbf{Hint:} Also for \texttt{add\_to\_opt} \var{mkfli4l} will first
    \jump{subsec:konfigspezdatei}{check} if the file to be copied can be found
    in the configuration directory.\\

    Filenames as well flags passed as string constants are subject of
    \jump{subsec:dev:string-rewrite}{variable substitution}.

\marklabel{subsec:dev:control}{
\subsubsection{Control Flow}
}

\begin{example}
\begin{verbatim}
    if (expr)
    then
            statement
    else
            statement
    fi
\end{verbatim}
\end{example}

    A classic case distinction, as we know it. If the condition is true,
    the \texttt{then} part is executed, if the condition is wrong the \texttt{else} part.

    If you want to run tests on array variables, you have to test every single
    variable. The \texttt{foreach} loop in two variants for this.

    \begin{enumerate}
    \item Iterate over array variables:

\begin{example}
\begin{verbatim}
    foreach <control variable> in <array variable>
    do
            <instruction>
    done

    foreach <control variable> in <array variable-1> <array variable-2> ...
    do
            <instruction>
    done
\end{verbatim}
\end{example}

    This loop iterates over all of the specified array variables, each
    starting with the first to the last element, the number of elements
    in this array is taken from the N-variable associated with this array.
    The control variable takes the values of the respective array variables.
    It should be noted that when processing optional array variables that are
    not present in the configuration, an empty element is generated. You may
    have to take this into account in the script, for example like this:

\begin{example}
\begin{verbatim}
    foreach i in template_var_opt_%
    do
        if (i != "")
        then
            warning "%i is present (%i='$i')"
        else
            warning "%i is undefined (empty)"
        fi
    done
\end{verbatim}
\end{example}

    As you also can see in the example, the \emph{name} of the respective
    array variables can be determined with the \texttt{\%<control variable>}
    construction.

    The instruction in the loop may be one of the above
    control elements or functions (\texttt{if}, \texttt{foreach},
    \texttt{provides}, \texttt{depends}, \ldots).

    If you want to access exactly one element of an array, you can address it
    using the syntax \texttt{<Array>[<Index>]}. The index can be a normal variable,
    a numeric constant or again an indexed array.

    \item Iteration over N-variables:

\begin{example}
\begin{verbatim}
    foreach <control variable> in <N-variable>
    do
            <instruction>
    done
\end{verbatim}
\end{example}

    This loop executes from 1 to the value that is given in the N-variable. You can
    use the control variable to index array variables. So if you want to iterate over
    not only one but more array variables at the same time all controlled by the
    \emph{same} N-variable you take this variant of the loop and use the
    control variable for indexing multiple array variables. Example:

\begin{example}
\begin{verbatim}
    foreach i in host_n
    do
        set name=host_%_name[i]
        set ip4=host_%_ip4[i]
        warning "$i: name=$name ip4=$ip4"
    done
\end{verbatim}
\end{example}

    The resulting content of the \var{HOST\_\%\_NAME}- and \var{HOST\_\%\_IP4}-arrays
    for this example:

\begin{example}
\begin{verbatim}
    Warning: 1: name=berry ip4=192.168.11.226
    Warning: 2: name=fence ip4=192.168.11.254
    Warning: 3: name=sandbox ip4=192.168.12.254
\end{verbatim}
\end{example}

    \end{enumerate}

\subsubsection{Expressions}

    Expressions link values and operators to a new value. Such a value
    can be an normal variable, an array element, or a constant (Number,
    string or version number). All string constants in expressions are
    subject to \jump{subsec:dev:string-rewrite}{variable substitution}.

    Operators allow just about everything you could want from a
    programming language. A test for the equality of two variables
    could look like this:

\begin{example}
\begin{verbatim}
    var1 == var2
    "$var1" == "$var2"
\end{verbatim}
\end{example}

    It should be noted that the comparison is done depending on the type
    that was defined for the variable in \texttt{check/<PACKAGE>.txt}.
    If one of the two variables is \jump{subsec:dev:data-types}{numeric}
    the comparison is made numeric-based, meaning that the strings are
    converted to numbers and then compared. Otherwise, the comparison is done
    string-based; comparing \texttt{"05"\ == "5"} gives the result ``false'',
    a comparison \texttt{"18"\ < "9"} ``true'' due to the lexicographical string
    order: the digit ``1'' precedes the digit  ``9'' in the ASCII character set.

    For the comparison of version numbers the construct \texttt{numeric(version)}
    is introduced, which generates the numeric value of a version number for
    comparison purposes. Here applies:


\begin{example}
\begin{verbatim}
    numeric(version) := major * 10000 + minor * 1000 + sub
\end{verbatim}
\end{example}

    whereas ``major'' is the first component of the version number, ``minor'' the
    second and ``sub'' the third. If ``sub'' is missing the term in the addition above
    is omitted (in other words ``sub'' will be equalled to zero).

    A complete list of all expressions can be found in table \ref{tab:expr}. ``val''
    stands for any value of any type, ``number'' for a numeric value and ``string''for
    a string.

    \begin{table}[htb]
      \centering
      \caption{Logical Epressions}
      \label{tab:expr}
      \begin{tabular}{ll}
        \hline
        Expression &                     true if\\
        \hline
        \hline
       id                    &    id == ``yes''\\
       val  == val           &    values of identical type are equal\\
       val  != val           &    values of identical type are unequal\\
       val  == number        &    numeric value of val == number\\
       val  != number        &    numeric value of val != number\\
       val  $<$  number      &    numeric value of val $<$ number\\
       val  $>$  number      &    numeric value of val $>$ number\\
       val  == version       &    numeric(val) == numeric(version) \\
       val  $<$  version     &    numeric(val) $<$  numeric(version) \\
       val  $>$  version     &    numeric(val) $>$  numeric(version) \\
       val  =\verb?~? string &    regular expression in string matches val\\
       ( expr )              &    Expression in brackets is true\\
       expr \&\& expr        &    both expressions are true\\
       expr || expr          &    at least one of both expressions is true\\
       copy\_pending(id)     &    see description\\
       samenet (string1, string2) & string1 describes the same net as string2\\
       subnet (string1, string2)  & string1 describes a subnet of string2\\
        \hline
      \end{tabular}
    \end{table}

\subsubsection{Match-Operator}

With the match operator \verb?=~? you can check whether a regular
expression matches the value of a variable. Furthermore, one can
also use the operator to extract subexpressions from a variable.
After successfully applying a regular expression on a variable
the array \var{MATCH\_\%} contains the parts found. May look
like this:

\begin{example}
\begin{verbatim}
    set foo="foobar12"
    if ( foo =~ "(foo)(bar)([0-9]*)" )
    then
            foreach i in match_%
            do
                    warning "match %i: $i"
            done
    fi
\end{verbatim}
\end{example}

Calling \var{mkfli4l} then would lead to this output:

\begin{example}
\begin{verbatim}
    Warning: match MATCH_1: foo
    Warning: match MATCH_2: bar
    Warning: match MATCH_3: 12
\end{verbatim}
\end{example}

When using \verb?=~? you may take all existing regular expressions
into account. If you i.e. want to check whether a PCMCIA Ethernet driver
is selected without \var{OPT\_PCMCIA} being set to ``yes'', it might
look like this:

\begin{example}
\begin{verbatim}
    if (!opt_pcmcia)
    then
        foreach i in net_drv_%
        do
           if (i =~ "^(RE:PCMCIA_NET_DRV)$")
           then
               error "If you want to use ..."
           fi
        done
    fi
\end{verbatim}
\end{example}

As demonstrated in the example, it is important to \emph{anchor} the regular
expression with \texttt{\^} and \texttt{\$} if intending to apply the
expression on the \emph{complete} variable. Otherwise, the match-expression
already returns ``true'' if only a \emph{part} of the variable is covered by
the regular expression, which is certainly not desired in this case.

\subsubsection{Check if a File has been copied depending on the Value of a Variable: \texttt{copy\_pending}}

        With the information gained during the checking process the function \texttt{copy\_pending}
        tests if a file has been copied depending on the value of a variable or not.
        This can be used i.e. in order to test whether the driver specified by the user
        really exists and has been copied. \texttt{copy\_pending} accepts the name to be
        tested in the form of a variable or a string. \footnote{As described before
        the string is subject of variable substitution, i.e via a
        \jump{subsec:dev:control}{\texttt{foreach}-loop} and a
        \jump{subsec:dev:string-rewrite}{\texttt{\%<Name>}-subsitution}
        all elements of an array may be examined.} In order to accomplish this \texttt{copy\_pending}
        checks whether

        \begin{itemize}
        \item the variable is active (if it depends on an OPT it has to be set to ``yes''),

         \item the variable was referenced in an \texttt{opt/<PACKAGE>.txt}-file and

         \item whether a file was copied dependant on the current value.

        \end{itemize}

        \texttt{copy\_pending} will return ``true'' if it detects that during
        the last step \emph{no} file was copied, the copy process hence still is ``pending''.

    A small example of the use of all these functions
    can be found in \texttt{check/base.ext}:

\begin{example}
\begin{verbatim}
    foreach i in net_drv_%
    do
        if (copy_pending("%i"))
        then
            error "No network driver found for %i='$i', check config/base.txt"
        fi
    done
\end{verbatim}
\end{example}

    Alle elements of the array \var{NET\_DRV\_\%} are detected for
    which no copy action has been done because there is no corresponding
    entry existing in \texttt{opt/base.txt}.

\subsubsection{Comparison of Network Addresses: \texttt{samenet} und \texttt{subnet}}

For testing routes from time to time a test is needed whether two
networks are identical or if one is a subnet of the other. The two
functions \texttt{samenet} and \texttt{subnet} are of help here.

\begin{example}
\begin{verbatim}
    samenet (netz1, netz2)
\end{verbatim}
\end{example}

returns ``true'' if both nets are identical and

\begin{example}
\begin{verbatim}
    subnet (net1, net2)
\end{verbatim}
\end{example}

returns ``true'' if ``net1'' is a subnet of ``net2''.

\subsubsection{Expanding the Kernel Command Line}

If an OPT must pass other boot parameters to the kernel, in former times
the variable \var{KERNEL\_BOOT\_OPTION} had to be checked whether the required
parameter was included, and if necessary, a warning or error message had to be
displayed. With the internal variable \var{KERNEL\_BOOT\_OPTION\_EXT}
you may add a necessary but missing option directly in an ext-script. An
Example taken from \texttt{check/base.ext}:

\begin{example}
\begin{verbatim}
    if (powermanagement =~ "apm.*|none")
    then
        if ( ! kernel_boot_option =~ "acpi=off")
        then
            set kernel_boot_option_ext="${kernel_boot_option_ext} acpi=off"
        fi
    fi
\end{verbatim}
\end{example}

This passes ``acpi=off'' to the kernel if no or ``APM''-type power
management is desired.

\subsection{Support for Different Kernel Version Lines}

Different kernel version lines often differ in some details:
\begin{itemize}
\item changed drivers are provided, some are deleted, others are added
\item module names simply differ
\item module dependencies are different
\item modules are stored in different locations
\end{itemize}

These differences are mostly handled automatically by \var{mkfli4l}.
To describe the available modules you can, on one hand expand tests
dependant on the version
(\jump{sec:regexp-dependencies}{conditional regular expressions}), or,
on the other hand \var{mkfli4l} allows \emph{version dependant}
\texttt{opt/<PACKAGE>.txt}-files. These are then named
\texttt{opt/<PACKAGE>\_<Kernel-Version>.txt}, where the components of the
kernel version are separated from each other by underscores. An example:
the package ``base'' contains these files in its \texttt{opt}-directory:

\begin{itemize}
\item \texttt{base.txt}
\item \texttt{base\_3\_18.txt}
\item \texttt{base\_3\_19.txt}
\end{itemize}

the first file (\texttt{base.txt}) is \emph{always} considered. Both
other files are only considered if the kernel version is called ``3.18(.*)''
resp.\ ``3.19(.*)''. As seen here, some parts of the version may be omitted
in file names, if a group of kernels should be addressed.
If \verb+KERNEL_VERSION='3.18.9'+ is given, the following files (if existing)
are considered for the package \texttt{<PACKAGE>}:

\begin{itemize}
\item \texttt{<PACKAGE>.txt}
\item \texttt{<PACKAGE>\_3.txt}
\item \texttt{<PACKAGE>\_3\_18.txt}
\item \texttt{<PACKAGE>\_3\_18\_9.txt}
\end{itemize}

\subsection{Documentation}

    Documentation should be placed in the files

    \begin{itemize}
    \item \texttt{doc/<LANGUAGE>/opt/<PACKAGE>.txt}
    \item \texttt{doc/<LANGUAGE>/opt/<PACKAGE>.html}.
    \end{itemize}

    HTML-files may be splitted, meaning one for each OPT contained.
    Nevertheless a file \texttt{<PACKAGE>.html} has to be created
    linking to the other files. Changes should be documented in:

    \begin{itemize}
    \item \texttt{changes/<PACKAGE>.txt}
    \end{itemize}

    The entire text documentation may not contain any tabs and has
    to have a line feed no later than after 79 characters. This ensures
    that the documentation can also be read correctly with an editor
    without automatic line feed.

    Also a documentation in \LaTeX-format is possible, with HTML and
    PDF versions generated from it. The documentation of fli4l may serve as an example
    here. A documentation framework for required \LaTeX-macros can be found in
    the package ``template''. A brief description is to be found in the
    following subsections.

    The fli4l documentation is currently available in the following languages:
    German (\texttt{<LANGUAGE>} = ``deutsch''), English (\texttt{<LANGUAGE>} = ``english'') and
    French (\texttt{<LANGUAGE>} = ``french''). It is the package developer's
    decision to document his package in any language. For the purposes of
    clarity it is recommended to create a documentation in German and/or
    English (ideally in both languages).

\subsubsection{Prerequisites for Creating a \LaTeX Documentation}

  To create a documentation from \LaTeX-sources the following
  requirements apply:

  \begin{itemize}
  \item Linux/OS~X-Environment: For ease of production, a makefile
    exists to automate all other calls (Cygwin should work too, but
    is not tested by the fli4l team)
  \item LaTeX2HTML for the HTML version
  \item of course \LaTeX\ (Recommended: ``TeX Live'' for Linux/OS~X and
  ``MiKTeX'' for Microsoft Windows)  the ``pdftex''program and these
    \TeX-packages:
    \begin{itemize}
    \item current KOMA-Skript (at least version 2)
    \item all packages necessary for pdftex
    \item unpacked documentation package for fli4l, it provides the
      necessary makefiles and \TeX-styles
  \end{itemize}
  \end{itemize}


\subsubsection{File Names}

The documentation files are named according to the following scheme:

\begin{description}
\item [\texttt{<PACKAGE>\_main.tex}:] This file contains the main
  part of the documentation. \texttt{<PACKAGE>} stands for the name of the
  package to be described (in lowercase letters).
\item[\texttt{<PACKAGE>\_appendix.tex}:] If further comments should be
  added to the package, they should be placed there.
\end{description}

The files should be stored in the directory
\texttt{fli4l/<PACKAGE>/doc/<SPRACHE>/tex/<PACKAGE>}.
For the package sshd this looks like here:

\begin{verbatim}
    $ ls fli4l/doc/deutsch/tex/sshd/
    Makefile sshd_appendix.tex  sshd_main.tex  sshd.tex
\end{verbatim}

The Makefile is responsible for generating the documentation,
the \texttt{sshd.tex}-file provides a framework for the actual
documentation and the appendix, which is located in the other
two files. See an example in the documentation of the package
``template''.

\subsubsection{\LaTeX-Basics}

\LaTeX\ is, just like HTML, ``Tag-based'' , only that the tags
are called ``commands'' and have this format: \verb*?\command?
resp.\ \verb*?\begin{environment}? \ldots \verb*?\end{environment}?

By the help of commands you should rather emphasize the \emph{importance}
of the text less the \emph{display}. It is therefore of advantage to use

\begin{example}
\verb*?\warning{Please do not...}?
\end{example}

\noindent instead of

\begin{example}
\verb*?\emph{Please do not...}?
\end{example}

\noindent.

Each command rsp.\ each environment may take some more parameters
noted like this: \verb*?\command{parameter1}{parameter2}{parameterN}?.

Some commands have optional parameters in square (instead of curly)
brackets:\\ \verb*?\command[optionalParameter]{parameter1}?
\ldots\ Usually only one optional parameter is used, in
rare cases there may be more.

Individual paragraphs in the document are separated by blank lines.
Within these paragraphs \LaTeX\ itself takes care of line breaks and
hyphenation.

The following characters have special meaning in \LaTeX\ and, if
occuring in normal text, must be masked prefixed by a \verb*?\?:
\# \$ \& \_ \% \{ \}. ``\verb?~?'' and ``\verb?^?'' have to be
written as follows: \verb!\verb?~?! \verb!\verb?^?!

The main \LaTeX-commands are explained in the documentation of the
package ``template''.

\subsection{File Formats}

    All text files (both documentation and scripts, which later reside on
    the router) should be added to the package in DOS file format, with
    CR/LF instead of just LF at the end of a line. This ensures that Windows
    users can read the documentation even with ``notepad'' and that after changing
    a script under Windows everything still is executable on the router.

    The scripts are converted to the required format during archive creation
    (see the description of the flags in table~\ref{table:options}).

\subsection{Developer Documentation}

    If a program from the package defines a new interface that other programs
    can use, please store the documentation for this interface in a separate
    documentation in \texttt{doc/dev/<PACKAGE>.txt}.

\subsection{Client Programs}

    If a package also provides additional client programs, please store them
    in the directory \texttt{windows/} for Windows clients and in the directory
    \texttt{unix/} for *nix and Linux clients.

\subsection{Source Code}

    Customized programs and source code may be enclosed in the directory
    \texttt{src/<PACKAGE>/}. If the programs should be built like the rest
    of the fi4l programs, please have a look at the documentation of the
    \jump{buildroot}{``src''-package} .

\marklabel{sec:script_names}{
  \subsection{More Files}
}

    All files, which will be copied to the router have to be stored under
    \texttt{opt/}. Be under
    \begin{itemize}
    \item \texttt{opt/etc/boot.d/} and \texttt{opt/etc/rc.d/}: scripts, that should be
      executed on system start
    \item \texttt{opt/etc/rc0.d/}: scripts, that should be
      executed on system shutdown
    \item \texttt{opt/etc/ppp/}: scripts, that should be
      executed on dialin or hangup
    \item \texttt{opt/}: executable programs and other files
       according to their positions in the file system (for example the file
      \texttt{opt/bin/busybox} will later be situated in the directory \texttt{/bin}
      on the router)
    \end{itemize}

    Scripts in \texttt{opt/etc/boot.d/}, \texttt{opt/etc/rc.d/} and
    \texttt{opt/etc/rc0.d/}
    have the following naming scheme:

    \begin{example}
    \begin{verbatim}
    rc<number>.<name>
    \end{verbatim}
    \end{example}

    The number defines the order of execution, the name gives a hint on
    what program/package is processed by this script.


% Last Update: $Id: dev_main_scripting.tex 35981 2014-12-28 19:32:31Z kristov $

\section{Allgemeine Skript-Erstellung auf fli4l}


Hier folgt jetzt \emph{keine} allgemeine Einführung in Shell-Skripte, das
kann jeder im Internet selber nachlesen, es wird nur auf die spezielle
Gegebenheiten bei fli4l eingegangen. Informationen dazu gibt es in den diversen
Unix-/Linux-Hilfeseiten. Folgende Links können als Einstiegspunkte zu
diesem Thema dienen:
\begin{itemize}
\item Einführung in Shell-Skripte:
  \begin{itemize}
  \item \altlink{http://cip.physik.uni-freiburg.de/main/howtos/sh.php}
  \end{itemize}
 \item
   Hilfeseiten online:
   \begin{itemize}
   \item \altlink{http://linux.die.net/}
   \item \altlink{http://heapsort.de/man2web}
   \item \altlink{http://man.he.net/}
   \item \altlink{http://www.linuxcommand.org/superman_pages.php}
   \end{itemize}
\end{itemize}

\subsection{Aufbau}

    In der Unix-Welt ist es nötig, ein Skript mit dem Namen des Interpreters
    zu beginnen, daher steht in der ersten Zeile:
\begin{example}
\begin{verbatim}
      #!/bin/sh
\end{verbatim}
\end{example}

    Damit man später leichter erkennen kann, was ein Skript macht und wer es
    geschrieben hat, sollte jetzt ein kurzer Header folgen, in etwa so:

\begin{example}
\begin{verbatim}
      #--------------------------------------------------------------------
      # /etc/rc.d/rc500.dummy - start my cool dummy server
      #
      # Creation:     19.07.2001  Toller Hecht <toller-hecht@example.net>
      # Last Update:  11.11.2001  Süße Maus <suesse-maus@example.net>
      #--------------------------------------------------------------------
\end{verbatim}
\end{example}

    Nun kann das eigentliche Skript folgen...


\marklabel{dev:sec:config-variables}{
\subsection{Umgang mit Konfigurationsvariablen}
}


    Pakete werden über die Datei \texttt{config/<PACKAGE>.txt}
    konfiguriert. Die darin enthaltenen und
    \jump{subsec:dev:var-check}{aktiven Variablen} werden beim Erzeugen
    des Boot-Mediums in die Datei \texttt{rc.cfg} übernommen. Beim Booten des
    Routers wird diese Datei eingelesen, bevor irgend ein rc-Skript
    (Skripte unter \texttt{/etc/rc.d/}) gestartet wird. Diese Skripte können
    dadurch auf alle Konfigurationsvariablen einfach durch
    \var{\$<Variablenname>} zugreifen.

    Benötigt man Werte von Konfigurationsvariablen auch noch nach dem
    Booten, dann kann man sie aus der \texttt{/etc/rc.cfg} extrahieren, in
    welche während des Bootens die Konfiguration des Boot-Mediums geschrieben
    wurde. Möchte man beispielsweise den Wert der Variable \texttt{OPT\_DNS}
    in einem Skript auslesen, so kann man dies folgendermaßen tun:

\begin{example}
\begin{verbatim}
    eval $(grep "^OPT_DNS=" /etc/rc.cfg)
\end{verbatim}
\end{example}

    Das funktioniert auch mit mehreren Variablen effizient (d.\,h.\ mit nur
    einem Aufruf des \texttt{grep}-Programms):

\begin{example}
\begin{verbatim}
    eval $(grep "^\(HOSTNAME\|DOMAIN_NAME\|OPT_DNS\|DNS_LISTEN_N\)=" /etc/rc.cfg)
\end{verbatim}
\end{example}

\marklabel{dev:sec:persistent-data}{
\subsection{Persistente Speicherung von Daten}
}

Gelegentlich benötigt ein Paket die Möglichkeit, Daten persistent abzulegen,
die also einen Neustart des Routers überleben. Dazu existiert die Funktion
\texttt{map2persistent}, die von einem Skript in \texttt{/etc/rc.d/}
aufgerufen werden kann. Sie erwartet eine Variable, die einen Pfad enthält,
und ein Unterverzeichnis. Die Idee ist, dass die Variable entweder einen
tatsächlichen Pfad beschreibt~-- dann wird dieser Pfad auch genommen, denn der
Nutzer hat ihn so gewünscht, oder die Zeichenkette "`auto"'~-- dann wird
unterhalb eines Verzeichnisses auf einem persistenten Medium ein entsprechendes
Unterverzeichnis gemäß dem zweiten Parameter erzeugt. Die Funktion liefert
das Resultat in eben der Variable zurück, deren Name im ersten Parameter
übergeben wurde.

Ein Beispiel soll dies verdeutlichen. Sei \var{VBOX\_SPOOLPATH} eine Variable,
die einen Pfad oder die Zeichenkette "`auto"' enthält. Dann führt der Aufruf

\begin{example}
\begin{verbatim}
    begin_script VBOX "Configuring vbox ..."
    [...]
    map2persistent VBOX_SPOOLPATH /spool
    [...]
    end_script
\end{verbatim}
\end{example}

dazu, dass die Variable \var{VBOX\_SPOOLPATH} entweder gar nicht verändert
wird (falls sie einen Pfad enthält), oder dass sie durch den Pfad
\texttt{/var/lib/persistent/vbox/spool} ersetzt wird (falls sie die Zeichenkette
"`auto"' enthält). Dabei verweist\footnote{mit Hilfe eines so genannten
"`bind"'-Mounts} \texttt{/var/lib/persistent} auf ein
Verzeichnis auf einem beschreibbaren und nicht flüchtigen Speichermedium, und
\var{<SCRIPT>} stellt das aufrufende Skript in Kleinbuchstaben dar (dieser Name
wird vom ersten Argument des
\jump{subsec:dev:bug-searching}{\texttt{begin\_script}-Aufrufs} abgeleitet).
Falls kein geeignetes Medium existieren sollte (was durchaus sein kann),
ist \texttt{/var/lib/persistent} ein Verzeichnis in der RAM-Disk.

Zu beachten ist, dass der Pfad, der von \texttt{map2persistent} zurückgegeben
wird, \emph{nicht} automatisch erzeugt wird~-- das muss der Aufrufer selbst
erledigen (etwa durch einen Aufruf von \texttt{mkdir -p <Pfad>}).

In der Datei \texttt{/var/run/persistent.conf} kann nachgeschaut werden, ob
die persistente Speicherung von Daten möglich ist. Beispiel:

\begin{example}
\begin{verbatim}
    . /var/run/persistent.conf
    case $SAVETYPE in
    persistent)
        echo "Persistente Speicherung möglich!"
        ;;
    transient)
        echo "Persistente Speicherung NICHT möglich!"
        ;;
    esac
\end{verbatim}
\end{example}

\marklabel{subsec:dev:bug-searching}{
\subsection{Fehlersuche}
}

    Bei Start-Skripten ist es oft sinnvoll, diese bei Bedarf im Debug-Modus
    der Shell laufen zu lassen, um festzustellen, wo "`der Wurm drin ist"'.
    Dazu wird am Anfang und am Ende folgendes eingefügt:

\begin{example}
\begin{verbatim}
      begin_script <OPT-Name> "start message"
      <script code>
      end_script
\end{verbatim}
\end{example}

Im normalen Betrieb erscheint jetzt beim Start des Skriptes der
angegebene Text und am Ende der gleiche Text mit einem vorangestellten
"`finished"'.

Will man die Skripte debuggen, muss man zwei Dinge tun:

\begin{enumerate}

\item Man muss \jump{DEBUGSTARTUP}{\var{DEBUG\_\-STARTUP}} auf "`yes"'
  setzen.
\item Man muss das Debuggen für dieses OPT aktivieren. Das tut man in
  der Regel durch den Eintrag
\begin{example}
\begin{verbatim}
      <OPT-Name>_DO_DEBUG='yes'
\end{verbatim}
\end{example}
in der Konfigurationsdatei.\footnote{Manchmal werden mehrere
  Start-Skripte verwendet, die dann auch verschiedene Namen für ihre
  Debug-Variablen haben. Hier hilft ein kurzer Blick in die Skripte.}
    Jetzt wird während der Ausführung am Bildschirm genau dargestellt, was
    passiert.
\end{enumerate}


\subsubsection{Weitere beim Debuggen hilfreiche Variablen}

\begin{description}

  \config{DEBUG\_ENABLE\_CORE}{DEBUG\_ENABLE\_CORE}{DEBUGENABLECORE}

  Diese Variable gestattet das Erzeugen von "`Core-Dumps"' (Speicherauszügen).
  Stürzt ein Programm aufgrund eines Fehlers ab, wird ein Abbild des aktuellen
  Zustandes im Dateisystem abgelegt, der hinterher zur Analyse des
  Problems verwendet werden kann. Die Core-Dumps werden unter
  \texttt{/var/log/dumps/} abgelegt.

  \config{DEBUG\_IP}{DEBUG\_IP}{DEBUGIP}

  Wird diese Variable gesetzt, werden alle Aufrufe des Programms \texttt{ip}
  protokolliert.

  \config{DEBUG\_IPUP}{DEBUG\_IPUP}{DEBUGIPUP}

  Wird diese Variable auf "`yes"' gesetzt, werden während der
  Ausführung der \texttt{ip-up}/\texttt{ip-down}-Skripte die ausgeführten
  Anweisungen mitgezeichnet und im System-Protokoll gespeichert.

  \config{LOG\_BOOT\_SEQ}{LOG\_BOOT\_SEQ}{LOGBOOTSEQ}

  Wird diese Variable auf "`yes"' gesetzt, protokolliert der \texttt{bootlogd}
  während des Bootens alle auf der Konsole getätigten Ausgaben in der Datei
  \texttt{/var/tmp/boot.log}. Diese Variable hat standardmäßig den Wert "`yes"'.

  \config{DEBUG\_KEEP\_BOOTLOGD}{DEBUG\_KEEP\_BOOTLOGD}{DEBUGKEEPBOOTLOGD}

  Normalerweise wird der \texttt{bootlogd} am Ende des Bootvorganges
  beendet. Ein Setzen dieser Variable unterbindet das und erlaubt ein
  Protokollieren der Konsolenausgaben über den Bootvorgang hinaus.

  \config{DEBUG\_MDEV}{DEBUG\_MDEV}{DEBUGMDEV}

  Ein Setzen dieser Variable generiert ein Protokoll des \texttt{mdev}-Daemons,
  der für das Anlegen der Geräte-Dateien unter \texttt{/dev} zuständig ist.

\end{description}
\subsection{Hinweise}
\begin{itemize}
\item  Es ist \emph{immer} besser, geschweifte Klammern "`\{\ldots\}"' an Stelle von runden
      Klammern "`(\ldots)"' zu benutzen. Allerdings muss dabei darauf geachtet werden,
      dass nach der öffnenden Klammer ein Leerzeichen oder eine neue Zeile vor
      dem nächsten Befehl kommt und vor der schließenden Klammer ein
      Semikolon oder auch eine neue Zeile kommt. Beispielsweise ist

\begin{example}
\begin{verbatim}
        { echo "cpu"; echo "quit"; } | ...
\end{verbatim}
\end{example}

      \noindent gleichbedeutend mit:

\begin{example}
\begin{verbatim}
        {
                echo "cpu"
                echo "quit"
        } | ...
\end{verbatim}
\end{example}


      \item Ein Skript kann mit "`exit"' vorzeitig beendet
        werden. Dies ist aber bei den Start-Skripten (\texttt{opt/etc/boot.d/...},
        \texttt{opt/etc/rc.d/...}), den Stopp-Skripten (\texttt{opt/etc/rc0.d/...}) und den
        \texttt{ip-up}/\texttt{ip-down}-Skripten (\texttt{opt/etc/ppp/...}) geradezu tödlich, da
        auch nachfolgende Skripte nicht mehr ausgeführt werden. Im
        Zweifelsfall immer weglassen.


      \item KISS~-- Keep it small and simple. Du willst Perl als
        Skript-Sprache verwenden? Dir reichen die
        Skripting-Möglichkeiten von fli4l nicht?  Überdenke deine
        Einstellung! Ist dein OPT wirklich nötig? fli4l ist immer noch
        "`nur"' ein Router, ein Router sollte eigentlich keine
        Serverdienste anbieten.


      \item Die Fehlermeldung "`: not found"' heißt meistens, dass das
        Skript noch im DOS-Format vorliegt. Weitere Fehlerquelle: Das
        Skript ist nicht ausführbar. In beiden Fällen sollte die
\texttt{opt/<PACKAGE>.txt}-Datei daraufhin geprüft werden, ob sie die
korrekten Optionen (in Bezug auf "`mode"', "`gid"', "`uid"' und Flags) enthält. Wenn das Skript erst
bem Booten erzeugt wird, sollte ein "`chmod +x <Skriptname>"' ausgeführt
werden.

      \item Für temporäre Dateien sollte der Pfad \texttt{/tmp} genutzt werden.
        Es ist aber unbedingt darauf zu achten, dass hier nur wenig
        Platz ist, weil dies in der RootFS-RAM-Disk liegt! Wenn mehr
        Platz benötigt wird, muss man sich eine eigene RAM-Disk
        erstellen und mounten. Details dazu verrät der Abschnitt "`RAM-Disks"'
        in dieser Dokumentation.

    \item Damit temporäre Dateien eindeutige Namen erhalten, sollte man
      grundsätzlich die aktuelle Prozess-ID, die in der Shell-Variable "`\$"'
      gespeichert ist, an den Dateiname anhängen.
      \texttt{/tmp/<OPT-Name>.\$\$} stellt somit einen guten Dateinamen dar,
      \texttt{/tmp/<OPT-Name>} eher weniger, wobei \texttt{<OPT-Name>} natürlich
      nicht so stehen bleiben soll, sondern geeignet ersetzt werden muss.

\end{itemize}


% Last Update: $Id: dev_main_pf.tex 49532 2017-12-13 13:16:20Z kristov $

\providecommand{\fwaction}[1]{{\small\textsf{#1}}}
\providecommand{\fwchain}[1]{\texttt{#1}}
\providecommand{\fwtable}[1]{\textsc{#1}}
\providecommand{\fwmatch}[1]{\texttt{#1}}
\providecommand{\fwpktstate}[1]{\texttt{#1}}
\providecommand{\fwloglevel}[1]{\texttt{#1}}

\section{Arbeit mit dem Paketfilter}
\subsection{Hinzufügen von eigenen Ketten und Regeln}

Zur Manipulation des Paketfilters steht eine Reihe von Routinen zur
Verfügung, mit deren Hilfe man Ketten (engl. "`Chains"') und Regeln hinzufügen
und wieder löschen kann. Eine Kette ist eine benannte und geordnete Liste von
Regeln. Es gibt einen Satz vordefinierter Ketten (\fwchain{PREROUTING},
\fwchain{INPUT}, \fwchain{FORWARD}, \fwchain{OUTPUT}, \fwchain{POSTROUTING});
mit Hilfe dieser Funktionen können weitere Ketten nach Bedarf erstellt werden.

\begin{description}
\item [\texttt{add\_chain/add\_nat\_chain <chain>}:]
  Fügt eine Kette zur "`filter"'- oder "`nat"'-Ta\-bel\-le hinzu.
\item [\texttt{flush\_chain/flush\_nat\_chain <chain>}:]
  Entfernt alle Regeln aus einer Kette der "`filter"'- oder "`nat"'-Tabelle.
\item [\texttt{del\_chain/del\_nat\_chain <chain>}:]
  Entfernt eine Kette aus der "`filter"'- oder "`nat"'-Tabelle. Ketten müssen
  leer sein, bevor sie gelöscht werden können, und es darf auch keine Referenz
  mehr auf sie geben. Eine solche Referenz ist z.\,B. eine \fwaction{JUMP}-Aktion,
  deren Ziel eben diese Kette ist.
\item[\texttt{add\_rule/ins\_rule/del\_rule}:] Fügt Regeln
  am Ende einer Kette (\texttt{add\_rule}) bzw. an beliebiger Stelle in einer
  Kette (\texttt{ins\_rule}) ein bzw. löscht Regeln aus einer Kette
  (\texttt{del\_rule}). Ein Aufruf sieht wie folgt aus:

\begin{example}
\begin{verbatim}
    add_rule <table> <chain> <rule> <comment>
    ins_rule <table> <chain> <rule> <position> <comment>
    del_rule <table> <chain> <rule> <comment>
\end{verbatim}
\end{example}

  \noindent wobei die Parameter folgende Bedeutung haben:
  \begin{description}
  \item[table] Die Tabelle, in der sich die Kette befindet
  \item[chain] Die Kette, in welche die Regel eingefügt werden soll
  \item[rule] Die Regel, die eingefügt werden soll; das Format
    entspricht dem in der Konfigurationsdatei verwendeten
  \item[position] Die Position, an der die Regel eingefügt werden soll (nur
    bei \texttt{ins\_rule})
  \item[comment] Ein Kommentar, der zusammen mit der Regel angezeigt
    werden soll, wenn sich jemand den Paketfilter ansieht.
  \end{description}
\end{description}


\subsection{Einordnen in bestehende Regeln}

fli4l konfiguriert den Paketfilter mit einem gewissen
Standardregelsatz. Will man eigene Regeln einfügen, wird man diese in
der Regel nach dem Standardregelsatz einfügen wollen. Ebenfalls wird
man wissen wollen, was denn die vom Nutzer gewünschte Aktion beim
Verwerfen eines Paketes ist. Diese Informationen bekommt man für die
\fwchain{FORWARD}- und \fwchain{INPUT}-Ketten durch Aufruf zweier Funktionen,
\texttt{get\_defaults} und \texttt{get\_count}. Nach Aufruf von

\begin{example}
\begin{verbatim}
    get_defaults <chain>
\end{verbatim}
\end{example}

erhält man die folgenden Ergebnisse:

\begin{description}
\item[\var{drop}:] Diese Variable enthält die Kette, in die verzweigt wird,
  wenn ein Paket verworfen wird.
\item[\var{reject}:] Diese Variable enthält die Kette, in die verzweigt wird,
  wenn ein Paket abgelehnt wird.
\end{description}

Nach Aufruf von

\begin{example}
\begin{verbatim}
    get_count <chain>
\end{verbatim}
\end{example}

erhält man in der Variable \var{res} die Anzahl der Regeln in der Kette
\texttt{<chain>}. Diese Position ist insofern wichtig, als man \emph{nicht}
einfach \texttt{add\_rule} verwenden kann, um eine Regel am Ende der
vordefinierten "`filter"'-Ketten \fwchain{INPUT}, \fwchain{FORWARD} und
\fwchain{OUTPUT} einzufügen. Dies liegt daran, dass diese Ketten mit einer
Standardregel abgeschlossen werden, welche alle verbliebenen Pakete behandelt,
je nach Belegung der \var{PF\_<Kette>\_POLICY}-Variablen. Ein Einfügen
\emph{hinter} dieser letzten Regel hat also keine Auswirkungen. Die Funktion
\texttt{get\_count} erlaubt es nun hingegen, die Stelle direkt \emph{vor}
dieser letzten Regel zu ermitteln und die Position dann an die
\texttt{ins\_rule}-Funktion im Parameter \texttt{<position>} zu übergeben, um
die Regel wie gewünscht am Ende der jeweiligen Kette, aber vor der letzten
Auffang-Regel einzubauen.

Ein Beispiel aus dem Skript \texttt{opt/etc/rc.d/rc390.dns\_dhcp} des Pakets
"`dns\_dhcp"' soll dies verdeutlichen:

\begin{example}
\begin{verbatim}
    case $OPT_DHCPRELAY in
        yes)
            begin_script DHCRELAY "starting dhcprelay ..."

            idx=1
            interfaces=""
            while [ $idx -le $DHCPRELAY_IF_N ]
            do
                eval iface='$DHCPRELAY_IF_'$idx

                get_count INPUT
                ins_rule filter INPUT "prot:udp  if:$iface:any 68 67 ACCEPT" \
                    $res "dhcprelay access"

                interfaces=$interfaces' -i '$iface
                idx=`expr $idx + 1`
            done
            dhcrelay $interfaces $DHCPRELAY_SERVER

            end_script
        ;;
  esac
\end{verbatim}
\end{example}

Hier sieht man inmitten der Schleife einen Aufruf von \texttt{get\_count},
gefolgt von einem Aufruf der \texttt{ins\_rule}-Funktion, der unter anderem
die \var{res}-Variable als \texttt{position}-Parameter übergeben wird.


\subsection{Erweiterung der Paketfilter-Tests}

fli4l verwendet in den Paketfilterregeln die Syntax \fwmatch{match:params},
um zusätzliche Bedingungen an die Pakete zu stellen
(siehe \fwmatch{mac:}, \fwmatch{limit:}, \fwmatch{length:}, \fwmatch{prot:},
\ldots). Will man zusätzliche
Tests hinzufügen, wird das folgendermaßen gemacht:

\begin{enumerate}
\item Festlegen eines passenden Namens. Dieser Name muss mit einem
Kleinbuchstaben im Bereich a-z beginnen und ansonsten aus beliebigen
Buchstaben und Ziffern bestehen.

\achtung{Wenn der Paketfilter-Test in IPv6-Regeln verwendet werden soll,
dann muss darauf geachtet werden, dass der Name keine gültige
IPv6-Adresskomponente ist!}

\item Anlegen einer Datei \texttt{opt/etc/rc.d/fwrules-<name>.ext}.
In dieser Datei steht in etwa Folgendes:

\begin{example}
\begin{verbatim}
    # IPv4 extension is available
    foo_p=yes

    # the actual IPv4 extension, adding matches to match_opt
    do_foo()
    {
        param=$1
        get_negation $param
        match_opt="$match_opt -m foo $neg_opt --fooval $param"
    }

    # IPv6 extension is available
    foo6_p=yes

    # the actual IPv6 extension, adding matches to match_opt
    do6_foo()
    {
        param=$1
        get_negation6 $param
        match_opt="$match_opt -m foo $neg_opt --fooval $param"
    }
\end{verbatim}
\end{example}

Der Paketfilter-Test muss nicht zwingend sowohl für IPv4 als auch für IPv6
implementiert sein (obwohl dies zu bevorzugen ist, falls er für beide
Layer-3-Protokolle sinnvoll ist).

\item Testen der Erweiterung:

\begin{example}
\begin{verbatim}
    $ cd opt/etc/rc.d
    $ sh test-rules.sh 'foo:bar ACCEPT'
    add_rule filter FORWARD 'foo:bar ACCEPT'
    iptables -t filter -A FORWARD -m foo --fooval bar -s 0.0.0.0/0 \
        -d 0.0.0.0/0 -m comment --comment foo:bar ACCEPT -j ACCEPT
\end{verbatim}
\end{example}

\item Aufnahme der Erweiterung und aller noch benötigten Dateien
(\texttt{iptables}-Komponenten) ins Archiv über einen
der bekannten Mechanismen.
\item Zulassen der Erweiterung in der Konfiguration durch Erweitern
von \var{FW\_GENERIC\_MATCH} und/oder \var{FW\_GENERIC\_MATCH6} in einer
exp-Datei, z.\,B. wie folgt:

\begin{example}
\begin{verbatim}
    +FW_GENERIC_MATCH(OPT_FOO) = 'foo:bar' : ''
    +FW_GENERIC_MATCH6(OPT_FOO) = 'foo:bar' : ''
\end{verbatim}
\end{example}
\end{enumerate}


% Last Update: $Id$

\section{CGI-Erstellung für das \emph{httpd}-Paket}

\subsection{Allgemeines zum Webserver}
Der Webserver, der bei fli4l verwendet wird, ist der \texttt{mini\_httpd} von ACME
Labs. Die Quellen können unter
\altlink{http://www.acme.com/software/mini_httpd/} heruntergeladen werden.
Allerdings wurden für fli4l ein paar Änderungen vorgenommen. 
Die Anpassungen befinden sich im \emph{src}-Paket im Verzeichnis
\texttt{src/""fbr/""buildroot/""package/""mini\_httpd}.

\subsection{Skriptnamen}

Der Skriptname sollte möglichst vielsagend sein, damit er von
anderen Skripten leichter zu unterscheiden ist und es keine
Namensüberschneidungen bei verschiedenen OPTs gibt.

Damit die Skripte ausführbar gemacht werden und DOS-Zeilenumbrüche in
Unix-Zeilen\-um\-brüche umgewandelt werden, muss in der \texttt{opt/<PAKET>.txt}
ein entsprechender Eintrag gemacht werden, siehe
Tabelle~\jump{table:options}{\ref{table:options}}.

\subsection{Menü-Einträge}

Um einen Eintrag im Menü vorzunehmen, muss eine Eintragung in der Datei
\texttt{/etc/httpd/menu} vorgenommen werden. Dieser Mechanismus erlaubt es OPTs,
auch im laufendem Betrieb Änderungen am Menü vorzunehmen. Dies sollte nur mit
dem Skript \texttt{httpd-menu.sh} gemacht werden, da dieses darauf achtet, dass
das Dateiformat dieser Datei immer konsistent ist. Um neue Menüpunkte
einzufügen, wird es folgendermaßen aufgerufen:

\begin{example}
\begin{verbatim}
    httpd-menu.sh add [-p <priority>] <link> <name> [section] [realm]
\end{verbatim}
\end{example}

So wird ein Eintrag mit dem Namen \texttt{<name>} in den Abschnitt
\texttt{[section]} eingetragen. Wenn \texttt{[section]} weggelassen wird, wird
es standardmäßig in den Abschnitt "`OPT-Pakete"' eingetragen. \texttt{<link>}
gibt das Ziel des neuen Links an. \texttt{<priority>} spezifiziert die Priorität
eines Menüeintrags in seinem Abschnitt. Wird sie nicht angegeben, wird die
Standardpriorität 500 benutzt. Die Priorität sollte eine dreistellige Nummer
sein. Je kleiner die Priorität, desto weiter oben steht der Link in dem
Abschnitt. Soll ein Eintrag möglichst weit nach unten, so ist z.\,B. die Priorität
900 zu wählen. Bei gleicher Priorität werden die Einträge nach dem Ziel des
Links sortiert. Bei \texttt{[realm]} wird der Bereich angegeben, für den ein
angemeldeter Benutzer mindestens die Berechtigung \emph{view} haben muss, damit
der Menüpunkt angezeigt wird. Wird \texttt{[realm]} nicht angegeben, wird der
Menüpunkt immer angezeigt. Siehe hierzu auch den Abschnitt
\jump{sec:rights}{"`Benutzerrechte"'}.

Beispiel:

\begin{example}
\begin{verbatim}
    httpd-menu.sh add "neuedatei.cgi" "Hier klicken" "Tools" "tools"
\end{verbatim}
\end{example}

Dieses Beispiel erzeugt im Abschnitt "`Tools"' einen Link mit dem Titel 
"`Hier klicken"' und dem Link-Ziel "`neuedatei.cgi"' und legt den Abschnitt,
falls dieser nicht vorhanden ist, an.

Das Skript kann auch Einträge aus dem Menü wieder entfernen:

\begin{example}
\begin{verbatim}
    httpd-menu.sh rem <link>
\end{verbatim}
\end{example}

Mit diesem Aufruf wird der Eintrag mit dem Link \texttt{<link>} wieder
entfernt.

\wichtig{Wenn mehrere Menüeinträge auf die gleiche Datei verweisen,
werden alle diese Einträge aus dem Menü entfernt.}

Da Abschnitte auch Prioritäten haben können, können diese auch manuell angelegt
werden. Wird ein Abschnitt automatisch beim Hinzufügen eines Eintrages
angelegt, erhält er automatisch die Priorität 500. Der Syntax zum Anlegen von
Abschnitten lautet:

\begin{example}
\begin{verbatim}
    httpd-menu.sh addsec <priority> <name>
\end{verbatim}
\end{example}

Auch hier sollte \texttt{<priority>} nur dreistellige numerische Werte annehmen.

Um sinnvolle Prioritäten in Erfahrung zu bringen, lohnt es sich, auf
einem laufenden fli4l in die Datei \texttt{/etc/httpd/menu} zu schauen, die
Prioritäten stehen in der zweiten Spalte.

Der Vollständigkeit halber wird hier kurz auf das Dateiformat der Menüdatei
eingegangen. Wem die Funktion von \texttt{httpd-menu.sh} reicht, der kann diesen
Abschnitt überspringen. Die Datei \texttt{/etc/httpd/menu} hat den folgenden
Aufbau: Sie ist in vier Spalten eingeteilt. In der ersten Spalte steht ein
Kennbuchstabe, der Überschriften und Einträge unterscheidet. In der zweiten
Spalte steht die Sortierungspriorität. Die dritte Spalte enthält bei Einträgen
das Ziel des Links und bei Überschriften einen Bindestrich, da dieses Feld bei
Überschriften keine Bedeutung hat. Im Rest der Zeile steht der Text, der
nachher im Menü erscheint.

Überschriften benutzen den Kennbuchstaben "`t"', dann wird ein neuer
Menüabschnitt begonnen. Normale Menüeinträge benutzen den Kennbuchstaben "`e"'.
Ein Beispiel:

\begin{example}
\begin{verbatim}
    t 300 - Mein tolles OPT
    e 200 meinopt.cgi Mach etwas Tolles
    e 500 meinopt.cgi?mehr=ja Mach mehr Tolles
\end{verbatim}
\end{example}

Beim Bearbeiten dieser Datei muss man darauf achten, dass das
\texttt{httpd-menu.sh}-Skript die Datei immer sortiert abspeichert. Die
einzelnen Abschnitte sind sortiert und die Einträge pro Abschnitt sind in diesem
Abschnitt sortiert. Der genaue Sortieralgorithmus kann aus
\texttt{httpd-menu.sh} übernommen werden~-- besser wäre allerdings, dieses
Skript um mögliche neue Funktionen zu erweitern, damit alle Menü-Bearbeitungen
an zentraler Stelle passieren.

\subsection{Aufbau eines CGI-Skriptes}

\subsubsection{Die Kopfzeilen}
Alle Skripte des Webservers sind einfache Shell-Skripte (Interpreter wie z.\,B.
Perl, PHP, etc.\ sind viel zu groß für fli4l). Sie sollten mit dem
obligatorischen Skript-Header anfangen (Verweis auf den Interpreter,
Name, Sinn des Skriptes, Autor, Lizenz).

\subsubsection{Das Hilfsskript "'cgi-helper"'}
Gleich nach den Kopfzeilen sollte dann schon das Hilfsskript \texttt{cgi-helper}
mit folgendem Aufruf eingebunden werden:

\begin{example}
\begin{verbatim}
    . /srv/www/include/cgi-helper
\end{verbatim}
\end{example}

Wichtig ist ein Leerzeichen zwischen Punkt und Schrägstrich!

Dieses Skript stellt diverse Hilfsfunktionen bereit, die das Erstellen von CGIs
für fli4l wesentlich vereinfachen sollen. Außerdem werden mit dem Einbinden
des \texttt{cgi-helper} auch noch Standardaufgaben ausgeführt, wie
beispielsweise das Parsen von Variablen, die mit Formularen oder über die URL
übergebenen wurden, oder das Laden von Sprach- und CSS-Dateien.

Tabelle~\ref{tab:dev:cgi-helper} gibt einen Überblick über die Funktionen des
\texttt{cgi-helper}-Skriptes.

\begin{table}[htbp]
  \centering
  \caption{Funktionen des \texttt{cgi-helper}-Skriptes}
  \label{tab:dev:cgi-helper}
  \begin{small}
    \begin{tabular}{|l|p{0.8\textwidth}|}
      \hline
      Name                         & Funktion         \\
      \hline
      \texttt{check\_rights}      & Überprüfung der Benutzerrechte \\
      \texttt{http\_header}       & Ausgabe eines Standard-HTTP-Headers oder eines speziellen HTTP-Headers, beispielsweise zum Download von Dateien\\
      \texttt{show\_html\_header} & Ausgabe des kompletten Seitenheaders (inkl. HTTP-Header, Kopfzeile und Menü)\\
      \texttt{show\_html\_footer} & Ausgabe des Abschlusses der HTML-Seite \\
      \texttt{show\_tab\_header}  & Ausgabe eines Inhalt-Fensters mit Reitern\\
      \texttt{show\_tab\_footer}  & Ausgabe des Abschlusses des Inhaltsfensters\\
      \texttt{show\_error}        & Ausgabe einer Box für Fehlermeldungen (Hintergrundfarbe: rot)\\
      \texttt{show\_warn}         & Ausgabe einer Box für Warnmeldungen (Hintergrundfarbe: gelb)\\
      \texttt{show\_info}         & Ausgabe einer Box für Informationen oder Erfolgsmeldungen (Hintergrundfarbe: grün)\\
      \hline
    \end{tabular}
  \end{small}
\end{table}

\subsubsection{Der Inhalt eines CGI-Skriptes}

Um ein einheitliches Design und vor allem die Kompatibilität
mit zukünftigen fli4l-Versionen zu gewährleisten, ist es sehr zu empfehlen, die
Funktionen des Hilfsskriptes \texttt{cgi-helper} zu benutzen, auch wenn man in
einem CGI theoretisch alle Ausgaben selbst generieren kann.

Eine einfaches CGI-Skript könnte wie folgt aussehen:

\begin{example}
\begin{verbatim}
    #!/bin/sh
    # --------------------
    # Header (c) Autor Datum
    # --------------------
    # get main helper functions
    . /srv/www/include/cgi-helper

    show_html_header "Mein erstes CGI"
    echo '   <h2>Willkommen</h2>'
    echo '   <h3>Dies ist ein Beispiel-CGI-Skript</h3>'
    show_html_footer
\end{verbatim}
\end{example}

\subsubsection{Die Funktion \texttt{show\_html\_header}}

Die Funktion \texttt{show\_html\_header} erwartet eine Zeichenkette als
Parameter. Diese Zeichenkette stellt den Titel der zu generierenden Seite dar.
Die Funktion generiert automatisch das Menü und bindet ebenso automatisch zum
Skript gehörende CSS- und Sprachdateien ein. Voraussetzung dafür ist, dass
diese sich in den Verzeichnissen \texttt{/srv/www/css}
bzw.\ \texttt{/srv/www/lang} befinden und denselben Namen (aber natürlich eine
andere Endung) wie das Skript haben. Ein Beispiel:

\begin{example}
\begin{verbatim}
    /srv/www/admin/OpenVPN.cgi
    /srv/www/css/OpenVPN.css
    /srv/www/lang/OpenVPN.de
\end{verbatim}
\end{example}

Sowohl das Benutzen von Sprachdateien als auch von CSS-Dateien ist optional.
Immer eingebunden werden \texttt{css/main.css} und \texttt{lang/main.<lang>},
wobei \texttt{<lang>} der gewählten Sprache entspricht.

Der Funktion \texttt{show\_html\_header} können aber neben dem Titel noch
weitere Parameter übergeben werden. Ein Aufruf mit allen möglichen Parametern
könnte so aussehen:

\begin{example}
\begin{verbatim}
    show_html_header "Titel" "refresh=$time;url=$url;cssfile=$cssfile;showmenu=no"
\end{verbatim}
\end{example}

Alle zusätzlichen Parameter müssen, wie im Beispiel gezeigt, mit
Anführungszeichen umschlossen und durch ein Semikolon getrennt werden. Eine
Überprüfung der Syntax erfolgt \emph{nicht}! Es ist also notwendig, sehr genau
auf die Parameterübergabe zu achten.

Hier eine kurze Übersicht über die Funktion der Parameter:

\begin{itemize}
 \item \texttt{refresh=}\emph{time}: Zeit in Sekunden in der die Seite vom
    Browser neu geladen werden soll
 \item \texttt{url=}\emph{url}: die URL, die bei einem Refresh neu geladen wird
 \item \texttt{cssfile=}\emph{cssfile}: Name einer CSS-Datei, wenn diese vom
    Namen des CGIs abweicht
 \item \texttt{showmenu=no}: unterdrückt die Anzeige des Menüs und des Headers
\end{itemize}

Sonstige Richtlinien zum Inhalt:

\begin{itemize}
 \item Fasst euch kurz :-)
 \item Schreibt sauberes HTML (SelfHTML\footnote{siehe
    \altlink{http://de.selfhtml.org/}} ist da ein guter Ansatzpunkt).
 \item Verzichtet auf hochmodernen Schnick-Schnack (JavaScript ist i.\,O., wenn
    es nicht stört, sondern den Benutzer unterstützt, das Ganze muss auch ohne
    JavaScript funktionieren).
\end{itemize}

\subsubsection{Die Funktion \texttt{show\_html\_footer}}

Die Funktion \texttt{show\_html\_footer} schließt den Block im CGI-Skript ab,
der durch die Funktion \texttt{show\_html\_header} eingeleitet wurde.

\subsubsection{Die Funktion \texttt{show\_tab\_header"'}}

Damit der Inhalt eurer mit Hilfe des CGIs erzeugten Webseite auch hübsch
geordnet aussieht, könnt ihr die \texttt{cgi-helper}-Funktion
\texttt{show\_tab\_header} nutzen. Sie erzeugt dann anklickbare Reiter ("`Tabs"'
genannt), so dass ihr eure Seite in mehrere logisch voneinander getrennte
Bereiche unterteilen könnt.

Der \texttt{show\_tab\_header}-Funktion werden Parameter immer in Paaren
übergeben. Der erste Wert entspricht dem Titel eines Reiters, der zweite dem
Link. Wird als Link die Zeichenkette "`no"' übergeben, wird lediglich der Titel
ausgegeben und der Reiter ist nicht anklickbar (und blau).

Im folgenden Beispiel wird ein "`Fenster"' mit dem Titel "`Ein tolles Fenster"'
erzeugt. Im Fenster steht "`foo bar"':

\begin{example}
\begin{verbatim}
    show_tab_header "Ein tolles Fenster" "no"
    echo "foo"
    echo "bar"
    show_tab_footer
\end{verbatim}
\end{example}

Im nächsten Beispiel werden zwei anklickbare Reiter generiert, die dem
Skript die Variable \var{action} mit verschiedenen Werten übergibt.

\begin{example}
\begin{verbatim}
    show_tab_header "1. Auswahltab" "$myname?action=machdies" \
                    "2. Auswahltab" "$myname?action=machjenes"
    echo "foo"
    echo "bar"
    show_tab_footer
\end{verbatim}
\end{example}

Nun kann das Skript den Inhalt der Variablen \var{FORM\_action} (siehe weiter
unten zur Variablenauswertung) auswerten und je nachdem unterschiedliche Inhalte
bereitstellen. Damit der angeklickte Reiter auch ausgewählt erscheint und nicht
mehr angeklickt werden kann, müsste der Funktion statt dem Link wie schon
erwähnt ein "`no"' übergeben werden. Das geht aber auch einfacher, wenn man sich
an die Konvention in folgendem Beispiel hält:

\begin{example}
\begin{verbatim}
    _opt_machdies="1. Auswahltab"
    _opt_machjenes="2. Auswahltab"
    show_tab_header "$_opt_machdies" "$myname?action=opt_machdies" \
                    "$_opt_machjenes" "$myname?action=opt_machjenes"
    case $FORM_action in
        opt_machdies) echo "foo" ;;
        opt_machjenes) echo "bar" ;;
    esac
    show_tab_footer
\end{verbatim}
\end{example}

Wird also für den Titel eine Variable verwendet, deren Namen dem Inhalt der
Variablen \var{action} mit führendem Unterstrich (\texttt{\_}) entspricht, wird
der entsprechende Reiter ausgewählt dargestellt.

\subsubsection{Die Funktion \texttt{show\_tab\_footer}}

Die Funktion \texttt{show\_tab\_footer} schließt den Block im CGI-Skript ab,
der durch die Funktion \texttt{show\_tab\_header} eingeleitet wurde.

\subsubsection{Mehrsprachfähigkeit}

Das Hilfsskript \texttt{cgi-helper} enthält weiterhin Funktionen, um CGI-Skripte
mehrsprachfähig zu machen. Dazu müssen "`nur"' für alle Textausgaben Variablen
mit führenden Unterstrichen (\texttt{\_}) verwendet und diese Variablen in den
entsprechenden Sprachdateien definiert werden.

Beispiel:

\texttt{lang/opt.de} enthalte:

\begin{example}
\begin{verbatim}
    _opt_machdies="Eine Ausgabe"
\end{verbatim}
\end{example}

\texttt{lang/opt.en} enthalte:

\begin{example}
\begin{verbatim}
    _opt_machdies="An Output"
\end{verbatim}
\end{example}

\texttt{admin/opt.cgi} enthalte:

\begin{example}
\begin{verbatim}
    ...
    echo $_opt_machdies
    ...
\end{verbatim}
\end{example}


\subsubsection{Formular-Auswertung}

Um Formulare zu verarbeiten, muss man noch einige Dinge wissen. Egal ob die
Formular-Methode \var{GET} oder \var{POST} verwendet wird, die Parameter finden
sich nach dem Einbinden des \texttt{cgi-helper}-Skripts (welches wiederum das
Hilfsprogramm proccgi aufruft) in den Variablen \var{FORM\_<Parameter>} wieder.
Wenn das Formularfeld also den Namen "`eingabe"' hatte, kann im CGI-Skript
mit \var{\$FORM\_eingabe} darauf zugegriffen werden.

Weitere Informationen zum Programm \texttt{proccgi} finden sich unter
\altlink{http://www.fpx.de/fp/Software/ProcCGI.html}.

\marklabel{sec:rights}{
    \subsubsection{Benutzerrechte: Die Funktion \texttt{check\_rights}}
}

Um zu prüfen, ob der Benutzer ausreichende Rechte zur Nutzung eines CGI-Skripts
besitzt, muss am Anfang des CGI-Skripts die Funktion \texttt{check\_rights}
wie folgt aufgerufen werden:

\begin{example}
\begin{verbatim}
    check_rights <Bereich> <Aktion>
\end{verbatim}
\end{example}

Das CGI-Skript wird dann nur ausgeführt, wenn der aktuell angemeldete Benutzer
\begin{itemize}
\item alle Rechte hat (\verb+HTTPD_RIGHTS_x='all'+), oder
\item alle Rechte für den angegebenen Bereich hat
    (\verb+HTTPD_RIGHTS_x='<Bereich>:all'+), oder
\item das Recht hat, die angegebene Aktion im angegebenen Bereich auszuführen\\
    (\verb+HTTPD_RIGHTS_x='<Bereich>:<Aktion>'+).
\end{itemize}

% More examples?

\subsubsection{Die Funktion \texttt{show\_error}}

Diese Funktion gibt eine Fehlermeldung in einer roten Box aus. Sie erwartet
zwei Parameter: einen Titel sowie eine Meldung. Beispiel:

\begin{example}
\begin{verbatim}
    show_error "Error: No key" "No key was specified!"
\end{verbatim}
\end{example}

\subsubsection{Die Funktion \texttt{show\_warn}}

Diese Funktion gibt eine Warnmeldung in einer gelben Box aus. Sie erwartet
zwei Parameter: einen Titel sowie eine Meldung. Beispiel:

\begin{example}
\begin{verbatim}
    show_info "Warnung" "Derzeit besteht keine Verbindung!"
\end{verbatim}
\end{example}

\subsubsection{Die Funktion \texttt{show\_info}}

Diese Funktion gibt eine Informations- oder Erfolgsmeldung in einer grünen Box
aus. Sie erwartet zwei Parameter: einen Titel sowie eine Meldung. Beispiel:

\begin{example}
\begin{verbatim}
    show_info "Info" "Aktion wurde erfolgreich ausgeführt!"
\end{verbatim}
\end{example}

\subsubsection{Das Hilfsskript "'cgi-helper-ip4"'}

Gleich nach den Hilfsskript "cgi-helper" kann dann das Hilfsskript cgi-helper-ip4
mit folgendem Aufruf eingebunden werden:

\begin{example}
\begin{verbatim}
. /srv/www/include/cgi-helper-ip4
\end{verbatim}
\end{example}

Wichtig ist ein Leerzeichen zwischen Punkt und Schrägstrich!

Dieses Skript stellt Hilfsfunktionen bereit, um Prüfungen von IPv4-Adressen
vornehmen zu können.

\subsubsection{Die Funktion \texttt{ip4\_isvalidaddr}}

Diese Funktion überprüft, ob eine gültige IPv4-Adresse übergeben wurde.
Beispiel:

\begin{example}
\begin{verbatim}
    if ip4_isvalidaddr ${FORM_inputip}
    then
        ...
    fi
\end{verbatim}
\end{example}

\subsubsection{Die Funktion \texttt{ipv4\_normalize}}

Diese Funktion entfernt aus der übergebenen IPv4-Adresse führende Nullen.
Beispiel:

\begin{example}
\begin{verbatim}
    ip4_normalize ${FORM_inputip}
    IP=$res
    if [ -n "$IP" ]
    then
        ...
    fi
\end{verbatim}
\end{example}

\subsubsection{Die Funktion \texttt{ipv4\_isindhcprange}}

Diese Funktion prüft, ob die übergebene IPv4-Adresse sich im Bereich der
übergebenen Start- und Zieladresse befindet. Beispiel:
 
\begin{example}
\begin{verbatim}
    if ip4_isindhcprange $FORM_inputip $ip_start $ip_end
    then
        ...
    fi
\end{verbatim}
\end{example}

\subsection{Sonstiges}

Dies und das (ja, das ist auch noch wichtig!):

\begin{itemize}
 \item Der \texttt{mini\_httpd} schützt Unterverzeichnisse nicht mit einem
    Passwort. Es muss für jedes Verzeichnis eine eigene \texttt{.htaccess}-Datei
    oder ein Link auf eine andere \texttt{.htaccess}-Datei angelegt werden.
 \item KISS - Keep it simple, stupid!
 \item Diese Angaben können sich jederzeit ohne Vorankündigung ändern!
\end{itemize}

\subsection{Fehlersuche}

Um die Fehlersuche innerhalb eines CGI-Skripts zu erleichtern, kann man vor
der Einbindung des \texttt{cgi-helper}-Skripts den Debugging-Modus aktivieren.
Dazu muss die Variable \var{set\_debug} auf den Wert "`yes"' gesetzt werden.
Dies führt zur Erstellung der Datei \texttt{debug.log}, die über die URL
\texttt{http://<fli4l-Host>/admin/debug.log} heruntergeladen werden kann.
Diese enthält alle Aufrufe des CGIs. Die \texttt{set\_debug}-Variable ist nicht
global, muss also in jedem Problem-CGI erneut gesetzt werden. Beispiel:

\begin{example}
\begin{verbatim}
    set_debug="yes"
    . /srv/www/include/cgi-helper
\end{verbatim}
\end{example}

Alternativ kann auch die jeweilige CGI-URL um den Parameter \texttt{debug=yes}
ergänzt werden, etwa \texttt{http://<fli4l-Host>/admin/meinopt.cgi?debug=yes}.

Des Weiteren eignet sich cURL\footnote{siehe
\altlink{http://de.wikipedia.org/wiki/CURL}} hervorragend zur Fehlersuche,
insbesondere wenn die HTTP-Kopfzeilen nicht korrekt zusammengesetzt werden oder
der Browser nur weiße Seiten anzeigt. Auch ist das Cachingverhalten moderner
Webbrowser bei der Fehlersuche hinderlich.

Beispiel: Mit dem folgenden Aufruf wird der HTTP-Header ("`\emph{d}ump"',
\texttt{-D}) sowie die normale Ausgabe des CGIs \texttt{admin/mein.cgi}
ausgegeben. Es soll der Benutzername ("`\emph{u}ser"', \texttt{-u}) "`admin"'
verwendet werden.

\begin{example}
\begin{verbatim}
    curl -D - http://fli4l/admin/mein.cgi -u admin
\end{verbatim}
\end{example}


% Synchronized to r38663

\section{Démarrer, arrêter, se connecter et se déconnecter avec fli4l}

\subsection{Concept de Boot}

fli4l 2.0 peut être installé sur différent média, disque-dur ou
carte-Compact-Flash(TM), il est aussi possible de l’installer sur un support
Zip ou sur un CD-ROM amorçable. En outre, l’installation d’une version sur un
disque-dur n'est pas fondamentalement différente que sur une disquette.
\footnote{ À l'origine fli4l pouvait s'exécuter à partir d'une disquette. Depuis
fli4l est devenue trop volumineux, la disquette ne peut plus est utilisée.}

Ces exigences ont été réalisé grâce à l’archive \texttt{opt.img}, jusqu'ici il était
installé sur un disque-RAM maintenant il peut être placé sur d’autre média.
Il peut s'agir d'une partition sur un disque dur ou une carte-CF. Pour ce second
volume, le répertoire \texttt{/opt} sera monté et les programmes auront des liens symboliques
et seront intégrés dans le rootfs. La structure apparaissant dans le système de fichier
RootFS correspond au répertoire \texttt{opt} de la distribution fli4l à une exception prète~--
le préfixe des \texttt{fichiers} seront omis. Le fichier \texttt{opt/etc/rc} se trouve
directement dans \texttt{/etc/rc} et pour le fichier \texttt{opt/bin/busybox} il est
dans \texttt{/bin/busybox}. Ces fichiers sont seulement des liens dans le média monté en
lecture seule, on peut les ignorer, tant que les fichiers ne sont pas modifiés. Si vous
voulez les modifier, il faut d'abord rendre ces fichiers accessibles en écriture
avec la commande \texttt{mk\_writable} (voir ci-dessous).


\subsection{Scripts de démarrage et d'arrêt}

Les Scripts qui sont exécutés lors du boot système, sont dans les répertoires
\texttt{opt/etc/boot.d} et \texttt{opt/etc/rc.d} ils sont également exécutés
dans l'ordre.

\wichtig{Quand ces scripts sont exécutés aucun processus particulier
n'est produit, ils ne peuvent pas être arrêtés avec la commande \og{}exit\fg{}.
Cette commande conduirait à une rupture du processus de Boot~!}


\subsubsection{Scripts de démarrage dans \texttt{opt/etc/boot.d/}}

    Les scripts de ce répertoire sont exécutés les premiers. Ils ont pour mission,
	de monter le périphérique de boot, le fichier de configuration \texttt{rc.cfg}
	se trouve sur le support de boot et décompresse l'archive \texttt{opt.img}. Selon
	le \jump{BOOTTYPE}{type de Boot} il est plus ou moins complexe et font les choses
	suivantes~:

\begin{itemize}
\item Charge les pilotes du matériels (optionnel)
\item Monte le volume de boot (optionnel)
\item Le fichier de configuration \texttt{rc.cfg} est lu depuis le volume de boot
  et sera écrit dans \texttt{/etc/rc.cfg}
\item Monte le volume Opt (optionnel)
\item Extrait les archives Opt (optionnel)
\end{itemize}

Lors de la construction des scripts, vous avez la chance d'en apprendre davantage sur
la configuration de fli4l, le fichier de configuration est également intégré dans
l'archive rootfs, dans ce fichier \texttt{/etc/rc.cfg} vous trouverez les variables
de configuration qui seront analysées puis les scripts de démarrage seront exécutés
depuis le répertoire \texttt{opt/etc/boot.d/}. Après le montage du volume de boot,
le fichier \texttt{/etc/rc.cfg} est remplacé par le fichier de configuration dans le
volume de boot, de sorte que les scripts de démarrage dans \texttt{opt/etc/rc.d/} soit
disponible pour l'actuel configutation du volume de boot (voir ci-dessous).
\footnote{Normalement, ces deux fichiers sont identiques. Un changement n'est possible
que si le fichier de configuration sur le volume de démarrage a été modifié manuellement,
par exemple pour modifier une configuration qui sera utilisé plus tard, sans avoir à
reconstruire l'archive fli4l.}


\subsubsection{Scripts de démarrage dans \texttt{opt/etc/rc.d/}}

    C'est les commandes qui sont exécutées à chaque démarrage du routeur, elles peuvent
	être stockés dans le répertoire \texttt{opt/etc/rc.d/}. Les conventions suivantes
	s'appliquent~:

    \begin{enumerate}
    \item Il faut classer les noms de script comme ceci~:

\begin{example}
\begin{verbatim}
    rc<nombre à trois chiffres>.<nom de l'OPT>
\end{verbatim}
\end{example}

    Les scripts sont démarrés dans l'ordre croissant des numéros.
    Si plusieurs scripts ont le même numéro attribué, c’est le caractère
    alphabétique après le point qui détermine l'ordre. L’installation des
    paquetages s’effectue les uns après les autres, ils sont définis par un numéro.

          Voici une estimation approximative, des numéros pouvant être utilisé
          pour une installation~:

          \begin{table}[htbp]
          \centering
          \begin{tabular}{ll}
                  \hline
                  Numéro        &       Fonction \\
                  \hline
                  \hline
                  000-099       &       Système de base (hardware, fuseau horaire, système de fichiers) \\
                  100-199       &       Module Kernel (drivers) \\
                  200-299       &       Connexions externe (PPPoE, ISDN4Linux, PPtP) \\
                  300-399       &       Réseau (routage, interface, filtrage de paquet) \\
                  400-499       &       Serveur (DHCP, HTTPD, Proxy, etc.) \\
                  500-900       &       Tout le reste \\
                  900-997       &       Tout ce qui peut générer un dial-up \\
                  998-999       &       Réservé (ne pas utiliser!) \\
                  \hline
          \end{tabular}
          \end{table}

    \item Vous devez \emph{placer} dans ces scripts, toutes les fonctions
    nécessaires pour changer le RootFS. (par ex. pour la création d'un
    répertoire \texttt{/var/log/lpd}).

    \item Vous ne \emph{devez} pas effectuer d'écriture dans les fichiers
    scripts qui font partie de l'archive-opt, car ces fichiers sont en
    lecture seule sur le média. Pour modifier de tel fichier, il faut, au
    préalable rendre accessible ce fichier en écriture via \texttt{mk\_writable}
	(voir ci-dessous). En exécutant cette fonction, le fichier si nécessaire,
	sera copié et sera accessible en écriture sur le RootFS. Si le fichier est
	déjà en écriture, rien ne se passera lors de l’exécution de la fonction \texttt{mk\_writable}.

    \wichtig{\texttt{mk\_writable} doit être utilisé sur les fichiers qui sont dans le
    rootfs et non pas par le biais du dossier \texttt{/opt}. Si vous voulez modifier
    \texttt{/usr/local/bin/foo} vous exécutez \texttt{mk\_writable /usr/local/bin/foo}.}

    \item Ces scripts ont besoin de vérifiés, avant d'exécuter les commandes
    réelles, si l'OPT correspondant est actif. Cela se fait habituellement par
	une simple distinction de cas~:

\begin{example}
\begin{verbatim}
    if [ "$OPT_<OPT-Name>" = "yes" ]
    then
        ...
        # ici OPT start!
        ...
    fi
\end{verbatim}
\end{example}

    \item Pour pouvoir déboguer plus facile, vous devez inséter dans le script
      les fonctions \texttt{begin\_script} et \texttt{end\_script}~:

\begin{example}
\begin{verbatim}
    if [ "$OPT_<OPT-Name>" = "yes" ]
    then
        begin_script FOO "configuring foo ..."
        ...
        end_script
    fi
\end{verbatim}
\end{example}

    Pour juste déboguer de script au démarrage, vous devez simplement activer
    la variable \verb+FOO_DO_DEBUG='yes'+.

  \item Toutes les variables de configuration sont directement disponible par les scripts.
	Des explications pour accéder à d'autres scripts à partir des variables de configuration
	peuvent être trouvés dans la section \jump{dev:sec:config-variables}{\og{}Gestion
    des variables de configuration\fg{}}

  \item Le dossier \texttt{/opt} ne doit pas être utilisé comme stocker des données OPTs.
	Si de l'espace supplémentaire est nécessaire, l'utilisateur doit de définir un chemin
	approprié en utilisant la variable de configuration. Selon le type de données à stocker
	(données persistante ou transitoire) vous devez utiliser différentes affectations par défaut.
	Le chemin \texttt{/var/run/} est logique pour les données transitoires, tandis que pour
	les données persistant, il est conseillé d'utiliser la fonction
	\jump{dev:sec:persistent-data}{map2persistent} combiné avec une variable de configuration
	appropriée.
    \end{enumerate}


\subsubsection{Scripts d'arrêt dans \texttt{opt/etc/rc0.d/}}

Chaque ordinateur a besoin d’un temps pour s’arrêter ou redémarrer. Il se
pourrait bien que vous pouvez avoir à effectuer des opérations avant que
l'ordinateur s'éteigne ou redémarre. Les commandes officielles sont
\og{}halt\fg{} et \og{}reboot\fg{}. Ces commandes sont également placées
dans IMONC ou dans le Web-GUI si vous l’avez installé, un clique sur
le bouton suffira pour arrêter ou redémarrer le routeur.

Tous les scripts d'arrêt se trouvent dans le répertoire \texttt{opt/etc/rc0.d/}.
Le nom du fichier est analogue à celui du script de démarrage. Ils sont également exécutées
dans ordre \emph{croissant} des numéros.


\subsection{Fonctions auxiliaires}

Dans \texttt{/etc/boot.d/base-helper} différentes fonctions sont mises à disposition,
elles peuvent être utilisées pour les scripts de démarrage. Cela se applique pour
certaines choses comme pour un support de débogage, pour le chargement des modules
du Kernel ou pour la sortie des messages.

Les différentes fonctions sont les suivantes et brièvement décrites.


\subsubsection{Contrôle des scripts}

\begin{description}

\item[\texttt{begin\_script <Symbol> <Message>}~:] envoie un message et active
le débogueur de script en utilisant \texttt{set -x}, si \var{<Symbol>\_DO\_DEBUG}
est sur \og{}yes\fg{}.

\item[\texttt{end\_script}~:] envoie un message final et fournit l'état de débogage
si vous avez activé le \texttt{begin\_script}. Pour chaque activation du \texttt{begin\_script}
un \texttt{end\_script} sera associé et activé (et vice versa).

\end{description}


\subsubsection{Chargement des modules du Kernel}

\begin{description}

\item[\texttt{do\_modprobe [-q] <Module> <Paramètre>*}:]
Charge le module du Kernel, y compris ses paramètres, en résolvant en même temp
les dépendances du module. Le paramètre \og{}-q\fg{} empêche qu'un message erreur
soit mis. La fonction retourne en cas de succès une valeur nulle, dans le cas
d'une erreur une valeur non nulle. Cela permet décrire un code pour gérer les échecs
lors du chargement les modules du Kernel~:

\begin{example}
\begin{verbatim}
    if do_modprobe -q acpi-cpufreq
    then
        # pas de contrôle de fréquence du CPU via ACPI
        log_error "le contrôle de fréquence du CPU via ACPI n'est pas disponible."
        # [...]
    else
        log_info "le contrôle de fréquence du CPU via ACPI est activé."
        # [...]
    fi
\end{verbatim}
\end{example}

\item[\texttt{do\_modrobe\_if\_exists [-q] <Chemin> <Module> <Paramètre>*}~:]\mbox{}\\
vérifie d'abord si le module \texttt{/lib/modules/<version du kernel>/<Chemin>/<Module>}
existe et ensuite exécute \texttt{do\_modprobe}.

\wichtig{Le module doit exister précisément par ce nom, aucun alias ne peuvent être utilisé.
Lorsque vous utilisez un alias \texttt{do\_modprobe} sera exécuté immédiatement.}

\end{description}


\subsubsection{Messages et gestion des erreurs}

\begin{description}

\item[\texttt{log\_info <Message>}~:] envoi un message sur la console
et dans \texttt{/bootmsg.txt}. Si aucun message n'est enregistré dans ce paramètre
le fichier \texttt{log\_info} sera lu depuis l'entrée par défaut. La fonction renvoie
toujours en retour une valeur nulle.

\item[\texttt{log\_warn <Message>}~:] envoi un message d'avertissement sur la console
et dans \texttt{/bootmsg.txt}, la chaîne \texttt{WARN:} sera utilisée comme préfixe.
Si aucun message n'est enregistré dans ce paramètre le fichier \texttt{log\_warn}
sera lu depuis l'entrée par défaut. La fonction renvoie toujours en retour une valeur null.

\item[\texttt{log\_error <Message>}~:] envoi un message d'erreur sur la console
et dans \texttt{/bootmsg.txt}, la chaîne \texttt{ERR:} sera utilisée comme préfixe.
Si aucun message n'est enregistré dans ce paramètre le fichier \texttt{log\_error}
sera lu depuis l'entrée par défaut. La fonction renvoie toujours en retour une valeur null.

\item[\texttt{set\_error <Message>}~:] envoi un message d'erreur qui sera définit dans
  une variable d'erreur interne, ensuite elle pourra être examiné via \texttt{is\_error}.

\item[\texttt{is\_error}~:] réinitialise la variable d'erreur interne et renvoie true,
si elle a été précédemment défini par \texttt{set\_error}.

\end{description}


\subsubsection{Fonctions réseau}

\begin{description}

\item[\texttt{translate\_ip\_net <Valeur> <Nom de Variable> [<Variable de Résultat>]}~:]\mbox{}\\
remplace les références symboliques par des paramètres. Actuellement les traductions
suivantes ont lieu~:
\begin{description}
\item[*.*.*.*, \texttt{none}, \texttt{default}, \texttt{pppoe}] ne sont pas remplacé
\item[\texttt{any}] sera remplacé par \texttt{0.0.0.0/0}
\item[\texttt{dynamic}] sera remplacée par une adresse IP du routeur, à travers laquel
il existe une connexion Internet.
\item[\var{IP\_NET\_x}] sera remplacé par le réseau se trouvant dans la configuration.
\item[\var{IP\_NET\_x\_IPADDR}] sera remplacé par l'adresse IP se trouvant dans
la configuration.
\item[\var{IP\_ROUTE\_x}] sera remplacé par le réseau routé se trouvant dans la configuration
\item[\texttt{@<Hostname>}] sera remplacé par l'hôte ou par l'adresse IP se trouvant dans
la configuration.
\end{description}

Le résultat de la traduction est mémorisée dans la variable dont le nom est transmis dans
le troisième paramètre, si ce paramètre est manquant, le résultat sera stocké dans la variable 
\var{res}. Le nom de la variable qui est transmis dans le second paramètre est utilisé
uniquement pour les messages d'erreur si la traduction échoue, cela permet d'appeler la source
de la valeur à traduire. Si une erreur se produit, un message est alors exécuté.

\begin{example}
\begin{verbatim}
    Unable to translate value '<Valeur>' contained in <Nom de Variable>.
\end{verbatim}
\end{example}

La valeur nulle est retourée si la traduction a réussie et une valeur non nulle sera retounée
si une erreur s'est produite.

\end{description}


\subsubsection{Divers}

\begin{description}

\item[\texttt{mk\_writable <Fichier>}~:] pour vous assurer que le fichier transmis
sera accessible en écriture. Seulement si le fichier est en lecture seule dans
le fichier-système et juste monté par un lien symbolique, une copie locale
sera alors crée pour avoir le fichier en écrit.

\item[\texttt{list\_unique <Liste>}~:] supprime les doublons dans la liste transmise.
Le résultat est écrit dans la sortie standard.

\end{description}

\subsection{Règles mdev}

Il est possible d'établir des règles mdev supplémentaires qui exécutent des
actions spécifiques pour l'apparition ou la disparition de certains dispositifs
dans les paquetages OPT. Par exemple la variable \var{OPT\_AUTOMOUNT} dans le
paquetage hd, utilise une telle règle pour monter automatiquement un nouveau
support de stockage émergentes. Si vous souhaitez intégrer une règle mdev
supplémentaire, vous pouvez utiliser le script sous la forme~:

\begin{verbatim}
    /etc/mdev.d/mdev<Nummer>.<Name>
\end{verbatim}

Installation du rootFS, le nombre doit être similaire au début et à la fin du script
et constitué de trois chiffres, le nom peut être choisi arbitrairement. Dans ce
scénario, toutes les sorties sont intégrés dans le fichier standard
\texttt{/etc/mdev.conf}. Exemple avec \var{OPT\_AUTOMOUNT} mentionné ci-dessus~:

\begin{small}
\begin{verbatim}
#!/bin/sh
#----------------------------------------------------------------------------
# /etc/mdev.d/mdev500.automount - mdev HD automounting rules     __FLI4LVER__
#
#
# Last Update:  $Id$
#----------------------------------------------------------------------------

cat <<"EOF"
#
# mdev500.automount
#

-SUBSYSTEM=block;DEVTYPE=partition;.+       0:0 660 */lib/mdev/automount

EOF
\end{verbatim}
\end{small}

L'en-tête de la syntaxe des règles du fichier \texttt{/etc/mdev.conf}
et la documentation mdev, se trouve sur la page Web
\altlink{http://git.busybox.net/busybox/plain/docs/mdev.txt} tout y est référencées.
Si le script invoque une règle (comme dans l'exemple ci-dessus \texttt{/lib/mdev/automount}),
il déclenchera un accès à toutes les variables "événements" du kernel et en particulier~:

\begin{itemize}
\item \var{ACTION} (Typiquement \texttt{add} ou \texttt{remove}, plus rarement \texttt{change})
\item \var{DEVPATH} (le chemin sysfs pour le composant concerné)
\item \var{SUBSYSTEM} (le sous-système du kernel affecté, voir ci-dessous)
\item \var{DEVNAME} (le fichier de périphérique affecté dans \texttt{/dev}, si manquant,
					le périphérique ne sera pas créé ou supprimé, mais par exemple un module)
\item \var{MDEV} (est fixé par mdev le nom du fichier du périphérique sera finalement généré)
\end{itemize}

Exemple de sous-systèmes du Kernel~:

\begin{description}
\item[block] Bloque les périphériques (carte mémoire) comme \texttt{sda} pour le (premier
			 disque dur), \texttt{sr0} pour le (premier lecteur de CD) ou \texttt{ram1}
			 pour le (deuxième disque RAM)
\item[input] Périphériques d'entrée de clavier, de la souris, etc., comme le fichier
			 \texttt{input/event0} qui correspond à un périphérique, il doit être défini
			 avec sysfs c'est un système de fichiers virtuel.
\item[mem]   Périphériques pour accéder aux ports de la mémoire et du matériel, comme
			 \texttt{mem} et \texttt{ports}, il peut également comprendre les pseudo-périphériques
			 comme \texttt{zero} (délivre en permanence le caractère ASCII par la valeur null)
			 et \texttt{null} (pas de retours, tout sera absorbé)
\item[sound] Divers Périphériques pour les sorties audio, appellation hétérogène
\item[tty]   Périphériques pour l'accès aux consoles physiques ou virtuels, comme \texttt{tty1}
			 (première console virtuelle) ou \texttt{ttyS0} (première console série)
\end{description}

Un exemple pour les deux premiers ports série~:

\begin{scriptsize}
\begin{verbatim}
mdev[42]: 30.050644 add@/devices/pnp0/00:04/tty/ttyS0
mdev[42]:   ACTION=add
mdev[42]:   DEVPATH=/devices/pnp0/00:04/tty/ttyS0
mdev[42]:   SUBSYSTEM=tty
mdev[42]:   MAJOR=4
mdev[42]:   MINOR=64
mdev[42]:   DEVNAME=ttyS0
mdev[42]:   SEQNUM=613

mdev[42]: 30.051477 add@/devices/platform/serial8250/tty/ttyS1
mdev[42]:   ACTION=add
mdev[42]:   DEVPATH=/devices/platform/serial8250/tty/ttyS1
mdev[42]:   SUBSYSTEM=tty
mdev[42]:   MAJOR=4
mdev[42]:   MINOR=65
mdev[42]:   DEVNAME=ttyS1
mdev[42]:   SEQNUM=614
\end{verbatim}
\end{scriptsize}

Un exemple pour connecter un clavier MF II~:

\begin{scriptsize}
\begin{verbatim}
mdev[41]: 4.030653 add@/devices/platform/i8042/serio0/input/input0
mdev[41]:   ACTION=add
mdev[41]:   DEVPATH=/devices/platform/i8042/serio0/input/input0
mdev[41]:   SUBSYSTEM=input
mdev[41]:   PRODUCT=11/1/1/ab41
mdev[41]:   NAME="AT Translated Set 2 keyboard"
mdev[41]:   PHYS="isa0060/serio0/input0"
mdev[41]:   PROP=0
mdev[41]:   EV=120013
mdev[41]:   KEY=4 2000000 3803078 f800d001 feffffdf ffefffff ffffffff fffffffe
mdev[41]:   MSC=10
mdev[41]:   LED=7
mdev[41]:   MODALIAS=input:b0011v0001p0001eAB41-e0,1,4,11,14,k71,72,73,74,75,76,77,79,
7A,7B,7C,7D,7E,7F,80,8C,8E,8F,9B,9C,9D,9E,9F,A3,A4,A5,A6,AC,AD,B7,B8,B9,D9,E2,ram4,l0,
1,2,sfw
mdev[41]:   SEQNUM=604
\end{verbatim}
\end{scriptsize}

Un exemple pour charger un module kernel USB (\texttt{uhci\_hcd})~:

\begin{scriptsize}
\begin{verbatim}
mdev[41]: 6.537506 add@/module/uhci_hcd
mdev[41]:   ACTION=add
mdev[41]:   DEVPATH=/module/uhci_hcd
mdev[41]:   SUBSYSTEM=module
mdev[41]:   SEQNUM=633
\end{verbatim}
\end{scriptsize}

Un exemple pour connecter un disque dur~:

\begin{scriptsize}
\begin{verbatim}
mdev[41]: 7.267527 add@/devices/pci0000:00/0000:00:07.1/ata1/host0/target0:0:0/0:0:0:0/block/sda
mdev[41]:   ACTION=add
mdev[41]:   DEVPATH=/devices/pci0000:00/0000:00:07.1/ata1/host0/target0:0:0/0:0:0:0/block/sda
mdev[41]:   SUBSYSTEM=block
mdev[41]:   MAJOR=8
mdev[41]:   MINOR=0
mdev[41]:   DEVNAME=sda
mdev[41]:   DEVTYPE=disk
mdev[41]:   SEQNUM=688
\end{verbatim}
\end{scriptsize}

Ci-dessus est un disque dur ATA/IDE (ata1), le nom du périphérique \texttt{sda} sera utilisé
pour le traitement.

Un exemple pour un périphérique de blocage à distance (affecte le marquage d'un fichier d'image
de VM-fli4l et a été résolu via \texttt{virsh detach-device})~:

\begin{scriptsize}
\begin{verbatim}
mdev[42]: 52.600646 remove@/devices/pci0000:00/0000:00:0a.0/virtio5/block/vdb/vdb1
mdev[42]:   ACTION=remove
mdev[42]:   DEVPATH=/devices/pci0000:00/0000:00:0a.0/virtio5/block/vdb/vdb1
mdev[42]:   SUBSYSTEM=block
mdev[42]:   MAJOR=254
mdev[42]:   MINOR=17
mdev[42]:   DEVNAME=vdb1
mdev[42]:   DEVTYPE=partition
mdev[42]:   SEQNUM=776

mdev[42]: 52.644642 remove@/devices/virtual/bdi/254:16
mdev[42]:   ACTION=remove
mdev[42]:   DEVPATH=/devices/virtual/bdi/254:16
mdev[42]:   SUBSYSTEM=bdi
mdev[42]:   SEQNUM=777

mdev[42]: 52.644718 remove@/devices/pci0000:00/0000:00:0a.0/virtio5/block/vdb
mdev[42]:   ACTION=remove
mdev[42]:   DEVPATH=/devices/pci0000:00/0000:00:0a.0/virtio5/block/vdb
mdev[42]:   SUBSYSTEM=block
mdev[42]:   MAJOR=254
mdev[42]:   MINOR=16
mdev[42]:   DEVNAME=vdb
mdev[42]:   DEVTYPE=disk
mdev[42]:   SEQNUM=778

mdev[42]: 52.644777 remove@/devices/pci0000:00/0000:00:0a.0/virtio5
mdev[42]:   ACTION=remove
mdev[42]:   DEVPATH=/devices/pci0000:00/0000:00:0a.0/virtio5
mdev[42]:   SUBSYSTEM=virtio
mdev[42]:   MODALIAS=virtio:d00000002v00001AF4
mdev[42]:   SEQNUM=779

mdev[42]: 52.644973 remove@/devices/pci0000:00/0000:00:0a.0
mdev[42]:   ACTION=remove
mdev[42]:   DEVPATH=/devices/pci0000:00/0000:00:0a.0
mdev[42]:   SUBSYSTEM=pci
mdev[42]:   PCI_CLASS=10000
mdev[42]:   PCI_ID=1AF4:1001
mdev[42]:   PCI_SUBSYS_ID=1AF4:0002
mdev[42]:   PCI_SLOT_NAME=0000:00:0a.0
mdev[42]:   MODALIAS=pci:v00001AF4d00001001sv00001AF4sd00000002bc01sc00i00
mdev[42]:   SEQNUM=780
\end{verbatim}
\end{scriptsize}

Comme vous pouvez le voir, plusiers extration du sous-système du kernel ont été impliquès
(avec \texttt{block}, \texttt{bdi}, \texttt{virtio} et \texttt{pci}).

\subsection{Périphériques ttyI}

Au sujet les périphériques ttyI, vous pouvez utiliser (\texttt{/dev/ttyI0} \ldots \texttt{/dev/ttyI15}),
pour la création d’un \og{}émulateur de modem\fg{} avec plusieurs cartes ISDN (ou RNIS),
il existe un compteur pour les conflits entre les différents OPTs et les utilisations
de ces périphériques, c'est un dispositif à éviter. Ces émulateur seront créés lors
du démarrage du routeur, dans le fichier \texttt{/var/run/next\_ttyI}, avec l'utilisation
un compteur. Dans l'exemple de script suivant, la valeur est interrogé et peut être augmentée
de un, il sera exporté de nouveau dans le prochain OPT.

\begin{example}
\begin{verbatim}
    ttydev_error=
    ttydev=$(cat /var/run/next_ttyI)
    if [ $ttydev -le 16 ]
    then                                    # ttyI device available? yes
        ttydev=$((ttydev + 1))              # ttyI device + 1
        echo $ttydev >/var/run/next_ttyI    # save it
    else                                    # ttyI device available? no
        log_error "No ttyI device for <Nom de votre OPT> available!"
        ttydev_error=true                   # set error for later use
    fi

    if [ -z "$ttydev_error" ]               # start OPT only if next tty device
    then                                    # was available to minimize error
        ...                                 # messages and minimize the
                                            # risk of uncomplete boot
    fi
\end{verbatim}
\end{example}

\subsection{Scripts de connexion et de déconnexion par modem}


\subsubsection{Généralités}

Après avoir établi ou coupé une connexion par modem, les scripts \texttt{/etc/ppp/} sont
traitées. Voici les actions qui sont nécessaires pour activer ou déactiver une connexion,
qui seront stocker dans l'OPT. Le schéma des noms de fichier est le suivant~:

\begin{table}[htbp]
\centering
\begin{tabular}{l}
    \texttt{ip-up<numéro à trois chiffres>.<nom d'OPT>}\\
    \texttt{ip-down<numéro à trois chiffres>.<nom d'OPT>}\\
\end{tabular}
\end{table}

Le script \texttt{ip-up} est exécuté après la \emph{connexion} et le script
\texttt{ip-down} est exécuté après la \emph{déconnexion}

\wichtig{Dans le script \texttt{ip-down} aucune intervention ne doit être réalisé,
autrement cela conduirait à une nouvelle connexion Internet, avec uniquement un accès
à l’état online permanent, pour les utilisateurs qui non pas de forfait illimité,
cela peut couter très cher.}

\wichtig{Car pour ce script, il n’y a aucun processus qui est généré, en plus, ce
script ne peut \emph{pas} être arrêté avec la commande \og{}exit\fg{}!}

\textbf{Remarque~:} le scripts \texttt{ip-up} peut être examiné lors qu'il est exécuté,
pour cela le fichier \texttt{rc400} sera vérifier avec la variable \var{ip\_up\_events}.
Si c'est réglé sur "yes", une connexion par modem existe et le script \texttt{ip-up} sera
exécuté. Si c'est réglé sur "no", une connexion par modem n'existe pas et le script
\texttt{ip-up} ne sera pas exécuté. Il y a une exception pour cette règle~: Si un
routeur Ethernet pur n'est pas configuré pour des connexions commutées, mais configuré
pour une route par défaut (\texttt{0.0.0.0/0}), le script \texttt{ip-up} sera exécuté qu'une
fois, exactement à la fin du processus de boot. (De même le script \texttt{ip-down} sera
exécuté qu'une fois avant l'arrêt du routeur).


\subsubsection{Les variables}

Grâce au concept d'appel spécial les scripts \texttt{ip-up} et \texttt{ip-down} sont exécutés
et les variables suivantes sont utilisées~:

\begin{table}[htbp]
\centering
\begin{tabular}{lp{10cm}}
    \var{real\_interface}    & L'interface actuelle, par ex.
                               \texttt{ppp0}, \texttt{ippp0}, \ldots\\
    \var{interface}          & Interface-IMOND, avec \texttt{pppoe},
                               \texttt{ippp0}, \ldots\\
    \var{tty}                & Terminal de connexion, peut-être vide~!\\
    \var{speed}              & Vitesse de connexion, par ex. avec ISDN 64000\\
    \var{local}              & Adresse-IP spécifique\\
    \var{remote}             & Adresse-IP d'ordinatuer auquel vous êtes connecté\\
    \var{is\_default\_route} & Indique si l'actuel \texttt{ip-up}/\texttt{ip-down}
								est utilisé pour l'interface de la route par défaut
								peut-être \og{}yes\fg{} ou \og{}no\fg{})\\
\end{tabular}
\end{table}


\subsubsection{Route par défaut}

Depuis la version 2.1.0, les scripts \texttt{ip-up} et \texttt{ip-down} sont exécutés
non seulement pour l'interface sur laquelle la route par défaut est configurée, mais
aussi pour toutes les connexions qui ont besoin les scripts \texttt{ip-up} et \texttt{ip-down}.
Pour émuler des comportements anciens, vous devez inclure les éléments à déclencher
dans les scripts \texttt{ip-up} et \texttt{ip-down} la requête suivante doit être insérée~:

\begin{example}
\begin{verbatim}
    # is a default-route-interface going up?
    if [ "$is_default_route" = "yes" ]
    then
        # Les actions à déclencher
    fi
\end{verbatim}
\end{example}

Naturellement, les nouveaux comportements doivent être utilisés, pour des actions spécifiques.



\section{Paquetage Template}

Pour illustrer quelques-uns des objectifs décrits ci-dessus, vous avez un
paquetage avec des modèles pour la distribution fli4l. Dans ce paquetage vous avez
une série de petits exemples, tels que~:

\begin{itemize}
\item Voir un fichier config dans (\texttt{config/template.txt})
\item Un fichier de contrôle qui est écrit dans (\texttt{check/template.txt})
\item L’extension des fonctions de contrôle dans (\texttt{check/template.ext})
\item Des variables de configuration pour une utilisation ultérieure
  stockés dans\\ (\texttt{opt/etc/rc.d/rc999.template})
\item Des variables de configuration pour être lu à nouveau stockés dans\\
  (\texttt{opt/usr/bin/template\_show\_config})
\end{itemize}

\section{Construction du Boot sur un support de données}

Depuis de la version 1.5 fli4l utilise le programme \texttt{syslinux} pour booter.
Il a l'avantage d'avoir un système de fichier DOS compatible sur le support
de données.

Le support de données pour le boot contient les fichiers suivants~:

\begin{table}[htbp]
\centering
\begin{tabular}[h!]{lp{10cm}}
\texttt{ldlinux.sys}           & Le chargeur (\og{}Boot loader\fg{}) \texttt{syslinux} \\
\texttt{syslinux.cfg}          & Fichier de configuration pour \texttt{syslinux} \\
\texttt{kernel}                & Linux-Kernel\\
\texttt{rootfs.img}            & RootFS: contient les programmes nécessaires pour le Boot \\
\texttt{opt.img}               & Fichier Optionnel~: drivers et Opt-Paquetage \\
\texttt{rc.cfg}                & Fichier de configuration des variables utilisées depuis
le répertoire config \\
\texttt{boot.msg}              & Texte pour le menu de démarrage \texttt{syslinux} \\
\texttt{boot\_s.msg}           & Texte pour le menu de démarrage \texttt{syslinux} \\
\texttt{boot\_z.msg}           & Texte pour le menu de démarrage \texttt{syslinux} \\
\texttt{hd.cfg}                & Fichier de configuration pour l'attribution des partitions \\
\end{tabular}
\end{table}

Au démarrage du script \texttt{mkfli4l.sh} (ou \texttt{mkfli4l.bat} pour DOS) les fichiers
\texttt{opt.img}, \texttt{syslinux.cfg}, \texttt{rc.cfg}, ainsi que \texttt{rootfs.img}
sont d'abord exécutés. Le programme \var{mkfli4l} exécute les fichiers nécessaires des
(répertoires \texttt{unix} ou \texttt{windows} pour l'installation. Dans les deux archives,
le Kernel et les autres paquetages à installer sont inclus. Le fichier \texttt{rc.cfg} se trouve
à la fois dans l'archive opt et sur le boot disque (ou disque de démarrage).\footnote{Le
contenue du fichier dans l'archive opt est nécessaire au début de la phase de boot, car à ce
moment là le volume de boot n'est pas monté.}

Ensuite, l’ensemble des fichiers du Kernel, \texttt{rootfs.img}, \texttt{opt.img} et \texttt{rc.cfg},
sont copiés avec le fichier \texttt{syslinux} sur le support de données.

Lors du boot de fli4l le script \texttt{rc.cfg} dans \texttt{/etc/rc} est analisé
et l'archive \texttt{opt.img} compressée est intégrée à la racine du système de fichiers
du disque virtuel (selon le type d'installation, les fichiers seront décompressés directement
à la racine du système de fichiers du disque virtuel ou vers le lien symbolique inclus).
Pour terminer, les scripts dans le répertoire \texttt{/etc/rc.d} sont exécutés dans l'ordre
alphanumérique, ensuite les pilotes sont chargés et les services démarrent.

\section{Fichiers de configurations}

Voici les dossiers présélectionnés par le routeur fli4l on-the-fly, qui sont générés
au moment du boot.

\begin{enumerate}
\item Configuration du fournisseur
  \begin{itemize}
  \item         \texttt{etc/ppp/pap-secrets}

  \item         \texttt{etc/ppp/chap-secrets}
  \end{itemize}

\item Configuration du DNS
  \begin{itemize}
  \item         \texttt{etc/resolv.conf}

  \item         \texttt{etc/dnsmasq.conf}

  \item         \texttt{etc/dnsmasq\_dhcp.conf}

  \item         \texttt{tc/resolv.dnsmasq}
  \end{itemize}

\item Fichier hôte
  \begin{itemize}
  \item         \texttt{etc/hosts}
  \end{itemize}

\item Configuration de imond
  \begin{itemize}
  \item         \texttt{etc/imond.conf}
  \end{itemize}
\end{enumerate}


\subsection{Configuration du fournisseur}

Pour paramètrer l'ID de l'utilisateur et le mot de passe pour le fournisseur d'accés
cela doit se fait dans le fichier \texttt{etc/ppp/pap-secrets}.

Exemple pour le fournisseur Planet-Interkom~:

\begin{example}
\begin{verbatim}
# Secrets pour l'authentification en utilisant PAP
# client        server      secret           IP addresse
"anonymer"      *           "surfer"         *
\end{verbatim}
\end{example}

Dans cette exemple, l'ID de l'utilisateur est \og{}anonymer\fg{}. l'accés au serveur
distant sera permis à tous (avec \og{}*\fg{}). \og{}surfer\fg{} est le mot de passe
pour le fournisseur Planet-Interkom.


\subsection{Configuration DNS}

Vous pouvez utiliser le routeur fli4l comme un serveur DNS.
Pourquoi cette fontion est utile et même obligatoire dans un LAN avec des ordinateurs
Windows~? Tout est expliqués dans la documentation du paquetage \og{}base\fg{}.

Le fichier résolveur \texttt{etc/resolv.conf} contient les noms de domaine et
les utilisateurs du serveur de nom. Vous pouvez voir ci-dessous le contenu
(du \og{}domain.de\fg{}, vous avez seulement un espace réservé pour indiquer les
paramètres c'est dans la variable de configuration \var{DOMAIN\_\-NAME})~:

\begin{example}
\begin{verbatim}
        search domain.de
        nameserver 127.0.0.1
\end{verbatim}
\end{example}

Le serveur de noms dnsmasq est configuré en utilisant le fichier etc/dnsmasq.conf.
Il boot à partir du script rc040.dns-local et du rc370.dnsmasq qui sont générés
automatiquement cela pourrait ressembler à ceci~:

\begin{example}
\begin{verbatim}
user=dns
group=dns
resolv-file=/etc/resolv.dnsmasq
no-poll
no-negcache
bogus-priv
log-queries
domain-suffix=lan.fli4l
local=/lan.fli4l/
domain-needed
expand-hosts
filterwin2k
conf-file=/etc/dnsmasq_dhcp.conf
\end{verbatim}
\end{example}


\subsection{Fichier hôte}

    Ce fichier contient l'ensemble des noms d'hôtes avec leurs adresses IP.
	Donc, ce classement est applicable seulement pour un fli4l local, pour les autres
	ordinateurs dans le LAN, il ne sera pas visible. Ce fichier n'est pas vraiment
	nécessaire, si un serveur DNS local est déjà démarré.


\subsection{Configuration de imond}

Les variables de configurations pour le fichier \texttt{etc/imond}.conf doivent entre
autre être activées, \var{CIRC\_x\_NAME}, \var{CIRC\_x\_ROUTE}, \var{CIRC\_x\_CHARGEINT}
et \var{CIRC\_x\_TIMES}. Il peut être constituée d'un maximum de 32 lignes
(à l'exclusion des lignes de commentaires). Chaque ligne est composée de 8 colonnes~:

\begin{enumerate}
\item  Plage journalière
\item  Plage horraire
\item  Dispositif (\texttt{ippp}\emph{X} ou \texttt{isdn}\emph{X})
\item  Circuit pour Default-Route: \og{}yes\fg{}/\og{}no\fg{}
\item  Numéro de Téléphone
\item  Nom du Circuit
\item  Prix de l'unité Téléphonique par Minute en EU
\item  Compteur (interval d'unité-Tél) en seconde
\end{enumerate}

    Voici un exemple~:

\begin{example}
\begin{verbatim}
#day  hour  device  defroute  phone        name        charge  ch-int
Mo-Fr 18-09 ippp0   yes       010280192306 Addcom      0.0248  60
Sa-Su 00-24 ippp0   yes       010280192306 Addcom      0.0248  60
Mo-Fr 09-18 ippp1   yes       019160       Compuserve  0.019   180
Mo-Fr 09-18 isdn2   no        0221xxxxxxx  Firma       0.08    90
Mo-Fr 18-09 isdn2   no        0221xxxxxxx  Firma       0.03    90
Sa-Su 00-24 isdn2   no        0221xxxxxxx  Firma       0.03    90
\end{verbatim}
\end{example}

    D'autres explications peuvent être trouvées pour Least-Cost-Routing (ou routage au
	moindre coût) dans la documentation du paquetage \og{}base\fg{}.


\subsection{Le fichier \texttt{/etc/.profile}}

Le fichier \texttt{/etc/.profile} contient les paramètres par défaut du Shell.
Pour modifier le fichier \texttt{/etc/.profile} par défaut, il est nécessaire d'ajouter
les paramètres en dessous des paramètres existant. Ces paramètres peuvent être utilisés
pour paramètrer un raccourci d'une commande Prompt et ensuite (vous pourrez exécuter
\og{}ce raccourci\fg{}).

\wichtig{Ce fichier ne doit pas contenir le paramètre\texttt{exit}~!}

Exemple~:

\begin{example}
\begin{verbatim}
alias ll='ls -al'
\end{verbatim}
\end{example}


\subsection{Les scripts dans \texttt{/etc/profile.d/}}

Vous pouvez stocker un script dans le répertoire \texttt{/etc/profile.d/},
ce script sera exécuté au démarrage du shell et ainsi influencer l’environnement
du shell. Généralement, les développeurs de programme OPT, y placent des scripts
qui définissent des variables d'environnement spéciales nécessaires au programme OPT.

Si des scripts sont situés dans le répertoire \texttt{/etc/profile.d/} et s'il
existe un fichier script \texttt{/etc/.profile}, les scripts du répertoire
\texttt{/etc/profile.d/} seront exécutés après le fichier script \texttt{/etc/.profile}.
