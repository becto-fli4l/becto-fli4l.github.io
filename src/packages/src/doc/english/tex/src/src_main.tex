% Synchronized to r51513
marklabel{buildroot}{
\section{SRC - The fli4l Buildroot}
}

This chapter is mainly of interest for developers, who want to compile
binaries or the Linux kernel for the fli4l. If you only want to use
flil4 as a router and do not offer packages for the fli4l needing own
binaries, you may skip this chapter completely.

In general, for compiling program packages for the fli4l a
Linux system is required. Compilation under other operating systems
(Microsoft Windows, OS X, FreeBSD, etc.) is \emph{not} supported.

The requirements for a Linux system for fli4l development are as follows:
\begin{itemize}
\item GNU gcc and g++ in version 2.95 or higher
\item GNU gcc-multilib (depending on your host system)
\item GNU binutils (contains bind and other necessary programs)
\item GNU make in version 3.81 or higher
\item GNU bash
\item libncurses5-dev for \texttt{fbr-make *-menuconfig} (depending on your host system)
\item The programs sed, awk, which, flex, bison and patch
\item The programs makeinfo (package texinfo) and msgfmt (package gettext)
\item The programs tar, cpio, gzip, bzip2 and unzip
\item The programs wget, rsync, svn and git
\item The programs perl and python
\end{itemize}

In the following, characters printed \textbf{bold} represent keyboard
input, the \enter-character stands for the Enter key on your keyboard and
executes entered commands.

\subsection{The Sources - An Overview}

In the directory \texttt{src} you will find the following subdirectories:

\begin{longtable}{|l|p{10cm}|}
    \hline
    \multicolumn{1}{|l}{\textbf{Directory}} &
    \multicolumn{1}{|l|}{\textbf{Content}} \\
    \hline
    \endhead
    \hline
    \endfoot
    \endlastfoot
\texttt{fbr} &
    In this folder there is a custom build system based on the buildroot for
    uClibc (currently in version 0.9.33.2). FBR stands for ``flil4-Buildroot''.
    It is thus possible, to compile all programs used on fli4l (kernel,
    libraries and applications) anew.\\
\hline
\texttt{fli4l} &
    This directory contains the fli4l specific sources for packages sorted by
    their names. All sources that are included in this subdirectory,
    were either written specifically for use with fli4l or are
    at least strongly adapted. \\
\hline
\texttt{cross} &
    This directory contains scripts that create the cross-compilers
    necessary for compiling mkfli4l for various Platforms. \\
\hline
\end{longtable}

\subsection{Compile A Program For fli4l}

In the subdirectory ``fbr'' a script \texttt{fbr-make} exists which controls
the compilation of all the programs from the basic packages for fli4l. This
script takes care of downloading and compiling all required binaries for fli4l.
The script will save files in the directory \texttt{\~{}/.fbr}, if it does
not yet exist, it will be created.
(The directory may be changed by using the environment variable
\texttt{FBR\_BASEDIR}, see below.)

\emph{Note:} During the compilation process much space is needed (currently around
900~MiB for the downloaded archives and almost 30~GiB for the intermediate results
and the resulting compiled files). Hence, make sure  to have enough space under
\texttt{\~{}/.fbr}!
(Alternatively, you may also use the \var{FBR\_TIDY} option, see below.)

The directory structure below \texttt{\~{}/.fbr} is as follows:

\begin{longtable}{|l|p{10cm}|}
    \hline
    \multicolumn{1}{|l}{\textbf{Directory}} &
    \multicolumn{1}{|l|}{\textbf{Content}} \\
    \hline
    \endhead
    \hline
    \endfoot
    \endlastfoot
\texttt{fbr-<branch>-<arch>} &
    The uClibc buildroot is unpacked here. \texttt{<branch>} stands for the development
    branch the FBR is derived from, in this case (i.e. \texttt{trunk}).
    If the origin of the FBR is an unpacked \texttt{src} package, \texttt{fbr-custom}
    will be used. \texttt{<arch>} reflects the processor architecture in use
    (i.e. \texttt{x86} or \texttt{x86\_64}). More concerning this directory
    can be found below. \\
\hline
\texttt{dl}                            &
    Here the downloaded archives are stored. \\
\hline
\texttt{own}                       &
    Here own FBR packages which should be compiled can be stored. \\
\hline
\end{longtable}

Under the Buildroot directory
\texttt{\~{}/.fbr/fbr-<branch>-<arch>/buildroot} the following
directories are of interest:

\begin{longtable}{|l|p{10cm}|}
    \hline
    \multicolumn{1}{|l}{\textbf{Directory}} &
    \multicolumn{1}{|l|}{\textbf{Content}} \\
    \hline
    \endhead
    \hline
    \endfoot
    \endlastfoot
\texttt{output/sandbox} &
    In this directory a subdirectory exists for each FBR package that holds
    the files of the package after being compiled. In the directory
    \texttt{output/sandbox/<package>/target} the files for the fli4l router
    can be found in this case. In the directory \texttt{output/sandbox/<package>/staging}
    files needed for building \emph{other} FBR packages requiring this FBR package
    can be found.\\
    \hline
\texttt{output/target} &
    In this directory \emph{all} compiled programs for the
    fli4l-router are stored. This directory mirrors the directory structure
    on the fli4l router. By the help of \texttt{chroot} you can change to
    this directory and try the programs.\footnote{This is bound to some preconditions,
    see the paragraph \jump{sec:src:test}{``Testing Of A Compiled Program''}.} \\
\hline
\end{longtable}

\subsubsection{General Settings}

The operation of \texttt{fbr-make} can be influenced by various
environment variables:

\begin{longtable}{|l|p{10cm}|}
    \hline
    \multicolumn{1}{|l}{\textbf{Variable}} &
    \multicolumn{1}{|l|}{\textbf{Description}} \\
    \hline
    \endhead
    \hline
    \endfoot
    \endlastfoot
\texttt{FBR} &
    Specifies the path to the FBR explicitely. Per default the path
    \texttt{\~{}/.fbr/fbr-<branch>-<arch>} (see above) is used.\\
\hline
\texttt{FBR\_BASEDIR} &
    explicitely specifies the base path to the FBR. As a default the \texttt{\~{}/.fbr}
    (see above) will be used. This variable will be ignored if
    the environment variable \texttt{FBR} is set. \\
\hline
\texttt{FBR\_DLDIR} &
    Specifies the directory containing the source archives. Per default
    the path \texttt{\$\{FBR\}/../dl} (i.e. \texttt{\~{}/.fbr/dl})
    will be used. \\
\hline
\texttt{FBR\_BRANCH} &
    Specifies explicitly the name of the branch under which the packages below
    \texttt{\~{}/.fbr} (see above) are built. This variable will be ignored if
    the environment variable \texttt{FBR} is set. \\
\hline
\texttt{FBR\_CATEGORY} &
    Specifies explicitly the name of the category under which the packages below
    \texttt{\~{}/.fbr} (see above) are built. This variable will be ignored if
    the environment variable \texttt{FBR} is set. \\
\hline
\texttt{FBR\_OWNDIR} &
    Specifies the directory holding the own packages. Per default
    the path \texttt{\$\{FBR\}/../own} (i.e. \texttt{\~{}/.fbr/own})
    will be used. \\
\hline
\texttt{FBR\_TIDY} &
    if this variable is set to ``y'' intermediate results while compiling the
    FBR packages will be deletetd immediately after their installation to the directory
    \texttt{output/target}. This saves a lot of space and is always recommended if
    you do not feel the urge to have a look at \texttt{output/build/...}
    after the build process.
    If this variable 'contains the value `` k', only
    the intermediate results in the various Linux kernel directories are removed,
    because this saves a lot of space without losing any functionality. All other
    assignments (or if the variable is missing entirely) ensure that all intermediate
    results are kept.
    \\
\hline
\texttt{FBR\_ARCH} &
    This variable specifies the processor architecture for which the FBR (or
    FBR packages) should be built. If it is missing, \texttt{x86} will be
    used. The supported architectures can be found below. \\
\hline
\end{longtable}

The FBR currently supports the following architectures:

\begin{longtable}{|l|p{10cm}|}
    \hline
    \multicolumn{1}{|l}{\textbf{Architecture}} &
    \multicolumn{1}{|l|}{\textbf{Description}} \\
    \hline
    \endhead
    \hline
    \endfoot
    \endlastfoot
\texttt{x86} &
    Intel x86-Architecture (32-Bit), also known as IA-32. \\
\hline
\texttt{x86\_64} &
    AMD x86-64-Architecture (64-Bit), also called Intel 64 or EM64T by Intel. \\
\hline
\end{longtable}

\subsubsection{Compilation Of All FBR Packages}

If executing \texttt{fbr-make} with the argument \texttt{world},
it may last several hours to download and compile all source archives,
depending on the computer and Internet connections used.
\footnote{Downloading of source archives is of course done only once
as long as you don't update the FBR and thus need new package versions
and source archives.}

\subsubsection{Compiling The Toolchain}

If executing \texttt{fbr-make} with the argument \texttt{toolchain},
all FBR packages needed for building the fli4l binary programs will
be downloaded and compiled (Compiler, Binder, uClibc-library etc.).
Normally this command is not needed, because all FBR packages depend
on the toolchain and thus it has to be downloaded and build anyway.

\subsubsection{Compiling A Single FBR Package}

If you only want to compile a certain FBR package (i.e. the programs
of an OPT developed yourself), you may transfer the name of one or more
FBR packages to the \texttt{fbr-make} program (for example \texttt{fbr-make
openvpn} to download and compile the OpenVPN programs). All dependencies will
also be downloaded and compiled.

\subsubsection{Recompilation Of A Single FBR Package}

If you like to compile a FBR package again (for whatever reason),
you first need to remove the information on the FBR about the previous
compilation process. For this purpose use the command \texttt{fbr-make <package>-clean}
(eg \texttt{fbr-make openvpn-clean}). In this case informations about
all dependencies for the package are also reset causing their recompilation
with the next \texttt {fbr-make world} as well.

\subsubsection{Recompiling All FBR Packages}

If you like to recompile the entire FBR (eg because you want to use it as
a benchmark program for your new high-end developer system ;-),
you can use the command \texttt{fbr-make clean} and remove all artifacts
that have been generated during the last FBR build. You will have to
confirm this action. \footnote{The whole directory
\texttt{\~{}/.fbr/fbr-<branch>-<arch>/buildroot/output} will be removed.}
This is also useful to free used disk space.

\marklabel{sec:src:test}{\subsection{Testing Of A Compiled Program}}

If a program has been compiled with \texttt{fbr-make} it may also be tested
on the development machine. Such a test will of course only work if the processor
architecture of the developer machine matches the processor architecture the
fli4l programs were compiled for. (It is not possible, for example,
to run \texttt{x86\_64} flil4 programs on a \texttt{x86} operating system.)
If this condition is met, change to the fli4l target directory with

\begin{example}
\begin{Verbatim}[commandchars=\\\{\}]
\textbf{chroot ~/.fbr/fbr-<branch>-<arch>/buildroot/output/target /bin/sh}\enter
\end{Verbatim}
\end{example}

\noindent and test the compiled program(s) there immedeately. Please note that
executing \texttt{chroot} needs administrator rights and thus you have to use
\texttt{sudo} or \texttt{su}, depending on preference and system configuration!
In addition you will have to compile the FBR package \texttt{busybox}
(via \texttt{fbr-make busybox}), to have a working shell in the \texttt{chroot}
environment. A small example:

\begin{example}
\begin{Verbatim}[commandchars=\\\{\}]
$ \textbf{sudo chroot ~/.fbr/fbr-trunk-x86/buildroot/output/target /bin/sh}\enter
Passwort:\textbf{(Your password)}\enter


BusyBox v1.22.1 (fli4l) built-in shell (ash)
Enter 'help' for a list of built-in commands.

# \textbf{ls}\enter
THIS_IS_NOT_YOUR_ROOT_FILESYSTEM  mnt
bin                               opt
dev                               proc
etc                               root
home                              run
img                               sbin
include                           share
lib                               sys
lib32                             tmp
libexec                           usr
man                               var
media                             windows
# \textbf{bc --version}\enter
bc 1.06
Copyright 1991-1994, 1997, 1998, 2000 Free Software Foundation, Inc.
# \textbf{echo "42 - 23" | bc}\enter
19
#
\end{Verbatim}
\end{example}

\marklabel{sec:debugging}{\subsection{Debugging Of A Compiled Program}}

In case of problems with compiled fli4l programs (crashes) you have the option
to analyze the state of the program immediately before the crash (also called
``post-mortem debugging''). To do so, activate \var{DEBUG\_ENABLE\_CORE='yes'}
in the configuration of the \texttt{base} package. If case of a crash
a memory dump is generated in \texttt{/var/log/dumps/core.<PID>}. ``PID''
is the process ID of the crashed process. You may analyze the state of the
program on a Linux machine with a fully compiled FBR as described below.
The following example to be analyzed is the program \texttt{/usr/sbin/collectd},
which was terminated with a SIGBUS. The dump was stored in \texttt{/tmp/core.collectd}.

\begin{example}
\begin{Verbatim}[commandchars=\\\%!]
fli4l@eisler:~$ \textbf%.fbr/fbr-trunk-x86/buildroot/output/host/usr/bin/i586-linux-gdb!\enter
GNU gdb (GDB) 7.5.1
Copyright (C) 2012 Free Software Foundation, Inc.
License GPLv3+: GNU GPL version 3 or later <http://gnu.org/licenses/gpl.html>
This is free software: you are free to change and redistribute it.
There is NO WARRANTY, to the extent permitted by law.  Type "show copying"
and "show warranty" for details.
This GDB was configured as "--host=x86_64-unknown-linux-gnu --target=i586-buildr
oot-linux-uclibc".
For bug reporting instructions, please see:
<http://www.gnu.org/software/gdb/bugs/>.
(gdb) \textbf%set sysroot /project/fli4l/.fbr/fbr-trunk-x86/buildroot/output/target!\enter
(gdb) \textbf%set debug-file-directory /project/fli4l/.fbr/fbr-trunk-x86/buildroot/outpu!
\textbf%t/debug!\enter
(gdb) \textbf%file /project/fli4l/.fbr/fbr-trunk-x86/buildroot/output/target/usr/sbin/co!
\textbf%llectd!\enter
Reading symbols from /project/fli4l/.fbr/fbr-trunk-x86/buildroot/output/target/u
sr/sbin/collectd...Reading symbols from /project/fli4l/.fbr/fbr-trunk-x86/buildr
oot/output/debug/.build-id/8b/28ab573be4a2302e1117964edede2e54ebbdbf.debug...don
e.
done.
(gdb) \textbf%core /tmp/core.collectd!\enter
[New LWP 2250]
[New LWP 2252]
[New LWP 2259]
[New LWP 2257]
[New LWP 2255]
[New LWP 2232]
[New LWP 2235]
[New LWP 2238]
[New LWP 2242]
[New LWP 2244]
[New LWP 2245]
[New LWP 2231]
[New LWP 2243]
[New LWP 2251]
[New LWP 2248]
[New LWP 2239]
[New LWP 2229]
[New LWP 2249]
[New LWP 2230]
[New LWP 2247]
[New LWP 2233]
[New LWP 2256]
[New LWP 2236]
[New LWP 2246]
[New LWP 2240]
[New LWP 2241]
[New LWP 2237]
[New LWP 2234]
[New LWP 2253]
[New LWP 2254]
[New LWP 2258]
[New LWP 2260]
Failed to read a valid object file image from memory.
Core was generated by `collectd -f'.
Program terminated with signal 7, Bus error.
#0  0xb7705f5d in memcpy ()
   from /project/fli4l/.fbr/fbr-trunk-x86/buildroot/output/target/lib/libc.so.0
(gdb) \textbf%backtrace!\enter
#0  0xb7705f5d in memcpy ()
   from /project/fli4l/.fbr/fbr-trunk-x86/buildroot/output/target/lib/libc.so.0
#1  0xb768a251 in rrd_write (rrd_file=rrd_file@entry=0x808e930, buf=0x808e268,
    count=count@entry=112) at rrd_open.c:716
#2  0xb76834f3 in rrd_create_fn (
    file_name=file_name@entry=0x808d2f8 "/data/rrdtool/db/vm-fli4l-1/cpu-0/cpu-i
nterrupt.rrd.async", rrd=rrd@entry=0xacff2f4c) at rrd_create.c:727
#3  0xb7683d7b in rrd_create_r (
    filename=filename@entry=0x808d2f8 "/data/rrdtool/db/vm-fli4l-1/cpu-0/cpu-int
errupt.rrd.async", pdp_step=pdp_step@entry=10, last_up=last_up@entry=1386052459,
    argc=argc@entry=16, argv=argv@entry=0x808cf18) at rrd_create.c:580
#4  0xb76b77fd in srrd_create (
    filename=0xacff33f0 "/data/rrdtool/db/vm-fli4l-1/cpu-0/cpu-interrupt.rrd.asy
nc",
    pdp_step=10, last_up=1386052459, argc=16, argv=0x808cf18) at utils_rrdcreate
.c:377
#5  0xb76b78cb in srrd_create_thread (targs=targs@entry=0x808bab8)
    at utils_rrdcreate.c:559
#6  0xb76b7a8f in srrd_create_thread (targs=0x808bab8) at utils_rrdcreate.c:491
#7  0xb7763430 in ?? ()
   from /project/fli4l/.fbr/fbr-trunk-x86/buildroot/output/target/lib/libpthread
.so.0
#8  0xb775e672 in clone ()
   from /project/fli4l/.fbr/fbr-trunk-x86/buildroot/output/target/lib/libpthread
.so.0
(gdb) \textbf%frame 1!\enter
#1  0xb768a251 in rrd_write (rrd_file=rrd_file@entry=0x808e930, buf=0x808e268,
    count=count@entry=112) at rrd_open.c:716
716         memcpy(rrd_simple_file->file_start + rrd_file->pos, buf, count);
(gdb) \textbf%print (char*) buf!\enter
$1 = 0x808e268 "RRD"
(gdb) \textbf%print rrd_simple_file->file_start!\enter
value has been optimized out
(gdb) \textbf%list!\enter
711         if((rrd_file->pos + count) > old_size)
712         {
713             rrd_set_error("attempting to write beyond end of file");
714             return -1;
715         }
716         memcpy(rrd_simple_file->file_start + rrd_file->pos, buf, count);
717         rrd_file->pos += count;
718         return count;       /* mimmic write() semantics */
719     #else
720         ssize_t   _sz = write(rrd_simple_file->fd, buf, count);
(gdb) \textbf%list 700!\enter
695      * rrd_file->pos of rrd_simple_file->fd.
696      * Returns the number of bytes written or <0 on error.  */
697
698     ssize_t rrd_write(
699         rrd_file_t *rrd_file,
700         const void *buf,
701         size_t count)
702     {
703         rrd_simple_file_t *rrd_simple_file = (rrd_simple_file_t *)rrd_file->
pvt;
704     #ifdef HAVE_MMAP
(gdb) \textbf%print *(rrd_simple_file_t *)rrd_file->pvt!\enter
$2 = {fd = 9, file_start = 0xa67d0000 <Address 0xa67d0000 out of bounds>,
  mm_prot = 3, mm_flags = 1}
\end{Verbatim}
\end{example}

After some ``digging'', you will see that an invalid pointer is contained in
the \texttt{rrd\_simple\_file\_t} object (``Address ... out of bounds'').
In the further process of debugging, it became clear that a failed
\texttt{posix\_fallocate}-call  was the culprit for the crash.

It is important to pass \emph{all} paths fully qualified
(\texttt{/project/...}) and to use no ``shortcuts'' (i.e. in
\texttt{\~{}/...}). If you don't obey this it may happen that \texttt{gdb}
will not find the debug informations for the application and/or for the libraries
in use. Due to space reasons debug informations are not contained directly
in the program to be investigated but are saved in separate files in the directoy
\texttt{\~{}/.fbr/fbr-<branch>-<arch>/buildroot/output/debug/}.

\subsection{Informations On The FBR}

\subsubsection{Displaying Help}

Use the command \texttt{fbr-make help} to see what \texttt{fbr-make} can do for you.

\subsubsection{Displaying Program Informations}

By using the command \texttt{fbr-make show-versions} you can review all
FBR packages provided with version number:

\begin{example}
\begin{Verbatim}[commandchars=\\\{\}]
$ \textbf{fbr-make show-versions}\enter
Configured packages

acpid 2.0.20
actctrl 3.25+dfsg1
add-days undefined
[...]
\end{Verbatim}
\end{example}

\subsubsection{Display Of Dependencies On Libraries}

With \texttt{fbr-make links-against <soname>} all files in
\texttt{\~{}/.fbr/fbr-<branch>-<arch>/buildroot\\/output/target}
linked to a library named \texttt{soname} can be shown. If for example all
programs and libraries should be identified that use \texttt{libm} (Library
with mathematical functions) use \texttt{fbr-make links-against libm.so.0}
because \texttt{libm.so.0} is the name of the libm library.
A possible output would be:

\begin{example}
\begin{Verbatim}[commandchars=\\\{\}]
$ \textbf{fbr-make links-against librrd_th.so.4}\enter
Executing plugin links-against
Files linking against librrd_th.so.4
collectd usr/lib/collectd/rrdcached.so
collectd usr/lib/collectd/rrdtool.so
rrdtool usr/bin/rrdcached
\end{Verbatim}
\end{example}

In the first column is the package name and in the second the (relative)
path to the file that is linked against the library in question.

To find the library name for a library, you can use
\texttt{readelf} like this:

\begin{example}
\begin{Verbatim}[commandchars=\\\{\}]
$ \textbf{readelf -d ~/.fbr/fbr-trunk-x86/buildroot/output/target/lib/libm-0.9.33.2.so |}\enter
> \textbf{grep SONAME}\enter
 0x0000000e (SONAME)                     Library soname: [libm.so.0]
\end{Verbatim}
\end{example}

\subsubsection{Display Of Version Changes}

The command \texttt{fbr-make version-changes} is interesting (only) for fli4l-team developers
with write access to the fli4l SVN repository. It lists all FBR packages whose version has been
modified locally, i.e. those where the version in the working copy differs from the repository
version. This helps the developer to get an overview on updated FBR packages before writing
the changes to the repo. A possible output is:

\begin{example}
\begin{Verbatim}[commandchars=\\\{\}]
$ \textbf{fbr-make version-changes}\enter
Executing plugin version-changes
Package version changes
KAMAILIO: 4.0.5 --> 4.1.1
\end{Verbatim}
\end{example}

Here you can see that the package \texttt{kamailio} in FBR was updated from version
4.0.5 to version 4.1.1.

\subsection{Changing The FBR Configuration}

\subsubsection{Reconfiguration Of The FBR}

By using \texttt{fbr-make buildroot-menuconfig} it is possible to
select the FBR packages to be compiled. This is useful if you
want to compile other FBR packages for the fli4l that are not enabled
by default but are integrated in the uClibc buildroot, or if you
want to activate own FBR packages. On the other other hand  global
properties of FBR may be changed, such as the version of the used
GCC compiler. On successful exit of the configuration menu, the new
configuration is saved in the directory \texttt{src/fbr/buildroot/.config}.

\achtung{Please note, however, that such changes of the toolchain configuration
are \emph{not} officially supported because the resulting binaries will be
incompatible with the official fli4l distribution with a high probability.
So if you need binaries for your own OPT and want to publish this OPT, you
should not change the toolchain settings!}

\subsubsection{Reconfiguration Of The uClibc Library}

With \texttt{fbr-make uclibc-menuconfig} the funcionality of the uClibc
library in use may be changed. On successful exit of the configuration menu,
the new configuration is saved to \texttt{src/fbr/buildroot/package/uclibc/uclibc.config}.

\achtung{Like in the last paragraph also here applies: A change is most
likely not compatible with the official fli4l distribution and
is thus not supported!}

\subsubsection{Reconfiguration Of Busybox}

With \texttt{fbr-make busybox-menuconfig} the Busybox may be changed
in its funcionality. On successful exit of the configuration menu,
the new configuration is saved to \texttt{src/fbr/buildroot\\/package/busybox/busybox-<Version>.config}.

\achtung{Also here applies: A change is most likely not compatible with
the official fli4l distribution and is thus not supported! Adding new Busybox
applets causes no problems as long as you only use the modified Busybox
on your own flil4 router (and not require the users of your OPT to use a Busybox
modified in this way).}

\subsubsection{Reconfiguration Of The Linux Kernel Packages}

With \texttt{fbr-make linux-menuconfig} resp. \texttt{fbr-make linux-<version>-menuconfig}
the configuration of all activated Kernel packages resp. a specific Kernel package
may be changed. On successful exit of the configuration menu,
the new configuration is saved to
\texttt{src/fbr/buildroot\\/linux/linux-<version>/dot-config-<arch>}.
\footnote{This only applies to the standard Kernel. For variants of a Kernel package
a \texttt{diff} file will be saved in
\texttt{src/fbr/buildroot/linux/linux-<version>/linux-<version>\_<variante>/dot-config-<arch>.diff}
instead.}

\achtung{Like in the last paragraph also here applies:  A change is most likely
not compatible with the official fli4l distribution and is thus not supported!
At most, adding of new modules to the Linux kernel is easy, as long you only use the
modified kernel on your own flil4 router (and not require the users of your OPT to use
a Kernel modified in this way).}

\subsection{Updating The FBR}

Each of the commands outlined above is advanced by an an examination
of the actuality of the FBR. If a discrepancy between the sources in which
\texttt{fbr-make} is located (unpacked \texttt{src}-package or SVN-working copy)
and the FBR in \texttt{\~{}/.fbr/fbr-<branch>-<arch>/buildroot} is detected
the latter will be updated. New FBR packages will be integrated and old, no
longer contained FBR packages will be deleted. The configurations are compared:
FBR packages with modified configuration and all dependent FBR packages will be
rebuilt. This ensures that the FBR on your computer is always equal to the developer's
one (except for your own FBR packages under \texttt{\~{}/.fbr/own/}).
\textbf{However, This also means that changes to the official part of the
Buildroot configuration will be lost with the next update!} Therefore
it is not recommended to reconfigure FBR, at least not if you are using
\texttt{src} packages instead of a SVN working copy. (When updating a SVN
working copy your local configuration changes and the changes to the SVN
repository will be merged and the problem of lost configuration does not occur.)
However, your own FBR packages may be reconfigured easily, without data loss
occuring on an update.


\subsection{Integrating Own Programs Into The FBR}

Compilation of the individual FBR packages is controlled by small Makefiles.
If you want to develop your own FBR packages, you have to create a Makefile and
a configuration description in  \texttt{\~{}/.fbr/own/<package>/}.
How these are constructed and how to write own Makefiles derived from them
is decribed in detail in the documentation for the uClibc Buildroot
\altlink{http://buildroot.uclibc.org/downloads/manual/manual.html\#adding-packages}.
