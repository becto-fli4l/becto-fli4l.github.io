% Synchronized to r32928
\section {RRDTOOL - Collect Data And Display Graphs About It}

\subsection {Description}
The package RRDTOOL collects system data and stores it in rrd-databases by
the help of the 'collectd' daemon.
In the web interface of the fli4l router graphics for download or display
are generated from it by rrdtool.
\\
For example, the following data is recorded and displayed:
 \begin{itemize}
  \item Under System Status
  \begin{itemize}
   \item CPU load
   \item System load
   \item System uptime
   \item RAM usage
   \item Number of processes
  \end{itemize}
  \item Under Harddisk Status
   \begin{itemize}
   \item Partition usage of the root partition
   \item Usage of the partition /boot
   \item Usage of the partition /data (if existing)
   \item Usage of the partition /opt (if existing)
  \end{itemize}
  \item Under Network Status
  \begin{itemize}
   \item For each network interface amount of data sent and received
  \end{itemize}
  \item Under Interrupts
  \begin{itemize}
   \item The number of interrupts
  \end{itemize}
  \item Under Active Connections
  \begin{itemize}
   \item The number of connections
  \end{itemize}
 \end{itemize}

Optionally, also the acquisition and display of temperatures and voltages of the
motherboard, WLAN informations, the values from an APC UPS, PING values of hosts
or VPN endpoints a.s.o. is possible, depending on the configuration or the installed
packages.

\subsection {Hint concerning RRDTOOL versions}

  RRD database files that were created with the old version of rrdtool
  can not be used with the current version. The daemon uses a different data
  format and thus the files are incompatible.

\subsection {Hint concerning the use of RRDTOOL on different architectures}

  If fli4l's processor architecture is switched (eg from 32 bit to 64 bit)
  the rrdtool database files have to be converted manually by the user.
  Direct conversion is not possible.

  The old database has to be exported to XML files and imported then to the
  new architecture instead. It is important to do the XML export while
  the old architecture is still in use.

  A german HowTo article on this topic can be found at
  \altlink{https://ssl.nettworks.org/wiki/display/f/rrdtool-Datenbanken}.

\subsection {Configuration Of the Package RRDTOOL}

  The configuration is done by adapting the file
  \var{Path/fli4l-\version/$<$config$>$/rrdtool.txt}
  to your own needs, as with all FLI4L packages.

\begin{description}

\config {OPT\_RRDTOOL}{OPT\_RRDTOOL}{OPTRRDTOOL}

  The setting \var{'no'} deactivates OPT\_RRDTOOL completely. No changes
  to your \var{rootfs.img} resp. \var{opt.img} are made.
  Furthermore, OPT\_RRDTOOL will never overwrite other parts of the
  fli4l installation.\\
  To activate OPT\_RRDTOOL set the variable \var{OPT\_RRDTOOL} to
  \var{'yes'}.

\config {RRDTOOL\_DB\_PATH}{RRDTOOL\_DB\_PATH}{RRDTOOLDBPATH}

  Default Setting: \var{RRDTOOL\_DB\_PATH='/data/rrdtool/db'}

  Path to RRDTOOL's database files. These files should always be located on
  a persistent disk. It is no a problem to store the data on a
  CompactFlash card as appropriate caching mechanisms are used in the
  package RRDTOOL to minimize the number of disk accesses.
  When using OPT\_QOS please ensure to use ext2/ext3/ext4 as file system in the
  path's target because only these support the characters used in the file name.

\config {RRDTOOL\_CACHETIME}{RRDTOOL\_CACHETIME}{RRDTOOLCACHETIME}

  This optional configuration parameter can set after how many seconds cached values
  will be written to the RRD database by the rrdcached daemon. With smaller values,
  the cache file will be smaller in the ramdisk which is recommended for routers
  with a rather small amount of RAM, but the disk will be accessed more often then.
  Without activation of the variable, this is done after 3600 seconds or on
  shutdown of fli4l.

  The following values are possible:
  \begin{itemize}
    \item 3600
    \item 1800
    \item 1200
    \item 900
    \item 600
    \item 450
    \item 300
  \end{itemize}


\config {RRDTOOL\_NET}{RRDTOOL\_NET}{RRDTOOLNET}

  Setting this varable to \var{'yes'} activates the network plugin of collectd.
  This makes it possible to transfer the data detected/collected by collectd
  to another computer on the network with active collectd-plugin running in
  server mode.

\config {RRDTOOL\_NET\_HOST}{RRDTOOL\_NET\_HOST}{RRDTOOLNETHOST}

  FQDN or IP address of the computer running collectd with network plugin in
  server mode.

\config {RRDTOOL\_NET\_PORT}{RRDTOOL\_NET\_PORT}{RRDTOOLNETPORT}

  This optional variable configures the port on which the server is listening
  to incoming connections.

\config {RRDTOOL\_UNIXSOCK}{RRDTOOL\_UNIXSOCK}{RRDTOOLUNIXSOCK}

  Setting this to \var{'yes'} activiates the unixsock plugin of collectd.
  On this socket other data collecting services/processes can transfer
  data to collectd.

\config {RRDTOOL\_PING\_N}{RRDTOOL\_PING\_N}{RRDTOOLPINGN}

  Specifies the number of hosts where network ping times should
  be determined.

\config {RRDTOOL\_PING\_x}{RRDTOOL\_PING\_x}{RRDTOOLPINGx}

  Defines the host for which network ping times should be determined.
  Can be set as a FQDN or an IP address.

\config {RRDTOOL\_PING\_x\_LABEL}{RRDTOOL\_PING\_x\_LABEL}{RRDTOOLPINGxLABEL}

  Optionally defines a different description (label) for the ping target.

\config {RRDTOOL\_PING\_x\_GRPNR}{RRDTOOL\_PING\_x\_GRPNR}{RRDTOOLPINGxGRPNR}

  Assigns this ping target to the group defined in\\
  \var{RRDTOOL\_PINGGROUP\_x\_LABEL} by the number of the index.

\config {RRDTOOL\_PINGGROUP\_N}{RRDTOOL\_PINGGROUP\_N}{RRDTOOLPINGGROUPN}

  Number of ping target groups. Each group defined will be displayed on a
  separate tab in the web interface.

\config {RRDTOOL\_PINGGROUP\_x\_LABEL}{RRDTOOL\_PINGGROUP\_x\_LABEL}{RRDTOOLPINGGROUPxLABEL}

  Name of the ping target group.

\config {RRDTOOL\_APCUPS}{RRDTOOL\_APCUPS}{RRDTOOLAPCUPS}

  Activates resp. deactivates the collecting of data from an APC-USV.
  For data collection the apcupsd daemon has to be active on a host
  reachable via network.

\config {RRDTOOL\_APCUPS\_HOST}{RRDTOOL\_APCUPS\_HOST}{RRDTOOLAPCUPSHOST}

  Host on which the apcupsd daemon is running.

\config {RRDTOOL\_APCUPS\_PORT}{RRDTOOL\_APCUPS\_PORT}{RRDTOOLAPCUPSPORT}

  Network port on which the apcupsd daemon can be accessed.
  Normally this is port 3351.

\config{RRDTOOL\_PEERPING\_N}{RRDTOOL\_PEERPING\_N}{RRDTOOLPEERPINGN}

  Sets the number of Peer-Ping targets. A Peer-Ping target is i.e. the target
  of a VPN tunnel.

\config {RRDTOOL\_PEERPING\_x}{RRDTOOL\_PEERPING\_x}{RRDTOOLPEERPINGx}

  Defines the Peer-Ping target. \\
  Possible targets are for example tun0, tun1, pppoe, a.s.o. Alias- resp. circuit
  names can be used as well.

\config {RRDTOOL\_PEERPING\_x\_LABEL}{RRDTOOL\_PEERPING\_x\_LABEL}{RRDTOOLPEERPINGxLABEL}

  Optionally defines a different description (label) for the ping target.

\config {RRDTOOL\_OWFS}{RRDTOOL\_OWFS}{RRDTOOLOWFS}

  Activates resp. deactivates the collecting and graphical display of
  data generated from package OW.

\config {RRDTOOL\_OWFS\_HOST}{RRDTOOL\_OWFS\_HOST}{RRDTOOLOWFSHOST}

  Host the OWFS service is running on. Usually this is the router
  itself. Thus the value '127.0.0.1' has to be entered.

\config {RRDTOOL\_OWFS\_PORT}{RRDTOOL\_OWFS\_PORT}{RRDTOOLOWFSPORT}

  Network port on which the OWFS service is reachable.
  Usually this is port 4304.

\config {RRDTOOL\_NTP}{RRDTOOL\_NTP}{RRDTOOLNTP}

  Activates resp. deactivates the collecting and graphical display of
  data generated from package NTP.

\end{description}

