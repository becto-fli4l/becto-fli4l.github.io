% Synchronized to r30405
\marklabel{sec:opt-wol}
{
\section {WOL - Wake On LAN}
}
OPT\_WOL enables fli4l to start remote computers with a Wake on LAN
enabled network card by executing 'wol.sh' on the console or via the
router's web interface.

For this to work, the network card usually has a small three-wire cable
connected to the motherboard, so the network card stays powered from
the ATX power supply even when the computer is currently in standby mode.

\subsection {Configuration}

\begin{description}

\config{OPT\_WOL}{OPT\_WOL}{OPTWOL}

	Default Setting:  \var{OPT\_WOL}='no'

	\var{'no'} deactivates OPT\_WOL Paket completely. No changes will
	be made to the fli4l boot medium resp. the archive \var{opt.img}.\\
	\var{'yes'} activates the package OPT\_WOL.

	To power on a client via WOL its MAC address has to be entered in
	$<$config-dir$>$/dns\_dhcp.txt (HOST\_x\_MAC).
	All computers without MAC address defined are excluded from WOL automatically.

\config{WOL\_LIST}{WOL\_LIST}{WOLLIST}

    Configuration is made by black or whitelisting.
    blacklisting means that all clients on this list are excluded from WOL, whitelisting
    means that WOL is possible for clients on the list.

	Default Setting:  \var{WOL\_LIST}='black'

	Valid values:
	\begin{itemize}
		\item black - means that all clients on this list can not be powered on (woken)
		\item white - means that all clients on this list can be powered on (woken)
	\end{itemize}

\config{WOL\_LIST\_N}{WOL\_LIST\_N}{WOLLISTN}

	Default Setting:  \var{WOL\_LIST\_N}='0'

	As the default no client is on the Blacklist, thus each client can be powered via WOL.

\config{WOL\_LIST\_x}{WOL\_LIST\_x}{WOLLISTx}

	Default Setting:  \var{WOL\_LIST\_x}=''

	Valid values:
	\begin{itemize}
		\item IP\_NET\_1	- All clients reachable over \var{IP\_NET\_x} (here IP\_NET\_1)
		\item @client1		- Name of a client (\var{HOST\_x\_NAME}) here 'client1'
		\item IP-Address	- IP of a client (\var{HOST\_x\_IP4} or \var{HOST\_x\_IP6})
	\end{itemize}

Example:
\begin{example}
\begin{verbatim}
      WOL_LIST='black'            # black or white listing
      WOL_LIST_N='3'              # Number of list entries
      WOL_LIST_1='IP_NET_1'       # All clients in Network IP_NET_1
      WOL_LIST_2='@client1'       # Client with the name HOST_1_x
      WOL_LIST_3='192.168.6.3'    # Client with this IP
\end{verbatim}
\end{example}

\end{description}

\subsection{Wake On Lan At Router Boot}
\begin{description}

\config{WOL\_BOOT}{WOL\_BOOT}{WOLBOOT}

This setting should only be set to 'yes' if you want to boot a computer in your
network with Wake on LAN when starting the router. This configuration is independent
of \var{WOL\_LIST}, clients not listed in \var{WOL\_LIST} may be specified too.

\config{WOL\_BOOT\_N}{WOL\_BOOT\_N}{WOLBOOTN}

	Default Setting:  \var{WOL\_BOOT\_N}='0'

	As the default no clients are on the list, thus no clients will be powered
	via WOL when booting the router.

\config{WOL\_BOOT\_x}{WOL\_BOOT\_x}{WOLBOOTx}

	Default Setting:  \var{WOL\_BOOT\_x}=''

	Valid values:
	\begin{itemize}
		\item IP\_NET\_1	- All clients reachable over \var{IP\_NET\_x} (here IP\_NET\_1)
		\item @client1		- Name of a client (\var{HOST\_x\_NAME}) here 'client1'
		\item IP-Address	- IP of a client (\var{HOST\_x\_IP4} or \var{HOST\_x\_IP6})
	\end{itemize}

Example:
\begin{example}
\begin{verbatim}
      WOL_BOOT='yes'               # install WOL on Boot: yes or no
      WOL_BOOT_N='2'               # Number of computers
      WOL_BOOT_1='@client1'        # first client
      WOL_BOOT_2='192.162.6.17     # second client
\end{verbatim}
\end{example}

\end{description}

\subsection{Usage}

Log in to the console directly or via SSH and wake a computer like this:
'wol.sh~$<$computerername$>$' or 'wol.sh~$<$IP-Address$>$' or 'wol.sh~$<$MAC-Address$>$'.

Computers not contained in $<$config-dir$>$/wol.txt may be woken by
'etherwake~$<$MAC-Address$>$'.

% ##TRANSLATE## : FFL-1666: begin automatic webgui
\subsection {Web GUI}

    The package also provides a Web GUI for mini-httpd.
    The Web GUI is automatically activated with \var{OPT\_HTTPD='yes'}.
% ##TRANSLATE## : FFL-1666: end automatic webgui
