% Last Update: $Id: hwsupp_main.tex 59357 2020-11-21 15:01:39Z chfritsch $
\section {HWSUPP - Unterstützung von Hardware}

\subsection {Beschreibung}

Das Paket stellt die Unterstützung für die Nutzung spezieller Hardwarekomponenten bereit.

Unterstützte Hardwarekomponenten/-elemente:
\begin{itemize}
\item Temeratursensoren
\item LEDs
\item Spannungssensoren
\item Lüfterdrehzahlen
\item Taster
\item Watchdog
\item VPN-Karten
\end{itemize}

Unterstützung gibt es für die folgende Systeme/Mainboards/VPN-Karten:
\begin{itemize}
  \item Standard PC-Hardware
  \begin{itemize}
    \item LEDs einer PC-Tastatur
  \end{itemize}
  \item ACPI-PC-Hardware
  \item Embedded Systeme
  \begin{itemize}
    \item AEWIN SCB6971
    \item Fujitsu Siemens Futro S200
    \item Fujitsu Siemens Futro S900
    \item PC Engines ALIX
    \item PC Engines APU
    \item PC Engines WRAP
    \item Soekris net4801
    \item Soekris net5501
  \end{itemize}
  \item Mainboards
  \begin{itemize}
    \item Commell LE-575
    \item GigaByte GA-M521-S3
    \item GigaByte GA-N3150N-D3V
    \item LEX CV860A
    \item MSI MS-9803
    \item SuperMicro PDSME
    \item SuperMicro X7SLA
    \item Tyan S5112
    \item WinNet PC640
    \item WinNet PC680
  \end{itemize}
    \item VPN Karten (PCI, miniPCI and miniPCIe)
  \begin{itemize}
    \item vpn1401 vpn1411
  \end{itemize}
\end{itemize}


\subsection {Konfiguration des Paketes HWSUPP}
  
  Die Konfiguration erfolgt, wie bei allen fli4l Paketen, durch Anpassung der \\
  Datei \var{Pfad/fli4l-\version/$<$config$>$/hwsupp.txt} 
  an die eigenen Anforderungen.
  
\begin{description}

\config {OPT\_HWSUPP}{OPT\_HWSUPP}{OPTHWSUPP}

  Die Einstellung \var{'no'} deaktiviert das OPT\_HWSUPP vollständig. Es werden
  keine Änderungen am fli4l Archiv \var{rootfs.img} bzw. dem Archiv \var{opt.img}
  vorgenommen. Weiterhin überschreibt das OPT\_HWSUPP grundsätzlich keine anderen
  Teile der fli4l Installation.\\
  Um OPT\_HWSUPP zu aktivieren, ist die Variable \var{OPT\_HWSUPP} auf 
  \var{'yes'} zu setzen.
  
\config {HWSUPP\_TYPE}{HWSUPP\_TYPE}{HWSUPPTYPE}

  In dieser Konfigurationsvariable wird die zu unterstützende Hardware festgelegt.
  Folgende Werte stehen zur Verfügung:
  \begin{itemize}
    \item sim
    \item generic-pc
    \item generic-acpi
    \item generic-acpi-coretemp
    \item aewin-scb6971
    \item commell-le575
    \item fsc-futro-s200
    \item fsc-futro-s900
    \item gigabyte-ga-m52l-s3
    \item gigabyte-ga-n3150n-d3v
    \item lex-cv860a
    \item msi-ms-9803
    \item pcengines-alix
    \item pcengines-apu (APU-1)
    \item pcengines-apu2 (APU-2)
    \item pcengines-wrap 
    \item rpi
    \item soekris-net4801
    \item soekris-net5501
    \item supermicro-pdsme
    \item supermicro-x7sla
    \item tyan-s5112
    \item winnet-pc640
    \item winnet-pc680 
  \end{itemize}
  
\config {HWSUPP\_WATCHDOG}{HWSUPP\_WATCHDOG}{HWSUPPWATCHDOG}
  Die Einstellung \var{'yes'} aktiviert den Watchdog-Daemon falls die gewählte 
  Hardware einen Watchdog besitzt. Durch den Watchdog wird ein hängendes System 
  automatisch neu gestartet werden.

\config {HWSUPP\_CPUFREQ}{HWSUPP\_CPUFREQ}{HWSUPPCPUFREQ}
  Die Einstellung \var{'yes'} aktiviert die Anpassung der Prozessor-Taktfrequenz.

\config {HWSUPP\_CPUFREQ\_GOVERNOR}{HWSUPP\_CPUFREQ\_GOVERNOR}{HWSUPPCPUFREQGOVERNOR}
  Auswahl des CPU-Frequenz-Reglers. Die Auswahl des Reglers steuert das Verhalten
  der Anpassung der Prozessor-Taktfrequenz. Zur Auswahl stehen:

  \begin{itemize}
    \item performance

      Der Prozessor läuft immer mit der maximalen Taktfrequenz.

    \item ondemand

      Die CPU-Frequenz wird an die Rechenleistung angepasst. Dabei kann die
      CPU-Frequenz u.U. sprunghaft angehoben oder abgesenkt werden. 

    \item conservative

      Die CPU-Frequenz wird an die Rechenleistung angepasst. Die CPU-Frequenz
      wird schrittweise angehoben bzw. abgesenkt. 

    \item powersave

      Der Prozessor läuft immer mit der minimmalen Taktfrequenz.

    \item userspace

      Der Prozessortakt kann manuell oder von einem Anwenderskript über die
      sysfs-Variable \var{/devices/system/cpu/cpu\emph{<n>}/cpufreq/scaling\_setspeed}
      gesetzt werden.

  \end{itemize}

\config {HWSUPP\_LED\_N}{HWSUPP\_LED\_N}{HWSUPPLEDN}
  Definiert die Anzahl der LEDs. Hier sollte die Anzahl der LEDs die 
  die verwendete Hardware bereitstellt stehen.
  
\config {HWSUPP\_LED\_x}{HWSUPP\_LED\_x}{HWSUPPLEDx}
  Definiert die Information, die durch das LED angezeigt werden soll.
    Folgende Informationen sind möglich:
  \begin{itemize}
    \item ready - Der fli4l-Router ist betriebsbereit\footnotemark
    \item online - der fli4l-Router ist mit den Internet verbunden
    \item trigger - Anzeige wird durch einen LED-Trigger gesteuert
    \item user - Anzeige wird durch ein Benutzerscript gesteuert
  \end{itemize}

  \footnotetext{Ist \var{HWSUPP\_LED\_x='ready'}, so wird der Bootfortschritt
  durch eine Blink-folge angezeigt (siehe Anhang \ref{sec:blink}).}

  Die Liste der möglichen Information kann durch andere Pakete erweitert werden.
  So ist bei geladenem WLAN-Paket z.B. die Anzeige 
  \begin{itemize}
    \item wlan - das WLAN ist aktiviert
  \end{itemize}
  möglich.

  Im Anhang \ref{sec:developer} finden sich Hinweise für
  Paket-Entwickler wie eine solche Erweiterung anzulegen ist. 

\config {HWSUPP\_LED\_x\_DEVICE}{HWSUPP\_LED\_x\_DEVICE}{HWSUPPLEDxDEVICE}
  Gibt das LED-Device an.
  
  Hier wird entweder ein LED-Device eingetragen (zu finden unter \var{/sys/class/leds/}
  im Dateisystem des Routers) oder eine GPIO\footnotemark-Nummer. 

  \footnotetext{Ein GPIO (General Purpose Input/Output)
  ist ein Kontaktstift an einem integrierten Schaltkreis, dessen Verhalten durch Programmierung
  bestimmbar ist, unabhängig, ob als Ein- oder Ausgabekontakt.}

  Eine Liste gültiger Namen der LED-Devices für den jeweiligen \var{HSUPP\_TYPE}
  findet sich im Anhang \jump{sec:leddevices}{``Verfügbare LED-Devices''}.
  
  Die GPIO-Nummer muss im Format \var{gpio::x} eingegeben werden.
  Wenn man ein GPIO eingeträgt, so wird das dazugehörige LED-Device
  automatisch angelegt. Durch Voranstellen von \var{/} wird die
  Funktionsweise des GPIO invertiert.
  
  Beispiele:
  \begin{verbatim}
    HWSUPP_LED_1_DEVICE='apu::1'
    HWSUPP_LED_2_DEVICE='gpio::237'
    HWSUPP_LED_3_DEVICE='/gpio::245'
    HWSUPP_LED_4_DEVICE='led0'
  \end{verbatim}

\config {HWSUPP\_LED\_PARAM}{HWSUPP\_LED\_PARAM}{HWSUPPLEDPARAM}
  Definiert Parameter für die ausgewählte LED Anzeige.
  
  Je nach Auswahl in \var{HWSUPP\_LED\_x} hat \var{HWSUPP\_LED\_x\_PARAM} 
  eine unterschiedliche Bedeutung.  

  Ist \var{HWSUPP\_LED\_x='trigger'}, so ist der Name des LED-Triggers, 
  der die Ansteuerung der LED kontrolliert, in \var{HWSUPP\_LED\_x\_PARAM}
  einzutragen.

  Die verfügbaren Trigger können mit dem Shell-Kommando 
  \var{cat /sys/class/leds/*/trigger} angezeigt werden.
  
  Neben den Triggern die von z.B. netfilter oder Hardwaretreibern wie ath9k
  erzeugt werden, können weitere Trigger-Module über \var{HWSUPP\_DRIVER\_x}
  geladen werden.
  
  Beispiele:
  \begin{verbatim}
    HWSUPP_LED_1='trigger'
    HWSUPP_LED_1_PARAM='heartbeat'
    HWSUPP_LED_2='trigger'
    HWSUPP_LED_2_PARAM='netfilter-ssh'
  \end{verbatim}

  Ist \var{HWSUPP\_LED\_x='user'}, so ist in \var{HWSUPP\_LED\_PARAM} ein 
  gültiger Skriptname inclusive Pfad einzutragen.

  Beispiel:
  \begin{verbatim}
    HWSUPP_LED_1='user'
    HWSUPP_LED_1_PARAM='/usr/local/bin/myledscript'
  \end{verbatim}

% ##TODO## ? move to wlan doc begin
  Ist \var{HWSUPP\_LED\_x='wlan'}, so definiert \var{HWSUPP\_LED\_PARAM} ein
  oder mehrere WLAN Devices, deren Zustand angezeigt wird.
  Mehrere WLAN Devices sind durch Leerzeichen zu trennen.

  Wird der Zustand mehrerer WLAN Devices mit einer LED Angezeigt, so hat die 
  LED folgende Bedeutung:
  \begin{itemize}
    \item aus - alle WLAN Devices sind inaktiv
    \item blinkt - ein Teil der WLAN Devices ist aktiv
    \item an - alle WLAN Devices sind aktiv
  \end{itemize} 
  
  Beispiel:
  \begin{verbatim}
    HWSUPP_LED_1='wlan'
    HWSUPP_LED_1_PARAM='wlan0 wlan1'
  \end{verbatim}
% ##TODO## ? move to wlan doc end

\config {HWSUPP\_BOOT\_LED}{HWSUPP\_BOOT\_LED}{HWSUPPBOOTLED}
  Definiert eine LED die während des Bootvorgangs den Fortschritt durch
  eine Blinkfolge angezeigt. 

  Wenn für eine LED \var{HWSUPP\_LED\_x='ready'} gesetzt ist so hat diese
  Einstellung Vorrang und \var{HWSUPP\_BOOT\_LED} wird ignoriert.

\config {HWSUPP\_BUTTON\_N}{HWSUPP\_BUTTON\_N}{HWSUPPBUTTONN}
  Definiert die Anzahl der BUTTONs. Hier sollte die Anzahl der 
  Taster die die verwendete Hardware bereitstellt stehen.

\config {HWSUPP\_BUTTON\_x}{HWSUPP\_BUTTON\_x}{HWSUPPBUTTONx}
  Definiert die Aktion die durch drücken des Tasters durchgeführt werden soll.
  Folgende Aktionen sind möglich:
  \begin{itemize}
    \item reset - Startet den fli4l-Router neu
    \item online - Baut die Internetverbindung auf bzw. beendet diese
    \item user - Ein User-Script wird ausgeführt
  \end{itemize}
  
  Die Liste der möglichen Aktionen kann durch andere Pakete erweitert werden.
  So ist bei geladenem WLAN-Paket z.B. die Aktion 
  \begin{itemize}
    \item wlan - WLAN aktivieren bzw. deaktivieren
  \end{itemize}
  möglich.
  
\config {HWSUPP\_BUTTON\_x\_DEVICE}{HWSUPP\_BUTTON\_x\_DEVICE}{HWSUPPBUTTONxDEVICE}
  Gibt das Button-Device an. Hier gibt es zwei Möglichkeiten. Zum einen kann
  eine GPIO-Nummer eingetragen werden. Zum anderen kann ein so genannter
  Input-Pfad angegeben werden, der später über das evdev-Subsystem ausgelesen
  wird. Die zweite Möglichkeit ist, wenn möglich, immer vorzuziehen, weil die
  GPIO-Indizes nicht unbedingt über Kernelgrenzen hinweg stabil bleiben.

  Die GPIO-Nummer muss im Format \var{gpio::x} eingegeben werden. Normalerweise
  wird angenommen, dass die GPIO-Pins ``active-low'' sind, d.\,h. dass sie den
  Wert 0 annehmen, wenn ein Taster gedrückt wird, und 1, wenn der Taster nicht
  gedrückt ist. Liegt der umgekehrte Fall ``active-high'' vor (also 1 bei
  gedrücktem und 0 bei nicht gedrücktem Taster), kann dies durch Voranstellen
  von \texttt{/} angezeigt werden.

  Der Input-Pfad muss im Format \var{evdev:<Pfad>} eingegeben werden. In diesem
  Fall muss mit Hilfe der Variable \var{HWSUPP\_BUTTON\_x\_DEVICE\_KEY}
  spezifiziert werden, welche Taste verarbeitet werden soll. Dies ist nötig,
  weil im Gegensatz zur Anbindung via GPIO-Pin ein einziges evdev-Gerät viele
  verschiedene Buttons kapseln kann. Eine Übersicht der Tasten findet sich im
  Anhang \jump{sec:hwsupp:evdev-keys}{``Tasten-Codes''}.

  Eine Liste der vordefinierten Input-Pfade für den jeweiligen \var{HSUPP\_TYPE}
  findet sich im Anhang \jump{sec:buttondevices}{``Verfügbare Button-Devices''}.

  Beispiele:
  \begin{verbatim}
    HWSUPP_BUTTON_1_DEVICE='/gpio::237' # GPIO-Pin #237, active-high
    HWSUPP_BUTTON_2_DEVICE='evdev:isa0060/serio0/input0' # AT-Tastatur,
    HWSUPP_BUTTON_2_DEVICE_KEY='88'                      # F12-Taste
  \end{verbatim}

\config {HWSUPP\_BUTTON\_x\_PARAM}{HWSUPP\_BUTTON\_x\_PARAM}{HWSUPPBUTTONxPARAM}
  Definiert Parameter für die ausgewählte Aktion.
  
  Je nach Wert in  \var{HWSUPP\_BUTTON\_x} hat \var{HWSUPP\_BUTTON\_x\_PARAM}
  eine unterschiedliche Funktion.  

  Ist \var{HWSUPP\_BUTTON\_x='user'}, so definiert
  \var{HWSUPP\_BUTTON\_x\_PARAM} ein Script das beim Drücken des Tasters
  ausgeführt werden soll.
  
  Beispiel:
  \begin{verbatim}
    HWSUPP_BUTTON_1='user'
    HWSUPP_BUTTON_2_WLAN='/usr/local/bin/myscript'
  \end{verbatim}

% ##TODO## ? move to wlan doc begin
  Ist \var{HWSUPP\_BUTTON\_x\_ACTION='wlan'}, so sind in 
  \var{HWSUPP\_BUTTON\_x\_PARAM} das oder die WLAN Devices  einzutragen, 
  die durch Drücken des Tasters aktiviert bzw. deaktiviert werden.
  Mehrere WLAN Devices sind durch Leerzeichen zu trennen.
  
  Beispiel:
  \begin{verbatim}
    HWSUPP_BUTTON_2='wlan'
    HWSUPP_BUTTON_2_WLAN='wlan0 wlan1'
  \end{verbatim}
% ##TODO## ? move to wlan doc end

\subsection{Experten-Einstellungen}
  Die folgenden Einstellungen sollten nur gemacht werden, wenn man genau weiß
  \begin{itemize}
    \item welche Hardware man hat und welche zusätzlichen Treiber man dafür benötigt
    \item an welchen Adressen welche I\textsuperscript{2}C-Geräte\footnotemark liegen.
    
    \footnotetext{Ein I\textsuperscript{2}C-Bus oder SMBus ist ein serieller Bus
    der im PC z.B. zum Auslesen von Temperatur-Sensoren verwendet wird.
    Vielfach ist der I\textsuperscript{2}C-Bus oder SMBus auf einer Stiftleiste
    verfügbar und kann für eigene Hardwareerweiterungen genutzt werden.}
     
  \end{itemize}
  Nach dem Aktivieren der Experteneinstellungen erhält man eine Warnung beim
  mkfli4l Bau. 
  
\config {HWSUPP\_DRIVER\_N}{HWSUPP\_DRIVER\_N}{HWSUPPDRIVERN}
  Anzahl der zusätzlich zu ladenden Treiber. 
  Die Treiber in \var{HWSUPP\_DRIVER\_x} werden in der angegebenen Reihenfolge
  geladen.  

\config {HWSUPP\_DRIVER\_x}{HWSUPP\_DRIVER\_x}{HWSUPPDRIVERx}
  Name des Treibers (ohne Dateiendung \var{.ko}).

Beispiel:
\begin{verbatim}
HWSUPP_DRIVER_N='2'
HWSUPP_DRIVER_1='i2c-piix4'     # I2C Bus Treiber
HWSUPP_DRIVER_2='gpio-pcf857x'  # I2C GPIO Expander
\end{verbatim}
  
\config {HWSUPP\_I2C\_N}{HWSUPP\_I2C\_N}{HWSUPPI2CN}
  Anzahl der zu ladenden I\textsuperscript{2}C-Geräte.
  
  I\textsuperscript{2}C unterstützt keine PnP-Mechanismen. 
  Daher sind für jedes zu ladende I\textsuperscript{2}C-Gerät die Busnummer,
  die Geräteadresse und der Gerätetyp anzugeben.  

\config {HWSUPP\_I2C\_x\_BUS}{HWSUPP\_I2C\_x\_BUS}{HWSUPPI2CxBUS}
  I\textsuperscript{2}C-Busnummer an der das zu ladende Gerät angeschlossen ist.
  
  Die Busnummer ist im Format \var{i2c-x} anzugeben.
  
\config {HWSUPP\_I2C\_x\_ADDRESS}{HWSUPP\_I2C\_x\_ADDRESS}{HWSUPPI2CxADDRESS}
  I\textsuperscript{2}C-Busadresse des Geräts.
  
  Die Adresse ist als Hexadezimalzahl im Bereich von \var{0x03} bis \var{0x77}
  anzugeben. 

\config {HWSUPP\_I2C\_x\_DEVICE}{HWSUPP\_I2C\_x\_DEVICE}{HWSUPPI2CxDEVICE}
  Der Typ des I\textsuperscript{2}C-Geräts der vom einem zuvor geladenen Treiber
  erkannt wird.
  
Beispiel:
\begin{verbatim}
HWSUPP_I2C_N='1'
HWSUPP_I2C_1_BUS='i2c-1'
HWSUPP_I2C_1_ADDRESS='0x38'
HWSUPP_I2C_1_DEVICE='pcf8574a' # Unterstützt von gpio-pcf857x Treiber 
\end{verbatim}

\subsection {Unterstützung von VPN-Karten}

\config {OPT\_VPN\_CARD}{OPT\_VPN\_CARD}{OPTVPNCARD}
  Die Einstellung \var{'no'} deaktiviert das OPT\_VPN\_CARD vollständig. Es
  werden keine Änderungen am fli4l Archiv \var{rootfs.img} bzw. dem Archiv \var{opt.img}
  vorgenommen. Weiterhin überschreibt das OPT\_VPN\_CARD grundsätzlich keine
  anderen Teile der fli4l Installation.\\
  Um OPT\_VPN\_CARD zu aktivieren, ist die Variable \var{OPT\_VPN\_CARD} auf 
  \var{'yes'} zu setzen.

\config {VPN\_CARD\_TYPE}{VPN\_CARD\_TYPE}{VPNCARDTYPE}

  In dieser Konfigurationsvariable wird der zu unterstützende VPN Beschleuniger
  festgelegt. Folgende Werte stehen zur Verfügung:
  \begin{itemize}
    \item hifn7751 - Soekris vpn1401 und vpn1411
    \item hifnhipp
  \end{itemize}

\end{description}
