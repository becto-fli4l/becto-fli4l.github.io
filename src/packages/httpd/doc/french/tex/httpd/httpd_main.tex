% Synchronized to r48039

\section{HTTPD~- Statut du routeur avec le serveur web}

\marklabel{OPTHTTPD}{\subsection{OPT\_HTTPD~- Mini serveur web comme moniteur de statut}}\index{OPT\_HTTPD}

  Pour ceux qui n'ont pas la possibilité d'utiliser IMONC, pour certaines raison,
  par ex. parce qu'il utilise un Mac, ils peuvent utiliser le serveur web pour
  obtenir ou modifier le statut du routeur fli4l. En activant la variable
  \var{OPT\_HTTPD='yes'}, vous pouvez utiliser le serveur web et l'écran du
  statut fli4l. 

  Pour obtenir la page d'accueil du statut, il faut indiquer dans votre
  navigateur l'une des adresses suivantes~:

\begin{example}
\begin{verbatim}
    http://fli4l/
    http://fli4l.domain.lan/
    http://192.168.6.1/
\end{verbatim}
\end{example}

  Si votre routeur fli4l a un nom différent, celui-ci doit être utilisé à
  la place de "fli4l". Cela vaut aussi pour le nom de domaine et aussi pour
  l'adresse IP. Si vous avez configuré le serveur web sur un autre port
  (avec la variable \var{HTTPD\_\-PORT}), vous devez indiquer ceux-ci~:

\begin{example}
\begin{verbatim}
    http://fli4l:81/
\end{verbatim}
\end{example}

  Depuis la version 2.1.12 vous pouvez accéder à une page d'accueil pour le
  login et le mot de passe. Si vous voulez aller directement sur la page
  d'authentification avec le mot de passe et le login, cette page se trouve
  dans le sous-répertoire admin et vous devez spécifier alors~:

\begin{example}
\begin{verbatim}
    http://fli4l.domain.lan/admin/
\end{verbatim}
\end{example}

   Vous pouvez configurer le serveur web avec les variables suivantes~:

\begin{description}

\config{HTTPD\_GUI\_LANG}{HTTPD\_GUI\_LANG}{HTTPDGUILANG}

    {Vous pouvez régler avec cette variable la langue, dans laquelle
    l'interface web doit être affichée. Si vous enregistrez 'auto', le réglage
    linguistique de la variable \var{LOCALE} (dans base.txt) sera utilisé.}

\config{HTTPD\_LISTENIP}{HTTPD\_LISTENIP}{HTTPDLISTENIP}

    Normalement, le serveur web se connecte sur une adresse Wildcard (ou adresse
    pouvant prendre n'importe quelle valeur), de sorte qu'il puisse réagir sur
    n'importe quelle interface du routeur. On peut aussi connecter une seule
    adresse IP, cela peut être paramétré dans cette variable. Pour l'enregistrement
    d'une seul adresse IP il faut qu'elle soit indiquée dans~: \var{IP\_NET\_x\_IPADDR}.
    Normalement, le réglage de la variable doit rester vide, pour que la valeur
    par défaut puisse fonctionner (sur n'importe quel adresse IP réactif).

    Ce paramètre sert uniquement pour le httpd, il est associé seulement à une
    adresse IP, il ne peut pas être utilisé, pour accéder aux différents
    sous-réseaux de l'interface Web du routeur, car les accès seront bloqués.
    Si vous voulez utiliser d'autres adresses IP dans cette variable, vous devez
    utiliser le filtrage de paquets, pour que l'ensemble des adresses soient
    reconnus sur le routeur. Il est possible d'indiquer plusieurs adresses IP,
    en les séparant par un espace.

\config{HTTPD\_PORT}{HTTPD\_PORT}{HTTPDPORT}

    {Si le serveur web doit fonctionner sur un autre port que 80, cette variable
    doit être adaptée. Normalement cela n'est pas très recommandé, car nous
    devons alors interroger le serveur web avec http://fli4l:81/}

\config{HTTPD\_PORTFW}{HTTPD\_PORTFW}{HTTPDPORTFW}

    {Si l'on met cette variable sur 'yes', on peut effectuer des changements
    sur la transmission de port par l'intermédiaire de l'interface de web. Les
    règles peuvent être supprimées ou ajoutées, ces modifications entrent en
    vigueur immédiatement. Mais ces modifications ne sont valable que pour la
    durée de fonctionnement du routeur. Si le routeur est redémarré,
    les modifications sont annulées.

    Paramètre par défaut 'yes'.}

\config{HTTPD\_ARPING}{HTTPD\_ARPING}{HTTPDARPING}

    {Le serveur web montre l'état des hôtes en-ligne, ceux-ci sont énumérés dans
    la variable \var{HOST\_x}. Le serveur utilise le \emph{"cache-arp"}, pour
    enregistrer provisoirement dans la mémoire les adresses des hôtes locaux.
    Si un ordinateur n'a pas communiqué depuis longtemps avec le routeur, son
    adresse IP disparaît du \emph{"cache-arp"} et l'hôte semble être déconnecté.
    Vous devez tenir à jour le \emph{"cache-arp"} (cela empêchera les hotes qui
    ne sont pas réellement déconnectés de ne plus être vu), pour cela vous devez
    activer la variable \var{HTTPD\_ARPING} et mettre la valeur \emph{'yes'}.}

    \config{HTTPD\_ARPING\_IGNORE\_N}{HTTPD\_ARPING\_IGNORE\_N}{HTTPDARPINGIGNOREN}
  {Dans cette variable vous enregistrez le nombre de arping qui seront ignoré.}

    \config{HTTPD\_ARPING\_IGNORE\_x}{HTTPD\_ARPING\_IGNORE\_x}{HTTPDARPINGIGNOREx}
  {Dans cette variable vous indiquez l'adresse IP ou nom de l'hôte qui ne sera
  pas inclus dans le test arping. Cela peut être utile, pour les hôtes qui
  utilisent (le wifi du réseau local pour les ordinateurs portables ou les téléphones).
  Car les paquets de requêtes régulières passant par le réseau consomment plus
  rapidement la batterie.}
\end{description}

\subsection{Gestion des utilisateurs}

  Le serveur web offre à utilisateur une administration précise~:

\begin{description}
\config{HTTPD\_USER\_N}{HTTPD\_USER\_N}{HTTPDUSERN}

    {Avec cette variable on ajuste, le nombre d'utilisateurs. Si cette variable
    est placée sur 0, l'administration des utilisateurs est désactivée
    complètement et tous le monde a la possibilité d'interroger le serveur web.}

\configlabel{HTTPD\_USER\_x\_PASSWORD}{HTTPDUSERxPASSWORD}
\configlabel{HTTPD\_USER\_x\_RIGHTS}{HTTPDUSERxRIGHTS}
\config{HTTPD\_USER\_x\_USERNAME  HTTPD\_USER\_x\_PASSWORD  HTTPD\_USER\_x\_RIGHTS}
       {HTTPD\_USER\_x\_USERNAME}
       {HTTPDUSERxUSERNAME}

    {Avec ces variables on paramètre les différents utilisateurs avec, le nom
    d'utilisateur, le mot de passe et les droits. On règle Dans la variable
    \var{HTTPD\_\-RIGHTS\_\-x} les fonctions pour que chaque utilisateur puisse
    interroger le serveur web. Dans le cas le plus simple, on paramètre
    seulement 'all' ce qui signifie que l'utilisateur correspondant peut
    interroger toutes les fonctions. La variable peut avoir les constructions 
    suivantes~:

\begin{example}
\begin{verbatim}
       'fontion1:droit1,droit2,... fonction2:...'
\end{verbatim}
\end{example}

    Au lieu d'indiquer les différents droits pour chaque fonction, la valeur
    "all" peut être paramétrée, ainsi l'utilisateur aura tous les droits pour
    chaque fonction. Ci-dessous les fonctions et droit~:

      \begin{description}
      \item[Fonction "status"] Tout ce qui peut être vue dans le menu statut
        \begin{description}
        \item[view] L'utilisateur peut lancer tous les points du menu.
        \item[dial] L'utilisateur peut décrocher et raccrocher la ligne Tél.
        \item[boot] L'utilisateur peut arrêter \& redémarrer le routeur.
        \item[link] L'utilisateur peut ajouter ou mettre hors circuit un canal (ISDN)
        \item[circuit] L'utilisateur peut changer le circuit.
        \item[dialmode] L'utilisateur peut modifier le mode-dial (Auto, Manual, Off).
        \item[conntrack] L'utilisateur peut voir les connexions actuelles en
          fonctionnent sur le Routeur.
        \item[dyndns] L'utilisateur peut voir les informations du paquetage \jump{sec:dyndns}{\var{DYNDNS}}.
        \end{description}

      \item[Fonction "logs"] Tout se que l'on peut faire avec le fichier log (ou
        fichier journal) (connexion, appels, Syslog)
        \begin{description}
        \item[view] L'utilisateur peut voir le fichier journal des événements.
        \item[reset] L'utilisateur peut supprimer le fichiers journal des événements.
        \end{description}

      \item[Fonction "support"] Tout ce qui est utile, pour chercher des informations
        par exemple de l'aide dans le Newsgroup.
        \begin{description}
        \item[view] L'utilisateur peut utiliser les liens pour la documentation, aller
          sur à la page Web de fli4l, etc.
        \item[systeminfo] L'utilisateur peut voir les informations sur la configuration
          et l'état actuel du routeur (par ex. le pare-feu).
        \end{description}
      \end{description}

    Voici quelques exemples~:

      \begin{description}
      \item[HTTPD\_USER\_1\_RIGHTS='all']
        Avec ce paramètre l'utilisateur peut tous faire~!

      \item[HTTPD\_USER\_2\_RIGHTS='status:view logs:view support:all']
        Avec ces paramètres l'utilisateur peut tout voir, mais rien modifier.

      \item[HTTPD\_USER\_3\_RIGHTS='status:view,dial,link']
        Avec ces paramètres l'utilisateur peut suivre l'état de la connexion
        Internet, choisir l'agrégation des canaux (ISDN) ou de l'éteindre.

      \item[HTTPD\_USER\_4\_RIGHTS='status:all']
        Avec ce paramètre l'utilisateur peut tout faire avec les connexions
        Internet, aussi de redémarrer (et naturellement arrêter le routeur).
        Cependant il ne peut pas voir le fichier journal ou de l'effacer,
        il ne peut pas non plus voir les plages horaires des connexions Internet...
      \end{description}
    }

\end{description}

\subsection{OPT\_OAC~- Contrôle d'accès en ligne (OAC)}

\begin{description}

\config{OPT\_OAC}{OPT\_OAC}{OPTOAC} (Variable optionnelle)

    Avec cette variable vous activez le module pour le contrôle d'accès en ligne
    (OAC), l'accès Internet peut être configuré par rapport aux clients,
    sélectionnable dans le paquetage \jump{sec:dnsdhcp}{dns\_dhcp}, ainsi les
    ordinateurs auront une mobilité réduite.

    Cet outil existe également en ligne de commande, il permet de contrôler
    d'autres paquets, par ex. Easycron~:

    /usr/local/bin/oac.sh

    Les options sont affichées avec la commande ci-dessous.

\config{OAC\_WANDEVICE}{OAC\_WANDEVICE}{OACWANDEVICE} (Variable optionnelle)

    Avec cette variable vous indiquez le périphérique réseau qui sera utilisé
    pour restreindre ou verrouiller les accés. Par ex. 'pppoe'

\config{OAC\_INPUT}{OAC\_INPUT}{OACINPUT} (Variable optionnelle)

    Avec cette variable vous assurez la protection contre l'environnement via
    le Proxy.

    OAC\_INPUT='default' bloque les ports des configurations de~:
    Privoxy, Squid, Tor, SS5, Transproxy.

    OAC\_INPUT='tcp:8080 tcp:3128' bloque le Port TCP 8080 et 3128.
    Ici on bloque une liste de Ports avec le protocole qu'il l'accompagne
    (UDP, TCP), les ports seront séparés par un espace. Si le protocole udp ou tcp
    n'est pas indiqué, le port ne sera pas conforme.

    Vous pouvez omettre cette variable ou indiquer 'no' pour désactiver cette
    fonction.

\config{OAC\_ALL\_INVISIBLE}{OAC\_ALL\_INVISIBLE}{OACALLINVISIBLE} (Variable optionnelle)

    Avec cette variable on passe sur la vue d'ensemble, pour savoir s'il existe
    au moins un profil de groupe visible. S'il n'y a pas de profil de groupe
    visible, alors cette variable n'a aucune action.

\config{OAC\_LIMITS}{OAC\_LIMITS}{OACLIMITS} (Variable optionnelle)

    On indique dans cette variable une limite temps, que vous pouvez choisir
    dans la liste ci-dessous. Les limites temps sont donnés en minutes. De
    cette façon, vous pouvez bloquer ou le débloquer temporairement un accés
    sur le réseau.

    Par défaut~: '30 60 90 120 180 360 540'

\config{OAC\_MODE}{OAC\_MODE}{OACMODE} (Variable optionelle)

    Dans cette variable les valeurs possibles sont~: \var{'DROP'} ou \var{'REJECT'} (par défaut)

\config{OAC\_GROUP\_N}{OAC\_GROUP\_N}{OACGROUPN} (Variable optionnelle)

    Dans cette variable vous indiquez le nombre de groupes clients. Pour plus de
    clarté, vous pouvez également utiliser l'interface web pour créer l'ensemble
    des groupes, qui seront autoriser ou bloquer.

\config{OAC\_GROUP\_x\_NAME}{OAC\_GROUP\_x\_NAME}{OACGROUPxNAME} (Variable optionelle)

    Avec cette variable vous indiquez le nom du groupe. Ce nom sera affiché dans
    l'interface web et sera également utilisable avec le script 'oac.sh' en
    ligne de commande.

\config{OAC\_GROUP\_x\_BOOTBLOCK}{OAC\_GROUP\_x\_BOOTBLOCK}{OACGROUPxBOOTBLOCK} (Variable optionnelle)

    Si vous indiquez dans cette variable 'yes', tous les clients du groupe
    seront bloqués lors du boot. En règle générale ces ordinateurs seront
    verrouillés et pas seulement dans certains cas exceptionnels.

\config{OAC\_GROUP\_x\_INVISIBLE}{OAC\_GROUP\_x\_INVISIBLE}{OACGROUPxINVISIBLE} (Variable optionnelle)

    Dans cette variable vous pouvez marquer le groupe comme invisible. Si ces
    ordinateurs sont bloqués à l'avance, ces groupes ne seront pas visibles
    dans l'interface web.

\config{OAC\_GROUP\_x\_CLIENT\_N}{OAC\_GROUP\_x\_CLIENT\_N}{OACGROUPxCLIENTN} (Variable optionelle)

    Dans cette variable vous indiquez le nombre de clients dans le groupe.

\config{OAC\_GROUP\_x\_CLIENT\_x}{OAC\_GROUP\_x\_CLIENT\_x}{OACGROUPxCLIENTx} (Variable optionelle)

    Dans cette variable vous indiquez le nom du client, qui est paramétré dans
	la variable {\var{HOST\_x\_NAME}} du paquetage \jump{sec:dnsdhcp}{dns\_dhcp.}

\config{OAC\_BLOCK\_UNKNOWN\_IF\_x}{OAC\_BLOCK\_UNKNOWN\_IF\_x}{OACBLOCKUNKNOWNIF} (Variable optionelle)

    Dans cette variable vous indiquez la liste des interfaces définies dans
    le fichier base.txt, sur lesquels, seuls les hôtes définis dans le fichier
    (dns\_dhcp.txt) pourront accéder à Internet. Les hôtes non définis seront
    généralement bloqués.

\end{description}

