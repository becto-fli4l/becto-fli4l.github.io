% Last Update: $Id: mailsend_appendix.tex 48225 2017-06-05 11:46:00Z cspiess $
 
\marklabel{sec:mailsendexample}{
\section{Beispiele}}
\subsection {\var{mailsend} mit Text als Parameter}
\begin{verbatim}
mailsend -t "test@test.xy" -s "test" "das ist ein Test"
\end{verbatim}

\subsection {\var{mailsend} mit Text per Pipe}
\begin{verbatim}
echo "das ist ein Test" | mailsend -t "test@test.xy" -s "test"
\end{verbatim}

\subsection {\var{mailsend} mit Text per Datei}
\begin{verbatim}
echo "das ist ein Test" > /tmp/mail.txt
mailsend -t "test@test.xy" -s "test" -B "/tmp/mail.txt"
\end{verbatim}

\subsection {\var{mailsend} mit Variablen}
\begin{verbatim}
_to=#test@test.xy'
_subject='test'
_body='das ist ein Test'
mailsend -t "${_to}" -s "${_subject}" -B "${_body}"
\end{verbatim}

\subsection {\var{mailsend} mit Anhang}
\begin{verbatim}
mailsend -t "test@test.xy" -s "syslog" -a "/var/log/syslog" "syslog"
\end{verbatim}

\marklabel{sec:developer}{
\section{MAILSEND - Hinweise für Paket-Entwickler}}
    Im Folgenden ist beschrieben was ein Paket-Entwickler zu tun hat, 
    um MAILSEND-Funktionalität zu einem Paket hinzuzufügen.
 
\subsection{Konfiguration}
    In der Datei \var{check/myopt.txt} ist die Liste der Konfigurationsvariablen 
    um mindestens zwei Variablen zu erweitern:
\begin{verbatim}
MYOPT_MAIL_ACCOUNT      OPT_MYOPT       -       LABEL "default"
MYOPT_MAIL_TO           OPT_MYOPT       -       MAILADDR_LIST 
\end{verbatim}

    Ggf. ist die indizierte Variante zu verwenden:
\begin{verbatim}
MYOPT_%_MAIL_ACCOUNT    OPT_MYOPT   MYOPT_N     LABEL "default"
MYOPT_%_MAIL_TO         OPT_MYOPT   MYOPT_N     MAILADDR_LIST 
\end{verbatim}
    
    Weitere Variablen, wie z.B. CC- oder BCC-Adressen und Betreff können
    ebenfalls ergänzt werden.
     
    Die Konfigurationsdatei \var{config/myopt.txt} ist um die entsprechenden
    Variablen zu erweitern.
    
\begin{verbatim}
MYOPT_MAIL_ACCOUNT='default'
MYOPT_MAIL_TO='receiver@domain.xyz' 
\end{verbatim}
     
\subsection{Parameterprüfung}
    In der Datei \var{check/myopt.ext} wird die Installation von MAILSEND geprüft:
  
\begin{verbatim}
if (my_opt)
then
    [...]
    if (opt_mailsend)
    then
        depends on mailsend version 4.0
        [...]
    else
        error "OPT_MAILSEND must be 'yes' when ...'"
    fi
fi
\end{verbatim}

    Ebenso sollte das Mail-Konto auf Gültigkeit geprüft werden:
\begin{verbatim}
    if (myopt_mail_account != "default")
    then
        set found="no"
        foreach m in MAILSEND_N
        do
            if (mailsend_%_account[m] == myopt_mail_account)
            then
                set found="yes" 
            fi
        done
        if (found != "yes")
        then
            error "Account '${MYOPT_MAIL_ACCOUNT}' not found in MAILSEND_x_ACCOUNT"
        fi
    fi
\end{verbatim}

\subsection{Mail-Versand}
    In einem Skript sollte der Betreff und der Text der Mail generiert werden
    und dann an \var{mailsend} übergeben werden:
    
\begin{verbatim}
. /var/run/myopt.conf
_subject='myopt mail subject'
_message='myopt mail message'
mailsend -A "${MYOPT_MAIL_ACCOUNT}" -t "${MYOPT_MAIL_TO}" \
    -s "${_subject}" "${_message}"
\end{verbatim}

\marklabel{sec:mailsendappendix}{
\section{Anhang}}
 
\subsection{Dank}

  Das Paket MAILSEND basiert auf dem Paket MSMTP 
  von Christoph Fritsch {[\ref{opt_msmtp}]}.

  Die Mails werden mit msmtp {[\ref{msmtp}]} versendet.
  
\subsection{Referenzen}

\newcounter{ref}
\begin{list}{\textbf{[\arabic{ref}]}}{\usecounter{ref}}
  
  \item \label{opt_msmtp}
  \altlink{http://extern.fli4l.de/fli4l_opt-db3/download.pl?file=2780-opt_msmtp-3.10.1-0.9.13_x86.zip}
  
  \item \label{msmtp}
  \altlink{http://msmtp.sourceforge.net/}

    \item \label{rfc5322}
    \altlink{https://tools.ietf.org/pdf/rfc5322}
\end{list}
