% Last Update: $Id: mailsend_main.tex 44114 2016-01-18 22:17:38Z kristov $
\section {MAILSEND - eMails versenden}

\subsection {Beschreibung}
    Das Paket MAILSEND stellt eine API zum Versenden von eMails für ander Pakete
    zur Verfügung.
    
    Es wird zusätzlich das Paket EASYCRON benötigt (\var{OPT\_EASYCRON='yes'}).

\subsection {Konfiguration des Paketes MAILSEND}
    Die Konfiguration erfolgt, wie bei allen fli4l Paketen, durch Anpassung der \\
    Datei \var{Pfad/fli4l-\version/\emph{\flq config\frq}/mailsend.txt}
    an die eigenen Anforderungen.

\subsubsection {Allgemeines}
\begin{description}

\config {OPT\_MAILSEND}{OPT\_MAILSEND}{OPTMAILSEND}

    Die Einstellung \var{'no'} deaktiviert das OPT\_MAILSEND vollständig.
    Es werden keine Änderungen am fli4l Archiv \var{rootfs.img} bzw. 
    dem Archiv \var{opt.img} vorgenommen. 
    Weiterhin überschreibt das OPT\_MAILSEND grundsätzlich keine anderen
    Teile der fli4l Installation.
    
    Um OPT\_MAILSEND zu aktivieren, ist die Variable \var{OPT\_MAILSEND} auf
    \var{'yes'} zu setzen.

\config {MAILSEND\_SPOOL\_DIR}{MAILSEND\_SPOOL\_DIR}{MAILSENDSPOOLDIR}
    
    Diese optionale Einstellung legt das Spoolverzeichnis fest.
    Das Spoolverzeichnis sollte auf einer persistenten Partition liegen.
    
    Als Default wird \var{'/data/spool/mail'} verwendet.
     
\config {MAILSEND\_LOGFILE}{MAILSEND\_LOGFILE}{MAILSENDLOGFILE}

    Diese optionale Einstellung legt die Logdatei fest.
    Diese Einstellung muss auf \var{'syslog'} gesetzt werden, wenn in
    das System Logbuch geloggt werden soll. 
    
    Diese Einstellung muss auf \var{'{}'} gesetzt werden, wenn nicht geloggt 
    werden soll (Defaultwert).  

\config {MAILSEND\_SEND\_DELAYED}{MAILSEND\_SEND\_DELAYED}{MAILSENDSENDDELAYED}
    
    Wird diese Einstellung auf \var{'yes'} gesetzt, so wird die Mail 
    erst verzögert durch einen cron-Job oder ein ip-up Event gesendet.
    Bei \var{'no'} (Defaultwert) wird die Mail sofort versendet.
    
\config {MAILSEND\_CHARSET}{MAILSEND\_CHARSET}{MAILSENDCHARSET}
    
    Diese optionale Einstellung legt den Zeichensatz für die String-Kodierung
    fest. Als Defaultwert wird \var{'ISO-8859-1'} verwendet, es kann auch z.B. 
    \var{'UTF-8'} eingestellt werden. 
                                
\config {MAILSEND\_N}{MAILSEND\_N}{MAILSENDN}
  
  Legt die Anzahl der Mail-Konten fest. 

\end{description}

\subsubsection {Mail-Konten}
    Einstellungen für die Mail-Konten, diese Einstellungen werden für 
    jedes Mail-Konto separat vorgenommen.
     
\begin {description}

\config {MAILSEND\_\emph{x}\_ACCOUNT}{MAILSEND\_\emph{x}\_ACCOUNT}{MAILSENDACCOUNT}
    
    Legt den logischen Namen des Mail-Kontos fest.
    
    Hierüber können mehrere Mail-Konten unterschieden werden, 
    daher muss dieser Name für alle Konten jeweils unterschiedlich sein.
    
    Wenn kein Konto den Namen \var{'default'} hat, so kann das Konto mit 
    dem Index \var{'1'} zusätzlich unter diesem Namen angesprochen werden.
    
\config {MAILSEND\_\emph{x}\_AUTH}{MAILSEND\_\emph{x}\_AUTH}{MAILSENDAUTH}
    
    Legt die Authentifizierungsmethode fest. Mögliche Werte sind:
    
    \begin {itemize}
      \item [\var{'none'}] keine Authentifizierung
      \item [\var{'pop'}] POP3 vor SMTP
      \item [\var{'plain'}] unverschlüsseltes Passwort
      \item [\var{'cram-md5'}] MD5 Hash-Wert
      \item [\var{'login'}] unverschlüsseltes Passwort (ähnlich \var{'plain'})
    \end {itemize}
    
\config {MAILSEND\_\emph{x}\_CRYPT}{MAILSEND\_\emph{x}\_CRYPT}{MAILSENDCRYPT}

    Legt die Verschlüsselungsmethode fest. Mögliche Werte sind:
    
    \begin {itemize}
      \item [\var{'none'}] keine Verschlüsselung
      \item [\var{'TLS'}] SSL/TLS
      \item [\var{'STARTTLS'}] TLS mit STARTTLS
    \end {itemize}

\config {MAILSEND\_\emph{x}\_FROM}{MAILSEND\_\emph{x}\_FROM}{MAILSENDFROM}

    Diese optional Einstellung legt die Absenderadresse fest.
    
    Wenn diese Einstellung leer ist wird 
    \smalljump{MAILSENDSMTPUSERNAME}{\var{MAILSEND\_\emph{x}\_SMTP\_USERNAME}} verwendet.
    
    Die Absenderadresse kann mit dem \smalljump{sec:mailsendapi}{API} Aufruf 
    überschrieben werden.
 
\end{description}

\subsubsection {SMTP-Server}
\begin {description}

\config {MAILSEND\_\emph{x}\_SMTP\_SERVER}{MAILSEND\_\emph{x}\_SMTP\_SERVER}{MAILSENDSMTPSERVER}
    
    Legt den SMTP-Server fest, hier kann entweder der Hostname oder
    die IP-Adresse eingetragen werden.
            
\config {MAILSEND\_\emph{x}\_SMTP\_PORT}{MAILSEND\_\emph{x}\_SMTP\_PORT}{MAILSENDSMTPPORT}
    
    Mit dieser optionalen Einstellung kann die SMTP-Portnummer festgelegt
    werden, als Defaultwert wird der SMTP-Standardport \var{'25'} verwendet.
    
\config {MAILSEND\_\emph{x}\_SMTP\_USERNAME}{MAILSEND\_\emph{x}\_SMTP\_USERNAME}{MAILSENDSMTPUSERNAME}
    
    Der Username für die SMTP-Authentifizierung.
                
\config {MAILSEND\_\emph{x}\_SMTP\_PASSWORD}{MAILSEND\_\emph{x}\_SMTP\_PASSWORD}{MAILSENDSMTPPASSWORD}
    
    Das Passwort für die SMTP-Authentifizierung.

\end {description}

\subsubsection {Optional: POP3-Server}

    Diese Einstellungen sind nur notwendig wenn 
    \smalljump{MAILSENDAUTH}{\var{MAILSEND\_\emph{x}\_AUTH}} auf \var{'pop'} gesetzt ist. 

\begin {description}
\config {MAILSEND\_\emph{x}\_POP3\_SERVER}{MAILSEND\_\emph{x}\_POP3\_SERVER}{MAILSENDPOP3SERVER}

    Legt den POP3-Server fest, hier kann entweder der Hostname oder
    die IP-Adresse eingetragen werden.
            
\config {MAILSEND\_\emph{x}\_POP3\_PORT}{MAILSEND\_\emph{x}\_POP3\_PORT}{MAILSENDPOP3PORT}
    
    Mit dieser optionalen Einstellung kann die POP3-Portnummer festgelegt
    werden, als Defaultwert wird der POP3-Standardport \var{'110'} verwendet.

\config {MAILSEND\_\emph{x}\_POP3\_USERNAME}{MAILSEND\_\emph{x}\_POP3\_USERNAME}{MAILSENDPOP3USERNAME}
    
    Der Username für die POP3-vor-SMTP-Authentifizierung.
    
\config {MAILSEND\_\emph{x}\_POP3\_PASSWORD}{MAILSEND\_\emph{x}\_POP3\_PASSWORD}{MAILSENDPOP3PASSWORD}
    
    Das Passwort für die POP3-vor-SMTP-Authentifizierung.

\end {description}

\subsubsection {Erweiterte Features}

\paragraph {Server-Zertifikat}
Test des Server-Zertifikats

\begin {description}
    
\config {MAILSEND\_\emph{x}\_TEST\_SCERT}{MAILSEND\_\emph{x}\_TEST\_SCERT}{MAILSENDTESTSCERT}

    Wenn diese Einstellung \var{'yes'} ist, wird das SMTP-Server-Zertifikat 
    gegen das Zertifikat in \smalljump{MAILSENDSCERT}{MAILSEND\_\emph{x}\_SCERT}
    geprüft.

\config {MAILSEND\_\emph{x}\_SCERT}{MAILSEND\_\emph{x}\_SCERT}{MAILSENDSCERT}
    
    Dateiname inklusive Pfad des X.509-Server-Zertifikats im PEM-Format,
    z.B. \var{'/etc/mailsend/server-cert.pem'}
    
    Als Pfad empfiehlt sich \var{'/etc/mailsend/\emph{server}.pem'}
    Man kann dann das Zertifikat im Konfigurationsverzeichnis unter
    \var{\emph{\flq config\frq}/etc/mailsend/} ablegen.

\end {description}

\paragraph {Client-Zertifikat/Schlüssel}
\begin {description}

\config {MAILSEND\_\emph{x}\_USE\_CCERT}{MAILSEND\_\emph{x}\_USE\_CCERT}{MAILSENDUSECCERT}

    Wenn diese Einstellung \var{'yes'} ist, wird der SMTP-Client mit dem Paar
    aus Zertifikat in \smalljump{MAILSENDCCERT}{MAILSEND\_\emph{x}\_CCERT}
    und privatem Schlüssel in \smalljump{MAILSENDCKEY}{MAILSEND\_\emph{x}\_CKEY}
    authentifiziert. 

    
\config {MAILSEND\_\emph{x}\_CCERT}{MAILSEND\_\emph{x}\_CCERT}{MAILSENDCCERT}
    
    Dateiname des X.509-Client-Zertifikats im PEM-Format, \\
    z.B. \var{'/etc/mailsend/client-cert.pem'}

\config {MAILSEND\_\emph{x}\_CKEY}{MAILSEND\_\emph{x}\_CKEY}{MAILSENDCKEY}
    
    Dateiname des X.509-Client-Schlüssels im PEM-Format, \\
    z.B. \var{'/etc/mailsend/client-key.pem'}
    
\end {description}

\marklabel{sec:mailsendapi}{
\subsection {API}}
\subsubsection{\var{mailsend}}
    Der API \var {/usr/local/bin/mailsend} kann der Mailtext entweder 
    als Datei, als letzter Parameter in Anführungszeichen oder über die 
    Standard-Input-Pipe übergeben werden
    (siehe \smalljump{sec:mailsendexample}{Beispiele}).

    Die API kann mit den folgenden Parametern aufgerufen werden:
    
\begin {itemize}
  \item [\var{-t},] \var{-{}-to} \var{"\emph{\flq address\frq}"}
    \\erforderlich: Legt eine oder mehrere durch Komma getrennte
    Empfängeradressen fest an die die Mail gesendet wird.
    \\ Diese Option kann auch mehrfach gesetzt werden.

  \item [\var{-s},] \var{-{}-subject} \var{"\emph{\flq subject\frq}"}
    \\erforderlich: Legt den Betreff der Mail fest.
  
  \item [\var{-A},] \var{-{}-account} \var{"\emph{\flq account\frq}"}
    \\optional: Legt den \smalljump{MAILSENDACCOUNT}{Kontonamen}
    fest über den die Mail versendet wird. Fehlt dieser Parameter wird das
    Konto mit dem \smalljump{MAILSENDACCOUNT}{Namen} \var{default} verwendet.
  
  \item [\var{-f},] \var{-{}-from} \var{"\emph{\flq address\frq}"}
    \\optional: Legt die Absenderadresse fest mit der die Mail gesendet wird. 
    Fehlt dieser Parameter wird die Adresse aus 
    \smalljump{MAILSENDFROM}{\var{MAILSEND\_\emph{x}\_FROM}} verwendet.
  
  \item [\var{-c},] \var{-{}-cc} \var{"\emph{\flq address\frq}"}
    \\optional: Legt eine oder mehrere durch Komma getrennte
    CC-Empfängeradressen fest an die eine Kopie der Mail gesendet wird.
    \\ Diese Option kann auch mehrfach gesetzt werden.

  \item [\var{-b},] \var{-{}-bcc} \var{"\emph{\flq address\frq}"}
    \\optional: Legt eine oder mehrere durch Komma getrennte
    BCC-Empfängeradressen fest an die eine Blindkopie der Mail gesendet wird.
    \\ Diese Option kann auch mehrfach gesetzt werden.

  \item [\var{-r},] \var{-{}-reply-to} \var{"\emph{\flq address\frq}"}
    \\optional: Legt eine oder mehrere durch Komma getrennte
    Antwortadressen fest an die eine Antwort auf die Mail gesendet werden soll.
    \\ Diese Option kann auch mehrfach gesetzt werden.

  \item [\var{-B},] \var{-{}-body} \var{"\emph{\flq file\frq}"}
    \\optional: Lädt den Mailtext aus einer Datei (absoluter Pfad).
    \\Fehlt diese Option wird der Mailtext per Parameter oder Pipe übergeben 
    (siehe \smalljump{sec:mailsendexample}{Beispiele}).

  \item [\var{-a},] \var{-{}-attach-to} \var{"\emph{\flq file\frq}"}
    \\optional: Bindet eine Datei als Mailanhang ein (absoluter Pfad).
    \\ Diese Option kann auch mehrfach gesetzt werden.
    \\ Es können auch mehrere durch Semikolon (;) getrennte Dateien 
    angegeben werden.
  
  \item [\var{-C},] \var{-{}-charset} \var{"\emph{\flq charset\frq}"}
    \\optional: legt den Zeichensatz für die String-Kodierung
    fest. Fehlt dieser Parameter wird der Zeichensatz aus 
    \smalljump{MAILSENDCHARSET}{\var{MAILSEND\_CHARSET}} verwendet.
\end{itemize}

\subsubsection{\var{mailsend} Maildatei}
    Alternativ kann eine \smalljump{rfc5322}{RFC5322} kompatible
    Maildatei versendet werden. Es werden dann alle Parameter aus dieser 
    Datei verwendet. 
    
    Weitere Parameter (wie z.B. \var{-{}-to}) dürfen dann nicht verwendet werden!  

\begin {itemize}
  \item [\var{-m},] \var{-{}-mailfile} \var{"\emph{\flq file\frq}"}
    \\erforderlich: Name der Maildatei (absoluter Pfad).
  
  \item [\var{-A},] \var{-{}-account} \var{"\emph{\flq account\frq}"}
    \\optional: Legt den \smalljump{MAILSENDACCOUNT}{Kontonamen}
    fest über den die Mail versendet wird. Fehlt dieser Parameter wird das
    Konto mit dem \smalljump{MAILSENDACCOUNT}{Namen} \var{default} verwendet.
\end{itemize}

