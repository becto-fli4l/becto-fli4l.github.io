% Synchronized to r29817

\marklabel{sec:hdinstall}{
  \section{HD - Support For Harddisks, Flash-Cards, USB-Sticks, ...}
  }

\subsection {OPT\_HDINSTALL - Installation On A Harddisk/CompactFlash}
\configlabel{OPT\_HDINSTALL}{OPTHDINSTALL}

    fli4l supports a large variety of boot media (CD, HD, Network, 
    Compact-Flash,...). Floppy Disks are not supported anymore due to  
    filesize regressions.\\

    All steps necessary to install to a harddisk are explained below.\\

    The usual way is installation via a physical boot medium. Installing via 
    network boot is possible too. \var{OPT\_\-HDINSTALL} prepares the harddisk. 
    If boot medium and installation target share the same \verb*?BOOT_TYPE='hd'? 
    installation files will be transferred immedeately. If direct transfer is 
    not possible the files will be transferred later via scp or remote update 
    using Imonc.
    
    An Overview to the different harddisk installation variants A or B is found 
    at the \jump{VARIANTEN}{Beginning of the documentation for fli4l}. Please read 
    carefully before proceeding!

    \subsubsection{HD-Installation In Six Simple Steps}

\begin{enumerate}
    \item create a bootable fli4l medium with package \verb*?BASE? and \var{OPT\_\-HDINSTALL}. 
    This medium must be able to perform remote updates. Either \var{OPT\_\-SSHD}
    must be activated or \var{OPT\_\-IMOND} is set to \verb*?'yes'?. If additional 
    drivers not contained in the standard installation are necessary to access the 
    harddisk, configure \var{OPT\_\-HDDRV} as well.
    \item boot the router with the installation medium.
    \item log in to the router console and execute ``hdinstall.sh''.
    \item transfer the files syslinux.cfg, kernel, rootfs.img, opt.img and rc.cfg 
    via scp or Imonc to the router's /boot directory if prompted to. It is recommended to 
    work with two fli4l directories, one for the setup and a second for the hd 
    installation. In the HD version, set the variable \verb*?BOOT_TYPE='hd'? 
    and for the boot media type according to its type.
    
    \achtung{During remote update the files for the hd-version have to be copied!}
    \item unload boot medium, shut down the router and reboot again (by using 
    halt/reboot/poweroff). The router will boot from harddisk now.
    \item if problems occur please refer to the following text.
\end{enumerate}


    \subsubsection{HD-Installation Explained In Detail (including examples)}

    First, a router boot media containing the installation scripts and additional 
    drivers (eventually) has to be created. Activate \var{OPT\_\-HDINSTALL} in
    config/hd.txt and \var{OPT\_\-HDDRV} (only if additional drivers are needed).
    Please read the section on \var{OPT\_\-HDDRV} thoroughly!
    
    The variable \var{BOOT\_\-TYPE} in base.txt has to be set in accordance with the 
    chosen setup media. After all, a setup has to be performed\dots
    The variable \var{MOUNT\_\-BOOT} in base.txt has to set to \verb*?'rw'?,
    in order to allow saving new archives (*. img) loaded over the network.
    
    Then boot the router from this setup medium. Run the installation program by 
    executing ``hdinstall.sh'' at the fli4l sonsole . After answering a few questions
    the installion on the hard drive is performed. Eventually you will be prompted to 
    load files needed for the router via remote update.

    \achtung{Don't forget this remote update, otherwise the router won't boot from 
    harddisk. To reboot the routers after remote update use reboot/halt/poweroff,
    otherwise your remote update changes will be lost.}
    
    The installation script may be started directly at the router console or via ssh 
    from another PC. This way you have to log in by giving a password. As an ssh client
    you may use the freeware 'putty'.

\subsubsection{Configuration Of The Setup Bootmedia}

\begin{table}[htb]
  \begin{center}
    \begin{small}
    \begin{tabular}[h!]{lp{9cm}}
    \verb*?BOOT_TYPE? & set according to type of bootmedia for the installation\\
    \verb*?MOUNT_BOOT='rw'? & necessary to copy new archives (*.img) to the harddisk
        over network\\
    \verb*?OPT_HDINSTALL='yes'? & necessary to have the setup script and tools for  
        formatting of partitions on the bootmedia\\
    (\verb*?OPT_HDDRV='yes'?) & only necessary if harddisk needs special drivers\\
    \verb*?OPT_SSHD='yes'? & after preparation of the harddisk eventually files 
        have to be copied via remote update. You will either need sshd, imond 
        (\verb*?IMOND='yes'?) or another package allowing file transfers. 
    \end{tabular}
    \end{small}
    \caption{Configuration example of a setup media}
    \marklabel{tab:hd-setupdisk}{}
  \end{center}
\end{table}

    At this point network configuration has to be completed in order to be able to copy 
    files over the network later. Please do not activate DNS\_DHCP at this point because 
    this may cause all kinds of errors (the DHCP-server maybe already have a lease file for 
    the router to be installed). For a remote update via scp (package SSHD) please set up
    \verb*?OPT_SSHD='yes'?. As an alternative you may transfer files via IMOND. This needs 
    a complete and working configuration for DSL or ISDN. Please omit all packages not 
    mentioned, i.e. no  DNS\_DHCP, SAMBA\_LPD, LCD,Portforwarding a.s.o.\\

    In case that the installations stops with the error message
\begin{example}
\begin{verbatim}
    *** ERROR: can't create new partition table, see docu ***
\end{verbatim}
\end{example}
       several problems may have occurred:
        \begin{itemize}
        \item harddisk is in use, maybe by an installation canceled before.
        Reboot and try again.
        \item additional drivers are needed, see \verb*?OPT_HDDRV?
        \item hardware problems, see appendix.
        \end{itemize}
    In the last step the final configuration files can be set up and all other 
    packages needed may be added to the router.

\subsubsection{Examples For Completed Installations Type A and B:}

An example for each configuration is to be found in table
\ref{tab:hd-typ-a}.

\begin{table}[htb]
  \begin{center}
    \begin{small}
    \begin{tabular}{lp{9cm}}
    \verb*?BOOT_TYPE='hd'? & necessary because we are booting from harddisk now\\
    \verb*?MOUNT_BOOT='rw|ro|no'? &
                        choose one. For copying new fli4l archives to harddisk over 
                        network 'rw' is needed.\\

    \verb*?OPT_HDINSTALL='no'? & not needed after successful installation.\\


    \verb*?OPT_MOUNT? & activate only if a data partition was configured\\

    (\verb*?OPT_HDDRV='yes'?) & only necessary if harddisk can't be used without additional 
		       drivers.\\
    
    \end{tabular}
    \end{small}
    \caption{Example for an installation according to type A or B}
    \marklabel{tab:hd-typ-a}{}
  \end{center}
\end{table}

    Creation of a swap-partition will only be available if the router has less than 32MB 
    RAM and the installation target is NO flash media!

\subsection {OPT\_MOUNT - Automatic Mounting Of Filesystems}
\configlabel{OPT\_MOUNT}{OPTMOUNT}

    OPT\_MOUNT mounts data partitions created during installation to /data, file system 
    checks will be performed automatically when needed. CD-ROMs will be mounted to /cdrom 
    if a CD is inserted. For swap-partitions OPT\_MOUNT is not needed!

    \achtung{OPT\_MOUNT reads the configuration file hd.cfg on the boot-partition 
    and mounts partitions mentioned there. If \var{OPT\_\-MOUNT} was transferred via 
    remote update to an already installed router this file has to be edited manually.}
   
    \achtung{While booting from CD-ROM OPT\_MOUNT can't be used. The CD may be mounted 
    by setting \var{MOUNT\_\-BOOT='ro'}.}
    
    The file hd.cfg on the DOS-partition for a router according to type B with swap and 
    data partition looks like this (example):
    \begin{verbatim}
        hd_boot='sda1'
        hd_opt='sda2'
        hd_swap='sda3'
        hd_data='sda4'
        hd_boot_uuid='4A32-0C15'
        hd_opt_uuid='c1e2bfa4-3841-4d25-ae0d-f8e40a84534d'
        hd_swap_uuid='5f75874c-a82a-6294-c695-d301c3902844'
        hd_data_uuid='278a5d12-651b-41ad-a8e7-97ccbc00e38f'
    \end{verbatim}
    
    Just omit non-existent partitions, a router according to type A 
    without further partitions looks like this:
    \begin{verbatim}
        hd_boot='sda1'
        hd_boot_uuid='4863-65EF'
     \end{verbatim}

\subsection {OPT\_EXTMOUNT - Manual Mounting Of File Systems}
\configlabel{OPT\_EXTMOUNT}{OPTEXTMOUNT}

    OPT\_EXTMOUNT mounts data partitions to any chosen mountpoint in
    file system. This allows to mount file systems created manually 
    and for example provide a rsync-server directory.

\begin{description}
\config{EXTMOUNT\_N}{EXTMOUNT\_N}{EXTMOUNTN}

    Number of manually created data partitions to be mounted.

\config{EXTMOUNT\_x\_VOLUMEID}{EXTMOUNT\_x\_VOLUMEID}{EXTMOUNTxVOLUMEID}

    Device, label or UUID of the volume to be mounted. By executing
    'blkid' device, label and UUID of all volumes can be displayed.

\config{EXTMOUNT\_x\_FILESYSTEM}{EXTMOUNT\_x\_FILESYSTEM}{EXTMOUNTxFILESYSTEM}

    The file system used for the partition. fli4l supports isofs, fat, 
    vfat, ext2, ext3 und ext4 at the time of writing.\\
    (The default setting \verb*?EXTMOUNT_x_FILESYSTEM='auto'? automatically tries to determine 
    the file system used.)

\config{EXTMOUNT\_x\_MOUNTPOINT}{EXTMOUNT\_x\_MOUNTPOINT}{EXTMOUNTxMOUNTPOINT}

    The path (Mountpoint) to where the device should be mounted. It does not 
    have to exist and will be created automatically.

\config{EXTMOUNT\_x\_OPTIONS}{EXTMOUNT\_x\_OPTIONS}{EXTMOUNTxOPTIONS}

    Specify special options to be passed to the 'mount' command here.

% #EXTMOUNT_x_HOTPLUG# ##TRANSLATE## : FFL-827: translate new variable
\config{EXTMOUNT\_x\_HOTPLUG}{EXTMOUNT\_x\_HOTPLUG}{EXTMOUNTxHOTPLUG}

    \mtr{Wenn diese Variable den Wert `yes' enthält, ist es kein Fehler, wenn zur
    Boot-Zeit die Datenpartition nicht existiert. In diesem Fall wird davon
    ausgegangen, dass der zugehörige Datenträger fehlt und ggf. später
    eingebunden wird (z.B. via SATA-Hotplugging oder als USB-Stick). Das
    Aktivieren dieser Option erfordert zwingend \var{OPT\_AUTOMOUNT='yes'}. Des
    Weiteren muss zur Identifikation der gewünschten Datenpartition die
    eindeutige Kennung (UUID) des Dateisystems in \var{EXTMOUNT\_x\_VOLUMEID}
    eingetragen werden; andere IDs wie Gerätename oder Label werden \emph{nicht}
    unterstützt.}
% EOT

\end{description}

Example:
\begin{example}
\begin{verbatim}
       EXTMOUNT_1_VOLUMEID='sda2'        # device
       EXTMOUNT_1_FILESYSTEM='ext3'      # filesystem
       EXTMOUNT_1_MOUNTPOINT='/mnt/data' # mountpoint for device
       EXTMOUNT_1_OPTIONS=''             # extra mount options passed via mount -o
       EXTMOUNT_1_HOTPLUG='no'           # device must exist at boot time
\end{verbatim}
\end{example}


% #OPT_AUTOMOUNT# ##TRANSLATE## : FFL-827: translate new subsection
\subsection {OPT\_AUTOMOUNT -- automatisches Einhängen von Datenpartitionen}
\configlabel{OPT\_AUTOMOUNT}{OPTAUTOMOUNT}

    \mtr{\var{OPT\_AUTOMOUNT='yes'} erlaubt es, Datenpartitionen automatisch und
    dynamisch während der Laufzeit einzuhängen. Es gibt zwei
    Konfigurationsvarianten. Die erste arbeitet mit \var{OPT\_EXTMOUNT}
    zusammen und hängt nur Datenpartitionen ein, die beim Booten gefehlt haben.
    Die zweite ist unabhängig von \var{OPT\_EXTMOUNT} und hängt \emph{alle}
    lesbaren Datenpartitionen ein, egal ob bereits während des Bootens oder
    erst später. Steuern lässt sich das Verhalten mit Hilfe der Variablen
    \var{AUTOMOUNT\_UNKNOWN}:}

\begin{description}
\config{AUTOMOUNT\_UNKNOWN}{AUTOMOUNT\_UNKNOWN}{AUTOMOUNTUNKNOWN}

    \mtr{Diese Variable steuert, ob unbekannte Datenpartitionen eingehängt werden.
    Mit \var{AUTOMOUNT\_UNKNOWN='no'} werden nur Datenpartitionen dynamisch
    während der Laufzeit eingehängt, die einem \var{EXTMOUNT\_x}-Eintrag
    entsprechen. Dazu muss zusätzlich \var{EXTMOUNT\_x\_HOTPLUG='yes'}
    definiert sein, damit \var{OPT\_EXTMOUNT} nicht meckert, wenn die
    Datenpartition beim Booten fehlen sollte. Mit \var{AUTOMOUNT\_UNKNOWN='yes'}
    werden auch unbekannte Datenpartitionen eingehängt. Dies funktioniert aber
    nur, wenn das Dateisystem auf der Datenpartition eine eindeutige Kennung
    (UUID) besitzt. In diesem Fall wird die Datenpartition in dem Verzeichnis
    \texttt{/media/<UUID>} eingehängt (dieses Verzeichnis wird bei Bedarf
    erzeugt).

    Standard-Einstellung: \var{AUTOMOUNT\_UNKNOWN='no'}}

\config{AUTOMOUNT\_UNKNOWN\_OPTS}{AUTOMOUNT\_UNKNOWN\_OPTS}{AUTOMOUNTUNKNOWNOPTS}

    \mtr{Diese Variable gibt die \texttt{mount}-Optionen an, die bei unbekannten
    Datenpartitionen beim Einhängen verwendet werden. Ist die Datenpartition
    über \var{OPT\_EXTMOUNT} in der \texttt{/etc/fstab} identifizierbar, dann
    werden die hier angegebenen Optionen \emph{nicht} benutzt; vielmehr werden
    die Optionen im passenden \var{EXTMOUNT\_x\_OPTIONS}-Eintrag genutzt.

    Standard-Einstellung: \var{AUTOMOUNT\_UNKNOWN\_OPTS='ro'} (damit werden
    Schreibzugriffe auf unbekannte Datenpartitionen standardmäßig verhindert)}

\end{description}

    \mtr{Jede Datenpartition wird vor dem Einhängen mit Hilfe des für das jeweilige
    Dateisystem verfügbaren Prüfprogramms auf Fehler überprüft (\texttt{e2fsck}
    für ext2/ext3/ext4-Dateisysteme und \texttt{fsck.fat} für
    (V)FAT-Dateisysteme). Schlägt die Prüfung oder die automatische Korrektur
    fehl, wird das Dateisystem \emph{nicht} eingehängt, um Datenkorruption zu
    vermeiden.

    Wird ein Gerät entfernt, auf dem ein Dateisystem eingehängt war, wird dies
    nachträglich via \texttt{umount} ausgehängt. Natürlich können dabei
    womöglich noch nicht geschriebene Daten nicht mehr gesichert werden
    (schließlich ist der Datenträger nicht mehr da), aber immerhin kann nicht
    mehr versucht werden, auf den nicht mehr vorhandenen Datenträger weiter
    zuzugreifen. Die korrekte Vorgehensweise ist natürlich \emph{erst} das
    Dateisystem auszuhängen und \emph{dann} den Datenträger zu entfernen. Weil
    nicht alle Gerätetypen ein Entfernen verhindern, wenn das Dateisystem
    eingehängt ist (beispielsweise funktioniert dies gut bei CD-Laufwerken),
    muss man sich unter Umständen selbst um die korrekte Reihenfolge der+
    Aktionen kümmern.

    Alle Aktivitäten von \var{OPT\_AUTOMOUNT} werden in der Datei
    \texttt{/var/log/automount.log} protokolliert. Ein beispielhafter
    Ausschnitt einer solchen Log-Datei wird im Folgenden gezeigt. Zuerst kommt
    der Abschnitt, der die Aktivitäten für Datenpartitionen aufzeigt, die
    bereits während des Bootens verfügbar sind (\texttt{ACTION=change}):}

\begin{tiny}
\begin{verbatim}
[2015-04-25 00:33:35] [INFO   ] ACTION=change SUBSYSTEM=block DEVNAME=vda1 DEVPATH=/devices/pci0000:00/0000:00:08.0/virtio4/block/vda/vda1 MDEV=vda1
[2015-04-25 00:33:35] [INFO   ] TYPE: vfat
[2015-04-25 00:33:35] [INFO   ] UUID: 442e-93ba
[2015-04-25 00:33:35] [INFO   ] mount point: /media/442e-93ba
[2015-04-25 00:33:35] [ERROR  ] /dev/vda1 already mounted on /boot, giving up
[2015-04-25 00:33:35] [INFO   ] ACTION=change SUBSYSTEM=block DEVNAME=vda2 DEVPATH=/devices/pci0000:00/0000:00:08.0/virtio4/block/vda/vda2 MDEV=vda2
[2015-04-25 00:33:35] [INFO   ] TYPE: ext3
[2015-04-25 00:33:35] [INFO   ] UUID: 77ab35b3-029e-42c9-93a0-d197c01e6e89
[2015-04-25 00:33:35] [INFO   ] mount point: /media/77ab35b3-029e-42c9-93a0-d197c01e6e89
[2015-04-25 00:33:35] [INFO   ] /dev/vda2: clean, 671/26208 files, 57544/104420 blocks
[2015-04-25 00:33:35] [NOTICE ] /dev/vda2 mounted on /media/77ab35b3-029e-42c9-93a0-d197c01e6e89
[2015-04-25 00:33:36] [INFO   ] ACTION=change SUBSYSTEM=block DEVNAME=vda3 DEVPATH=/devices/pci0000:00/0000:00:08.0/virtio4/block/vda/vda3 MDEV=vda3
[2015-04-25 00:33:36] [INFO   ] TYPE: ext3
[2015-04-25 00:33:35] [INFO   ] UUID: 1580b80c-92b1-4492-abfa-92a12a7d2027
[2015-04-25 00:33:35] [INFO   ] mount point: /media/1580b80c-92b1-4492-abfa-92a12a7d2027
[2015-04-25 00:33:35] [ERROR  ] /dev/vda3 already mounted on /data, giving up
[2015-04-25 00:33:35] [INFO   ] ACTION=change SUBSYSTEM=block DEVNAME=vdb1 DEVPATH=/devices/pci0000:00/0000:00:0a.0/virtio5/block/vdb/vdb1 MDEV=vdb1
[2015-04-25 00:33:35] [INFO   ] TYPE: ext3
[2015-04-25 00:33:35] [INFO   ] UUID: 4c1a03e1-3a0c-4835-88dc-a51879def464
[2015-04-25 00:33:35] [INFO   ] mount point: /mnt/extra
[2015-04-25 00:33:35] [ERROR  ] /dev/vdb1 already mounted on /mnt/extra, giving up
[2015-04-25 00:33:35] [INFO   ] ACTION=change SUBSYSTEM=block DEVNAME=vdc1 DEVPATH=/devices/pci0000:00/0000:00:1f.0/virtio6/block/vdc/vdc1 MDEV=vdc1
[2015-04-25 00:33:35] [INFO   ] TYPE: vfat
[2015-04-25 00:33:35] [INFO   ] UUID: ba6e-9ebd
[2015-04-25 00:33:35] [INFO   ] mount point: /media/ba6e-9ebd
[2015-04-25 00:33:35] [INFO   ] fsck.fat 3.0.26 (2014-03-07)
[2015-04-25 00:33:35] [INFO   ] /dev/vdc1: 0 files, 0/32672 clusters
[2015-04-25 00:33:35] [NOTICE ] /dev/vdc1 mounted on /media/ba6e-9ebd
\end{verbatim}
\end{tiny}

    \mtr{Zwei Datenpartitionen wurden eingehängt (\texttt{/dev/vda2} und
    \texttt{/dev/vdc1}), davon wurden beide nicht via \var{OPT\_EXTMOUNT}
    konfiguriert und somit unterhalb von \texttt{/media} eingehängt. Die
    verbliebenen drei Datenpartitionen \texttt{/dev/vda1}, \texttt{/dev/vda3}
    und \texttt{/dev/vdb1} wurden bereits von anderen Boot-Skripten eingehängt
    und entsprechen der Boot- und der Datenpartition sowie einer
    benutzerdefinierten \var{OPT\_EXTMOUNT}-Datenpartition.

    Jetzt werden \texttt{/dev/vdb1} und \texttt{/dev/vdc1} ausgehängt
    (\texttt{ACTION=remove}; die Warnung, dass \texttt{/dev/vdb1} beim
    Aushängen nicht in der Volumen-Datenbank gefunden wurde, ist harmlos und
    weist darauf hin, dass diese Datenpartition bereits während des Bootens
    von \var{OPT\_EXTMOUNT} und nicht von \var{OPT\_AUTOMOUNT} eingehängt
    wurde)...}

\begin{tiny}
\begin{verbatim}
[2015-04-25 00:34:52] [INFO   ] ACTION=remove SUBSYSTEM=block DEVNAME=vdb1 DEVPATH=/devices/pci0000:00/0000:00:0a.0/virtio5/block/vdb/vdb1 MDEV=vdb1
[2015-04-25 00:34:52] [WARNING] /dev/vdb1 not found in volume database
[2015-04-25 00:34:52] [INFO   ] mount point: /mnt/extra
[2015-04-25 00:34:52] [NOTICE ] /dev/vdb1 unmounted from /mnt/extra
[2015-04-25 00:34:55] [INFO   ] ACTION=remove SUBSYSTEM=block DEVNAME=vdc1 DEVPATH=/devices/pci0000:00/0000:00:1f.0/virtio6/block/vdc/vdc1 MDEV=vdc1
[2015-04-25 00:34:55] [INFO   ] UUID: ba6e-9ebd
[2015-04-25 00:34:55] [INFO   ] mount point: /media/ba6e-9ebd
[2015-04-25 00:34:55] [NOTICE ] /dev/vdc1 unmounted from /media/ba6e-9ebd
\end{verbatim}
\end{tiny}

    \mtr{...und in umgekehrter Reihenfolge wieder eingehängt (\texttt{ACTION=add}):}

\begin{tiny}
\begin{verbatim}
[2015-04-25 00:35:14] [INFO   ] ACTION=add SUBSYSTEM=block DEVNAME=vdb1 DEVPATH=/devices/pci0000:00/0000:00:0b.0/virtio5/block/vdb/vdb1 MDEV=vdb1
[2015-04-25 00:35:14] [INFO   ] TYPE: vfat
[2015-04-25 00:35:14] [INFO   ] UUID: ba6e-9ebd
[2015-04-25 00:35:14] [INFO   ] mount point: /media/ba6e-9ebd
[2015-04-25 00:35:15] [INFO   ] fsck.fat 3.0.26 (2014-03-07)
[2015-04-25 00:35:15] [INFO   ] /dev/vdb1: 0 files, 0/32672 clusters
[2015-04-25 00:35:15] [NOTICE ] /dev/vdb1 mounted on /media/ba6e-9ebd
[2015-04-25 00:35:18] [INFO   ] ACTION=add SUBSYSTEM=block DEVNAME=vdc1 DEVPATH=/devices/pci0000:00/0000:00:0c.0/virtio6/block/vdc/vdc1 MDEV=vdc1
[2015-04-25 00:35:18] [INFO   ] TYPE: ext3
[2015-04-25 00:35:18] [INFO   ] UUID: 4c1a03e1-3a0c-4835-88dc-a51879def464
[2015-04-25 00:35:18] [INFO   ] mount point: /mnt/extra
[2015-04-25 00:35:18] [INFO   ] /dev/vdc1: recovering journal
[2015-04-25 00:35:18] [INFO   ] /dev/vdc1: clean, 11/16384 files, 7477/65488 blocks
[2015-04-25 00:35:18] [NOTICE ] /dev/vdc1 mounted on /mnt/extra
\end{verbatim}
\end{tiny}

    \mtr{Das ``unsaubere'' Aushängen des ext3-Dateisystems auf \texttt{/dev/vdc1}
    hat zu einer ``recovering journal''-Meldung beim Einhängen geführt, die
    aber nicht kritisch ist, da keine weiteren Fehler gefunden wurden.}
% EOT


\subsection {OPT\_HDSLEEP -- Setting Automatic Sleep Mode For Harddisks}
\configlabel{OPT\_HDSLEEP}{OPTHDSLEEP}

    A harddisk can power down after a certain time period without activity. 
    The disk will save power and operate quiet then. Accessing the harddisk 
    will cause it to automatically spin up again.

    \achtung{Not all harddisks tolerate frequent spinup. Don't set the time for 
    spindown too short. Older IDE-disks don't even have this function. This  
    setting is also senseless for Flash-Media.}

\begin{description}
\config{HDSLEEP\_TIMEOUT}{HDSLEEP\_TIMEOUT}{HDSLEEPTIMEOUT}

        The variable specifies after what time period without access the disk should 
        power down. It will power down after that time period and come up again 
        with the next access. Sleep timeouts can be specified in minutes from one to 20 
        and in periods of 30 minutes from half an hour up to 5 hours. A sleep timeout 
        of 21 or 25 minutes will be rounded to 30 minutes.
        Some harddisks ignore values too high and stop after some minutes then. 
        Please test the settings thoroughly because proper functioning depends
        on the hardware used!

\begin{example}
\begin{verbatim}
        HDSLEEP_TIMEOUT='2'              # wait 2 minutes until power down
\end{verbatim}
\end{example}

\end{description}

\subsection {OPT\_RECOVER -- Emergency Option}
\configlabel{OPT\_RECOVER}{OPTRECOVER}

        This variable determines if functions for creating an emergency option 
        will be available.
        If activated the option copies the command ``mkrecover.sh'' to the router. 
        By executing it you can activate the emergency option at the console. With package
        ``HTTPD'' installed the action of copying an existing installation to an emergency 
        instance can be achieved conveniantly in the menu ``recover''.

        To use the recovery installation choose ``r'' for recovery in the boot menu at 
        the next reboot.

\begin{example}
\begin{verbatim}
        OPT_RECOVER='yes'
\end{verbatim}
\end{example}

\subsection {OPT\_HDDRV - Additional Drivers For Harddisk Controllers}
\configlabel{OPT\_HDDRV}{OPTHDDRV}

    By setting \var{OPT\_\-HDDRV}='yes' you may activate drivers additionally needed.
    Generally this is NOT needed for IDE und SATA, package 'Base' will load all necessary files.

\begin{description}
\config{HDDRV\_N}{HDDRV\_N}{HDDRVN}
{
	Set the number of drivers to be loaded.
}

\config{HDDRV\_x}{HDDRV\_x}{HDDRVx}
{
        \var{HDDRV\_\-1} a.s.o. Set drivers to be chosen for the host-adapters
        in use. A list of all supported adapters is provided with the initial 
        config file for package hd.
}

\config{HDDRV\_x\_OPTION}{HDDRV\_x\_OPTION}{HDDRVxOPTION}
{
        With \var{HDDRV\_\-x\_\-OPTION} additional options can be passed that some 
        drivers need for proper operation (for example an IO-address). This 
        variable can be empty for the most drivers.
}

    In the \jump{sec:hd-errors}{Appendix} you may find an overview of the most common 
    errors concerning harddisk and CompactFlash operation.

    Example 1: Access to a SCSI-harddisk on an Adaptec 2940 controller

\begin{example}
\begin{verbatim}
   OPT_HDDRV='yes'             # install Drivers for Harddisk: yes or no
   HDDRV_N='1'                 # number of HD drivers
   HDDRV_1='aic7xxx'           # various aic7xxx based Adaptec SCSI 
   HDDRV_1_OPTION=''           # no need for options yet
\end{verbatim}
\end{example}

    Example 2: Accelerated IDE-Access with PC-Engines ALIX

\begin{example}
\begin{verbatim}
   OPT_HDDRV='yes'             # install Drivers for Harddisk: yes or no
   HDDRV_N='1'                 # number of HD drivers
   HDDRV_1='pata_amd'          # AMD PCI IDE/ATA driver (e.g. ALIX) 
   HDDRV_1_OPTION=''           # no need for options yet
\end{verbatim}
\end{example}

\end{description}
