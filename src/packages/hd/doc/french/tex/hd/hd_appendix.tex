% Do not remove the next line
% Synchronized to r30143

\marklabel{sec:hd-errors} 
{
  \section{HD~- Rapport d'erreur sur les disques durs/CompactFlashs}
}
    \textbf{Problème~:}

    \begin{itemize}
    \item Le routeur ne reconnaît pas le disque dur
    \end{itemize}

    Causes possibles~:
    \begin{itemize}
    \item Avec la variable \var{OPT\_\-HDDRV} on peut avoir
    besoin de pilotes supplémentaires pour le contrôleur HD. 
    \item Le disque est mal configuré dans le BIOS.
    \item Le contrôleur est désactivé ou défectueux.
    \item Lors de l'installation il est spécifié un mauvais disque
    \item Le contrôleur n'est pas pris en charge par fli4l. Certains contrôleurs
    nécessitent des pilotes spécifiques qui ne sont pas inclus dans fli4l.
    \end{itemize}

    \textbf{Problème~:}
    \begin{itemize}
    \item L'installation est interrompt
    \item Après une mise à jour du fichier opt-Archives le Routeur
      ne boot plus
    \item Il y a des messages d'erreurs au partitionnement ou au
      formatage du disque dur
    \end{itemize}

    Causes possibles~:
    \begin{itemize}
    \item Le câble du disques dur IDE est peut être inadapté ou trop long.
    \item Sur les disques durs plus anciens, le réglage de la vitesse du
    transfert/PIO-mode dans le BIOS ou sur le contrôleur est peut-être trop
    rapide pour le disque.
    \item Le chipset est inadapté.
    \end{itemize}

    Remarques~:
    \begin{itemize}
    \item Problèmes avec le DMA il peut éventuellement être résolu en indiquant
    la valeur \verb*?LIBATA_NODMA='no'~? (La valeur par défaut est 'yes')
    cela active le DMA avec les périphériques ATA.
    \end{itemize}

    \textbf{Problème~:}
    \begin{itemize}
    \item Après l'installation, fli4l ne démarre pas sur le disque dur
    \end{itemize}

    Causes possibles~:
    \begin{itemize}
    \item Si le démarrage à partir d'un module-CF a échoue vérifier si la CF
    a bien été reconnu en tant que LBA ou LARGE dans le Bios. Le réglage
    correct pour les petits modules de 512 Mo est NORMAL ou CHS.
    \item Si vous utilisez un contrôleur Adaptec 2940 avec un vieux BIOS et
    si l'affectation étendu pour les disques durs de plus de 1 Go est actif.
    Mettez à jour le BIOS de la carte SCSI ou affecter les micros-interrupteur.\\
    \achtung{En modifiant l'affectation les micros-interrupteur toutes les
    données du disque seront perdues~!}
    \end{itemize}

    \textbf{Problème~:}
    \begin{itemize}
    \item Message d'erreur de Windows lors de la préparation d'une carte CF~:
    \flqq{}Le lecteur (X:) ne comporte pas de partition FAT. [Annuler]\frqq{}
    \end{itemize}

    Causes possibles~:
    \begin{itemize}
    \item La carte a été retirée du lecteur trop tôt / sans être démontage
    (ou éjecté). Windows n'avait pas terminé l'écriture et le système de
    fichiers est endommagé. Préparer à nouveau la carte CompactFlash avec
    fli4l via HD-install.
    \end{itemize}